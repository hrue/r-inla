\documentclass[a4paper,11pt]{report}
\usepackage[scale={0.6,0.9},centering,includeheadfoot]{geometry}
\usepackage[T1]{fontenc}
\usepackage[utf8]{inputenc}
\usepackage{amsmath,upquote,Rd,Sweave}

\newcommand{\bA}{\mbox{\protect \boldmath $A$}}
\newcommand{\bB}{\mbox{\protect \boldmath $B$}}
\newcommand{\bF}{\mbox{\protect \boldmath $F$}}
\newcommand{\bD}{\mbox{\protect \boldmath $D$}}
\newcommand{\bW}{\mbox{\protect \boldmath $W$}}
\newcommand{\bC}{\mbox{\protect \boldmath $C$}}
\newcommand{\bG}{\mbox{\protect \boldmath $G$}}
\newcommand{\bK}{\mbox{\protect \boldmath $K$}}
\newcommand{\bQ}{\mbox{\protect \boldmath $Q$}}
\newcommand{\bM}{\mbox{\protect \boldmath $M$}}

\newcommand{\bby}{\mbox{\protect \boldmath $y$}}
\newcommand{\bbu}{\mbox{\protect \boldmath $u$}}
\newcommand{\bbv}{\mbox{\protect \boldmath $v$}}
\newcommand{\bbx}{\mbox{\protect \boldmath $x$}}

\title{\textbf{The R-INLA tutorial: SPDE models}\\
Warning: work in progress...\\
Suggestions are welcome to \texttt{elias@r-inla.org}}
\author{Elias T. Krainski}

\begin{document}
\maketitle

\begin{abstract}
  In this tutorial we presents how we fit models 
  to spatial point refered data, the called 
  geostatistical model, using INLA and SPDE. 
  Before a fast introduction on such models, 
  we start with a toy example to show details. 
  In forward examples we presents some funcionalities 
  to fit more complex models, such as  non-Gaussian,  
  models with two likelihoods, ... 
\end{abstract} 

\tableofcontents

%%% first section (need it on tutoriaḷ?)
%%% OOOPS: modify the .Rnw and not the .tex 



\chapter{Introduction}\label{ch:intro}

\section{Introduction and main results from \cite{lindgrenRL:2011}}\label{sec:intro}

The R source code for this introduction is available at  
\url{http://www.math.ntnu.no/inla/r-inla.org/tutorials/spde/R/spde-tutorial-introduction.R}

\subsection{Spatial variation}\label{sec:spatial}

A point-referenced dataset is made up of any 
data measured at known locations. 
These locations may be in any coordinate reference system, 
most often longitude and latitude. 
Point-referenced data are common in 
many areas of science. 
This type of data appears in mining, climate modeling, 
ecology, agriculture and elsewhere. 
If we want to incorporate the influence 
of location in a model for the data 
we need a model for geo-referenced data. 

A regression model can be built using 
the location's coordinates as a covariates. 
In some cases it will be necessary a 
complicated function based on the coordinates to 
adequately describe the effect of the location. 
For example, we can consider basis functions 
on the coordinates and use it as covariates 
in order to build a complex function. 
This model explictly the trend on the mean. 

Alternatively, it may be more natural to model 
explictly the variations of the outcome considering 
that it maybe similar at nearby locations. 
The first law of geography asserts: 
``Everything is related to everything else, 
but near things are more related than distant things'', 
\cite{tobler:1970}. 
We need a model that incorporates the property that an 
observation is more correlated with an observation 
collected at a neighboring location than with another 
observation that is collected from farther away. 
A spatially-structured random effects model 
incorporates such spatial dependency rather than 
simply one consistent spatial trend across a domain. 
Sometimes it is appropriate to include both terms in a model. 

The models that accounts for spatial dependency 
may be defined depending whether the locations 
are areas (cities for example) or points. 
The latter case is usually divided when 
considering the locations as fixed 
(geostatistics or point-referenced data) 
or random (point process), \cite{cressie:1993}. 

\subsection{The Gaussian random field}

To introduce some notation, let $s$ be any 
location in a study area and let $U(s)$ be 
the random (spatial) effect at that location. 
$U(s)$ is a stochastic process, with $s\in \bD$, 
where $\bD$ is the domain of the study area and $\bD\in \Re^d$. 
Suppose, for example, that $\bD$ is one country 
and we have data measured at geographical 
locations, over $d=2$ dimensions within this country. 

Suppose we assume that we have a realization 
of $u(\textbf{s}_i)$, $i=1,2,...,n$, a realization of $U(s)$ 
in $n$ locations. It is commonly assumed that 
$u(s)$ has a multivariate Gaussian distribution. 
If we assume that $U(s)$ is continuous over space, 
we have a continuously-indexed Gaussian field (GF). 
This implies that it is possible to collect these 
data at any finite set of locations within the 
study region. 
To complete the specification of the distribution of 
$u(s)$, it is necessary to define its mean and covariance. 

A very simple option is to define a 
correlation function based only on the Euclidean 
distance between locations. 
This assumes that if we have two pairs of 
points separated by the same distance $h$, 
both pairs have same degree of correlation. 
It is intuitive to choose a function that 
decrease with distance, $h$. 
\cite{abrahamsen:1997} presents Gaussian 
random fields and correlation functions. 

Now suppose that we have data $y_i$ observed 
at locations $\textbf{s}_i$, $i=1,...,n$. 
If an underlying GF generated these data, 
we can fit the parameters of this process 
considering $y(\textbf{s}_i) = u(\textbf{s}_i)$, where the 
observation $y(\textbf{s}_i)$ is assumed to be a 
realization of the GF at the location $\textbf{s}_i$. 
If we assume $y(\textbf{s}_i) = \mu + u(\textbf{s}_i)$, 
we have one more parameter to estimate. 
It is worth mentioning that the distribution for 
$u(s)$ at a finite number of points is considered 
a realization of a multivariate distribution. 
In this case, the likelihood function 
is the multivariate distribution with 
covariance $\Sigma$. 

In many situations we assume that there is an 
underlying GF but we cannot observe direclty. 
Instead we observe data with a measurement error 
\begin{align}\label{eq:toy0}
y(\textbf{s}_i) = u(\textbf{s}_i) + e_i .
\end{align}
It is common to assume that $e_i$ is 
independent of $e_j$ for all $i\neq j$ and 
$e_i \sim N(0, \sigma_e)$. 
This additional parameter, $\sigma_e$, measures 
the remaining noise and is called the nugget effect. 
The covariance of the marginal distribution 
of $y(s)$ at a finite number of locations 
is $\Sigma_y = \Sigma + \sigma^2_e\mathbf{I}$.
This is a short extension of the basic GF model,
and gives one additional parameter to estimate.    
For more about this model see 
\cite{diggleribeiro:2007}. 

To usual way to evaluate the likelihood function, 
which is just a multivariate Gaussian density for 
the model in Eq.~\ref{eq:toy0}, usually considers 
a Cholesky factorization of the covariance matrix. 
Because this matrix is a dense, 
this is a operation of order $O(n^3)$, 
so this is a 'big n problem'. 
Some software for geostatistical analysis 
uses an empirical variogram to fit the 
parameters of the correlation function. 
However, this option does not make any 
assumption about a likelihood function for the 
data or uses a multivariate Gaussian distribution 
for the spatially structured random effect. 
A good description of these techniques is 
available in \cite{cressie:1993}. 

When extending the model to deal with non-Gaussian 
data, it is usual to assume a likelihood for the 
data conditional on an unobserved random effect, 
which is GF to model the spatial dependence. 
Such spatial mixed effects model is under the 
the model based geostatistics approach, 
cite{diggleribeiro:2007}. 
It is possible to describe the model in Eq.~\ref{eq:toy0}
within a larger class of models, hierarchical models.
Suppose that we have observations $y_i$ for locations 
$\textbf{s}_i$, $i=1,...,n$. We start with 
\begin{equation}\begin{array}{c} 
y_i|\theta,\beta,u_i,\mathbf{F}_i \sim P(y_i|\mu_i,\phi) \\ 
\mathbf{u} \sim GF(0, \Sigma) 
\end{array}\end{equation}\label{eq:hm0} 
where $\mu_i = h(\mathbf{F}_i^{T}\beta + u_i)$, 
$\mathbf{F}$ is a matrix of covariates, 
$\mathbf{u}$ is the random effect, 
$\theta$ are parameters for the random effect, 
$\beta$ are covariate coefficients, 
$h()$ is a function mapping the linear predictor 
$\mathbf{F}_i^{T}\beta + u_i$ to E$(y_i) = \mu_i$ and 
$\phi$ is a dispersion parameter of the distribution, 
in the exponential family, which is assumed for $y_i$. 
To write the GF with a nugget effect, 
we replace $\beta_0$ with $\mathbf{F}_i^{T}\beta$, 
assume a Gaussian distribution for $y_i$, with 
variance $\sigma_e^2$ and $\mathbf{u}$ as a GF. 
However, at times one does consider a multivariate 
Gaussian distribution for the random effect, 
it becomes impractical to use covariance directly 
for model-based inference. 

In another area of spatial statistics, the 
analysis of areal data, there are models specified 
by conditional distributions that imply a joint 
distribution with a sparse precision matrix. 
These models are called Gaussian Markov random 
fields (GMRF) and a good reference is 
\cite{RueHeld:2005}. 
It is computationally easier to make Bayesian 
inference when we use a GMRF than when we use a 
GF, because the cost in computation of working 
with a sparse precision matrix in GMRF models is 
$O(n^{3/2})$.  
This makes it easier to conduct analyses 
with big 'n'. 

We can extend this basic hierarchical model in many ways,
and we return to some extensions later. 
If we know the properties of the GF, 
we can study all the practical models 
that contain or are based on, this random effect. 

\subsection{The Mat\'{e}rn covariance}\label{sec:matern} 

A very popular correlation function is the 
Mat\'ern correlation function. 
It has a scale parameter $\kappa>0$ 
and a smoothness parameter $\nu>0$. 
For two locations $\textbf{s}_i$ and $\textbf{s}_j$, the stationary 
and isotropic Mat\'ern correlation function is: 
\begin{equation}
Cor_M(U(\textbf{s}_i), U(\textbf{s}_j)) = 
\frac{2^{1-\nu}}{\Gamma(\nu)}
(\kappa \parallel \textbf{s}_i - \textbf{s}_j\parallel)^\nu 
K_\nu(\kappa \parallel \textbf{s}_i - \textbf{s}_j \parallel)
\end{equation}
where $\parallel . \parallel$ denotes 
the Euclidean distance and $K_\nu$ is the modified 
Bessel function of the second order. 
The Mat\'ern covariance function is
$\sigma_u Cor(U(\textbf{s}_i), U(\textbf{s}_j))$, where 
$\sigma_u$ is the marginal variance of the process. 

If we have a realization $u(s)$ from $U(s)$ 
at $n$ locations, $\textbf{s}_1, ..., \textbf{s}_n$, 
we can define its joint covariance matrix. 
Each entry of this joint covariance matrix $\Sigma$ is 
$\Sigma_{i,j} = \sigma_uCor_M(u(\textbf{s}_i), u(\textbf{s}_j))$. 
It is common to assume that $U(.)$ has a zero mean. 
We have now completely defined a multivariate 
distribution for $u(s)$. 

To gain a better feel about the Mat\'{e}rn 
correlation we can drawn samples from the 
RF process and look at it. 
A sample $\mathbf{u}$ is drawn considering 
$\mathbf{u} = \mathbf{Lz}$
where $\mathbf{L}$ is the Cholesky decomposition 
of the covariance at $n$ locations and 
$z$ is a vector with $n$ samples drawn from a 
standard Gaussian distribution. 
It implies that 
$\textrm{E}(\mathbf{u}) = 
\textrm{E}(\mathbf{Lz}) = \mathbf{L}\textrm{E}(\mathbf{z})=0$ 
and $Var(\mathbf{u}) = \mathbf{L}'\textrm{Var}(\mathbf{z})\mathbf{L}=\mathbf{L}'\mathbf{L}$. 
We define functions to do the sampling bellow 
\begin{knitrout}
\definecolor{shadecolor}{rgb}{0.969, 0.969, 0.969}\color{fgcolor}\begin{kframe}
\begin{alltt}
\hlstd{cMatern} \hlkwb{<-} \hlkwa{function}\hlstd{(}\hlkwc{h}\hlstd{,} \hlkwc{nu}\hlstd{,} \hlkwc{kappa}\hlstd{)} \hlcom{### Matern correlation}
    \hlkwd{besselK}\hlstd{(h} \hlopt{*} \hlstd{kappa, nu)} \hlopt{*}
        \hlstd{(h}\hlopt{*}\hlstd{kappa)}\hlopt{^}\hlstd{nu} \hlopt{/} \hlstd{(}\hlkwd{gamma}\hlstd{(nu)} \hlopt{*} \hlnum{2}\hlopt{^}\hlstd{(nu}\hlopt{-}\hlnum{1}\hlstd{))}
\hlcom{### function to sample from zero mean multivariate normal}
\hlstd{rmvnorm0} \hlkwb{<-} \hlkwa{function}\hlstd{(}\hlkwc{n}\hlstd{,} \hlkwc{cov}\hlstd{,} \hlkwc{L}\hlstd{=}\hlkwa{NULL}\hlstd{) \{}
    \hlkwa{if} \hlstd{(}\hlkwd{is.null}\hlstd{(L)) L} \hlkwb{<-} \hlkwd{chol}\hlstd{(cov)}
    \hlkwd{return}\hlstd{(}\hlkwd{crossprod}\hlstd{(L,} \hlkwd{matrix}\hlstd{(}\hlkwd{rnorm}\hlstd{(n}\hlopt{*}\hlkwd{ncol}\hlstd{(L)),} \hlkwd{ncol}\hlstd{(L))))}
\hlstd{\}}
\end{alltt}
\end{kframe}
\end{knitrout}
These steps are collected into the function 
\texttt{rMatern} that is available at 
\url{http://www.math.ntnu.no/inla/r-inla.org/tutorials/spde/R/spde-tutorial-functions.R}


In order to simplify the visualization of the 
properties, we consider a set of $n=249$ locations 
in the one-dimensional space from 0 to $25$. 
\begin{knitrout}
\definecolor{shadecolor}{rgb}{0.969, 0.969, 0.969}\color{fgcolor}\begin{kframe}
\begin{alltt}
\hlcom{### define locations and distance matrix}
\hlstd{loc} \hlkwb{<-} \hlnum{1}\hlopt{:}\hlnum{249}\hlopt{/}\hlnum{25}
\hlstd{mdist} \hlkwb{<-} \hlkwd{as.matrix}\hlstd{(}\hlkwd{dist}\hlstd{(loc))}
\end{alltt}
\end{kframe}
\end{knitrout}

We consider four values for the 
smoothness parameter $\nu$. 
The values for the $\kappa$ parameter 
was determined from the practical range 
expression $\sqrt{8\nu}/\kappa$, which is the 
distance that gives correlation near $0.13$. 
By combining the four values for $\nu$ with two 
values for the practical range we have 
eight parameter configurations. 
\begin{knitrout}
\definecolor{shadecolor}{rgb}{0.969, 0.969, 0.969}\color{fgcolor}\begin{kframe}
\begin{alltt}
\hlcom{### define parameters}
\hlstd{nu} \hlkwb{<-} \hlkwd{c}\hlstd{(}\hlnum{0.5}\hlstd{,}\hlnum{1}\hlstd{,}\hlnum{2}\hlstd{,}\hlnum{5}\hlstd{)}
\hlstd{pract.range} \hlkwb{=} \hlkwd{c}\hlstd{(}\hlnum{1}\hlstd{,}\hlnum{4}\hlstd{)}
\hlstd{kappa} \hlkwb{<-} \hlkwd{c}\hlstd{(}\hlkwd{sqrt}\hlstd{(}\hlnum{8}\hlopt{*}\hlstd{nu)}\hlopt{/}\hlstd{pract.range[}\hlnum{1}\hlstd{],}
           \hlkwd{sqrt}\hlstd{(}\hlnum{8}\hlopt{*}\hlstd{nu)}\hlopt{/}\hlstd{pract.range[}\hlnum{2}\hlstd{])}
\hlcom{### covariance parameter scenarios}
\hlstd{params} \hlkwb{<-} \hlkwd{cbind}\hlstd{(}\hlkwc{nu}\hlstd{=}\hlkwd{rep}\hlstd{(nu,} \hlkwd{length}\hlstd{(pract.range)),}
                \hlkwc{kappa}\hlstd{=kappa,}
                \hlkwc{r}\hlstd{=}\hlkwd{rep}\hlstd{(pract.range,} \hlkwc{each}\hlstd{=}\hlkwd{length}\hlstd{(nu)))}
\end{alltt}
\end{kframe}
\end{knitrout}

The sample value depends on the covariance  
matrix and on the noise considered 
from the standard Gaussian distribution, $z$. 
We consider a set of five vectors of size $n$ 
drawn from the standard Gaussian distribution. 
These five standard Gaussian were the same 
among the eight parameter configurations 
in order to keep track what the different 
parameter configuration are doing. 
\begin{knitrout}
\definecolor{shadecolor}{rgb}{0.969, 0.969, 0.969}\color{fgcolor}\begin{kframe}
\begin{alltt}
\hlcom{### sample error}
\hlkwd{set.seed}\hlstd{(}\hlnum{123}\hlstd{)}
\hlstd{z} \hlkwb{<-} \hlkwd{matrix}\hlstd{(}\hlkwd{rnorm}\hlstd{(}\hlkwd{nrow}\hlstd{(mdist)}\hlopt{*}\hlnum{5}\hlstd{),} \hlkwc{ncol}\hlstd{=}\hlnum{5}\hlstd{)}
\end{alltt}
\end{kframe}
\end{knitrout}

Therefore, we have a set of $40$ different 
realizations, five for each parameter configuration 
\begin{knitrout}
\definecolor{shadecolor}{rgb}{0.969, 0.969, 0.969}\color{fgcolor}\begin{kframe}
\begin{alltt}
\hlcom{### compute the correlated samples}
\hlstd{yy} \hlkwb{<-} \hlkwd{lapply}\hlstd{(}\hlnum{1}\hlopt{:}\hlkwd{nrow}\hlstd{(params),} \hlkwa{function}\hlstd{(}\hlkwc{j}\hlstd{) \{} \hlcom{## scenarios}
    \hlstd{v} \hlkwb{<-} \hlkwd{cMatern}\hlstd{(mdist, params[j,}\hlnum{1}\hlstd{], params[j,}\hlnum{2}\hlstd{])}
    \hlkwd{diag}\hlstd{(v)} \hlkwb{<-} \hlnum{1} \hlopt{+} \hlnum{1e-10}
    \hlkwd{return}\hlstd{(}\hlkwd{list}\hlstd{(}\hlkwc{params}\hlstd{=params[j,],} \hlcom{### parameter scenario}
                \hlkwc{y}\hlstd{=}\hlkwd{crossprod}\hlstd{(}\hlkwd{chol}\hlstd{(v), z)))} \hlcom{### compute sample}
\hlstd{\})}
\end{alltt}
\end{kframe}
\end{knitrout}

These samples are shown in the eight
plots in Figure~\ref{fig:maternsamples}.       

\begin{knitrout}
\definecolor{shadecolor}{rgb}{0.969, 0.969, 0.969}\color{fgcolor}\begin{kframe}
\begin{alltt}
\hlcom{### visualize}
\hlstd{(ry} \hlkwb{<-} \hlkwd{range}\hlstd{(}\hlkwd{unlist}\hlstd{(}\hlkwd{lapply}\hlstd{(yy, tail,} \hlnum{1}\hlstd{))))}
\hlkwd{par}\hlstd{(}\hlkwc{mfcol}\hlstd{=}\hlkwd{c}\hlstd{(}\hlnum{4}\hlstd{,}\hlnum{2}\hlstd{),} \hlkwc{mar}\hlstd{=}\hlkwd{c}\hlstd{(}\hlnum{2}\hlstd{,}\hlnum{2}\hlstd{,}\hlnum{1}\hlstd{,}\hlnum{.1}\hlstd{),} \hlkwc{mgp}\hlstd{=}\hlkwd{c}\hlstd{(}\hlnum{1.5}\hlstd{,}\hlnum{0.7}\hlstd{,}\hlnum{0}\hlstd{),} \hlkwc{las}\hlstd{=}\hlnum{1}\hlstd{)}
\hlkwa{for} \hlstd{(i} \hlkwa{in} \hlnum{1}\hlopt{:}\hlkwd{length}\hlstd{(yy)) \{} \hlcom{### each scenario}
    \hlkwd{plot}\hlstd{(loc, yy[[i]]}\hlopt{$}\hlstd{y[,}\hlnum{1}\hlstd{],} \hlkwc{ylim}\hlstd{=ry,}
         \hlkwc{xlab}\hlstd{=}\hlstr{''}\hlstd{,} \hlkwc{ylab}\hlstd{=}\hlstr{''}\hlstd{,} \hlkwc{type}\hlstd{=}\hlstr{'n'}\hlstd{,}
         \hlkwc{main}\hlstd{=}\hlkwd{as.expression}\hlstd{(}\hlkwd{bquote}\hlstd{(}\hlkwd{paste}\hlstd{(}
             \hlstd{nu}\hlopt{==}\hlkwd{.}\hlstd{(yy[[i]]}\hlopt{$}\hlstd{params[}\hlnum{1}\hlstd{]),} \hlstr{', '}\hlstd{,}
             \hlstd{kappa}\hlopt{==}\hlkwd{.}\hlstd{(}\hlkwd{round}\hlstd{(yy[[i]]}\hlopt{$}\hlstd{params[}\hlnum{2}\hlstd{],}\hlnum{2}\hlstd{)),} \hlstr{', '}\hlstd{,}
             \hlstd{r}\hlopt{==}\hlkwd{.}\hlstd{(yy[[i]]}\hlopt{$}\hlstd{params[}\hlnum{3}\hlstd{])))))}
    \hlkwa{for} \hlstd{(k} \hlkwa{in} \hlnum{1}\hlopt{:}\hlnum{5}\hlstd{)}
        \hlkwd{lines}\hlstd{(loc, yy[[i]]}\hlopt{$}\hlstd{y[,k],} \hlkwc{col}\hlstd{=k)} \hlcom{### each sample }
\hlstd{\}}
\end{alltt}
\end{kframe}
\end{knitrout}
\begin{figure}\centering
\begin{knitrout}
\definecolor{shadecolor}{rgb}{0.969, 0.969, 0.969}\color{fgcolor}
\includegraphics[width=0.97\textwidth]{figs/intromaternsamplesfig-1} 

\end{knitrout}
\caption{Five samples from the one-dimensional 
  Mat\'{e}rn correlation function for two different 
  ranges (each column of plots) and four different 
  value for the smoothness parameter (each line of plots).}
\end{figure}\label{fig:maternsamples}

One important point to observe 
in Figure~\ref{fig:maternsamples} 
is the main feature on the samples. 
It almost does not depends on the smoothness parameter. 
To see it consider one of the five samples 
(one of the colors) and compare it for 
different smoothness. 
Also, image if we add a noise on a smooth process, 
it will becomes hard them to distinguish what is 
due to noise from what is due to smoothness. 
Therefore, in practice we usually fix the 
smoothness parameter and add a noise term. 

\subsection{Simulation of a toy data set}\label{sec:simulatoy} 

We will now draw a sample from the model in Eq.~\ref{eq:toy0} 
and use in Section~\ref{sec:toyexample}. 
We consider a set of $n=100$ locations 
within a square of area one and bottom left and 
top right limits of (0,0) and (1,1). 
We choose a higher density of locations in 
the bottom left corner than in the top right corner. 
The \pkg{R} code to do this is:
\begin{knitrout}
\definecolor{shadecolor}{rgb}{0.969, 0.969, 0.969}\color{fgcolor}\begin{kframe}
\begin{alltt}
\hlstd{n} \hlkwb{<-} \hlnum{200}\hlstd{;}  \hlkwd{set.seed}\hlstd{(}\hlnum{123}\hlstd{)}
\hlstd{pts} \hlkwb{<-} \hlkwd{cbind}\hlstd{(}\hlkwc{s1}\hlstd{=}\hlkwd{sample}\hlstd{(}\hlnum{1}\hlopt{:}\hlstd{n}\hlopt{/}\hlstd{n}\hlopt{-}\hlnum{0.5}\hlopt{/}\hlstd{n)}\hlopt{^}\hlnum{2}\hlstd{,} \hlkwc{s2}\hlstd{=}\hlkwd{sample}\hlstd{(}\hlnum{1}\hlopt{:}\hlstd{n}\hlopt{/}\hlstd{n}\hlopt{-}\hlnum{0.5}\hlopt{/}\hlstd{n)}\hlopt{^}\hlnum{2}\hlstd{)}
\end{alltt}
\end{kframe}
\end{knitrout}
To get a (lower triangle) matrix of distances we call
\begin{knitrout}
\definecolor{shadecolor}{rgb}{0.969, 0.969, 0.969}\color{fgcolor}\begin{kframe}
\begin{alltt}
\hlstd{dmat} \hlkwb{<-} \hlkwd{dist}\hlstd{(pts)}
\end{alltt}
\end{kframe}
\end{knitrout}

We choose parameter values for the Mat\'ern as 
$\sigma^2_u=5$, $\kappa=7$ and $\nu=1$. 
We define the mean $\beta_0=10$  and 
the nugget parameter $\sigma^2_e=0.3$. 
We declare values for these parameters using
\begin{knitrout}
\definecolor{shadecolor}{rgb}{0.969, 0.969, 0.969}\color{fgcolor}\begin{kframe}
\begin{alltt}
\hlstd{beta0} \hlkwb{<-} \hlnum{10}\hlstd{; sigma2e} \hlkwb{<-} \hlnum{0.3}\hlstd{; sigma2u} \hlkwb{<-} \hlnum{5}\hlstd{; kappa} \hlkwb{<-} \hlnum{7}\hlstd{; nu} \hlkwb{<-} \hlnum{1}
\end{alltt}
\end{kframe}
\end{knitrout}

Now we need to sample from a multivariate distribution 
with constant mean equals $\beta_0$ and 
covariance $\sigma^2_eI + \Sigma$, which is the 
marginal covariance of the observations. 
\begin{knitrout}
\definecolor{shadecolor}{rgb}{0.969, 0.969, 0.969}\color{fgcolor}\begin{kframe}
\begin{alltt}
\hlstd{mcor} \hlkwb{<-} \hlkwd{as.matrix}\hlstd{(}\hlnum{2}\hlopt{^}\hlstd{(}\hlnum{1}\hlopt{-}\hlstd{nu)}\hlopt{*}\hlstd{(kappa}\hlopt{*}\hlstd{dmat)}\hlopt{^}\hlstd{nu} \hlopt{*}
                  \hlkwd{besselK}\hlstd{(dmat}\hlopt{*}\hlstd{kappa,nu)}\hlopt{/}\hlkwd{gamma}\hlstd{(nu))}
\hlkwd{diag}\hlstd{(mcor)} \hlkwb{<-} \hlnum{1}\hlstd{;   mcov} \hlkwb{<-} \hlstd{sigma2e}\hlopt{*}\hlkwd{diag}\hlstd{(n)} \hlopt{+} \hlstd{sigma2u}\hlopt{*}\hlstd{mcor}
\end{alltt}
\end{kframe}
\end{knitrout}
We can now sample considering the Cholesky factor times 
a unit variance noise and add the mean 
\begin{knitrout}
\definecolor{shadecolor}{rgb}{0.969, 0.969, 0.969}\color{fgcolor}\begin{kframe}
\begin{alltt}
\hlstd{L} \hlkwb{<-} \hlkwd{chol}\hlstd{(mcov);}   \hlkwd{set.seed}\hlstd{(}\hlnum{234}\hlstd{)}
\hlstd{y1} \hlkwb{<-} \hlstd{beta0} \hlopt{+} \hlkwd{drop}\hlstd{(}\hlkwd{crossprod}\hlstd{(L,} \hlkwd{rnorm}\hlstd{(n)))}
\end{alltt}
\end{kframe}
\end{knitrout}

We show these simulated data in a graph of 
the locations where the size of the points 
is proportional to the simulated values. 
Figure~\ref{fig:grf1} was produced with the code below 
\begin{knitrout}
\definecolor{shadecolor}{rgb}{0.969, 0.969, 0.969}\color{fgcolor}\begin{kframe}
\begin{alltt}
\hlkwd{par}\hlstd{(}\hlkwc{mar}\hlstd{=}\hlkwd{c}\hlstd{(}\hlnum{3}\hlstd{,}\hlnum{3}\hlstd{,}\hlnum{1}\hlstd{,}\hlnum{1}\hlstd{),} \hlkwc{mgp}\hlstd{=}\hlkwd{c}\hlstd{(}\hlnum{1.7}\hlstd{,} \hlnum{0.7}\hlstd{,} \hlnum{0}\hlstd{),} \hlkwc{las}\hlstd{=}\hlnum{1}\hlstd{)}
\hlkwd{plot}\hlstd{(pts,} \hlkwc{asp}\hlstd{=}\hlnum{1}\hlstd{,} \hlkwc{xlim}\hlstd{=}\hlkwd{c}\hlstd{(}\hlnum{0}\hlstd{,}\hlnum{1.2}\hlstd{),} \hlkwc{cex}\hlstd{=y1}\hlopt{/}\hlnum{10}\hlstd{)}
\hlstd{q} \hlkwb{<-} \hlkwd{quantile}\hlstd{(y1,} \hlnum{0}\hlopt{:}\hlnum{5}\hlopt{/}\hlnum{5}\hlstd{)}
\hlkwd{legend}\hlstd{(}\hlstr{'topright'}\hlstd{,} \hlkwd{format}\hlstd{(q,} \hlkwc{dig}\hlstd{=}\hlnum{2}\hlstd{),} \hlkwc{pch}\hlstd{=}\hlnum{1}\hlstd{,} \hlkwc{pt.cex}\hlstd{=q}\hlopt{/}\hlnum{10}\hlstd{)}
\end{alltt}
\end{kframe}
\end{knitrout}
\setkeys{Gin}{width=0.5\textwidth}
\begin{figure}\centering
\begin{knitrout}
\definecolor{shadecolor}{rgb}{0.969, 0.969, 0.969}\color{fgcolor}
\includegraphics[width=\maxwidth]{figs/introplot1-1} 

\end{knitrout}
\caption{The simulated toy example data.}
\end{figure}\label{fig:grf1}

This data will be used as a toy example in this tutorial. 
It is available in the \texttt{R-INLA} package and 
can be loaded by 
\begin{knitrout}
\definecolor{shadecolor}{rgb}{0.969, 0.969, 0.969}\color{fgcolor}\begin{kframe}
\begin{alltt}
\hlkwd{data}\hlstd{(SPDEtoy)}
\end{alltt}
\end{kframe}
\end{knitrout}

\subsection{The SPDE approach}\label{sec:spde} 

In this section we summarizes the main results 
in \cite{lindgrenRL:2011}. 
If your purpose does not includes understanding 
the methodology, you can skip this section. 
If you keep reading this section and have difficulty, 
do not be discouraged. You may still be able to use 
INLA effectively even if you have only a limited 
grasp of what's 'under the hood.' 

\cite{rueT:2002} proposed to approximate a continuous 
field using a Gaussian Markov Random Field - GMRF. 
This idea is not so strange since there are 
continuous random fields that are Markov. 
This is the case when the continuous field is a 
solution of a linear stochastic partial 
differential equation, see \cite{rozanov:1977}. 
\cite{lindgrenRL:2011} provided that the 
GRF with Mat\'{e}rn correlation is a solution 
for an stochastic partial differential equation (SPDE). 

If you want to get an intuition about the approach, 
we tried to provide it in this section. 
However, if you want to know all the details, 
they are in the Appendix of \cite{lindgrenRL:2011}. 
In few words it uses the Finite Element Method (FEM)
along with basis functions carefully chosen to 
preserve the sparse structure of the resulting precision 
matrix for the random field at a set of mesh nodes. 
This provides an explicit link between a 
continuous random field and a GMRF representation, 
which allows efficient computations. 

\subsubsection{First result}\label{sec:firstres}

Lindgren's first main provided that a GF with a generalized 
covariance function, obtained when $\nu \rightarrow 0$ in 
the Mat\'ern correlation function, is a solution of a SPDE. 
This extends the result obtained by \cite{besag:1981}. 
A more statistical way of considering this result is when 
taking a regular two-dimensional lattice with number of 
sites tending to infinity. 
In this case the full conditional at the site $ij$ has 
\begin{equation}
E(u_{ij}|u_{-ij}) = \frac{1}{a}(u_{i-1,j}+u_{i+1,j}+u_{i,j-1}+u_{i,j+1})
\end{equation}
and $Var(u_{ij}|u_{-ij}) = 1/a$ for $|a|>4$. 
In the representation using a precision matrix, 
we have, for a single site, only the upper right 
quadrant and with $a$ as the central element, such that 
\begin{equation}\label{eq:q0}
\begin{tabular}{|cc} 
-1  & \\
$a$ & -1 \\
\hline
\end{tabular}
\end{equation}

A GF $U(s)$ with Mat\'ern covariance is a solution 
to the following linear fractional SPDE 
\begin{equation}
(\kappa^2 - \Delta )^{\alpha/2}u(\mathbf{s}) = \bW(\mathbf{s}),\;\;\;
\mathbf{s} \in \mathbb{R}^d,\;\;\;\alpha=\nu+d/2,\;\;\kappa>0,\;\;\nu>0,
\end{equation}
\cite{lindgrenRL:2011} show that for $\nu=1$ and $\nu=2$, 
the GMRF representations are convolutions of (\ref{eq:q0}). 
So, for $\nu=1$ in that representation we have:
\begin{equation}
\begin{tabular}{|ccc} 
1  & & \\
-2$a$ & 2 & \\
4+$a^2$ & -2$a$ & 1 \\
\hline
\end{tabular}
\end{equation}\label{eq:q1}
and, for $\nu=2$:
\begin{equation}
\begin{tabular}{|cccc} 
-1  & & & \\
3$a$ & -3 & & \\
-3($a^2$+3) & 6$a$ & -3 & \\
$a$($a^2$+12) & -3($a^2$+3) & 3$a$ & -1 \\
\hline 
\end{tabular}
\end{equation}\label{eq:q2}

Interpreting the result intuitively, as 
the smoothness parameter $\nu$ 
increases the precision matrix in the GMRF 
representation becomes less sparse. 
Greater density of the matrix is because the conditional 
distributions depend on a wider neighborhood. 
The precision matrix for $\alpha=2$, 
$\bQ_2=\bQ_1\bC^{-1}\bQ_1$, is a standardized  
square of the precision matrix for 
$\alpha=1$, $\bQ_1$. 

However, the denser matrix does not imply that the 
conditional mean is an average over a wider neighborhood. 
The conceptual parallel is going from a 
first order random walk to a second order one. 
To understand this point let us consider the precision 
matrix for the first order random walk, 
its square and the precision matrix for the 
second order random walk. 
\begin{knitrout}
\definecolor{shadecolor}{rgb}{0.969, 0.969, 0.969}\color{fgcolor}\begin{kframe}
\begin{alltt}
\hlstd{(q1} \hlkwb{<-} \hlstd{INLA}\hlopt{:::}\hlkwd{inla.rw1}\hlstd{(}\hlkwc{n}\hlstd{=}\hlnum{5}\hlstd{))}
\end{alltt}
\begin{verbatim}
## 5 x 5 sparse Matrix of class "dgTMatrix"
##                    
## [1,]  1 -1  .  .  .
## [2,] -1  2 -1  .  .
## [3,]  . -1  2 -1  .
## [4,]  .  . -1  2 -1
## [5,]  .  .  . -1  1
\end{verbatim}
\begin{alltt}
\hlkwd{crossprod}\hlstd{(q1)} \hlcom{### same inner pattern as for RW2}
\end{alltt}
\begin{verbatim}
## 5 x 5 sparse Matrix of class "dsCMatrix"
##                    
## [1,]  2 -3  1  .  .
## [2,] -3  6 -4  1  .
## [3,]  1 -4  6 -4  1
## [4,]  .  1 -4  6 -3
## [5,]  .  .  1 -3  2
\end{verbatim}
\begin{alltt}
\hlstd{INLA}\hlopt{:::}\hlkwd{inla.rw2}\hlstd{(}\hlkwc{n}\hlstd{=}\hlnum{5}\hlstd{)}
\end{alltt}
\begin{verbatim}
## 5 x 5 sparse Matrix of class "dgTMatrix"
##                    
## [1,]  1 -2  1  .  .
## [2,] -2  5 -4  1  .
## [3,]  1 -4  6 -4  1
## [4,]  .  1 -4  5 -2
## [5,]  .  .  1 -2  1
\end{verbatim}
\end{kframe}
\end{knitrout}
We can see that the difference 
is only in the corners.

\subsubsection{Second result}\label{sec:secres}

It is common point data are not located on a grid, 
but instead are distributed irregularly.  
\cite{lindgrenRL:2011} provide a second set of results 
that provide a solution for the case irregular grids. 
They use the finite element method technique (FEM), 
a tool that is used widely in 
engineering and applied mathematics 
to solve differential equations. 

The domain can be divided into a set of 
non-intersecting triangles, which may be 
irregular, where any two triangles meet 
in at most a common edge or corner. 
The three corners of a triangle are named 
vertices or nodes. 
The solution for the SPDE and its properties 
will depend on the basis functions used. 
\cite{lindgrenRL:2011} choose basis functions 
carefully in order to preserve the sparse structure 
of the resulting precision matrix. 

The approximation is 
\[u(\mathbf{s}) = \sum_{k=1}^{m}\psi_k(\mathbf{s})w_k\]
where $\psi_k$ are basis functions, 
$w_k$ are Gaussian distributed weights, 
$k=1,...,m$ with $m$ the number of vertices in the triangulation.
Because $\psi_k$ is piecewise linear within each triangle, 
with $\psi_k$ is equal to 1 at vertices $k$ 
and 0 at all other vertices, 
we $w_k$ is the value of the field at the vertex $k$. 
A stochastic weak solution was considered to show that 
the joint distribution for the weights determines 
the full distribution in the continuous domain. 
These weights can be interpolated for 
any point inside the triangulated domain. 

We will now focus on the resulting precision matrix. 
It does consider the triangulation and the basis functions. 
It matches the first result when applying for a regular grid. 
Consider the set of $m\times m$ matrices 
$\bC$, $\bG$ and $\bK_{\kappa}$ with entries 
\begin{equation}
\bC_{i,j} = \langle \psi_i, \psi_j\rangle, \;\;\;\;\;
\bG_{i,j} = \langle \nabla \psi_i, \nabla \psi_j \rangle, \;\;\;\;\;
(\bK_{\kappa})_{i,j} = \kappa^2 C_{i,j} + G_{i,j}\;.
\end{equation}
The precision matrix $\bQ_{\alpha,\kappa}$ 
as a function of $\kappa^2$ and $\alpha$ can be written as 
\begin{equation}\label{eq:Qalpha}\begin{array}{c}
\bQ_{1,\kappa} = \bK_{\kappa}, \\
\bQ_{2,\kappa} = \bK_{\kappa}\bC^{-1}\bK_{\kappa}, \\
\bQ_{\alpha,\kappa} = \bK_{\kappa}\bC^{-1}Q_{\alpha-2,\kappa}\bC^{-1}\bK_{\kappa}, 
\;\;\;\mbox{for} \;\;\alpha = 3,4,...\;.
\end{array}\end{equation}
The actual $\bK_{\kappa}$ matrix consider 
\[\tilde{\bC}_{i,j} = \langle \psi_i, 1 \rangle\]
instead, which is common when working with FEM.
Since $\tilde{\bC}$ is diagonal  
$\bK_{\kappa}$ is as sparse as $\bG$. 

The projection of the weight for any 
location inside the mesh domain 
considers a linear interpolation in 2D. 
It uses the barycentric coordinates of the point 
with respect to the coordinates of the triangle vertices. 
For this particular case they are known also as 
areal coordinates. 
When a point is inside a triangle 
we have three non-zero values in the corresponding line of $\bA$. 
When it is along an edge, we have two non-zeros 
and when the point is on top of a triangle vertex 
we have only one non-zero which equals one. 



We can gain intuition about this result by 
considering the structure of each matrix 
which is detailed in Appendix A.2 in \cite{lindgrenRL:2011}. 
It may be easier to understand it by considering 
the plots in Figure~\ref{fig:mesh0}. 
In this figure we have a mesh
with 8 nodes,
shown in thicker border lines. 
The corresponding dual mesh form 
a collection of polygons 
around each mesh vertex. 

\begin{knitrout}
\definecolor{shadecolor}{rgb}{0.969, 0.969, 0.969}\color{fgcolor}\begin{kframe}
\begin{alltt}
\hlkwd{par}\hlstd{(}\hlkwc{mfrow}\hlstd{=}\hlkwd{c}\hlstd{(}\hlnum{1}\hlstd{,}\hlnum{3}\hlstd{),} \hlkwc{mar}\hlstd{=}\hlkwd{c}\hlstd{(}\hlnum{2}\hlstd{,}\hlnum{2}\hlstd{,}\hlnum{1}\hlstd{,}\hlnum{1}\hlstd{))}
\hlkwd{plot}\hlstd{(mesh,} \hlkwc{asp}\hlstd{=}\hlnum{1}\hlstd{,} \hlkwc{lwd}\hlstd{=}\hlnum{2}\hlstd{,} \hlkwc{edge.color}\hlstd{=}\hlnum{1}\hlstd{)}
\hlkwd{box}\hlstd{();} \hlkwd{axis}\hlstd{(}\hlnum{1}\hlstd{);} \hlkwd{axis}\hlstd{(}\hlnum{2}\hlstd{)}
\hlkwd{points}\hlstd{(mesh}\hlopt{$}\hlstd{loc,} \hlkwc{cex}\hlstd{=}\hlnum{3}\hlstd{)}
\hlkwd{text}\hlstd{(mesh}\hlopt{$}\hlstd{loc[,}\hlnum{1}\hlstd{]}\hlopt{-}\hlkwd{rep}\hlstd{(}\hlkwd{c}\hlstd{(}\hlnum{0}\hlstd{,}\hlnum{0.1}\hlstd{),}\hlkwd{c}\hlstd{(m}\hlopt{-}\hlnum{1}\hlstd{,}\hlnum{1}\hlstd{))}\hlopt{*}\hlstd{s,}
     \hlstd{mesh}\hlopt{$}\hlstd{loc[,}\hlnum{2}\hlstd{]}\hlopt{+}\hlnum{.2}\hlopt{*}\hlstd{s,} \hlnum{1}\hlopt{:}\hlstd{m,} \hlkwc{cex}\hlstd{=}\hlnum{2}\hlstd{)}

\hlkwd{plot}\hlstd{(dmesh,} \hlkwc{asp}\hlstd{=}\hlnum{1}\hlstd{,} \hlkwc{lwd}\hlstd{=}\hlnum{2}\hlstd{,} \hlkwc{main}\hlstd{=}\hlstr{'Dual mesh overlayed'}\hlstd{)}
\hlkwd{plot}\hlstd{(mesh,} \hlkwc{add}\hlstd{=}\hlnum{TRUE}\hlstd{)}
\hlkwd{box}\hlstd{();} \hlkwd{axis}\hlstd{(}\hlnum{1}\hlstd{);} \hlkwd{axis}\hlstd{(}\hlnum{2}\hlstd{)}
\hlkwd{points}\hlstd{(mesh}\hlopt{$}\hlstd{loc,} \hlkwc{cex}\hlstd{=}\hlnum{3}\hlstd{)}

\hlkwd{plot}\hlstd{(mesh,} \hlkwc{asp}\hlstd{=}\hlnum{1}\hlstd{,} \hlkwc{lwd}\hlstd{=}\hlnum{2}\hlstd{,} \hlkwc{edge.color}\hlstd{=}\hlnum{1}\hlstd{,} \hlkwc{main}\hlstd{=}\hlstr{''}\hlstd{)}
\hlkwd{title}\hlstd{(}\hlkwc{main}\hlstd{=}\hlstr{'Mesh and points'}\hlstd{)}
\hlkwd{box}\hlstd{();} \hlkwd{axis}\hlstd{(}\hlnum{1}\hlstd{);} \hlkwd{axis}\hlstd{(}\hlnum{2}\hlstd{)}
\hlkwd{points}\hlstd{(mesh}\hlopt{$}\hlstd{loc,} \hlkwc{cex}\hlstd{=}\hlnum{3}\hlstd{)}
\hlkwd{points}\hlstd{(pts,} \hlkwc{pch}\hlstd{=}\hlnum{8}\hlstd{,} \hlkwc{cex}\hlstd{=}\hlnum{2}\hlstd{,} \hlkwc{lwd}\hlstd{=}\hlnum{2}\hlstd{)}
\hlkwd{text}\hlstd{(pts[,}\hlnum{1}\hlstd{], pts[,}\hlnum{2}\hlstd{]}\hlopt{+}\hlnum{0.2}\hlopt{*}\hlstd{s,} \hlnum{1}\hlopt{:}\hlstd{n,} \hlkwc{cex}\hlstd{=}\hlnum{2}\hlstd{)}

\hlstd{A} \hlkwb{<-} \hlkwd{as}\hlstd{(}\hlkwd{Matrix}\hlstd{(}\hlkwd{round}\hlstd{(}\hlkwd{as.matrix}\hlstd{(A),} \hlnum{10}\hlstd{)),} \hlcom{### force zero as zero}
        \hlstr{'dgTMatrix'}\hlstd{)}
\hlstd{cc} \hlkwb{<-} \hlkwd{as}\hlstd{(fem}\hlopt{$}\hlstd{c0,} \hlstr{'dgTMatrix'}\hlstd{)}
\hlstd{gg} \hlkwb{<-} \hlkwd{as}\hlstd{(fem}\hlopt{$}\hlstd{g1,} \hlstr{'dgTMatrix'}\hlstd{)}
\hlkwd{library}\hlstd{(gridExtra)}
\hlkwd{library}\hlstd{(latticeExtra)}
\hlkwd{grid.arrange}\hlstd{(}
    \hlkwd{plot}\hlstd{(cc,} \hlkwc{colorkey}\hlstd{=}\hlnum{FALSE}\hlstd{,} \hlkwc{xlab}\hlstd{=}\hlstr{''}\hlstd{,} \hlkwc{ylab}\hlstd{=}\hlstr{''}\hlstd{,} \hlkwc{sub}\hlstd{=}\hlstr{''}\hlstd{)} \hlopt{+}
    \hlkwd{layer}\hlstd{(}\hlkwd{panel.text}\hlstd{(cc}\hlopt{@}\hlkwc{j}\hlopt{+}\hlnum{1L}\hlstd{, cc}\hlopt{@}\hlkwc{i}\hlopt{+}\hlnum{1L}\hlstd{,} \hlkwd{paste0}\hlstd{(}\hlkwd{round}\hlstd{(cc}\hlopt{@}\hlkwc{x}\hlstd{)),}
                     \hlkwc{col}\hlstd{=}\hlkwd{gray}\hlstd{(cc}\hlopt{@}\hlkwc{x}\hlopt{>}\hlnum{30}\hlstd{))),}
    \hlkwd{plot}\hlstd{(gg,} \hlkwc{colorkey}\hlstd{=}\hlnum{FALSE}\hlstd{,} \hlkwc{xlab}\hlstd{=}\hlstr{''}\hlstd{,} \hlkwc{ylab}\hlstd{=}\hlstr{''}\hlstd{,} \hlkwc{sub}\hlstd{=}\hlstr{''}\hlstd{)} \hlopt{+}
    \hlkwd{layer}\hlstd{(}\hlkwd{panel.text}\hlstd{(gg}\hlopt{@}\hlkwc{j}\hlopt{+}\hlnum{1L}\hlstd{, gg}\hlopt{@}\hlkwc{i}\hlopt{+}\hlnum{1L}\hlstd{,} \hlkwd{round}\hlstd{(gg}\hlopt{@}\hlkwc{x}\hlstd{,}\hlnum{2}\hlstd{),} \hlkwc{col}\hlstd{=}\hlkwd{gray}\hlstd{(}\hlkwd{abs}\hlstd{(gg}\hlopt{@}\hlkwc{x}\hlstd{)}\hlopt{>}\hlnum{1.5}\hlstd{))),}
    \hlkwd{plot}\hlstd{(A,} \hlkwc{colorkey}\hlstd{=}\hlnum{FALSE}\hlstd{,} \hlkwc{xlab}\hlstd{=}\hlstr{''}\hlstd{,} \hlkwc{ylab}\hlstd{=}\hlstr{''}\hlstd{,} \hlkwc{sub}\hlstd{=}\hlstr{''}\hlstd{)} \hlopt{+}
    \hlkwd{layer}\hlstd{(}\hlkwd{panel.text}\hlstd{(A}\hlopt{@}\hlkwc{j}\hlopt{+}\hlnum{1L}\hlstd{, A}\hlopt{@}\hlkwc{i}\hlopt{+}\hlnum{1L}\hlstd{,} \hlkwd{round}\hlstd{(A}\hlopt{@}\hlkwc{x}\hlstd{,} \hlnum{2}\hlstd{),}
                     \hlkwc{col}\hlstd{=}\hlkwd{gray}\hlstd{(A}\hlopt{@}\hlkwc{x}\hlopt{>}\hlnum{0.5}\hlstd{))),} \hlkwc{ncol}\hlstd{=}\hlnum{3}\hlstd{)}
\end{alltt}
\end{kframe}
\end{knitrout}
\setkeys{Gin}{width=0.99\textwidth}
\begin{figure}\centering
\begin{knitrout}
\definecolor{shadecolor}{rgb}{0.969, 0.969, 0.969}\color{fgcolor}
\includegraphics[width=0.99\linewidth]{figs/intromesh0fig-1} 

\includegraphics[width=0.99\linewidth]{figs/intromesh0fig-2} 

\end{knitrout}
\caption{A mesh and its nodes identified (top left), 
  the dual mesh polygons (top mid) and
  the mesh with some points identified (top right). 
  The associated $\bC$ matrix (bottom left), 
  $\bG$ matrix (bottom mid) and $\bA$ matrix (bottom right).} 
\end{figure}\label{fig:mesh0}

The $\tilde{\bC}$ matrix is diagonal for $\tilde{\bC}_{ii}$ 
equals the area of the polygons formed from the dual mesh. 
The $\tilde{\bC}_{ii}$ is equal the sum of one third 
the area of each triangle that the vertice $i$ is part of. 
Notice that each polygon around each mesh node is formed 
by one third of the triangles that it is part of. 

The $\bG$ matrix reflects the connectivity of the mesh nodes. 
Nodes not connected by edges have corresponding entry as zero. 
The values do not depend on the size of the triangles as 
they are scaled by the area of the triangles. 
For more detailed information, 
see A.2 in \cite{lindgrenRL:2011}. 

We have seen that the resulting precision matrix for 
increasing $\nu$ is a convolution of the precision matrix 
for $\nu-1$ with a scaled $\bK_{\kappa}$. 
It still implies denser the precision matrix 
when working with $\kappa\bC + \bG$.



The $\bQ$ precision matrix is generalized for a fractional values 
of $\alpha$ (or $\nu$) using a Taylor approximation. 
See the author's discussion response in \cite{lindgrenRL:2011}. 
From this approximation, we have the polynomial of 
order $p=\lceil \alpha \rceil$ for the precision matrix 
\begin{equation}\label{eq:Qfrac}
\bQ = \sum_{i=0}^p b_i \bC(\bC^{-1}\bG)^i.
\end{equation}
For $\alpha=1$ and $\alpha=2$ we have (\ref{eq:Qalpha}). 
For $\alpha=1$, we have $b_0=\kappa^2$ and $b_1=1$, 
and for $\alpha=2$, we have $b_0=\kappa^4$, 
$b_1=\alpha\kappa^4$ and $b_2=1$. 
For fractional $\alpha=1/2$, 
$b_0=3\kappa/4$ and $b_1=\kappa^{-1}3/8$. 
And for $\alpha=3/2$ ($\nu=0.5$, the exponential case), 
$b_0=15\kappa^3/16$, $b_1=15\kappa/8$, 
$b_2=15\kappa^{-1}/128$. 
Using these results combined with recursive construction, 
for $\alpha>2$, we have GMRF approximations for all positive 
integers and half-integers. 


\input{spde-tutorial-likelihoodintro}
%%% next sections are one file each
\input{spde-tutorial-toy}
\input{spde-tutorial-rain}
\input{spde-tutorial-semicontinuous}
\input{spde-tutorial-jcovar}
\input{spde-tutorial-nonstationar}

\bibliographystyle{apalike}
\bibliography{spde-tutorial}

\end{document}

