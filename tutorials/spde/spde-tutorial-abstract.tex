
\begin{abstract}
This tutorial will show you how to fit models 
that contains at least one effect specified 
from an SPDE using the `R-INLA`. 
Up to now, it can an SPDE based model can be applied to model 
random effects over continuous one- or two- dimensional domains. 
However, the theory works for higher dimensional cases. 
The usual application is data whose geographical 
location is explicitly considered in the analysis. 
This tutorial explores 'R-INLA' functionalities by using examples. 
It starts with simple models and increases in complexity. 

In Chapter~\ref{ch:intro} we will introduce 
the Gaussian random fields and the SPDE framework. 
We consider the Mat\'{e}rn random fields characteristics and 
introduce the main results in \cite{lindgrenRL:2011} 
intuitively linking to images of the matrices involved. 
We work with a toy example and work through 
some examples of building a mesh. 

We consider three detailed examples in Chapter~\ref{ch:ngns}. 
They address non-Gaussian data, survival analysis and 
inclusion of covariates in the covariance parameters. 
The example on accumulated rainfall includes code to compute 
geographical covariates and also include smoothed effect. 
The example case we consider the parametric Weibull case and 
also the non-parametric Cox proportional hazard case where it 
can be seen as a Poisson regression. 
The example on covariates in the covariance shows how 
it works in a simulated example. 

In Chapter~\ref{ch:manipula} we have a collection of examples 
were copy random fields to model two or more outcomes jointly. 
It includes a measurement error model in order to account for 
joint model of a covariate which can be misaligned over space. 
A coregionalization model consider the case for three outcomes 
were the fist outcome is in the linear predictor for the second 
one and both are in the predictor for the third outcome. 
A example considering part or the entire linear predictor from one 
outcome in a linear predictor to another one is also considered 
in this chapter.  

The log Cox point process model is considered in Chapter~\ref{ch:lcox}. 
In this case we show how to fit a Log-cox point process using 
the direct approximation for the likelihood. 
We also take the opportunity to show how to consider the 
preferential sampling modelling the locations jointly. 

Finally, Chapter~\ref{ch:spacetime} presents several cases to 
example analysis of space-time data. 
We start by an example having discrete time domain and an example 
having continuous time domain and consider the case where the temporal 
domain can be lowered by having temporal knots and temporal function basis. 
We have the space-time version of the coregionalization model to 
model three outcomes jointly. 
The space-time model is also applied for modeling regression coefficients 
in a dynamic regression example having the regression coefficients varying over space-time. 
We consider the space-time version of the log-Cox point process model for 
a dataset and also consider it for the case of a large point process data.

Since this tutorial is more a collection of examples, one should 
start with the tutorial marked as \textbf{Read this first!} at the 
tutorials link in the R-INLA web page, \url{http://www.r-inla.org}, 
more precisely at  
\url{http://www.r-inla.org/examples/tutorials/spde-tutorial-from-jss}.
If you are in a rush to fit a simple geostatistical model, 
we made a short tutorial without the details as a vignette in the \textbf{\textsf{INLA}}. Thus one can have it just typing   
\texttt{vignette(SPDEhowto)} for a two dimensional example 
or \texttt{vignette(SPDE1d)} for a one dimensional example. 
We built a Shiny application to help one to understand the mesh building. 
It depends on the \textbf{\textsf{shiny}} package. 
This application opens by typing \texttt{demo(mesh2d)}. 

\end{abstract}
