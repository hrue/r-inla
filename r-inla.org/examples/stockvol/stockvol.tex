\documentclass[a4paper,11pt]{article}
\usepackage[compat2]{geometry}
\usepackage{amstext}
\usepackage{listings}
\usepackage{amsmath,amssymb}

\def\mm#1{\ensuremath{\boldsymbol{#1}}} % version: amsmath


\begin{document}
\bibliographystyle{apalike}

\section*{Stochastic volatility models}

 The data consist in 945 observed logarithms of the daily
    difference of the dollar-pound exchange rate from October 1st, to
    June 28th, 1985. 
    We analyse this data set using a univariate stochastic volatility
    model (\cite{book1000}).  The likelihood of the data, conditional
    on the latent variables is:
    \begin{equation}\label{vol_gaus}
        y_t|\eta_t\sim\mathcal{N}(0,\exp(\eta_t)), \quad t=0,\dots,n_d-1
    \end{equation}
    and the model for the latent variables:
    \begin{equation}\label{vol_latent}
        \eta_t=\mu+f_t \quad t=0,\quad,n_{\eta}-1
    \end{equation}
    where $\mu$ is an unknown common mean with vague Gaussian prior
    and $\mm{f}=(f_0,\dots,f_{n_{\eta}-1})$ is modelled as an auto
    regressive process of order $1$ (AR1) with persistence parameter
    $\phi\in(-1,1)$ to ensure stationarity, and precision parameter
    $\lambda_f$.
  The model has two hyperparameters, $(\log\lambda_f,\phi)$. We
    re-parametrise the persistence parameter $\phi$ as
    \[
    \kappa=\text{logit}\left(\frac{\phi+1}{2}\right)
    \]
    and assign the following prior distributions
    \[
    \begin{array}{l}
        \log\lambda_f\sim\text{LogGamma}(1,0.0005)\\
        \kappa\sim\mathcal{N}(0,1/0.0001)\\
    \end{array}
    \]
\subsection*{Student-$t$  distribution}
An alternative model for the response variable $y_t$ is a
Student-$t$. This allows heavier tail, a feature which is often
observed in financial time series. The observation model in equation
(\ref{vol_gaus}) then becomes
\begin{equation}\label{vol_stud}
    y_t = \exp(\eta_t/2)\ \mathcal{T}_t(\nu)\quad t=1,\dots,T
\end{equation}
where $\mathcal{T}_t(\nu)$ is a random variable having a Student-$t$
distribution having $\nu$ degree of freedom and standardised so that
its variance is $1$ for any value of $\nu>2$.
\subsection*{Student-$t$  NIG distribution}

Yet another model is the \emph{normal inverse Gaussian} (NIG)
distribution, for which
\begin{equation}\label{eq4}%
    y_{t} = \exp(\eta_{t}/2)\; NIG, \quad t=1,\ldots, T
\end{equation}
where $NIG$ is a standardised NIG distribution with two parameters,
which (essentially) are skewness and shape-parameters.

\small\bibliography{../mybib} \newpage


\end{document}


% LocalWords: 

%%% Local Variables: 
%%% TeX-master: t
%%% End: 
