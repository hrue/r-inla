\documentclass[a4paper,11pt]{article}
\usepackage[scale={0.8,0.9},centering,includeheadfoot]{geometry}
\usepackage{amstext}
\usepackage{listings}
\begin{document}

\section*{PC prior for precision}

\subsection*{Parametrization}
The PC prior for the precision $\tau$ has density $$
\pi(\tau) = \frac{\lambda}{2}\tau^{-3/2} \exp\left(-\lambda \tau^{-1/2}\right), \quad \tau >0
$$
for $\lambda>0$ where
\begin{displaymath}
    \lambda = -\frac{\ln(\alpha)}{u}
\end{displaymath}
and $(u, \alpha{})$ are the parameters to this prior. The
interpretation of $(u,\alpha)$ is that
\begin{displaymath}
    \text{Prob}(\sigma > u) = \alpha, \qquad u>0, \quad 0<\alpha<1,
\end{displaymath}
where the standard deviation is $\sigma = 1/\sqrt{\tau}$.  The density, cumulative distribution function, quantile function, and a random number generator for this distribution are implemented in the  \texttt{inla.pc.\{d,p,q,r\}prec} functions.


Internally, R-INLA uses the log-precision rather than the precision and the corresponding  PC prior for the log-precision $x$ has density 
\begin{equation}
    \pi(x) = \frac{\lambda}{2} \exp\left(
      -\lambda\exp\left(-\frac{x}{2}\right) - \frac{x}{2}
    \right).
\end{equation}

\subsection*{Specification}
This prior for the hyperparameters is specified inside the
\texttt{hyper}-spesification, as
\begin{center}
    \texttt{hyper = list(<theta> =
        list(prior="pc.prec", param=c(<u>,<alpha>)))}
\end{center}

\subsection*{Example}

\subsection*{Notes}

See also functions \texttt{inla.pc.\{d,p,q,r\}prec}

\end{document}


% LocalWords: 

%%% Local Variables: 
%%% TeX-master: t
%%% End: 
