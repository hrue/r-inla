\documentclass[a4paper,11pt]{article}
\usepackage[scale={0.8,0.9},centering,includeheadfoot]{geometry}
\usepackage{amstext}
\usepackage{verbatim}
\begin{document}

\section*{Prior for the intercept in the skew-normal link}

\subsection*{Parametrisation}
The skew-normal link parameterise the intercept in terms of a quantile
level $q$, which for zero skewness equals the probit link,
\begin{equation}
    \mu = \Phi^{-1}(q).
\end{equation}
Further the quantile level $q$ is represented by
\begin{equation}
    q = \frac{\exp(\theta)}{1+\exp(\theta)}
\end{equation}
for the (internal) hyperparameter $\theta$. The
\texttt{linksnintercept} prior is the implied prior for $\theta$ when
$\mu$ is Normal with a given mean and precision.

Note that zero precision is interpreted that the Normal density is
uniform with unit density.

\subsection*{Specification}
\begin{center}
    \texttt{..., prior="linksnintercept", param=c(<mean>, <precision>),...}
\end{center}

\subsection*{Example}

\subsection*{Notes}

This is the default prior for the intercept in the skew-normal link. 

\end{document}


% LocalWords:  hyperparameters param gaussian hyperparameter

%%% Local Variables: 
%%% TeX-master: t
%%% End: 
