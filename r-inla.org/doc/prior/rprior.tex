\documentclass[a4paper,11pt]{article}
\usepackage[scale={0.8,0.9},centering,includeheadfoot]{geometry}
\usepackage{amstext}
\usepackage{verbatim}
\begin{document}

\section*{Defining priors in R; `rprior`}

This prior allow the user to use a prior for $\theta$ in a form of an
R-function which return $\log\pi(\theta)$ as a function of $\theta$.
The function needs \texttt{not} to vectorize. The prior function needs
to be prepared using \texttt{inla.rprior.define()} before use (see below
for an example).

\subsection*{Example}

As an example, we implement the \texttt{loggamma}-prior, which for
$\theta = \log(\text{precision})$ is
\begin{verbatim}
rprior.func = function(lprec) {
    return (dgamma(exp(lprec), a, b, log = TRUE) + lprec)
}
\end{verbatim}
The last term is the log-Jacobian for the change of variable. Note
that the prior-parameters, $a$ and $b$, can be passed on when
\emph{preparing} the prior (a required step), using
\begin{verbatim}
rprior <- inla.rprior.define(rprior.func, a=1.0, b=0.1)
\end{verbatim}
\texttt{rprior} can then be used as the argument \texttt{prior =
    rprior} in the \texttt{hyper} argument; see the following example.

\verbatiminput{example-table.R}

\end{document}
%%% Local Variables: 
%%% TeX-master: t
%%% End: 


% LocalWords:  rprior vectorize loggamma Jacobian
