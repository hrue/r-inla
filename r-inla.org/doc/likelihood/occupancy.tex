\documentclass[a4paper,11pt]{article}
\usepackage[scale={0.8,0.9},centering,includeheadfoot]{geometry}
\usepackage{amstext}
\usepackage{amsmath}
\usepackage{verbatim}

\begin{document}
\section*{Occupancy likelihood}

\subsection*{Parametrisation}

This is a specialized likelihood to for occupancy models.

\subsubsection*{Details}

An observation is an vector $y=(y_1, \ldots, y_m)$ of binary
observations, each depending on spesific covariates, with additional
zero-inflation. If there fewer than $m$ observations, like $m' < m$,
say, then the last $m-m'$ observations must be set to \texttt{NA}.
The likelihood for one observation(-vector) is
\begin{displaymath}
    f(y) = \phi \prod_{i=1}^{m} p_i^{y_i} (1-p_i)^{1-y_i} + (1-\phi)
    1_{\text{[$y_i=0, \forall i$]}}
\end{displaymath}
where 
\begin{displaymath}
    \text{logit}(p_i) = x_i^{T} \beta
\end{displaymath}
and $x_i= (x_{i1}, \ldots, x_{ik})$ are the $k>0$ covariate associated
to $y_i$, and $\beta=(\beta_1, \ldots, \beta_k)$ are the regression
coefficients. The linear predictor from the formula $\eta$, goes into
$\phi$, as
\begin{displaymath}
    \text{logit}(\phi) = \eta
\end{displaymath}

\subsection*{Link-function}

The link-function for the $p_i$-model is given by argument
\texttt{link.simple} in the \texttt{control.family}-argument. The
link-function for the $\phi$-model is given as normal. Both defaults
to the logit-link.

\subsection*{Hyperparameters}

The regression coefficients $\beta$ are treated as hyperparameters,
and $k$ is maximum 10. An intercept must be defined manually by adding
a constant covariate vector.

\subsection*{Specification}

\begin{itemize}
\item \texttt{family="occupancy"}
\item Required arguments: A matrix $Y$ with the observations and a matrix
    $X$ with the covariates.
\end{itemize}
The matrix $Y$ is $n\times m$, where $m$ is then the \emph{maximum}
number of observations over all locations. If there fewer than $m$
observations, like $m' < m$, at the $i$th location (ie $i$th row of
$Y$), then last $m-m'$ columns of the $i$th row must be \texttt{NA}.

The matrix $X$ stored the covariates. Since there are $k$
covariate for each observation, then the dimension of $X$ is
$n \times m k$. For the $i$th observation(-vector), then the
$i$th row of $X$ is $(x_{i1}, x_{i2}, \ldots, x_{im})$, where $x_{ij}$
is the covariate vector for $j$th observation at location $i$. In the
case of $m'<m$, then the last $m-m'$ covariate(-vectors) are not used.

\clearpage
Here is a simple example with $n=5$ locations, maximum $m=3$
observations, and $k=2$ covariates.
\begin{verbatim}
> Y
     [,1] [,2] [,3]
[1,]    1    1   NA
[2,]    0    1    0
[3,]    0    0    0
[4,]    0    1    0
[5,]    0    0   NA
> round(dig=3,X)
     [,1]   [,2] [,3]   [,4] [,5]   [,6]
[1,]    1 -0.129    1 -0.248   NA     NA
[2,]    1 -0.030    1  0.151    1  0.100
[3,]    1 -0.148    1 -0.061    1  0.023
[4,]    1 -0.055    1  0.081    1 -0.112
[5,]    1  0.164    1  0.027   NA     NA
\end{verbatim}
For $Y[1,1]$ the covariates are $(1, -0.129)$, for $Y[1,2]$ the
covariates are $(1, -0.248)$, for $Y[1,3]$ is \texttt{NA} hence not
used, for $Y[2,1]$ the covariates are $(1, -0.030)$, etc. We pass both
$Y$ and $X$ in the \texttt{inla.mdata()} in the formula, as
\begin{quote}
    \texttt{inla.mdata(Y,X) $\sim$ ...}
\end{quote}

\subsubsection*{Hyperparameter spesification and default values}
{\small \documentclass[a4paper,11pt]{article}
\usepackage[scale={0.8,0.9},centering,includeheadfoot]{geometry}
\usepackage{amstext}
\usepackage{amsmath}
\usepackage{verbatim}

\begin{document}
\section*{Occupancy likelihood}

\subsection*{Parametrisation}

This is a specialized likelihood to for occupancy models.

\subsubsection*{Details}

An observation is an vector $y=(y_1, \ldots, y_m)$ of binary
observations, each depending on spesific covariates, with additional
zero-inflation. If there fewer than $m$ observations, like $m' < m$,
say, then the last $m-m'$ observations must be set to \texttt{NA}.
The likelihood for one observation(-vector) is
\begin{displaymath}
    f(y) = \phi \prod_{i=1}^{m} p_i^{y_i} (1-p_i)^{1-y_i} + (1-\phi)
    1_{\text{[$y_i=0, \forall i$]}}
\end{displaymath}
where 
\begin{displaymath}
    \text{logit}(p_i) = x_i^{T} \beta
\end{displaymath}
and $x_i= (x_{i1}, \ldots, x_{ik})$ are the $k>0$ covariate associated
to $y_i$, and $\beta=(\beta_1, \ldots, \beta_k)$ are the regression
coefficients. The linear predictor from the formula $\eta$, goes into
$\phi$, as
\begin{displaymath}
    \text{logit}(\phi) = \eta
\end{displaymath}

\subsection*{Link-function}

The link-function for the $p_i$-model is given by argument
\texttt{link.simple} in the \texttt{control.family}-argument. The
link-function for the $\phi$-model is given as normal. Both defaults
to the logit-link.

\subsection*{Hyperparameters}

The regression coefficients $\beta$ are treated as hyperparameters,
and $k$ is maximum 10. An intercept must be defined manually by adding
a constant covariate vector.

\subsection*{Specification}

\begin{itemize}
\item \texttt{family="occupancy"}
\item Required arguments: A matrix $Y$ with the observations and a matrix
    $X$ with the covariates.
\end{itemize}
The matrix $Y$ is $n\times m$, where $m$ is then the \emph{maximum}
number of observations over all locations. If there fewer than $m$
observations, like $m' < m$, at the $i$th location (ie $i$th row of
$Y$), then last $m-m'$ columns of the $i$th row must be \texttt{NA}.

The matrix $X$ stored the covariates. Since there are $k$
covariate for each observation, then the dimension of $X$ is
$n \times m k$. For the $i$th observation(-vector), then the
$i$th row of $X$ is $(x_{i1}, x_{i2}, \ldots, x_{im})$, where $x_{ij}$
is the covariate vector for $j$th observation at location $i$. In the
case of $m'<m$, then the last $m-m'$ covariate(-vectors) are not used.

\clearpage
Here is a simple example with $n=5$ locations, maximum $m=3$
observations, and $k=2$ covariates.
\begin{verbatim}
> Y
     [,1] [,2] [,3]
[1,]    1    1   NA
[2,]    0    1    0
[3,]    0    0    0
[4,]    0    1    0
[5,]    0    0   NA
> round(dig=3,X)
     [,1]   [,2] [,3]   [,4] [,5]   [,6]
[1,]    1 -0.129    1 -0.248   NA     NA
[2,]    1 -0.030    1  0.151    1  0.100
[3,]    1 -0.148    1 -0.061    1  0.023
[4,]    1 -0.055    1  0.081    1 -0.112
[5,]    1  0.164    1  0.027   NA     NA
\end{verbatim}
For $Y[1,1]$ the covariates are $(1, -0.129)$, for $Y[1,2]$ the
covariates are $(1, -0.248)$, for $Y[1,3]$ is \texttt{NA} hence not
used, for $Y[2,1]$ the covariates are $(1, -0.030)$, etc. We pass both
$Y$ and $X$ in the \texttt{inla.mdata()} in the formula, as
\begin{quote}
    \texttt{inla.mdata(Y,X) $\sim$ ...}
\end{quote}

\subsubsection*{Hyperparameter spesification and default values}
{\small \documentclass[a4paper,11pt]{article}
\usepackage[scale={0.8,0.9},centering,includeheadfoot]{geometry}
\usepackage{amstext}
\usepackage{amsmath}
\usepackage{verbatim}

\begin{document}
\section*{Occupancy likelihood}

\subsection*{Parametrisation}

This is a specialized likelihood to for occupancy models.

\subsubsection*{Details}

An observation is an vector $y=(y_1, \ldots, y_m)$ of binary
observations, each depending on spesific covariates, with additional
zero-inflation. If there fewer than $m$ observations, like $m' < m$,
say, then the last $m-m'$ observations must be set to \texttt{NA}.
The likelihood for one observation(-vector) is
\begin{displaymath}
    f(y) = \phi \prod_{i=1}^{m} p_i^{y_i} (1-p_i)^{1-y_i} + (1-\phi)
    1_{\text{[$y_i=0, \forall i$]}}
\end{displaymath}
where 
\begin{displaymath}
    \text{logit}(p_i) = x_i^{T} \beta
\end{displaymath}
and $x_i= (x_{i1}, \ldots, x_{ik})$ are the $k>0$ covariate associated
to $y_i$, and $\beta=(\beta_1, \ldots, \beta_k)$ are the regression
coefficients. The linear predictor from the formula $\eta$, goes into
$\phi$, as
\begin{displaymath}
    \text{logit}(\phi) = \eta
\end{displaymath}

\subsection*{Link-function}

The link-function for the $p_i$-model is given by argument
\texttt{link.simple} in the \texttt{control.family}-argument. The
link-function for the $\phi$-model is given as normal. Both defaults
to the logit-link.

\subsection*{Hyperparameters}

The regression coefficients $\beta$ are treated as hyperparameters,
and $k$ is maximum 10. An intercept must be defined manually by adding
a constant covariate vector.

\subsection*{Specification}

\begin{itemize}
\item \texttt{family="occupancy"}
\item Required arguments: A matrix $Y$ with the observations and a matrix
    $X$ with the covariates.
\end{itemize}
The matrix $Y$ is $n\times m$, where $m$ is then the \emph{maximum}
number of observations over all locations. If there fewer than $m$
observations, like $m' < m$, at the $i$th location (ie $i$th row of
$Y$), then last $m-m'$ columns of the $i$th row must be \texttt{NA}.

The matrix $X$ stored the covariates. Since there are $k$
covariate for each observation, then the dimension of $X$ is
$n \times m k$. For the $i$th observation(-vector), then the
$i$th row of $X$ is $(x_{i1}, x_{i2}, \ldots, x_{im})$, where $x_{ij}$
is the covariate vector for $j$th observation at location $i$. In the
case of $m'<m$, then the last $m-m'$ covariate(-vectors) are not used.

\clearpage
Here is a simple example with $n=5$ locations, maximum $m=3$
observations, and $k=2$ covariates.
\begin{verbatim}
> Y
     [,1] [,2] [,3]
[1,]    1    1   NA
[2,]    0    1    0
[3,]    0    0    0
[4,]    0    1    0
[5,]    0    0   NA
> round(dig=3,X)
     [,1]   [,2] [,3]   [,4] [,5]   [,6]
[1,]    1 -0.129    1 -0.248   NA     NA
[2,]    1 -0.030    1  0.151    1  0.100
[3,]    1 -0.148    1 -0.061    1  0.023
[4,]    1 -0.055    1  0.081    1 -0.112
[5,]    1  0.164    1  0.027   NA     NA
\end{verbatim}
For $Y[1,1]$ the covariates are $(1, -0.129)$, for $Y[1,2]$ the
covariates are $(1, -0.248)$, for $Y[1,3]$ is \texttt{NA} hence not
used, for $Y[2,1]$ the covariates are $(1, -0.030)$, etc. We pass both
$Y$ and $X$ in the \texttt{inla.mdata()} in the formula, as
\begin{quote}
    \texttt{inla.mdata(Y,X) $\sim$ ...}
\end{quote}

\subsubsection*{Hyperparameter spesification and default values}
{\small \documentclass[a4paper,11pt]{article}
\usepackage[scale={0.8,0.9},centering,includeheadfoot]{geometry}
\usepackage{amstext}
\usepackage{amsmath}
\usepackage{verbatim}

\begin{document}
\section*{Occupancy likelihood}

\subsection*{Parametrisation}

This is a specialized likelihood to for occupancy models.

\subsubsection*{Details}

An observation is an vector $y=(y_1, \ldots, y_m)$ of binary
observations, each depending on spesific covariates, with additional
zero-inflation. If there fewer than $m$ observations, like $m' < m$,
say, then the last $m-m'$ observations must be set to \texttt{NA}.
The likelihood for one observation(-vector) is
\begin{displaymath}
    f(y) = \phi \prod_{i=1}^{m} p_i^{y_i} (1-p_i)^{1-y_i} + (1-\phi)
    1_{\text{[$y_i=0, \forall i$]}}
\end{displaymath}
where 
\begin{displaymath}
    \text{logit}(p_i) = x_i^{T} \beta
\end{displaymath}
and $x_i= (x_{i1}, \ldots, x_{ik})$ are the $k>0$ covariate associated
to $y_i$, and $\beta=(\beta_1, \ldots, \beta_k)$ are the regression
coefficients. The linear predictor from the formula $\eta$, goes into
$\phi$, as
\begin{displaymath}
    \text{logit}(\phi) = \eta
\end{displaymath}

\subsection*{Link-function}

The link-function for the $p_i$-model is given by argument
\texttt{link.simple} in the \texttt{control.family}-argument. The
link-function for the $\phi$-model is given as normal. Both defaults
to the logit-link.

\subsection*{Hyperparameters}

The regression coefficients $\beta$ are treated as hyperparameters,
and $k$ is maximum 10. An intercept must be defined manually by adding
a constant covariate vector.

\subsection*{Specification}

\begin{itemize}
\item \texttt{family="occupancy"}
\item Required arguments: A matrix $Y$ with the observations and a matrix
    $X$ with the covariates.
\end{itemize}
The matrix $Y$ is $n\times m$, where $m$ is then the \emph{maximum}
number of observations over all locations. If there fewer than $m$
observations, like $m' < m$, at the $i$th location (ie $i$th row of
$Y$), then last $m-m'$ columns of the $i$th row must be \texttt{NA}.

The matrix $X$ stored the covariates. Since there are $k$
covariate for each observation, then the dimension of $X$ is
$n \times m k$. For the $i$th observation(-vector), then the
$i$th row of $X$ is $(x_{i1}, x_{i2}, \ldots, x_{im})$, where $x_{ij}$
is the covariate vector for $j$th observation at location $i$. In the
case of $m'<m$, then the last $m-m'$ covariate(-vectors) are not used.

\clearpage
Here is a simple example with $n=5$ locations, maximum $m=3$
observations, and $k=2$ covariates.
\begin{verbatim}
> Y
     [,1] [,2] [,3]
[1,]    1    1   NA
[2,]    0    1    0
[3,]    0    0    0
[4,]    0    1    0
[5,]    0    0   NA
> round(dig=3,X)
     [,1]   [,2] [,3]   [,4] [,5]   [,6]
[1,]    1 -0.129    1 -0.248   NA     NA
[2,]    1 -0.030    1  0.151    1  0.100
[3,]    1 -0.148    1 -0.061    1  0.023
[4,]    1 -0.055    1  0.081    1 -0.112
[5,]    1  0.164    1  0.027   NA     NA
\end{verbatim}
For $Y[1,1]$ the covariates are $(1, -0.129)$, for $Y[1,2]$ the
covariates are $(1, -0.248)$, for $Y[1,3]$ is \texttt{NA} hence not
used, for $Y[2,1]$ the covariates are $(1, -0.030)$, etc. We pass both
$Y$ and $X$ in the \texttt{inla.mdata()} in the formula, as
\begin{quote}
    \texttt{inla.mdata(Y,X) $\sim$ ...}
\end{quote}

\subsubsection*{Hyperparameter spesification and default values}
{\small \input{../hyper/likelihood/occupancy.tex}}

\clearpage
\subsection*{Example}
{\small \verbatiminput{example-occupancy.R}}


\end{document}

% LocalWords:  np Hyperparameters Ntrials gaussian hyperparameter

%%% Local Variables: 
%%% TeX-master: t
%%% End: 
}

\clearpage
\subsection*{Example}
{\small \verbatiminput{example-occupancy.R}}


\end{document}

% LocalWords:  np Hyperparameters Ntrials gaussian hyperparameter

%%% Local Variables: 
%%% TeX-master: t
%%% End: 
}

\clearpage
\subsection*{Example}
{\small \verbatiminput{example-occupancy.R}}


\end{document}

% LocalWords:  np Hyperparameters Ntrials gaussian hyperparameter

%%% Local Variables: 
%%% TeX-master: t
%%% End: 
}

\clearpage
\subsection*{Example}
{\small \verbatiminput{example-occupancy.R}}


\end{document}

% LocalWords:  np Hyperparameters Ntrials gaussian hyperparameter

%%% Local Variables: 
%%% TeX-master: t
%%% End: 
