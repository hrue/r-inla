\documentclass[a4paper,11pt]{article}
\usepackage[scale={0.8,0.9},centering,includeheadfoot]{geometry}
\usepackage{amstext}
\usepackage{amsmath}
\usepackage{verbatim}

\begin{document}
\section*{Occupancy likelihood}

\subsection*{Parametrisation}

This is a specialized likelihood to for occupancy models.

\subsubsection*{Details}

An observation is an vector $y=(y_1, \ldots, y_m)$ of binary
observations, each depending on spesific covariates, with additional
zero-inflation. If there fewer than $m$ observations, like $m' < m$,
then ``observations not observed'' must  be set to \texttt{NA}.
The likelihood for one observation(-vector) is
\begin{displaymath}
    f(y) = \phi \prod_{i=1}^{m} p_i^{y_i} (1-p_i)^{1-y_i} + (1-\phi)
    1_{\text{[$y_i=0, \forall i$]}}
\end{displaymath}
with the convension that if $y_i$=\texttt{NA}, the contribution from
$y_i$ is ignored.
Further,
\begin{displaymath}
    \text{logit}(p_i) = x_i^{T} \beta
\end{displaymath}
and $x_i= (x_{i1}, \ldots, x_{ik})$ are the $k>0$ covariate associated
to $y_i$, and $\beta=(\beta_1, \ldots, \beta_k)$ are the regression
coefficients. The linear predictor from the formula $\eta$, goes into
$\phi$, as
\begin{displaymath}
    \text{logit}(\phi) = \eta
\end{displaymath}

\subsection*{Link-function}

The link-function for the $p_i$-model is given by argument
\texttt{link.simple} in the \texttt{control.family}-argument. The
link-function for the $\phi$-model is given as normal. Both defaults
to the logit-link.

\subsection*{Hyperparameters}

The regression coefficients $\beta$ are treated as hyperparameters,
and $k$ is maximum 10. An intercept must be defined manually by adding
a constant covariate vector.

\subsection*{Specification}

\begin{itemize}
\item \texttt{family="occupancy"}
\item Required arguments: A matrix $Y$ with the observations and a matrix
    $X$ with the covariates.
\end{itemize}
The matrix $Y$ is $n\times m$, where $m$ is then the \emph{maximum}
number of observations over all locations. If there fewer than $m$
observations in one location, then \texttt{NA} must be added to reach
$m$.

The matrix $X$ stored the covariates. Since there are $k$ covariate
for each observation, then the dimension of $X$ is $n \times m k$. For
the $i$th observation(-vector), then the $i$th row of $X$ is
$(x_{i1}, x_{i2}, \ldots, x_{im})$, where $x_{ij}$ is the covariate
vector for $j$th observation at location $i$. In a single observation
is \texttt{NA}, the corresponding covariate(-vectors) is not used (but
it still needs to be given).

\clearpage
Here is a simple example with $n=5$ locations, maximum $m=3$
observations, and $k=2$ covariates.
\begin{verbatim}
> Y
     [,1] [,2] [,3]
[1,]    1    1   NA
[2,]    0    1    0
[3,]    0    0    0
[4,]    0    1    0
[5,]   NA    0    0
> round(dig=3,X)
     [,1]   [,2] [,3]   [,4] [,5]   [,6]
[1,]    1 -0.129    1 -0.248   NA     NA
[2,]    1 -0.030    1  0.151    1  0.100
[3,]    1 -0.148    1 -0.061    1  0.023
[4,]    1 -0.055    1  0.081    1 -0.112
[5,]   NA     NA    1  0.164    1  0.027
\end{verbatim}
For $Y[1,1]$ the covariates are $(1, -0.129)$, for $Y[1,2]$ the
covariates are $(1, -0.248)$, for $Y[1,3]$ is \texttt{NA} hence not
used, for $Y[2,1]$ the covariates are $(1, -0.030)$, etc. We pass both
$Y$ and $X$ in the \texttt{inla.mdata()} in the formula, as
\begin{quote}
    \texttt{inla.mdata(Y,X) $\sim$ ...}
\end{quote}

\subsubsection*{Hyperparameter spesification and default values}
{\small %% DO NOT EDIT!
%% This file is generated automatically from models.R
\begin{description}
	\item[doc] \verb!Occupancy likelihood!
	\item[hyper]\ 
	 \begin{description}
	 	\item[theta1]\ 
	 	 \begin{description}
	 	 	\item[hyperid] \verb!56601!
	 	 	\item[name] \verb!beta1!
	 	 	\item[short.name] \verb!beta1!
	 	 	\item[output.name] \verb!beta1 for occupancy observations!
	 	 	\item[output.name.intern] \verb!beta1 for occupancy observations!
	 	 	\item[initial] \verb!-2!
	 	 	\item[fixed] \verb!FALSE!
	 	 	\item[prior] \verb!normal!
	 	 	\item[param] \verb!-2 10!
	 	 	\item[to.theta] \verb!function(x) x!
	 	 	\item[from.theta] \verb!function(x) x!
	 	 \end{description}
	 	\item[theta2]\ 
	 	 \begin{description}
	 	 	\item[hyperid] \verb!56602!
	 	 	\item[name] \verb!beta2!
	 	 	\item[short.name] \verb!beta2!
	 	 	\item[output.name] \verb!beta2 for occupancy observations!
	 	 	\item[output.name.intern] \verb!beta2 for occupancy observations!
	 	 	\item[initial] \verb!0!
	 	 	\item[fixed] \verb!FALSE!
	 	 	\item[prior] \verb!normal!
	 	 	\item[param] \verb!0 10!
	 	 	\item[to.theta] \verb!function(x) x!
	 	 	\item[from.theta] \verb!function(x) x!
	 	 \end{description}
	 	\item[theta3]\ 
	 	 \begin{description}
	 	 	\item[hyperid] \verb!56603!
	 	 	\item[name] \verb!beta3!
	 	 	\item[short.name] \verb!beta3!
	 	 	\item[output.name] \verb!beta3 for occupancy observations!
	 	 	\item[output.name.intern] \verb!beta3 for occupancy observations!
	 	 	\item[initial] \verb!0!
	 	 	\item[fixed] \verb!FALSE!
	 	 	\item[prior] \verb!normal!
	 	 	\item[param] \verb!0 10!
	 	 	\item[to.theta] \verb!function(x) x!
	 	 	\item[from.theta] \verb!function(x) x!
	 	 \end{description}
	 	\item[theta4]\ 
	 	 \begin{description}
	 	 	\item[hyperid] \verb!56604!
	 	 	\item[name] \verb!beta4!
	 	 	\item[short.name] \verb!beta4!
	 	 	\item[output.name] \verb!beta4 for occupancy observations!
	 	 	\item[output.name.intern] \verb!beta4 for occupancy observations!
	 	 	\item[initial] \verb!0!
	 	 	\item[fixed] \verb!FALSE!
	 	 	\item[prior] \verb!normal!
	 	 	\item[param] \verb!0 10!
	 	 	\item[to.theta] \verb!function(x) x!
	 	 	\item[from.theta] \verb!function(x) x!
	 	 \end{description}
	 	\item[theta5]\ 
	 	 \begin{description}
	 	 	\item[hyperid] \verb!56605!
	 	 	\item[name] \verb!beta5!
	 	 	\item[short.name] \verb!beta5!
	 	 	\item[output.name] \verb!beta5 for occupancy observations!
	 	 	\item[output.name.intern] \verb!beta5 for occupancy observations!
	 	 	\item[initial] \verb!0!
	 	 	\item[fixed] \verb!FALSE!
	 	 	\item[prior] \verb!normal!
	 	 	\item[param] \verb!0 10!
	 	 	\item[to.theta] \verb!function(x) x!
	 	 	\item[from.theta] \verb!function(x) x!
	 	 \end{description}
	 	\item[theta6]\ 
	 	 \begin{description}
	 	 	\item[hyperid] \verb!56606!
	 	 	\item[name] \verb!beta6!
	 	 	\item[short.name] \verb!beta6!
	 	 	\item[output.name] \verb!beta6 for occupancy observations!
	 	 	\item[output.name.intern] \verb!beta6 for occupancy observations!
	 	 	\item[initial] \verb!0!
	 	 	\item[fixed] \verb!FALSE!
	 	 	\item[prior] \verb!normal!
	 	 	\item[param] \verb!0 10!
	 	 	\item[to.theta] \verb!function(x) x!
	 	 	\item[from.theta] \verb!function(x) x!
	 	 \end{description}
	 	\item[theta7]\ 
	 	 \begin{description}
	 	 	\item[hyperid] \verb!56607!
	 	 	\item[name] \verb!beta7!
	 	 	\item[short.name] \verb!beta7!
	 	 	\item[output.name] \verb!beta7 for occupancy observations!
	 	 	\item[output.name.intern] \verb!beta7 for occupancy observations!
	 	 	\item[initial] \verb!0!
	 	 	\item[fixed] \verb!FALSE!
	 	 	\item[prior] \verb!normal!
	 	 	\item[param] \verb!0 10!
	 	 	\item[to.theta] \verb!function(x) x!
	 	 	\item[from.theta] \verb!function(x) x!
	 	 \end{description}
	 	\item[theta8]\ 
	 	 \begin{description}
	 	 	\item[hyperid] \verb!56608!
	 	 	\item[name] \verb!beta8!
	 	 	\item[short.name] \verb!beta8!
	 	 	\item[output.name] \verb!beta8 for occupancy observations!
	 	 	\item[output.name.intern] \verb!beta8 for occupancy observations!
	 	 	\item[initial] \verb!0!
	 	 	\item[fixed] \verb!FALSE!
	 	 	\item[prior] \verb!normal!
	 	 	\item[param] \verb!0 10!
	 	 	\item[to.theta] \verb!function(x) x!
	 	 	\item[from.theta] \verb!function(x) x!
	 	 \end{description}
	 	\item[theta9]\ 
	 	 \begin{description}
	 	 	\item[hyperid] \verb!56609!
	 	 	\item[name] \verb!beta9!
	 	 	\item[short.name] \verb!beta9!
	 	 	\item[output.name] \verb!beta9 for occupancy observations!
	 	 	\item[output.name.intern] \verb!beta9 for occupancy observations!
	 	 	\item[initial] \verb!0!
	 	 	\item[fixed] \verb!FALSE!
	 	 	\item[prior] \verb!normal!
	 	 	\item[param] \verb!0 10!
	 	 	\item[to.theta] \verb!function(x) x!
	 	 	\item[from.theta] \verb!function(x) x!
	 	 \end{description}
	 	\item[theta10]\ 
	 	 \begin{description}
	 	 	\item[hyperid] \verb!56610!
	 	 	\item[name] \verb!beta10!
	 	 	\item[short.name] \verb!beta10!
	 	 	\item[output.name] \verb!beta10 for occupancy observations!
	 	 	\item[output.name.intern] \verb!beta10 for occupancy observations!
	 	 	\item[initial] \verb!0!
	 	 	\item[fixed] \verb!FALSE!
	 	 	\item[prior] \verb!normal!
	 	 	\item[param] \verb!0 10!
	 	 	\item[to.theta] \verb!function(x) x!
	 	 	\item[from.theta] \verb!function(x) x!
	 	 \end{description}
	 \end{description}
	\item[survival] \verb!FALSE!
	\item[discrete] \verb!TRUE!
	\item[link] \verb!default logit!
	\item[link.simple] \verb!default logit!
	\item[pdf] \verb!occupancy!
\end{description}
}

\clearpage
\subsection*{Example}
{\small \verbatiminput{example-occupancy.R}}


\end{document}

% LocalWords:  np Hyperparameters Ntrials gaussian hyperparameter

%%% Local Variables: 
%%% TeX-master: t
%%% End: 
