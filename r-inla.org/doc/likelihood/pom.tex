\documentclass[a4paper,11pt]{article}
\usepackage[scale={0.8,0.9},centering,includeheadfoot]{geometry}
\usepackage{amstext}
\usepackage{amsmath}
\usepackage{verbatim}

\begin{document}
\section*{Proportional odds model}

\subsection*{Parametrisation}

The proportional odds model, is for discrete observations
\begin{displaymath}
    y \in \{ 1, 2, \ldots, K\}, \qquad K>1,
\end{displaymath}
defined via the cummulative distribution function
\begin{displaymath}
    F(k) = \text{Prob}(y \le k)  = \frac{\exp(\gamma_k)}{1 + \exp(\gamma_k)}
\end{displaymath}
where
\begin{displaymath}
    \gamma_k = \alpha_k - \eta.
\end{displaymath}
$\{\alpha_k\}$ is here increasing sequence of $K-1$ cut-off points, 
\begin{displaymath}
    \alpha_{0} = -\infty < \alpha_1 < \alpha_2 < \ldots < \alpha_{K-1}
    < \alpha_K=1,
\end{displaymath}
and $\eta$ is the linear predictor. The likelihood for an observation
is then
\begin{displaymath}
    \text{Prob}(y = k) = F(k) - F(k-1).
\end{displaymath}

\subsection*{Link-function}

Not available.

\subsection*{Hyperparameters}

The hyperparameters are $\theta_1, \ldots, \theta_{K-1}$, where
\begin{displaymath}
    \alpha_1 = \theta_1,
\end{displaymath}
and 
\begin{displaymath}
    \alpha_{k} = \alpha_{k-1} + \exp(\theta_{k}) = \theta_1 +
    \sum_{j=2}^{k} \exp(\theta_j)
\end{displaymath}
for $k=2, \ldots, K-1$. The posteriors for $\{\alpha_k\}$ must be
found through simulations as shown in the example below.

\subsection*{Specification}

\begin{itemize}
\item $\text{family}=\texttt{pom}$
\item Required arguments: $y$ (observations)
\end{itemize}
Number of classes, $K$ is determined as the maximum of the
observations. Empty classes are not allowed.

\subsection*{Example}

In the following example we estimate the parameters in a simulated
example.
{\small
\verbatiminput{example-pom.R}
}

\subsection*{Notes}

The prior for $\{\theta_k\}$ are fixed to the Dirichlet distribution
for
\begin{displaymath}
    \left(F^{-1}(\alpha_1),\quad
    F^{-1}(\alpha_2)- F^{-1}(\alpha_1),\quad
    F^{-1}(\alpha_3)- F^{-1}(\alpha_2),\quad
    \ldots,\quad
    1-F^{-1}(\alpha_{K-1})\right)
\end{displaymath}    
with a common scale parameter; see \texttt{inla.doc("dirichlet", sec="prior")}

\end{document}

%%% Local Variables: 
%%% TeX-master: t
%%% End: 
