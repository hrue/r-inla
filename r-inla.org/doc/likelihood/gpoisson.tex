\documentclass[a4paper,11pt]{article}
\usepackage[scale={0.8,0.9},centering,includeheadfoot]{geometry}
\usepackage{amstext}
\usepackage{amsmath,amssymb}
\usepackage{verbatim}

\begin{document}

\section*{Generalized Poisson}

The generalized Poisson distribution is given by
\begin{displaymath}
    f(y|\lambda,w) = \frac{\lambda(\lambda+wy)^{y-1}}{y!}
    \exp(-(\lambda+wy))
\end{displaymath}
for $y=0, 1, 2, \ldots$ and where $\lambda>0$ and
$\max(-1,-\lambda/4)\leq w\leq 1$. The mean and variance of $y$ are
\begin{displaymath}
    \mu =\lambda(1-w)^{-1} \qquad\text{and}\qquad
    \sigma^{2} = \lambda(1-w)^{-3}= \mu(1-w)^{-2}.
\end{displaymath}
Since the dispersion parameter $w$ influence the mean as well as the
variance, we will use the following parameterisation (ADD REFERENCES)
\begin{displaymath}
    w=\frac{\varphi\mu^{p-1}}{1+\varphi\mu^{p-1}}, 
\end{displaymath}
for a fixed $p$, which gives the following density
\begin{displaymath} 
    f(y|\mu,\varphi,p) =\frac{ \mu(\mu+\varphi\mu^{p-1} y)^{y-1}}%%
    {(1+\varphi\mu^{p-1})^{y}y!}
    \exp\left(-\frac{\mu+\varphi\mu^{p-1} y}{1+\varphi\mu^{p-1}}\right)
\end{displaymath}
for $y=0, 1, 2, \ldots$. We assume $\varphi \ge 0$.

\subsection*{Link-function}

The mean and variance of $y$ are given as
\begin{displaymath}
    \text{E}(y|.) = \mu \qquad\text{and}\qquad
    \text{Var}(y|.) = \mu\left(1+\varphi\mu^{p-1}\right)^{2}
\end{displaymath}
and the mean is linked to the linear predictor by
\begin{displaymath}
    \mu = E \exp(\eta)
\end{displaymath}

\subsection*{Hyperparameters}
The overdispersion parameter $\varphi \ge 0$ is represented as
\begin{displaymath}
    \varphi = \exp(\theta) 
\end{displaymath}
The prior is defined on $\theta$. 

\subsection*{Specification}

\begin{itemize}
\item \texttt{family="gpoisson"}
\item \texttt{control.family = list(gpoisson.p = <p>)} defines the
    fixed parameter \texttt{p} (default 1).
\end{itemize}


\subsubsection*{Hyperparameter spesification and default values}
\documentclass[a4paper,11pt]{article}
\usepackage[scale={0.8,0.9},centering,includeheadfoot]{geometry}
\usepackage{amstext}
\usepackage{amsmath,amssymb}
\usepackage{verbatim}

\begin{document}

\section*{Generalized Poisson}

The generalized Poisson distribution is given by
\begin{displaymath}
    f(y|\lambda,w) = \frac{\lambda(\lambda+wy)^{y-1}}{y!}
    \exp(-(\lambda+wy))
\end{displaymath}
for $y=0, 1, 2, \ldots$ and where $\lambda>0$ and
$\max(-1,-\lambda/4)\leq w\leq 1$. The mean and variance of $y$ are
\begin{displaymath}
    \mu =\lambda(1-w)^{-1} \qquad\text{and}\qquad
    \sigma^{2} = \lambda(1-w)^{-3}= \mu(1-w)^{-2}.
\end{displaymath}
Since the dispersion parameter $w$ influence the mean as well as the
variance, we will use the following parameterisation (ADD REFERENCES)
\begin{displaymath}
    w=\frac{\varphi\mu^{p-1}}{1+\varphi\mu^{p-1}}, 
\end{displaymath}
for a fixed $p$, which gives the following density
\begin{displaymath} 
    f(y|\mu,\varphi,p) =\frac{ \mu(\mu+\varphi\mu^{p-1} y)^{y-1}}%%
    {(1+\varphi\mu^{p-1})^{y}y!}
    \exp\left(-\frac{\mu+\varphi\mu^{p-1} y}{1+\varphi\mu^{p-1}}\right)
\end{displaymath}
for $y=0, 1, 2, \ldots$. We assume $\varphi \ge 0$.

\subsection*{Link-function}

The mean and variance of $y$ are given as
\begin{displaymath}
    \text{E}(y|.) = \mu \qquad\text{and}\qquad
    \text{Var}(y|.) = \mu\left(1+\varphi\mu^{p-1}\right)^{2}
\end{displaymath}
and the mean is linked to the linear predictor by
\begin{displaymath}
    \mu = E \exp(\eta)
\end{displaymath}

\subsection*{Hyperparameters}
The overdispersion parameter $\varphi \ge 0$ is represented as
\begin{displaymath}
    \varphi = \exp(\theta) 
\end{displaymath}
The prior is defined on $\theta$. 

\subsection*{Specification}

\begin{itemize}
\item \texttt{family="gpoisson"}
\item \texttt{control.family = list(gpoisson.p = <p>)} defines the
    fixed parameter \texttt{p} (default 1).
\end{itemize}


\subsubsection*{Hyperparameter spesification and default values}
\documentclass[a4paper,11pt]{article}
\usepackage[scale={0.8,0.9},centering,includeheadfoot]{geometry}
\usepackage{amstext}
\usepackage{amsmath,amssymb}
\usepackage{verbatim}

\begin{document}

\section*{Generalized Poisson}

The generalized Poisson distribution is given by
\begin{displaymath}
    f(y|\lambda,w) = \frac{\lambda(\lambda+wy)^{y-1}}{y!}
    \exp(-(\lambda+wy))
\end{displaymath}
for $y=0, 1, 2, \ldots$ and where $\lambda>0$ and
$\max(-1,-\lambda/4)\leq w\leq 1$. The mean and variance of $y$ are
\begin{displaymath}
    \mu =\lambda(1-w)^{-1} \qquad\text{and}\qquad
    \sigma^{2} = \lambda(1-w)^{-3}= \mu(1-w)^{-2}.
\end{displaymath}
Since the dispersion parameter $w$ influence the mean as well as the
variance, we will use the following parameterisation (ADD REFERENCES)
\begin{displaymath}
    w=\frac{\varphi\mu^{p-1}}{1+\varphi\mu^{p-1}}, 
\end{displaymath}
for a fixed $p$, which gives the following density
\begin{displaymath} 
    f(y|\mu,\varphi,p) =\frac{ \mu(\mu+\varphi\mu^{p-1} y)^{y-1}}%%
    {(1+\varphi\mu^{p-1})^{y}y!}
    \exp\left(-\frac{\mu+\varphi\mu^{p-1} y}{1+\varphi\mu^{p-1}}\right)
\end{displaymath}
for $y=0, 1, 2, \ldots$. We assume $\varphi \ge 0$.

\subsection*{Link-function}

The mean and variance of $y$ are given as
\begin{displaymath}
    \text{E}(y|.) = \mu \qquad\text{and}\qquad
    \text{Var}(y|.) = \mu\left(1+\varphi\mu^{p-1}\right)^{2}
\end{displaymath}
and the mean is linked to the linear predictor by
\begin{displaymath}
    \mu = E \exp(\eta)
\end{displaymath}

\subsection*{Hyperparameters}
The overdispersion parameter $\varphi \ge 0$ is represented as
\begin{displaymath}
    \varphi = \exp(\theta) 
\end{displaymath}
The prior is defined on $\theta$. 

\subsection*{Specification}

\begin{itemize}
\item \texttt{family="gpoisson"}
\item \texttt{control.family = list(gpoisson.p = <p>)} defines the
    fixed parameter \texttt{p} (default 1).
\end{itemize}


\subsubsection*{Hyperparameter spesification and default values}
\documentclass[a4paper,11pt]{article}
\usepackage[scale={0.8,0.9},centering,includeheadfoot]{geometry}
\usepackage{amstext}
\usepackage{amsmath,amssymb}
\usepackage{verbatim}

\begin{document}

\section*{Generalized Poisson}

The generalized Poisson distribution is given by
\begin{displaymath}
    f(y|\lambda,w) = \frac{\lambda(\lambda+wy)^{y-1}}{y!}
    \exp(-(\lambda+wy))
\end{displaymath}
for $y=0, 1, 2, \ldots$ and where $\lambda>0$ and
$\max(-1,-\lambda/4)\leq w\leq 1$. The mean and variance of $y$ are
\begin{displaymath}
    \mu =\lambda(1-w)^{-1} \qquad\text{and}\qquad
    \sigma^{2} = \lambda(1-w)^{-3}= \mu(1-w)^{-2}.
\end{displaymath}
Since the dispersion parameter $w$ influence the mean as well as the
variance, we will use the following parameterisation (ADD REFERENCES)
\begin{displaymath}
    w=\frac{\varphi\mu^{p-1}}{1+\varphi\mu^{p-1}}, 
\end{displaymath}
for a fixed $p$, which gives the following density
\begin{displaymath} 
    f(y|\mu,\varphi,p) =\frac{ \mu(\mu+\varphi\mu^{p-1} y)^{y-1}}%%
    {(1+\varphi\mu^{p-1})^{y}y!}
    \exp\left(-\frac{\mu+\varphi\mu^{p-1} y}{1+\varphi\mu^{p-1}}\right)
\end{displaymath}
for $y=0, 1, 2, \ldots$. We assume $\varphi \ge 0$.

\subsection*{Link-function}

The mean and variance of $y$ are given as
\begin{displaymath}
    \text{E}(y|.) = \mu \qquad\text{and}\qquad
    \text{Var}(y|.) = \mu\left(1+\varphi\mu^{p-1}\right)^{2}
\end{displaymath}
and the mean is linked to the linear predictor by
\begin{displaymath}
    \mu = E \exp(\eta)
\end{displaymath}

\subsection*{Hyperparameters}
The overdispersion parameter $\varphi \ge 0$ is represented as
\begin{displaymath}
    \varphi = \exp(\theta) 
\end{displaymath}
The prior is defined on $\theta$. 

\subsection*{Specification}

\begin{itemize}
\item \texttt{family="gpoisson"}
\item \texttt{control.family = list(gpoisson.p = <p>)} defines the
    fixed parameter \texttt{p} (default 1).
\end{itemize}


\subsubsection*{Hyperparameter spesification and default values}
\input{../hyper/likelihood/gpoisson.tex}
    
\subsection*{Example}

In the following example we estimate the parameters in a simulated
example with generalized Poisson responses.
\verbatiminput{example-gpoisson.R}

\end{document}

    
\subsection*{Example}

In the following example we estimate the parameters in a simulated
example with generalized Poisson responses.
\verbatiminput{example-gpoisson.R}

\end{document}

    
\subsection*{Example}

In the following example we estimate the parameters in a simulated
example with generalized Poisson responses.
\verbatiminput{example-gpoisson.R}

\end{document}

    
\subsection*{Example}

In the following example we estimate the parameters in a simulated
example with generalized Poisson responses.
\verbatiminput{example-gpoisson.R}

\end{document}
