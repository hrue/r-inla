\documentclass[a4paper,11pt]{article}
\usepackage[scale={0.8,0.9},centering,includeheadfoot]{geometry}
\usepackage{ifpdf}
\usepackage{amstext}
\usepackage{amsmath}
\usepackage{verbatim}
\newcommand{\vect}[1]{\boldsymbol{#1}}
\begin{document}
\section*{Gompertz}

\subsection*{Parametrisation}

The Gompertz distribution has log survial function
\begin{displaymath}
    \log S(y) = -\frac{\mu}{\alpha}\left(\exp(\alpha y) -1\right)
\end{displaymath}
for response $y\ge 0$, $\mu>0$ and $\alpha>0$. The cummulative
distribution function and the density then follows as
\begin{displaymath}
    F(y) = 1 - \exp\left[ -\frac{\mu}{\alpha}\left(\exp(\alpha y)
        -1\right) \right]
\end{displaymath}
and
\begin{displaymath}
    f(y) = \mu \exp\left[ \alpha y -\frac{\mu}{\alpha}\left(\exp(\alpha y)
        -1\right) \right].
\end{displaymath}

\subsection*{Link-function}
The parameter $\mu$ is linked to the linear predictor $\eta$ as:
\[
    \mu = \exp(\eta)
\]

\subsection*{Hyperparameters}

The shape parameter $\alpha$ is represented as
\begin{displaymath}
    \alpha = \exp(S\theta)
\end{displaymath}
and the prior is defined on $\theta$. The constant $S$ currently set
to $0.1$ to avoid numerical instabilities in the optimization, since
small changes of $\alpha$ can make a huge difference.

\subsection*{Specification}

\begin{itemize}
\item \texttt{family="gompertz"} for regression models and
    \texttt{family="gompertz.surv"} for survival models.
\item Required arguments: $y$ (to be given in a format by using
    $\texttt{inla.surv()}$ for survival models )
\end{itemize}

\subsubsection*{Hyperparameter spesification and default values}
%% DO NOT EDIT!
%% This file is generated automatically from models.R
\begin{description}
	\item[doc] gompertz distribution
	\item[hyper]\ 
	 \begin{description}
	 	\item[theta]\ 
	 	 \begin{description}
	 	 	\item[hyperid] 105101
	 	 	\item[name] shape
	 	 	\item[short.name] alpha
	 	 	\item[initial] 0
	 	 	\item[fixed] FALSE
	 	 	\item[prior] loggamma
	 	 	\item[param] 0.001 0.001
	 	 	\item[to.theta] \verb!function(x, sc = 0.1) log(x) / sc!
	 	 	\item[from.theta] \verb!function(x, sc = 0.1) exp(sc * x)!
	 	 \end{description}
	 \end{description}
	\item[status] experimental
	\item[survival] FALSE
	\item[discrete] FALSE
	\item[link] default log
	\item[pdf] gompertz
\end{description}

%% DO NOT EDIT!
%% This file is generated automatically from models.R
\begin{description}
	\item[doc] gompertz distribution
	\item[hyper]\ 
	 \begin{description}
	 	\item[theta1]\ 
	 	 \begin{description}
	 	 	\item[hyperid] 106101
	 	 	\item[name] shape
	 	 	\item[short.name] alpha
	 	 	\item[initial] -10
	 	 	\item[fixed] FALSE
	 	 	\item[prior] normal
	 	 	\item[param] 0 1
	 	 	\item[to.theta] \verb!function(x, sc = 0.1) log(x) / sc!
	 	 	\item[from.theta] \verb!function(x, sc = 0.1) exp(sc * x)!
	 	 \end{description}
	 	\item[theta2]\ 
	 	 \begin{description}
	 	 	\item[hyperid] 106102
	 	 	\item[name] beta1
	 	 	\item[short.name] beta1
	 	 	\item[initial] -5
	 	 	\item[fixed] FALSE
	 	 	\item[prior] normal
	 	 	\item[param] -4 100
	 	 	\item[to.theta] \verb!function(x) x!
	 	 	\item[from.theta] \verb!function(x) x!
	 	 \end{description}
	 	\item[theta3]\ 
	 	 \begin{description}
	 	 	\item[hyperid] 106103
	 	 	\item[name] beta2
	 	 	\item[short.name] beta2
	 	 	\item[initial] 0
	 	 	\item[fixed] FALSE
	 	 	\item[prior] normal
	 	 	\item[param] 0 100
	 	 	\item[to.theta] \verb!function(x) x!
	 	 	\item[from.theta] \verb!function(x) x!
	 	 \end{description}
	 	\item[theta4]\ 
	 	 \begin{description}
	 	 	\item[hyperid] 106104
	 	 	\item[name] beta3
	 	 	\item[short.name] beta3
	 	 	\item[initial] 0
	 	 	\item[fixed] FALSE
	 	 	\item[prior] normal
	 	 	\item[param] 0 100
	 	 	\item[to.theta] \verb!function(x) x!
	 	 	\item[from.theta] \verb!function(x) x!
	 	 \end{description}
	 	\item[theta5]\ 
	 	 \begin{description}
	 	 	\item[hyperid] 106105
	 	 	\item[name] beta4
	 	 	\item[short.name] beta4
	 	 	\item[initial] 0
	 	 	\item[fixed] FALSE
	 	 	\item[prior] normal
	 	 	\item[param] 0 100
	 	 	\item[to.theta] \verb!function(x) x!
	 	 	\item[from.theta] \verb!function(x) x!
	 	 \end{description}
	 	\item[theta6]\ 
	 	 \begin{description}
	 	 	\item[hyperid] 106106
	 	 	\item[name] beta5
	 	 	\item[short.name] beta5
	 	 	\item[initial] 0
	 	 	\item[fixed] FALSE
	 	 	\item[prior] normal
	 	 	\item[param] 0 100
	 	 	\item[to.theta] \verb!function(x) x!
	 	 	\item[from.theta] \verb!function(x) x!
	 	 \end{description}
	 	\item[theta7]\ 
	 	 \begin{description}
	 	 	\item[hyperid] 106107
	 	 	\item[name] beta6
	 	 	\item[short.name] beta6
	 	 	\item[initial] 0
	 	 	\item[fixed] FALSE
	 	 	\item[prior] normal
	 	 	\item[param] 0 100
	 	 	\item[to.theta] \verb!function(x) x!
	 	 	\item[from.theta] \verb!function(x) x!
	 	 \end{description}
	 	\item[theta8]\ 
	 	 \begin{description}
	 	 	\item[hyperid] 106108
	 	 	\item[name] beta7
	 	 	\item[short.name] beta7
	 	 	\item[initial] 0
	 	 	\item[fixed] FALSE
	 	 	\item[prior] normal
	 	 	\item[param] 0 100
	 	 	\item[to.theta] \verb!function(x) x!
	 	 	\item[from.theta] \verb!function(x) x!
	 	 \end{description}
	 	\item[theta9]\ 
	 	 \begin{description}
	 	 	\item[hyperid] 106109
	 	 	\item[name] beta8
	 	 	\item[short.name] beta8
	 	 	\item[initial] 0
	 	 	\item[fixed] FALSE
	 	 	\item[prior] normal
	 	 	\item[param] 0 100
	 	 	\item[to.theta] \verb!function(x) x!
	 	 	\item[from.theta] \verb!function(x) x!
	 	 \end{description}
	 	\item[theta10]\ 
	 	 \begin{description}
	 	 	\item[hyperid] 106110
	 	 	\item[name] beta9
	 	 	\item[short.name] beta9
	 	 	\item[initial] 0
	 	 	\item[fixed] FALSE
	 	 	\item[prior] normal
	 	 	\item[param] 0 100
	 	 	\item[to.theta] \verb!function(x) x!
	 	 	\item[from.theta] \verb!function(x) x!
	 	 \end{description}
	 	\item[theta11]\ 
	 	 \begin{description}
	 	 	\item[hyperid] 106111
	 	 	\item[name] beta10
	 	 	\item[short.name] beta10
	 	 	\item[initial] 0
	 	 	\item[fixed] FALSE
	 	 	\item[prior] normal
	 	 	\item[param] 0 100
	 	 	\item[to.theta] \verb!function(x) x!
	 	 	\item[from.theta] \verb!function(x) x!
	 	 \end{description}
	 \end{description}
	\item[status] experimental
	\item[survival] TRUE
	\item[discrete] FALSE
	\item[link] default log neglog
	\item[pdf] gompertz
\end{description}



\subsection*{Example}

In the following example we estimate the parameters in a simulated
case%%
\verbatiminput{example-gompertz.R}

\subsection*{Notes}

\end{document}


%%% Local Variables: 
%%% TeX-master: t
%%% End: 
