\documentclass[a4paper,11pt]{article}
\usepackage[scale={0.8,0.9},centering,includeheadfoot]{geometry}
\usepackage{amstext}
\usepackage{amsmath}
\usepackage{verbatim}

\begin{document}
\section*{Censored Poisson (version 2)}

\subsection*{Parametrisation}

The Poisson distribution is
\begin{displaymath}
    \text{Prob}(y) = \frac{\lambda^{y}}{y!}\exp(-\lambda)
\end{displaymath}
for responses $y=0, 1, 2, \ldots$, where $\lambda$ is the expected
value. The cencored version is that reponse response in the interval
$L \le y \le H$ are cencored (and reported as $y=L$, say), whereas
other values are reported as is. This is often due to privacy issue,
for example using $L=1$ and $H=5$, for example. 

The ``cenpoisson'' probability distribution is then, for 
$y=0, 1, \ldots, $
\begin{displaymath}
    \text{Prob}^{*}(y) =
    \begin{cases}
      \sum_{z=L}^{H}
      \frac{\lambda^{z}}{z!}\exp(-\lambda) & L
                                             \leq
                                             y
                                             \leq
                                             H \\
      \frac{\lambda^{y}}{y!}\exp(-\lambda) & \text{otherwise}
    \end{cases}
\end{displaymath}


\subsection*{Link-function}

The mean-parameter is $\lambda$ and is linked to the linear predictor $\eta$
by
\begin{displaymath}
    \lambda = E \exp(\eta)
\end{displaymath}
where $E>0$ is a known constant (or $\log(E)$ is the offset of $\eta$).

\subsection*{Hyperparameters}

None.

\subsection*{Specification}

\begin{itemize}
\item \texttt{family="cenpoisson2"}
\item The \texttt{cenpoisson2} differ from \texttt{cenpoisson}, in
    that $L$ and $H$ are vectors and not scalars, hence different
    observations can have different censoring.
\item Required arguments: $y$, $E$, $L$ and $H$. The vector of the
    triplet $(y_i,L_i,H_i)$ must be given as a
    \texttt{inla.mdata}-object. $L$ and $H$ are vectors of same length
    as $y$ hence the cencoring can be different for each observation.
    $L$ and $H$ must be integer valued or \texttt{Inf}.

    $L[i]=$\texttt{Inf} and/or $H[i]=$\texttt{Inf} are allowed, which
    is equivalent to $L[i]= -1$ and/or $H[i] = -1$. See the example
    for details.
\end{itemize}

\subsection*{Example}

In the following example we estimate the parameters in a simulated
example with Poisson responses.
{\small
\verbatiminput{example-cenpoisson2.R}
}

\subsection*{Notes}

For censored values, then $y$ must be one arbitrary value in the interval; \texttt{NA} does not work!!!

\end{document}


% LocalWords:  np Hyperparameters Ntrials poisson

%%% Local Variables: 
%%% TeX-master: t
%%% End: 



% LocalWords: 

%%% Local Variables: 
%%% TeX-master: t
%%% End: 



% LocalWords: 

%%% Local Variables: 
%%% TeX-master: t
%%% End: 
