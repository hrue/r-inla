\documentclass[a4paper,11pt]{article}
\usepackage[scale={0.8,0.9},centering,includeheadfoot]{geometry}
\usepackage{amstext}
\usepackage{amsmath}
\usepackage{verbatim}

\begin{document}
\section*{New $0$inflated models: Poisson \& Binomial}

\subsection*{Parametrisation}

This is a new implementation (Nov'22) of zero-inflated Poisson and
Binomial likelihood, where we will allow for a linear predictor in
both the zero-inflattion and in the mean, but one of them needs to
consists of fixed effects only. This means the setup will be
somewhat different that for other likelihood models. 

\subsubsection*{Details}

The zero-inflated likelihood $f_0(y|\ldots)$ is defined as
\begin{displaymath}
    f_0(y | \eta_1, \eta_2) = p(\eta_1) 1_{[y=0]} +
    (1-p(\eta_1)) f(y | \eta_2)
\end{displaymath}
where $f(y|\ldots)$ is either Poisson or Binomial. We allow for two
linear predictors in the model, but one needs to be ``simple'' (i.e.\
only consists of fixed effects). The other is general and defined via
the formula. Normally, the zero-inflation probability is simpler (\texttt{family="0..."})
\begin{displaymath}
    \eta_1 = \text{simple} \qquad \eta_2 = \text{formula}
\end{displaymath}
but they can also be swapped (\texttt{family="0...S"})
\begin{displaymath}
    \eta_1 = \text{formula} \qquad \eta_2 = \text{simple}
\end{displaymath}

\subsection*{Link-function}

This is similar to Poisson and Binomial.

The link-function for the 'simple'-model must be given by argument
\texttt{link.simple} in the \texttt{control.family}-argument. Only
link-models without covariates/parameters are currently available. 
The examples later on shows how this is done.

\subsection*{Hyperparameters}

All parameters in the simple model are treated as hyperparameters.
The $j$'th element of $\eta_1$ is
\begin{displaymath}
    (\eta_1)_j = \sum_{i=1}^{m} \beta_i x_{ij}
\end{displaymath}
for covariates $x_1, \ldots, $, where $m$ is maximum 10. An intercept
in this model have to be defined manually by adding a constant
covariate vector.

\subsection*{Specification}

\begin{itemize}
\item \texttt{family="0poisson"}
\item \texttt{family="0poissonS"}
\item \texttt{family="0binomial"}
\item \texttt{family="0binomialS"}
\item Required arguments: As for the Poisson and Binomial (but how
    these arguments are given, will differ). Optional argument
    \texttt{link.simple}.
\end{itemize}

\subsubsection*{Hyperparameter spesification and default values}
\paragraph{0poisson}
{\small %% DO NOT EDIT!
%% This file is generated automatically from models.R
\begin{description}
	\item[doc] New 0-inflated Poisson
	\item[hyper]\ 
	 \begin{description}
	 	\item[theta1]\ 
	 	 \begin{description}
	 	 	\item[hyperid] 56201
	 	 	\item[name] beta1
	 	 	\item[short.name] beta1
	 	 	\item[initial] -4
	 	 	\item[fixed] FALSE
	 	 	\item[prior] normal
	 	 	\item[param] -4 10
	 	 	\item[to.theta] \verb!function(x) x!
	 	 	\item[from.theta] \verb!function(x) x!
	 	 \end{description}
	 	\item[theta2]\ 
	 	 \begin{description}
	 	 	\item[hyperid] 56202
	 	 	\item[name] beta2
	 	 	\item[short.name] beta2
	 	 	\item[initial] 0
	 	 	\item[fixed] FALSE
	 	 	\item[prior] normal
	 	 	\item[param] 0 10
	 	 	\item[to.theta] \verb!function(x) x!
	 	 	\item[from.theta] \verb!function(x) x!
	 	 \end{description}
	 	\item[theta3]\ 
	 	 \begin{description}
	 	 	\item[hyperid] 56203
	 	 	\item[name] beta3
	 	 	\item[short.name] beta3
	 	 	\item[initial] 0
	 	 	\item[fixed] FALSE
	 	 	\item[prior] normal
	 	 	\item[param] 0 10
	 	 	\item[to.theta] \verb!function(x) x!
	 	 	\item[from.theta] \verb!function(x) x!
	 	 \end{description}
	 	\item[theta4]\ 
	 	 \begin{description}
	 	 	\item[hyperid] 56204
	 	 	\item[name] beta4
	 	 	\item[short.name] beta4
	 	 	\item[initial] 0
	 	 	\item[fixed] FALSE
	 	 	\item[prior] normal
	 	 	\item[param] 0 10
	 	 	\item[to.theta] \verb!function(x) x!
	 	 	\item[from.theta] \verb!function(x) x!
	 	 \end{description}
	 	\item[theta5]\ 
	 	 \begin{description}
	 	 	\item[hyperid] 56205
	 	 	\item[name] beta5
	 	 	\item[short.name] beta5
	 	 	\item[initial] 0
	 	 	\item[fixed] FALSE
	 	 	\item[prior] normal
	 	 	\item[param] 0 10
	 	 	\item[to.theta] \verb!function(x) x!
	 	 	\item[from.theta] \verb!function(x) x!
	 	 \end{description}
	 	\item[theta6]\ 
	 	 \begin{description}
	 	 	\item[hyperid] 56206
	 	 	\item[name] beta6
	 	 	\item[short.name] beta6
	 	 	\item[initial] 0
	 	 	\item[fixed] FALSE
	 	 	\item[prior] normal
	 	 	\item[param] 0 10
	 	 	\item[to.theta] \verb!function(x) x!
	 	 	\item[from.theta] \verb!function(x) x!
	 	 \end{description}
	 	\item[theta7]\ 
	 	 \begin{description}
	 	 	\item[hyperid] 56207
	 	 	\item[name] beta7
	 	 	\item[short.name] beta7
	 	 	\item[initial] 0
	 	 	\item[fixed] FALSE
	 	 	\item[prior] normal
	 	 	\item[param] 0 10
	 	 	\item[to.theta] \verb!function(x) x!
	 	 	\item[from.theta] \verb!function(x) x!
	 	 \end{description}
	 	\item[theta8]\ 
	 	 \begin{description}
	 	 	\item[hyperid] 56208
	 	 	\item[name] beta8
	 	 	\item[short.name] beta8
	 	 	\item[initial] 0
	 	 	\item[fixed] FALSE
	 	 	\item[prior] normal
	 	 	\item[param] 0 10
	 	 	\item[to.theta] \verb!function(x) x!
	 	 	\item[from.theta] \verb!function(x) x!
	 	 \end{description}
	 	\item[theta9]\ 
	 	 \begin{description}
	 	 	\item[hyperid] 56209
	 	 	\item[name] beta9
	 	 	\item[short.name] beta9
	 	 	\item[initial] 0
	 	 	\item[fixed] FALSE
	 	 	\item[prior] normal
	 	 	\item[param] 0 10
	 	 	\item[to.theta] \verb!function(x) x!
	 	 	\item[from.theta] \verb!function(x) x!
	 	 \end{description}
	 	\item[theta10]\ 
	 	 \begin{description}
	 	 	\item[hyperid] 56210
	 	 	\item[name] beta10
	 	 	\item[short.name] beta10
	 	 	\item[initial] 0
	 	 	\item[fixed] FALSE
	 	 	\item[prior] normal
	 	 	\item[param] 0 10
	 	 	\item[to.theta] \verb!function(x) x!
	 	 	\item[from.theta] \verb!function(x) x!
	 	 \end{description}
	 \end{description}
	\item[status] experimental
	\item[survival] FALSE
	\item[discrete] TRUE
	\item[link] default log quantile
	\item[link.simple] default logit cauchit probit cloglog
	\item[pdf] 0inflated
\end{description}
}
\paragraph{0poissonS}
{\small %% DO NOT EDIT!
%% This file is generated automatically from models.R
\begin{description}
	\item[doc] \verb!New 0-inflated Poisson Swap!
	\item[hyper]\ 
	 \begin{description}
	 	\item[theta1]\ 
	 	 \begin{description}
	 	 	\item[hyperid] \verb!56301!
	 	 	\item[name] \verb!beta1!
	 	 	\item[short.name] \verb!beta1!
	 	 	\item[output.name] \verb!beta1 for 0poissonS observations!
	 	 	\item[output.name.intern] \verb!beta1 for 0poissonS observations!
	 	 	\item[initial] \verb!-4!
	 	 	\item[fixed] \verb!FALSE!
	 	 	\item[prior] \verb!normal!
	 	 	\item[param] \verb!-4 10!
	 	 	\item[to.theta] \verb!function(x) x!
	 	 	\item[from.theta] \verb!function(x) x!
	 	 \end{description}
	 	\item[theta2]\ 
	 	 \begin{description}
	 	 	\item[hyperid] \verb!56302!
	 	 	\item[name] \verb!beta2!
	 	 	\item[short.name] \verb!beta2!
	 	 	\item[output.name] \verb!beta2 for 0poissonS observations!
	 	 	\item[output.name.intern] \verb!beta2 for 0poissonS observations!
	 	 	\item[initial] \verb!0!
	 	 	\item[fixed] \verb!FALSE!
	 	 	\item[prior] \verb!normal!
	 	 	\item[param] \verb!0 10!
	 	 	\item[to.theta] \verb!function(x) x!
	 	 	\item[from.theta] \verb!function(x) x!
	 	 \end{description}
	 	\item[theta3]\ 
	 	 \begin{description}
	 	 	\item[hyperid] \verb!56303!
	 	 	\item[name] \verb!beta3!
	 	 	\item[short.name] \verb!beta3!
	 	 	\item[output.name] \verb!beta3 for 0poissonS observations!
	 	 	\item[output.name.intern] \verb!beta3 for 0poissonS observations!
	 	 	\item[initial] \verb!0!
	 	 	\item[fixed] \verb!FALSE!
	 	 	\item[prior] \verb!normal!
	 	 	\item[param] \verb!0 10!
	 	 	\item[to.theta] \verb!function(x) x!
	 	 	\item[from.theta] \verb!function(x) x!
	 	 \end{description}
	 	\item[theta4]\ 
	 	 \begin{description}
	 	 	\item[hyperid] \verb!56304!
	 	 	\item[name] \verb!beta4!
	 	 	\item[short.name] \verb!beta4!
	 	 	\item[output.name] \verb!beta4 for 0poissonS observations!
	 	 	\item[output.name.intern] \verb!beta4 for 0poissonS observations!
	 	 	\item[initial] \verb!0!
	 	 	\item[fixed] \verb!FALSE!
	 	 	\item[prior] \verb!normal!
	 	 	\item[param] \verb!0 10!
	 	 	\item[to.theta] \verb!function(x) x!
	 	 	\item[from.theta] \verb!function(x) x!
	 	 \end{description}
	 	\item[theta5]\ 
	 	 \begin{description}
	 	 	\item[hyperid] \verb!56305!
	 	 	\item[name] \verb!beta5!
	 	 	\item[short.name] \verb!beta5!
	 	 	\item[output.name] \verb!beta5 for 0poissonS observations!
	 	 	\item[output.name.intern] \verb!beta5 for 0poissonS observations!
	 	 	\item[initial] \verb!0!
	 	 	\item[fixed] \verb!FALSE!
	 	 	\item[prior] \verb!normal!
	 	 	\item[param] \verb!0 10!
	 	 	\item[to.theta] \verb!function(x) x!
	 	 	\item[from.theta] \verb!function(x) x!
	 	 \end{description}
	 	\item[theta6]\ 
	 	 \begin{description}
	 	 	\item[hyperid] \verb!56306!
	 	 	\item[name] \verb!beta6!
	 	 	\item[short.name] \verb!beta6!
	 	 	\item[output.name] \verb!beta6 for 0poissonS observations!
	 	 	\item[output.name.intern] \verb!beta6 for 0poissonS observations!
	 	 	\item[initial] \verb!0!
	 	 	\item[fixed] \verb!FALSE!
	 	 	\item[prior] \verb!normal!
	 	 	\item[param] \verb!0 10!
	 	 	\item[to.theta] \verb!function(x) x!
	 	 	\item[from.theta] \verb!function(x) x!
	 	 \end{description}
	 	\item[theta7]\ 
	 	 \begin{description}
	 	 	\item[hyperid] \verb!56307!
	 	 	\item[name] \verb!beta7!
	 	 	\item[short.name] \verb!beta7!
	 	 	\item[output.name] \verb!beta7 for 0poissonS observations!
	 	 	\item[output.name.intern] \verb!beta7 for 0poissonS observations!
	 	 	\item[initial] \verb!0!
	 	 	\item[fixed] \verb!FALSE!
	 	 	\item[prior] \verb!normal!
	 	 	\item[param] \verb!0 10!
	 	 	\item[to.theta] \verb!function(x) x!
	 	 	\item[from.theta] \verb!function(x) x!
	 	 \end{description}
	 	\item[theta8]\ 
	 	 \begin{description}
	 	 	\item[hyperid] \verb!56308!
	 	 	\item[name] \verb!beta8!
	 	 	\item[short.name] \verb!beta8!
	 	 	\item[output.name] \verb!beta8 for 0poissonS observations!
	 	 	\item[output.name.intern] \verb!beta8 for 0poissonS observations!
	 	 	\item[initial] \verb!0!
	 	 	\item[fixed] \verb!FALSE!
	 	 	\item[prior] \verb!normal!
	 	 	\item[param] \verb!0 10!
	 	 	\item[to.theta] \verb!function(x) x!
	 	 	\item[from.theta] \verb!function(x) x!
	 	 \end{description}
	 	\item[theta9]\ 
	 	 \begin{description}
	 	 	\item[hyperid] \verb!56309!
	 	 	\item[name] \verb!beta9!
	 	 	\item[short.name] \verb!beta9!
	 	 	\item[output.name] \verb!beta9 for 0poissonS observations!
	 	 	\item[output.name.intern] \verb!beta9 for 0poissonS observations!
	 	 	\item[initial] \verb!0!
	 	 	\item[fixed] \verb!FALSE!
	 	 	\item[prior] \verb!normal!
	 	 	\item[param] \verb!0 10!
	 	 	\item[to.theta] \verb!function(x) x!
	 	 	\item[from.theta] \verb!function(x) x!
	 	 \end{description}
	 	\item[theta10]\ 
	 	 \begin{description}
	 	 	\item[hyperid] \verb!56310!
	 	 	\item[name] \verb!beta10!
	 	 	\item[short.name] \verb!beta10!
	 	 	\item[output.name] \verb!beta10 for 0poissonS observations!
	 	 	\item[output.name.intern] \verb!beta10 for 0poissonS observations!
	 	 	\item[initial] \verb!0!
	 	 	\item[fixed] \verb!FALSE!
	 	 	\item[prior] \verb!normal!
	 	 	\item[param] \verb!0 10!
	 	 	\item[to.theta] \verb!function(x) x!
	 	 	\item[from.theta] \verb!function(x) x!
	 	 \end{description}
	 \end{description}
	\item[survival] \verb!FALSE!
	\item[discrete] \verb!TRUE!
	\item[link] \verb!default logit loga cauchit probit cloglog ccloglog loglog log sslogit logitoffset quantile pquantile robit sn powerlogit!
	\item[link.simple] \verb!default log!
	\item[pdf] \verb!0inflated!
\end{description}
}
\paragraph{0binomial}
{\small %% DO NOT EDIT!
%% This file is generated automatically from models.R
\begin{description}
	\item[doc] New 0-inflated Binomial
	\item[hyper]\ 
	 \begin{description}
	 	\item[theta1]\ 
	 	 \begin{description}
	 	 	\item[hyperid] 56401
	 	 	\item[name] beta1
	 	 	\item[short.name] beta1
	 	 	\item[initial] -4
	 	 	\item[fixed] FALSE
	 	 	\item[prior] normal
	 	 	\item[param] -4 10
	 	 	\item[to.theta] \verb!function(x) x!
	 	 	\item[from.theta] \verb!function(x) x!
	 	 \end{description}
	 	\item[theta2]\ 
	 	 \begin{description}
	 	 	\item[hyperid] 56402
	 	 	\item[name] beta2
	 	 	\item[short.name] beta2
	 	 	\item[initial] 0
	 	 	\item[fixed] FALSE
	 	 	\item[prior] normal
	 	 	\item[param] 0 10
	 	 	\item[to.theta] \verb!function(x) x!
	 	 	\item[from.theta] \verb!function(x) x!
	 	 \end{description}
	 	\item[theta3]\ 
	 	 \begin{description}
	 	 	\item[hyperid] 56403
	 	 	\item[name] beta3
	 	 	\item[short.name] beta3
	 	 	\item[initial] 0
	 	 	\item[fixed] FALSE
	 	 	\item[prior] normal
	 	 	\item[param] 0 10
	 	 	\item[to.theta] \verb!function(x) x!
	 	 	\item[from.theta] \verb!function(x) x!
	 	 \end{description}
	 	\item[theta4]\ 
	 	 \begin{description}
	 	 	\item[hyperid] 56404
	 	 	\item[name] beta4
	 	 	\item[short.name] beta4
	 	 	\item[initial] 0
	 	 	\item[fixed] FALSE
	 	 	\item[prior] normal
	 	 	\item[param] 0 10
	 	 	\item[to.theta] \verb!function(x) x!
	 	 	\item[from.theta] \verb!function(x) x!
	 	 \end{description}
	 	\item[theta5]\ 
	 	 \begin{description}
	 	 	\item[hyperid] 56405
	 	 	\item[name] beta5
	 	 	\item[short.name] beta5
	 	 	\item[initial] 0
	 	 	\item[fixed] FALSE
	 	 	\item[prior] normal
	 	 	\item[param] 0 10
	 	 	\item[to.theta] \verb!function(x) x!
	 	 	\item[from.theta] \verb!function(x) x!
	 	 \end{description}
	 	\item[theta6]\ 
	 	 \begin{description}
	 	 	\item[hyperid] 56406
	 	 	\item[name] beta6
	 	 	\item[short.name] beta6
	 	 	\item[initial] 0
	 	 	\item[fixed] FALSE
	 	 	\item[prior] normal
	 	 	\item[param] 0 10
	 	 	\item[to.theta] \verb!function(x) x!
	 	 	\item[from.theta] \verb!function(x) x!
	 	 \end{description}
	 	\item[theta7]\ 
	 	 \begin{description}
	 	 	\item[hyperid] 56407
	 	 	\item[name] beta7
	 	 	\item[short.name] beta7
	 	 	\item[initial] 0
	 	 	\item[fixed] FALSE
	 	 	\item[prior] normal
	 	 	\item[param] 0 10
	 	 	\item[to.theta] \verb!function(x) x!
	 	 	\item[from.theta] \verb!function(x) x!
	 	 \end{description}
	 	\item[theta8]\ 
	 	 \begin{description}
	 	 	\item[hyperid] 56408
	 	 	\item[name] beta8
	 	 	\item[short.name] beta8
	 	 	\item[initial] 0
	 	 	\item[fixed] FALSE
	 	 	\item[prior] normal
	 	 	\item[param] 0 10
	 	 	\item[to.theta] \verb!function(x) x!
	 	 	\item[from.theta] \verb!function(x) x!
	 	 \end{description}
	 	\item[theta9]\ 
	 	 \begin{description}
	 	 	\item[hyperid] 56409
	 	 	\item[name] beta9
	 	 	\item[short.name] beta9
	 	 	\item[initial] 0
	 	 	\item[fixed] FALSE
	 	 	\item[prior] normal
	 	 	\item[param] 0 10
	 	 	\item[to.theta] \verb!function(x) x!
	 	 	\item[from.theta] \verb!function(x) x!
	 	 \end{description}
	 	\item[theta10]\ 
	 	 \begin{description}
	 	 	\item[hyperid] 56410
	 	 	\item[name] beta10
	 	 	\item[short.name] beta10
	 	 	\item[initial] 0
	 	 	\item[fixed] FALSE
	 	 	\item[prior] normal
	 	 	\item[param] 0 10
	 	 	\item[to.theta] \verb!function(x) x!
	 	 	\item[from.theta] \verb!function(x) x!
	 	 \end{description}
	 \end{description}
	\item[status] experimental
	\item[survival] FALSE
	\item[discrete] TRUE
	\item[link] default logit loga cauchit probit cloglog loglog log
	\item[link.simple] default logit cauchit probit cloglog
	\item[pdf] 0inflated
\end{description}
}
\paragraph{0poisson}
{\small %% DO NOT EDIT!
%% This file is generated automatically from models.R
\begin{description}
	\item[doc] New 0-inflated Binomial Swap
	\item[hyper]\ 
	 \begin{description}
	 	\item[theta1]\ 
	 	 \begin{description}
	 	 	\item[hyperid] 56501
	 	 	\item[name] beta1
	 	 	\item[short.name] beta1
	 	 	\item[initial] -4
	 	 	\item[fixed] FALSE
	 	 	\item[prior] normal
	 	 	\item[param] -4 10
	 	 	\item[to.theta] \verb!function(x) x!
	 	 	\item[from.theta] \verb!function(x) x!
	 	 \end{description}
	 	\item[theta2]\ 
	 	 \begin{description}
	 	 	\item[hyperid] 56502
	 	 	\item[name] beta2
	 	 	\item[short.name] beta2
	 	 	\item[initial] 0
	 	 	\item[fixed] FALSE
	 	 	\item[prior] normal
	 	 	\item[param] 0 10
	 	 	\item[to.theta] \verb!function(x) x!
	 	 	\item[from.theta] \verb!function(x) x!
	 	 \end{description}
	 	\item[theta3]\ 
	 	 \begin{description}
	 	 	\item[hyperid] 56503
	 	 	\item[name] beta3
	 	 	\item[short.name] beta3
	 	 	\item[initial] 0
	 	 	\item[fixed] FALSE
	 	 	\item[prior] normal
	 	 	\item[param] 0 10
	 	 	\item[to.theta] \verb!function(x) x!
	 	 	\item[from.theta] \verb!function(x) x!
	 	 \end{description}
	 	\item[theta4]\ 
	 	 \begin{description}
	 	 	\item[hyperid] 56504
	 	 	\item[name] beta4
	 	 	\item[short.name] beta4
	 	 	\item[initial] 0
	 	 	\item[fixed] FALSE
	 	 	\item[prior] normal
	 	 	\item[param] 0 10
	 	 	\item[to.theta] \verb!function(x) x!
	 	 	\item[from.theta] \verb!function(x) x!
	 	 \end{description}
	 	\item[theta5]\ 
	 	 \begin{description}
	 	 	\item[hyperid] 56505
	 	 	\item[name] beta5
	 	 	\item[short.name] beta5
	 	 	\item[initial] 0
	 	 	\item[fixed] FALSE
	 	 	\item[prior] normal
	 	 	\item[param] 0 10
	 	 	\item[to.theta] \verb!function(x) x!
	 	 	\item[from.theta] \verb!function(x) x!
	 	 \end{description}
	 	\item[theta6]\ 
	 	 \begin{description}
	 	 	\item[hyperid] 56506
	 	 	\item[name] beta6
	 	 	\item[short.name] beta6
	 	 	\item[initial] 0
	 	 	\item[fixed] FALSE
	 	 	\item[prior] normal
	 	 	\item[param] 0 10
	 	 	\item[to.theta] \verb!function(x) x!
	 	 	\item[from.theta] \verb!function(x) x!
	 	 \end{description}
	 	\item[theta7]\ 
	 	 \begin{description}
	 	 	\item[hyperid] 56507
	 	 	\item[name] beta7
	 	 	\item[short.name] beta7
	 	 	\item[initial] 0
	 	 	\item[fixed] FALSE
	 	 	\item[prior] normal
	 	 	\item[param] 0 10
	 	 	\item[to.theta] \verb!function(x) x!
	 	 	\item[from.theta] \verb!function(x) x!
	 	 \end{description}
	 	\item[theta8]\ 
	 	 \begin{description}
	 	 	\item[hyperid] 56508
	 	 	\item[name] beta8
	 	 	\item[short.name] beta8
	 	 	\item[initial] 0
	 	 	\item[fixed] FALSE
	 	 	\item[prior] normal
	 	 	\item[param] 0 10
	 	 	\item[to.theta] \verb!function(x) x!
	 	 	\item[from.theta] \verb!function(x) x!
	 	 \end{description}
	 	\item[theta9]\ 
	 	 \begin{description}
	 	 	\item[hyperid] 56509
	 	 	\item[name] beta9
	 	 	\item[short.name] beta9
	 	 	\item[initial] 0
	 	 	\item[fixed] FALSE
	 	 	\item[prior] normal
	 	 	\item[param] 0 10
	 	 	\item[to.theta] \verb!function(x) x!
	 	 	\item[from.theta] \verb!function(x) x!
	 	 \end{description}
	 	\item[theta10]\ 
	 	 \begin{description}
	 	 	\item[hyperid] 56510
	 	 	\item[name] beta10
	 	 	\item[short.name] beta10
	 	 	\item[initial] 0
	 	 	\item[fixed] FALSE
	 	 	\item[prior] normal
	 	 	\item[param] 0 10
	 	 	\item[to.theta] \verb!function(x) x!
	 	 	\item[from.theta] \verb!function(x) x!
	 	 \end{description}
	 \end{description}
	\item[status] experimental
	\item[survival] FALSE
	\item[discrete] TRUE
	\item[link] default logit loga cauchit probit cloglog ccloglog loglog log
	\item[link.simple] default logit cauchit probit cloglog ccloglog
	\item[pdf] 0inflated
\end{description}
}


\clearpage
\subsection*{Example: Poisson}
{\small \verbatiminput{example-0inflated-poisson.R}}

\clearpage
\subsection*{Example: Binomial}
{\small \verbatiminput{example-0inflated-binomial.R}}


\end{document}

% LocalWords:  np Hyperparameters Ntrials gaussian hyperparameter

%%% Local Variables: 
%%% TeX-master: t
%%% End: 

