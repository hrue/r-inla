\documentclass[a4paper,11pt]{article}
\usepackage[scale={0.8,0.9},centering,includeheadfoot]{geometry}
\usepackage{amstext}
\usepackage{amsmath}
\usepackage{verbatim}

\begin{document}
\section*{Fancy likelihood: ``fl'' (EXPERIMENTAL)}

\subsection*{Parametrisation}

This not a likelihood in the usual sense, but a artificial one to more
easily be able add missing terms into the likelihood contribution due
to various rewrites/reformulations. Obviously, it is only ment for
special occations and to be used only for those who have spesific
needs.

The ``loglikelihood'' is
\begin{displaymath}
    \log f(y) = c_1 + c_2 \eta -\frac{1}{2} c_3 (c_4 - \eta)^{2} -
    c_5 \exp(c_6 + c_7 \eta)
\end{displaymath}
for constants $c_1, \ldots, c_7$. In most cases, only a few of the
$c_i$'s will be non-zero. Note that there is no dependence on the
reponse $y$, as $y$ itself is not part of the spesification.

\subsection*{Link-function}

The identity link is used.

\subsection*{Hyperparameters}

None.

\subsection*{Specification}

\begin{itemize}
\item \texttt{family="fl"}
\item This family require the response to be a
    \texttt{inla.mdata}-object, where each row defines the vectorv
    $(c_1, \ldots, c_7)$ for each likelihood contribution. Any
    \texttt{NA}'s in the $c_i$'s will be converted to 0.
\end{itemize}

\subsubsection*{Hyperparameter spesification and default values}

\textbf{\texttt{family="fl"}}
%% DO NOT EDIT!
%% This file is generated automatically from models.R
\begin{description}
	\item[doc] \verb!The fl likelihood!
	\item[hyper]\ 
	\item[survival] \verb!FALSE!
	\item[discrete] \verb!TRUE!
	\item[link] \verb!default identity!
	\item[status] \verb!experimental!
	\item[pdf] \verb!fl!
\end{description}


\subsection*{Example}

\verbatiminput{example-fl.R}

\subsection*{Notes}

Since this is not a likelihood in the usual sense, it will not be used
for CPO/GCPO calculations and not be influenced by the
\texttt{control.inla=list(cmin=...)}-argument.

This likelihood is EXPERIMENTAL and only supported in using the
(default) \texttt{compact}-mode for the moment.

\end{document}


% LocalWords:  np Hyperparameters Ntrials gaussian

%%% Local Variables: 
%%% TeX-master: t
%%% End: 
