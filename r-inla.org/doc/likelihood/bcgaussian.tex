\documentclass[a4paper,11pt]{article}
\usepackage[scale={0.8,0.9},centering,includeheadfoot]{geometry}
\usepackage{amstext}
\usepackage{amsmath}
\usepackage{verbatim}

\begin{document}
\section*{Box-Cox Gaussian (DISABLED, NOT YET COMPLETE)}

\subsection*{Parametrisation}

The Gaussian distribution is
\begin{displaymath}
    f(y) = \frac{\sqrt{s\tau}}{\sqrt{2\pi}} y^{\lambda-1} \exp\left( -\frac{1}{2}
      s\left(f_{\lambda,\mu}(y)-\mu\right)^{2}\right) 
\end{displaymath}
for continuously responses $y>0$ where
\begin{description}
\item[$\mu$:] is the the mean
\item[$\tau$:] is the precision
\item[$s$:] is a fixed scaling, $s>0$.
\end{description}
and the Box-Cox transformation is
\begin{displaymath}
    f_{\lambda,\mu}(y) = \frac{y^{\lambda}-1}{\lambda} - \frac{\mu^{\lambda} -1}{\lambda}
\end{displaymath}


\subsection*{Link-function}

The mean and variance of $y$ are given as
\begin{displaymath}
    \mu \quad\text{and}\qquad \sigma^{2} = \frac{1}{s\tau}
\end{displaymath}
and the mean is linked to the linear predictor by
\begin{displaymath}
    \mu = \eta
\end{displaymath}

\subsection*{Hyperparameters}

The precision is $\tau$ and 
\begin{displaymath}
    \theta_1 = \log \tau
\end{displaymath}
and the prior is defined on $\theta_1$.

The Box-Cox transformation-parameter $\lambda$ is 
$\theta_2 = \lambda$
and the prior is defined on $\theta_2$.

\subsection*{Specification}

\begin{itemize}
\item \texttt{family="bcgaussian"}
\item Required arguments: $y$ and $s$ (argument \texttt{scale})
\end{itemize}
The scalings have default value 1.

\subsubsection*{Hyperparameter spesification and default values}
\documentclass[a4paper,11pt]{article}
\usepackage[scale={0.8,0.9},centering,includeheadfoot]{geometry}
\usepackage{amstext}
\usepackage{amsmath}
\usepackage{verbatim}

\begin{document}
\section*{Box-Cox Gaussian}

\subsection*{Parametrisation}

The Gaussian distribution is
\begin{displaymath}
    f(y) = \frac{\sqrt{s\tau}}{\sqrt{2\pi}} y^{\lambda-1} \exp\left( -\frac{1}{2}
      s\left(f_{\lambda,\mu}(y)-\mu\right)^{2}\right) 
\end{displaymath}
for continuously responses $y>0$ where
\begin{description}
\item[$\mu$:] is the the mean
\item[$\tau$:] is the precision
\item[$s$:] is a fixed scaling, $s>0$.
\end{description}
and the Box-Cox transformation is
\begin{displaymath}
    f_{\lambda,\mu}(y) = \frac{y^{\lambda}-1}{\lambda} - \frac{\mu^{\lambda} -1}{\lambda}
\end{displaymath}


\subsection*{Link-function}

The mean and variance of $y$ are given as
\begin{displaymath}
    \mu \quad\text{and}\qquad \sigma^{2} = \frac{1}{s\tau}
\end{displaymath}
and the mean is linked to the linear predictor by
\begin{displaymath}
    \mu = \eta
\end{displaymath}

\subsection*{Hyperparameters}

The precision is $\tau$ and 
\begin{displaymath}
    \theta_1 = \log \tau
\end{displaymath}
and the prior is defined on $\theta_1$.

The Box-Cox transformation-parameter $\lambda$ is 
$\theta_2 = \lambda$
and the prior is defined on $\theta_2$.

\subsection*{Specification}

\begin{itemize}
\item \texttt{family="bcgaussian"}
\item Required arguments: $y$ and $s$ (argument \texttt{scale})
\end{itemize}
The scalings have default value 1.

\subsubsection*{Hyperparameter spesification and default values}
\documentclass[a4paper,11pt]{article}
\usepackage[scale={0.8,0.9},centering,includeheadfoot]{geometry}
\usepackage{amstext}
\usepackage{amsmath}
\usepackage{verbatim}

\begin{document}
\section*{Box-Cox Gaussian}

\subsection*{Parametrisation}

The Gaussian distribution is
\begin{displaymath}
    f(y) = \frac{\sqrt{s\tau}}{\sqrt{2\pi}} y^{\lambda-1} \exp\left( -\frac{1}{2}
      s\left(f_{\lambda,\mu}(y)-\mu\right)^{2}\right) 
\end{displaymath}
for continuously responses $y>0$ where
\begin{description}
\item[$\mu$:] is the the mean
\item[$\tau$:] is the precision
\item[$s$:] is a fixed scaling, $s>0$.
\end{description}
and the Box-Cox transformation is
\begin{displaymath}
    f_{\lambda,\mu}(y) = \frac{y^{\lambda}-1}{\lambda} - \frac{\mu^{\lambda} -1}{\lambda}
\end{displaymath}


\subsection*{Link-function}

The mean and variance of $y$ are given as
\begin{displaymath}
    \mu \quad\text{and}\qquad \sigma^{2} = \frac{1}{s\tau}
\end{displaymath}
and the mean is linked to the linear predictor by
\begin{displaymath}
    \mu = \eta
\end{displaymath}

\subsection*{Hyperparameters}

The precision is $\tau$ and 
\begin{displaymath}
    \theta_1 = \log \tau
\end{displaymath}
and the prior is defined on $\theta_1$.

The Box-Cox transformation-parameter $\lambda$ is 
$\theta_2 = \lambda$
and the prior is defined on $\theta_2$.

\subsection*{Specification}

\begin{itemize}
\item \texttt{family="bcgaussian"}
\item Required arguments: $y$ and $s$ (argument \texttt{scale})
\end{itemize}
The scalings have default value 1.

\subsubsection*{Hyperparameter spesification and default values}
\documentclass[a4paper,11pt]{article}
\usepackage[scale={0.8,0.9},centering,includeheadfoot]{geometry}
\usepackage{amstext}
\usepackage{amsmath}
\usepackage{verbatim}

\begin{document}
\section*{Box-Cox Gaussian}

\subsection*{Parametrisation}

The Gaussian distribution is
\begin{displaymath}
    f(y) = \frac{\sqrt{s\tau}}{\sqrt{2\pi}} y^{\lambda-1} \exp\left( -\frac{1}{2}
      s\left(f_{\lambda,\mu}(y)-\mu\right)^{2}\right) 
\end{displaymath}
for continuously responses $y>0$ where
\begin{description}
\item[$\mu$:] is the the mean
\item[$\tau$:] is the precision
\item[$s$:] is a fixed scaling, $s>0$.
\end{description}
and the Box-Cox transformation is
\begin{displaymath}
    f_{\lambda,\mu}(y) = \frac{y^{\lambda}-1}{\lambda} - \frac{\mu^{\lambda} -1}{\lambda}
\end{displaymath}


\subsection*{Link-function}

The mean and variance of $y$ are given as
\begin{displaymath}
    \mu \quad\text{and}\qquad \sigma^{2} = \frac{1}{s\tau}
\end{displaymath}
and the mean is linked to the linear predictor by
\begin{displaymath}
    \mu = \eta
\end{displaymath}

\subsection*{Hyperparameters}

The precision is $\tau$ and 
\begin{displaymath}
    \theta_1 = \log \tau
\end{displaymath}
and the prior is defined on $\theta_1$.

The Box-Cox transformation-parameter $\lambda$ is 
$\theta_2 = \lambda$
and the prior is defined on $\theta_2$.

\subsection*{Specification}

\begin{itemize}
\item \texttt{family="bcgaussian"}
\item Required arguments: $y$ and $s$ (argument \texttt{scale})
\end{itemize}
The scalings have default value 1.

\subsubsection*{Hyperparameter spesification and default values}
\input{../hyper/likelihood/bcgaussian.tex}

\subsection*{Example}

\verbatiminput{example-gaussian.R}

\subsection*{Notes}


\end{document}


% LocalWords:  np Hyperparameters Ntrials gaussian

%%% Local Variables: 
%%% TeX-master: t
%%% End: 


\subsection*{Example}

\verbatiminput{example-gaussian.R}

\subsection*{Notes}


\end{document}


% LocalWords:  np Hyperparameters Ntrials gaussian

%%% Local Variables: 
%%% TeX-master: t
%%% End: 


\subsection*{Example}

\verbatiminput{example-gaussian.R}

\subsection*{Notes}


\end{document}


% LocalWords:  np Hyperparameters Ntrials gaussian

%%% Local Variables: 
%%% TeX-master: t
%%% End: 


\subsection*{Example}

\verbatiminput{example-gaussian.R}

\subsection*{Notes}


\end{document}


% LocalWords:  np Hyperparameters Ntrials gaussian

%%% Local Variables: 
%%% TeX-master: t
%%% End: 
