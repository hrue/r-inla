\documentclass[a4paper,11pt]{article}
\usepackage[scale={0.8,0.9},centering,includeheadfoot]{geometry}
\usepackage{amstext}
\usepackage{amsmath}
\usepackage{verbatim}

\begin{document}
\section*{Thinned Poisson (Experimental)}

\subsection*{Parametrisation}

The Poisson distribution is
\begin{displaymath}
    \text{Prob}(y) = \frac{\lambda^{y}}{y!}\exp(-\lambda)
\end{displaymath}
for responses $y=0, 1, 2, \ldots$, where $\lambda$ is the expected
value.

The thinned Poisson allow the observations to have a known or unknown
thinning: \texttt{event}$=1$ its observed as is. With
\texttt{event}$=0$ (or $\not=1$) its thinned, so the likelihood is
Poisson with mean $p(\cdot) \lambda$ where
\begin{displaymath}
    \text{logit}(p(\cdot)) = \text{offset} + \sum_{i=1} \beta_i x_i
\end{displaymath}

\subsection*{Link-function}

The mean $\lambda$ is linked to the linear predictor by
\begin{displaymath}
    \lambda(\eta) = E \exp(\eta)
\end{displaymath}
where $E>0$ is a known constant (or $\log(E)$ is the offset of $\eta$).

\subsection*{Hyperparameters}

$\beta_1, \beta_2, \ldots$ if in use. Maximum $10$. 

\subsection*{Specification}

\begin{itemize}
\item \texttt{family="tpoisson"}
\item Data are given as an \texttt{inla.mdata}-object, with format
    \begin{displaymath}
        \text{inla.mdata}(y, E, \text{event}, \text{offset}, x_1,
        x_2, \ldots)
    \end{displaymath}
    where maximum $10$ covariates can be given. Each argument is a
    vector. Note that the four first columns are required, and the
    covariates can be omitted if there are none.
\end{itemize}

\subsubsection*{Hyperparameter spesification and default values}
\documentclass[a4paper,11pt]{article}
\usepackage[scale={0.8,0.9},centering,includeheadfoot]{geometry}
\usepackage{amstext}
\usepackage{amsmath}
\usepackage{verbatim}

\begin{document}
\section*{Thinned Poisson (Experimental)}

\subsection*{Parametrisation}

The Poisson distribution is
\begin{displaymath}
    \text{Prob}(y) = \frac{\lambda^{y}}{y!}\exp(-\lambda)
\end{displaymath}
for responses $y=0, 1, 2, \ldots$, where $\lambda$ is the expected
value.

The thinned Poisson allow the observations to have a known or unknown
thinning: \texttt{event}$=1$ its observed as is. With
\texttt{event}$=0$ (or $\not=1$) its thinned, so the likelihood is
Poisson with mean $p(\cdot) \lambda$ where
\begin{displaymath}
    \text{logit}(p(\cdot)) = \text{offset} + \sum_{i=1} \beta_i x_i
\end{displaymath}

\subsection*{Link-function}

The mean $\lambda$ is linked to the linear predictor by
\begin{displaymath}
    \lambda(\eta) = E \exp(\eta)
\end{displaymath}
where $E>0$ is a known constant (or $\log(E)$ is the offset of $\eta$).

\subsection*{Hyperparameters}

$\beta_1, \beta_2, \ldots$ if in use. Maximum $10$. 

\subsection*{Specification}

\begin{itemize}
\item \texttt{family="tpoisson"}
\item Data are given as an \texttt{inla.mdata}-object, with format
    \begin{displaymath}
        \text{inla.mdata}(y, E, \text{event}, \text{offset}, x_1,
        x_2, \ldots)
    \end{displaymath}
    where maximum $10$ covariates can be given. Each argument is a
    vector. Note that the four first columns are required, and the
    covariates can be omitted if there are none.
\end{itemize}

\subsubsection*{Hyperparameter spesification and default values}
\documentclass[a4paper,11pt]{article}
\usepackage[scale={0.8,0.9},centering,includeheadfoot]{geometry}
\usepackage{amstext}
\usepackage{amsmath}
\usepackage{verbatim}

\begin{document}
\section*{Thinned Poisson (Experimental)}

\subsection*{Parametrisation}

The Poisson distribution is
\begin{displaymath}
    \text{Prob}(y) = \frac{\lambda^{y}}{y!}\exp(-\lambda)
\end{displaymath}
for responses $y=0, 1, 2, \ldots$, where $\lambda$ is the expected
value.

The thinned Poisson allow the observations to have a known or unknown
thinning: \texttt{event}$=1$ its observed as is. With
\texttt{event}$=0$ (or $\not=1$) its thinned, so the likelihood is
Poisson with mean $p(\cdot) \lambda$ where
\begin{displaymath}
    \text{logit}(p(\cdot)) = \text{offset} + \sum_{i=1} \beta_i x_i
\end{displaymath}

\subsection*{Link-function}

The mean $\lambda$ is linked to the linear predictor by
\begin{displaymath}
    \lambda(\eta) = E \exp(\eta)
\end{displaymath}
where $E>0$ is a known constant (or $\log(E)$ is the offset of $\eta$).

\subsection*{Hyperparameters}

$\beta_1, \beta_2, \ldots$ if in use. Maximum $10$. 

\subsection*{Specification}

\begin{itemize}
\item \texttt{family="tpoisson"}
\item Data are given as an \texttt{inla.mdata}-object, with format
    \begin{displaymath}
        \text{inla.mdata}(y, E, \text{event}, \text{offset}, x_1,
        x_2, \ldots)
    \end{displaymath}
    where maximum $10$ covariates can be given. Each argument is a
    vector. Note that the four first columns are required, and the
    covariates can be omitted if there are none.
\end{itemize}

\subsubsection*{Hyperparameter spesification and default values}
\documentclass[a4paper,11pt]{article}
\usepackage[scale={0.8,0.9},centering,includeheadfoot]{geometry}
\usepackage{amstext}
\usepackage{amsmath}
\usepackage{verbatim}

\begin{document}
\section*{Thinned Poisson (Experimental)}

\subsection*{Parametrisation}

The Poisson distribution is
\begin{displaymath}
    \text{Prob}(y) = \frac{\lambda^{y}}{y!}\exp(-\lambda)
\end{displaymath}
for responses $y=0, 1, 2, \ldots$, where $\lambda$ is the expected
value.

The thinned Poisson allow the observations to have a known or unknown
thinning: \texttt{event}$=1$ its observed as is. With
\texttt{event}$=0$ (or $\not=1$) its thinned, so the likelihood is
Poisson with mean $p(\cdot) \lambda$ where
\begin{displaymath}
    \text{logit}(p(\cdot)) = \text{offset} + \sum_{i=1} \beta_i x_i
\end{displaymath}

\subsection*{Link-function}

The mean $\lambda$ is linked to the linear predictor by
\begin{displaymath}
    \lambda(\eta) = E \exp(\eta)
\end{displaymath}
where $E>0$ is a known constant (or $\log(E)$ is the offset of $\eta$).

\subsection*{Hyperparameters}

$\beta_1, \beta_2, \ldots$ if in use. Maximum $10$. 

\subsection*{Specification}

\begin{itemize}
\item \texttt{family="tpoisson"}
\item Data are given as an \texttt{inla.mdata}-object, with format
    \begin{displaymath}
        \text{inla.mdata}(y, E, \text{event}, \text{offset}, x_1,
        x_2, \ldots)
    \end{displaymath}
    where maximum $10$ covariates can be given. Each argument is a
    vector. Note that the four first columns are required, and the
    covariates can be omitted if there are none.
\end{itemize}

\subsubsection*{Hyperparameter spesification and default values}
\input{../hyper/likelihood/tpoisson.tex}

\subsection*{Example}

In the following example we estimate the parameters in a simulated
example with Poisson responses.
\verbatiminput{example-tpoisson.R}

\subsection*{Notes}

\end{document}
% LocalWords:  np Hyperparameters Ntrials poisson

%%% Local Variables: 
%%% TeX-master: t
%%% End: 


\subsection*{Example}

In the following example we estimate the parameters in a simulated
example with Poisson responses.
\verbatiminput{example-tpoisson.R}

\subsection*{Notes}

\end{document}
% LocalWords:  np Hyperparameters Ntrials poisson

%%% Local Variables: 
%%% TeX-master: t
%%% End: 


\subsection*{Example}

In the following example we estimate the parameters in a simulated
example with Poisson responses.
\verbatiminput{example-tpoisson.R}

\subsection*{Notes}

\end{document}
% LocalWords:  np Hyperparameters Ntrials poisson

%%% Local Variables: 
%%% TeX-master: t
%%% End: 


\subsection*{Example}

In the following example we estimate the parameters in a simulated
example with Poisson responses.
\verbatiminput{example-tpoisson.R}

\subsection*{Notes}

\end{document}
% LocalWords:  np Hyperparameters Ntrials poisson

%%% Local Variables: 
%%% TeX-master: t
%%% End: 
