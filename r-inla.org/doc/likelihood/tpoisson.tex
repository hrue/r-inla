\documentclass[a4paper,11pt]{article}
\usepackage[scale={0.8,0.9},centering,includeheadfoot]{geometry}
\usepackage{amstext}
\usepackage{amsmath}
\usepackage{verbatim}

\begin{document}
\section*{Thinned Poisson (Experimental)}

\subsection*{Parametrisation}

The Poisson distribution is
\begin{displaymath}
    \text{Prob}(y) = \frac{\lambda^{y}}{y!}\exp(-\lambda)
\end{displaymath}
for responses $y=0, 1, 2, \ldots$, where $\lambda$ is the expected
value.

The thinned Poisson allow the observations to have a known or unknown
thinning: \texttt{event}$=1$ its observed as is. With
\texttt{event}$=0$ (or $\not=1$) its thinned, so the likelihood is
Poisson with mean $p(\cdot) \lambda$ where
\begin{displaymath}
    \text{logit}(p(\cdot)) = \text{offset} + \sum_{i=1} \beta_i x_i
\end{displaymath}

\subsection*{Link-function}

The mean $\lambda$ is linked to the linear predictor by
\begin{displaymath}
    \lambda(\eta) = E \exp(\eta)
\end{displaymath}
where $E>0$ is a known constant (or $\log(E)$ is the offset of $\eta$).

\subsection*{Hyperparameters}

$\beta_1, \beta_2, \ldots$ if in use. Maximum $10$. 

\subsection*{Specification}

\begin{itemize}
\item \texttt{family="tpoisson"}
\item Data are given as an \texttt{inla.mdata}-object, with format
    \begin{displaymath}
        \text{inla.mdata}(y, E, \text{event}, \text{offset}, x_1,
        x_2, \ldots)
    \end{displaymath}
    where maximum $10$ covariates can be given. Each argument is a
    vector. Note that the four first columns are required, and the
    covariates can be omitted if there are none.
\end{itemize}

\subsubsection*{Hyperparameter spesification and default values}
%% DO NOT EDIT!
%% This file is generated automatically from models.R
\begin{description}
	\item[doc] \verb!Thinned Poisson!
	\item[hyper]\ 
	 \begin{description}
	 	\item[theta1]\ 
	 	 \begin{description}
	 	 	\item[hyperid] \verb!66721!
	 	 	\item[name] \verb!beta1!
	 	 	\item[short.name] \verb!beta1!
	 	 	\item[output.name] \verb!beta1 tpoisson observations!
	 	 	\item[output.name.intern] \verb!beta1 tpoisson observations!
	 	 	\item[initial] \verb!0!
	 	 	\item[fixed] \verb!FALSE!
	 	 	\item[prior] \verb!normal!
	 	 	\item[param] \verb!0 100!
	 	 	\item[to.theta] \verb!function(x) x!
	 	 	\item[from.theta] \verb!function(x) x!
	 	 \end{description}
	 	\item[theta2]\ 
	 	 \begin{description}
	 	 	\item[hyperid] \verb!66722!
	 	 	\item[name] \verb!beta2!
	 	 	\item[short.name] \verb!beta2!
	 	 	\item[output.name] \verb!beta2 tpoisson observations!
	 	 	\item[output.name.intern] \verb!beta2 tpoisson observations!
	 	 	\item[initial] \verb!0!
	 	 	\item[fixed] \verb!FALSE!
	 	 	\item[prior] \verb!normal!
	 	 	\item[param] \verb!0 100!
	 	 	\item[to.theta] \verb!function(x) x!
	 	 	\item[from.theta] \verb!function(x) x!
	 	 \end{description}
	 	\item[theta3]\ 
	 	 \begin{description}
	 	 	\item[hyperid] \verb!66723!
	 	 	\item[name] \verb!beta3!
	 	 	\item[short.name] \verb!beta3!
	 	 	\item[output.name] \verb!beta3 tpoisson observations!
	 	 	\item[output.name.intern] \verb!beta3 tpoisson observations!
	 	 	\item[initial] \verb!0!
	 	 	\item[fixed] \verb!FALSE!
	 	 	\item[prior] \verb!normal!
	 	 	\item[param] \verb!0 100!
	 	 	\item[to.theta] \verb!function(x) x!
	 	 	\item[from.theta] \verb!function(x) x!
	 	 \end{description}
	 	\item[theta4]\ 
	 	 \begin{description}
	 	 	\item[hyperid] \verb!66724!
	 	 	\item[name] \verb!beta4!
	 	 	\item[short.name] \verb!beta4!
	 	 	\item[output.name] \verb!beta4 tpoisson observations!
	 	 	\item[output.name.intern] \verb!beta4 tpoisson observations!
	 	 	\item[initial] \verb!0!
	 	 	\item[fixed] \verb!FALSE!
	 	 	\item[prior] \verb!normal!
	 	 	\item[param] \verb!0 100!
	 	 	\item[to.theta] \verb!function(x) x!
	 	 	\item[from.theta] \verb!function(x) x!
	 	 \end{description}
	 	\item[theta5]\ 
	 	 \begin{description}
	 	 	\item[hyperid] \verb!66725!
	 	 	\item[name] \verb!beta5!
	 	 	\item[short.name] \verb!beta5!
	 	 	\item[output.name] \verb!beta5 tpoisson observations!
	 	 	\item[output.name.intern] \verb!beta5 tpoisson observations!
	 	 	\item[initial] \verb!0!
	 	 	\item[fixed] \verb!FALSE!
	 	 	\item[prior] \verb!normal!
	 	 	\item[param] \verb!0 100!
	 	 	\item[to.theta] \verb!function(x) x!
	 	 	\item[from.theta] \verb!function(x) x!
	 	 \end{description}
	 	\item[theta6]\ 
	 	 \begin{description}
	 	 	\item[hyperid] \verb!66726!
	 	 	\item[name] \verb!beta6!
	 	 	\item[short.name] \verb!beta6!
	 	 	\item[output.name] \verb!beta6 tpoisson observations!
	 	 	\item[output.name.intern] \verb!beta6 tpoisson observations!
	 	 	\item[initial] \verb!0!
	 	 	\item[fixed] \verb!FALSE!
	 	 	\item[prior] \verb!normal!
	 	 	\item[param] \verb!0 100!
	 	 	\item[to.theta] \verb!function(x) x!
	 	 	\item[from.theta] \verb!function(x) x!
	 	 \end{description}
	 	\item[theta7]\ 
	 	 \begin{description}
	 	 	\item[hyperid] \verb!66727!
	 	 	\item[name] \verb!beta7!
	 	 	\item[short.name] \verb!beta7!
	 	 	\item[output.name] \verb!beta7 tpoisson observations!
	 	 	\item[output.name.intern] \verb!beta7 tpoisson observations!
	 	 	\item[initial] \verb!0!
	 	 	\item[fixed] \verb!FALSE!
	 	 	\item[prior] \verb!normal!
	 	 	\item[param] \verb!0 100!
	 	 	\item[to.theta] \verb!function(x) x!
	 	 	\item[from.theta] \verb!function(x) x!
	 	 \end{description}
	 	\item[theta8]\ 
	 	 \begin{description}
	 	 	\item[hyperid] \verb!66728!
	 	 	\item[name] \verb!beta8!
	 	 	\item[short.name] \verb!beta8!
	 	 	\item[output.name] \verb!beta8 tpoisson observations!
	 	 	\item[output.name.intern] \verb!beta8 tpoisson observations!
	 	 	\item[initial] \verb!0!
	 	 	\item[fixed] \verb!FALSE!
	 	 	\item[prior] \verb!normal!
	 	 	\item[param] \verb!0 100!
	 	 	\item[to.theta] \verb!function(x) x!
	 	 	\item[from.theta] \verb!function(x) x!
	 	 \end{description}
	 	\item[theta9]\ 
	 	 \begin{description}
	 	 	\item[hyperid] \verb!66729!
	 	 	\item[name] \verb!beta9!
	 	 	\item[short.name] \verb!beta9!
	 	 	\item[output.name] \verb!beta9 tpoisson observations!
	 	 	\item[output.name.intern] \verb!beta9 tpoisson observations!
	 	 	\item[initial] \verb!0!
	 	 	\item[fixed] \verb!FALSE!
	 	 	\item[prior] \verb!normal!
	 	 	\item[param] \verb!0 100!
	 	 	\item[to.theta] \verb!function(x) x!
	 	 	\item[from.theta] \verb!function(x) x!
	 	 \end{description}
	 	\item[theta10]\ 
	 	 \begin{description}
	 	 	\item[hyperid] \verb!66730!
	 	 	\item[name] \verb!beta10!
	 	 	\item[short.name] \verb!beta10!
	 	 	\item[output.name] \verb!beta10 tpoisson observations!
	 	 	\item[output.name.intern] \verb!beta10 tpoisson observations!
	 	 	\item[initial] \verb!0!
	 	 	\item[fixed] \verb!FALSE!
	 	 	\item[prior] \verb!normal!
	 	 	\item[param] \verb!0 100!
	 	 	\item[to.theta] \verb!function(x) x!
	 	 	\item[from.theta] \verb!function(x) x!
	 	 \end{description}
	 \end{description}
	\item[status] \verb!experimental!
	\item[survival] \verb!FALSE!
	\item[discrete] \verb!TRUE!
	\item[link] \verb!default log!
	\item[pdf] \verb!tpoisson!
\end{description}


\subsection*{Example}

In the following example we estimate the parameters in a simulated
example with Poisson responses.
\verbatiminput{example-tpoisson.R}

\subsection*{Notes}

\end{document}
% LocalWords:  np Hyperparameters Ntrials poisson

%%% Local Variables: 
%%% TeX-master: t
%%% End: 
