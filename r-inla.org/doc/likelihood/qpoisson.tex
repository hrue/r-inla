\documentclass[a4paper,11pt]{article}
\usepackage[scale={0.8,0.9},centering,includeheadfoot]{geometry}
\usepackage{amstext}
\usepackage{amsmath}
\usepackage{verbatim}

\begin{document}
\section*{qPoisson}

\subsection*{Parametrisation}

The Poisson distribution is
\begin{displaymath}
    \text{Prob}(y) = \frac{\lambda^{y}}{y!}\exp(-\lambda)
\end{displaymath}
for responses $y=0, 1, 2, \ldots$, where
\begin{description}
\item[$\lambda$:] the expected value.
\end{description}

\subsection*{Link-function}

The mean and variance of $y$ are given as
\begin{displaymath}
    \mu = \lambda \qquad\text{and}\qquad \sigma^{2} = \lambda
\end{displaymath}
and the mean is linked to the linear predictor by
\begin{displaymath}
    \lambda(\eta) = E q_\alpha
\end{displaymath}
where $E>0$ is a known constant (or $\log(E)$ is an offset), and
$q_\alpha$ is the $\alpha$ quantile of the continous Poisson
distribution.

\subsection*{Hyperparameters}
None.


\subsection*{Specification}

\begin{itemize}
\item $\text{family}=\texttt{qpoisson}$
\item Required arguments: $y$, $E$ and $\alpha$ (given as
    \texttt{control.family = list(quantile = $\alpha$)}.
\end{itemize}

\subsubsection*{Hyperparameter spesification and default values}
%% DO NOT EDIT!
%% This file is generated automatically from models.R
\begin{description}
	\item[hyper]\ 
	\item[survival] FALSE
	\item[discrete] TRUE
	\item[link] default log
	\item[pdf] qpoisson
\end{description}


\subsection*{Example}

In the following example we estimate the parameters in a simulated
example with Poisson responses.
\verbatiminput{example-qpoisson.R}

\subsection*{Notes}



\end{document}


% LocalWords:  np Hyperparameters Ntrials poisson

%%% Local Variables: 
%%% TeX-master: t
%%% End: 
