\documentclass[a4paper,11pt]{article}
\usepackage[scale={0.8,0.9},centering,includeheadfoot]{geometry}
\usepackage{amstext}
\usepackage{amsmath}
\usepackage{verbatim}
\newcommand{\vect}[1]{\boldsymbol{#1}}
\begin{document}

\section*{qLogLogistic likelihood}

\subsection*{Parametrisation}

The LogLogistic distribution has cumulative distribution function
\begin{displaymath}
    F_0(y) = \frac{1}{1 + \lambda y^{-\alpha}}, \qquad y > 0
\end{displaymath}
if \texttt{variant=0}, or
\begin{displaymath}
    F_1(y) = \frac{1}{1 + (\lambda y)^{-\alpha}}, \qquad y > 0
\end{displaymath}
if \texttt{variant=1}, where
\begin{description}
\item[$\alpha > 0$] is a shape parameter, and
\item[$\lambda > 0$] is a scale parameter.
\end{description}
The $\lambda$ is defined implicitely through the quantile, as
\begin{displaymath}
    F_0(y_q) = q, \quad\text{or}\quad
    F_q(y_q) = q, \quad 0<q<1
\end{displaymath}
and the linear predictor is defined on $y_q$.

\subsection*{Link-functions}

The parameter $\lambda$ is linked to the linear predictor, implicitely through
\[
    y_q = \exp(\eta)
\]

\subsection*{Hyperparameters}

The $\alpha$ parameter is represented as
\[
    \theta = \log\alpha
\]
and the prior is defined on $\theta$.

\subsection*{Specification}

\begin{itemize}
\item \texttt{family} equals \texttt{qloglogistic} (regression) or
    \texttt{qloglogisticsurv} (survival)
\item \texttt{variant=0} (default) or 1, chosing between
    parameterisation $F_0$ or $F_1$.
\item Required arguments: $y$ (regression) or an
    \texttt{inla.surv}-object using \texttt{inla.surv()} (for survival
    data), and quantile$=q$.
\end{itemize}

\subsubsection*{Hyperparameter spesification and default values}
\textbf{Regression:} %% DO NOT EDIT!
%% This file is generated automatically from models.R
\begin{description}
	\item[doc] A quantile loglogistic likelihood
	\item[hyper]\ 
	 \begin{description}
	 	\item[theta]\ 
	 	 \begin{description}
	 	 	\item[hyperid] 60011
	 	 	\item[name] log alpha
	 	 	\item[short.name] alpha
	 	 	\item[initial] 1
	 	 	\item[fixed] FALSE
	 	 	\item[prior] loggamma
	 	 	\item[param] 25 25
	 	 	\item[to.theta] \verb!function(x) log(x)!
	 	 	\item[from.theta] \verb!function(x) exp(x)!
	 	 \end{description}
	 \end{description}
	\item[status] changed:Oct.25.2017
	\item[survival] FALSE
	\item[discrete] FALSE
	\item[link] default log neglog
	\item[pdf] qloglogistic
\end{description}


\textbf{Survival:} %% DO NOT EDIT!
%% This file is generated automatically from models.R
\begin{description}
	\item[doc] \verb!A quantile loglogistic likelihood (survival)!
	\item[hyper]\ 
	 \begin{description}
	 	\item[theta1]\ 
	 	 \begin{description}
	 	 	\item[hyperid] \verb!60021!
	 	 	\item[name] \verb!log alpha!
	 	 	\item[short.name] \verb!alpha!
	 	 	\item[output.name] \verb!alpha for qloglogisticsurv observations!
	 	 	\item[output.name.intern] \verb!log alpha for qloglogisticsurv observations!
	 	 	\item[initial] \verb!1!
	 	 	\item[fixed] \verb!FALSE!
	 	 	\item[prior] \verb!loggamma!
	 	 	\item[param] \verb!25 25!
	 	 	\item[to.theta] \verb!function(x) log(x)!
	 	 	\item[from.theta] \verb!function(x) exp(x)!
	 	 \end{description}
	 	\item[theta2]\ 
	 	 \begin{description}
	 	 	\item[hyperid] \verb!60022!
	 	 	\item[name] \verb!beta1!
	 	 	\item[short.name] \verb!beta1!
	 	 	\item[output.name] \verb!beta1 for qlogLogistic-Cure!
	 	 	\item[output.name.intern] \verb!beta1 for logLogistic-Cure!
	 	 	\item[initial] \verb!-5!
	 	 	\item[fixed] \verb!FALSE!
	 	 	\item[prior] \verb!normal!
	 	 	\item[param] \verb!-4 100!
	 	 	\item[to.theta] \verb!function(x) x!
	 	 	\item[from.theta] \verb!function(x) x!
	 	 \end{description}
	 	\item[theta3]\ 
	 	 \begin{description}
	 	 	\item[hyperid] \verb!60023!
	 	 	\item[name] \verb!beta2!
	 	 	\item[short.name] \verb!beta2!
	 	 	\item[output.name] \verb!beta2 for qlogLogistic-Cure!
	 	 	\item[output.name.intern] \verb!beta2 for logLogistic-Cure!
	 	 	\item[initial] \verb!0!
	 	 	\item[fixed] \verb!FALSE!
	 	 	\item[prior] \verb!normal!
	 	 	\item[param] \verb!0 100!
	 	 	\item[to.theta] \verb!function(x) x!
	 	 	\item[from.theta] \verb!function(x) x!
	 	 \end{description}
	 	\item[theta4]\ 
	 	 \begin{description}
	 	 	\item[hyperid] \verb!60024!
	 	 	\item[name] \verb!beta3!
	 	 	\item[short.name] \verb!beta3!
	 	 	\item[output.name] \verb!beta3 for qlogLogistic-Cure!
	 	 	\item[output.name.intern] \verb!beta3 for qlogLogistic-Cure!
	 	 	\item[initial] \verb!0!
	 	 	\item[fixed] \verb!FALSE!
	 	 	\item[prior] \verb!normal!
	 	 	\item[param] \verb!0 100!
	 	 	\item[to.theta] \verb!function(x) x!
	 	 	\item[from.theta] \verb!function(x) x!
	 	 \end{description}
	 	\item[theta5]\ 
	 	 \begin{description}
	 	 	\item[hyperid] \verb!60025!
	 	 	\item[name] \verb!beta4!
	 	 	\item[short.name] \verb!beta4!
	 	 	\item[output.name] \verb!beta4 for qlogLogistic-Cure!
	 	 	\item[output.name.intern] \verb!beta4 for qlogLogistic-Cure!
	 	 	\item[initial] \verb!0!
	 	 	\item[fixed] \verb!FALSE!
	 	 	\item[prior] \verb!normal!
	 	 	\item[param] \verb!0 100!
	 	 	\item[to.theta] \verb!function(x) x!
	 	 	\item[from.theta] \verb!function(x) x!
	 	 \end{description}
	 	\item[theta6]\ 
	 	 \begin{description}
	 	 	\item[hyperid] \verb!60026!
	 	 	\item[name] \verb!beta5!
	 	 	\item[short.name] \verb!beta5!
	 	 	\item[output.name] \verb!beta5 for qlogLogistic-Cure!
	 	 	\item[output.name.intern] \verb!beta5 for qlogLogistic-Cure!
	 	 	\item[initial] \verb!0!
	 	 	\item[fixed] \verb!FALSE!
	 	 	\item[prior] \verb!normal!
	 	 	\item[param] \verb!0 100!
	 	 	\item[to.theta] \verb!function(x) x!
	 	 	\item[from.theta] \verb!function(x) x!
	 	 \end{description}
	 	\item[theta7]\ 
	 	 \begin{description}
	 	 	\item[hyperid] \verb!60027!
	 	 	\item[name] \verb!beta6!
	 	 	\item[short.name] \verb!beta6!
	 	 	\item[output.name] \verb!beta6 for qlogLogistic-Cure!
	 	 	\item[output.name.intern] \verb!beta6 for qlogLogistic-Cure!
	 	 	\item[initial] \verb!0!
	 	 	\item[fixed] \verb!FALSE!
	 	 	\item[prior] \verb!normal!
	 	 	\item[param] \verb!0 100!
	 	 	\item[to.theta] \verb!function(x) x!
	 	 	\item[from.theta] \verb!function(x) x!
	 	 \end{description}
	 	\item[theta8]\ 
	 	 \begin{description}
	 	 	\item[hyperid] \verb!60028!
	 	 	\item[name] \verb!beta7!
	 	 	\item[short.name] \verb!beta7!
	 	 	\item[output.name] \verb!beta7 for qlogLogistic-Cure!
	 	 	\item[output.name.intern] \verb!beta7 for qlogLogistic-Cure!
	 	 	\item[initial] \verb!0!
	 	 	\item[fixed] \verb!FALSE!
	 	 	\item[prior] \verb!normal!
	 	 	\item[param] \verb!0 100!
	 	 	\item[to.theta] \verb!function(x) x!
	 	 	\item[from.theta] \verb!function(x) x!
	 	 \end{description}
	 	\item[theta9]\ 
	 	 \begin{description}
	 	 	\item[hyperid] \verb!60029!
	 	 	\item[name] \verb!beta8!
	 	 	\item[short.name] \verb!beta8!
	 	 	\item[output.name] \verb!beta8 for qlogLogistic-Cure!
	 	 	\item[output.name.intern] \verb!beta8 for qlogLogistic-Cure!
	 	 	\item[initial] \verb!0!
	 	 	\item[fixed] \verb!FALSE!
	 	 	\item[prior] \verb!normal!
	 	 	\item[param] \verb!0 100!
	 	 	\item[to.theta] \verb!function(x) x!
	 	 	\item[from.theta] \verb!function(x) x!
	 	 \end{description}
	 	\item[theta10]\ 
	 	 \begin{description}
	 	 	\item[hyperid] \verb!60030!
	 	 	\item[name] \verb!beta9!
	 	 	\item[short.name] \verb!beta9!
	 	 	\item[output.name] \verb!beta9 for qlogLogistic-Cure!
	 	 	\item[output.name.intern] \verb!beta9 for qlogLogistic-Cure!
	 	 	\item[initial] \verb!0!
	 	 	\item[fixed] \verb!FALSE!
	 	 	\item[prior] \verb!normal!
	 	 	\item[param] \verb!0 100!
	 	 	\item[to.theta] \verb!function(x) x!
	 	 	\item[from.theta] \verb!function(x) x!
	 	 \end{description}
	 	\item[theta11]\ 
	 	 \begin{description}
	 	 	\item[hyperid] \verb!60031!
	 	 	\item[name] \verb!beta10!
	 	 	\item[short.name] \verb!beta10!
	 	 	\item[output.name] \verb!beta10 for qlogLogistic-Cure!
	 	 	\item[output.name.intern] \verb!beta10 for qlogLogistic-Cure!
	 	 	\item[initial] \verb!0!
	 	 	\item[fixed] \verb!FALSE!
	 	 	\item[prior] \verb!normal!
	 	 	\item[param] \verb!0 100!
	 	 	\item[to.theta] \verb!function(x) x!
	 	 	\item[from.theta] \verb!function(x) x!
	 	 \end{description}
	 \end{description}
	\item[survival] \verb!TRUE!
	\item[discrete] \verb!FALSE!
	\item[link] \verb!default log neglog!
	\item[pdf] \verb!qloglogistic!
\end{description}



\subsection*{Example}

In the following example we estimate the parameters in a simulated
case \verbatiminput{example-qloglogistic.R}

\subsection*{Notes}

\begin{itemize}
\item Loglogisticsurv model can be used for right censored, left
    censored, interval censored data. If the observed times $y$ are
    large/huge, then this can cause numerical overflow in the
    likelihood routine. If you encounter this problem, try to scale
    the observatios, \verb|time = time / max(time)| or similar.
\end{itemize}

\end{document}



%%% Local Variables: 
%%% TeX-master: t
%%% End: 

