\documentclass[a4paper,11pt]{article}
\usepackage[scale={0.8,0.9},centering,includeheadfoot]{geometry}
\usepackage{amstext}
\usepackage{amsmath}
\usepackage{verbatim}

\begin{document}
\section*{The Gamma-distribution}

\subsection*{Parametrisation}

The Gamma-distribution has the following density
\begin{displaymath}
    \pi(y) = \frac{b^{a}}{\Gamma(a)} y^{a-1} \exp(-by), \qquad a>0,
    \quad b>0, \quad y >0,
\end{displaymath}
where $\text{E}(y) = \mu = a/b$ and $\text{Var}(y) = 1/\tau = a/b^{2}$,
where $\tau$ is the precision and $\mu$ is the mean. We will use the
following parameterisation for the precision
\begin{displaymath}
    \tau = (s\phi) / \mu^{2}
\end{displaymath}
where $\phi$ is the precision parameter (or $1/\phi$ is the dispersion
parameter) and $s>0$ is a fixed scaling (for the regression model),
which gives this density
\begin{displaymath}
    \pi(y) = \frac{1}{\Gamma(s\phi)}
    \left(\frac{(s\phi)}{\mu}\right)^{(s\phi)}
    y^{(s\phi) -1} \exp\left(-(s\phi) \frac{y}{\mu}\right)
\end{displaymath}


\subsection*{Link-function}

The linear predictor $\eta$ is linked to the mean $\mu$ using a
default log-link
\begin{displaymath}
    \mu = \exp(\eta)
\end{displaymath}

\subsection*{Hyperparameter}

The hyperparameter is the precision parameter $\phi$, which is
represented as
\begin{displaymath}
    \phi = \exp(\theta)
\end{displaymath}
and the prior is defined on $\theta$.

\subsection*{Specification}

\begin{itemize}
\item \texttt{family="gamma"} for regression models and
    \texttt{family="gamma.surv"} for survival models.
\item Required arguments: for \texttt{gamma.surv}, $y$ (to be given in
    a format by using $\texttt{inla.surv()}$), and for \texttt{gamma},
    $y$ and $s$ (default value 1).
\end{itemize}
The scalings $s$ is \textbf{not} used for \texttt{family="gamma.surv"}.

\subsubsection*{Hyperparameter spesification and default values}
%% DO NOT EDIT!
%% This file is generated automatically from models.R
\begin{description}
	\item[doc] \verb!The Gamma likelihood!
	\item[hyper]\ 
	 \begin{description}
	 	\item[theta]\ 
	 	 \begin{description}
	 	 	\item[hyperid] \verb!58001!
	 	 	\item[name] \verb!precision parameter!
	 	 	\item[short.name] \verb!prec!
	 	 	\item[output.name] \verb!Precision-parameter for the Gamma observations!
	 	 	\item[output.name.intern] \verb!Intern precision-parameter for the Gamma observations!
	 	 	\item[initial] \verb!4.60517018598809!
	 	 	\item[fixed] \verb!FALSE!
	 	 	\item[prior] \verb!loggamma!
	 	 	\item[param] \verb!1 0.01!
	 	 	\item[to.theta] \verb!function(x) log(x)!
	 	 	\item[from.theta] \verb!function(x) exp(x)!
	 	 \end{description}
	 \end{description}
	\item[survival] \verb!FALSE!
	\item[discrete] \verb!FALSE!
	\item[link] \verb!default log quantile!
	\item[pdf] \verb!gamma!
\end{description}

%% DO NOT EDIT!
%% This file is generated automatically from models.R
\begin{description}
	\item[doc] The Gamma likelihood (survival)
	\item[hyper]\ 
	 \begin{description}
	 	\item[theta1]\ 
	 	 \begin{description}
	 	 	\item[hyperid] 58101
	 	 	\item[name] precision parameter
	 	 	\item[short.name] prec
	 	 	\item[initial] 0
	 	 	\item[fixed] FALSE
	 	 	\item[prior] loggamma
	 	 	\item[param] 1 0.01
	 	 	\item[to.theta] \verb!function(x) log(x)!
	 	 	\item[from.theta] \verb!function(x) exp(x)!
	 	 \end{description}
	 	\item[theta2]\ 
	 	 \begin{description}
	 	 	\item[hyperid] 58102
	 	 	\item[name] beta1
	 	 	\item[short.name] beta1
	 	 	\item[initial] -7
	 	 	\item[fixed] FALSE
	 	 	\item[prior] normal
	 	 	\item[param] -4 100
	 	 	\item[to.theta] \verb!function(x) x!
	 	 	\item[from.theta] \verb!function(x) x!
	 	 \end{description}
	 	\item[theta3]\ 
	 	 \begin{description}
	 	 	\item[hyperid] 58103
	 	 	\item[name] beta2
	 	 	\item[short.name] beta2
	 	 	\item[initial] 0
	 	 	\item[fixed] FALSE
	 	 	\item[prior] normal
	 	 	\item[param] 0 100
	 	 	\item[to.theta] \verb!function(x) x!
	 	 	\item[from.theta] \verb!function(x) x!
	 	 \end{description}
	 	\item[theta4]\ 
	 	 \begin{description}
	 	 	\item[hyperid] 58104
	 	 	\item[name] beta3
	 	 	\item[short.name] beta3
	 	 	\item[initial] 0
	 	 	\item[fixed] FALSE
	 	 	\item[prior] normal
	 	 	\item[param] 0 100
	 	 	\item[to.theta] \verb!function(x) x!
	 	 	\item[from.theta] \verb!function(x) x!
	 	 \end{description}
	 	\item[theta5]\ 
	 	 \begin{description}
	 	 	\item[hyperid] 58105
	 	 	\item[name] beta4
	 	 	\item[short.name] beta4
	 	 	\item[initial] 0
	 	 	\item[fixed] FALSE
	 	 	\item[prior] normal
	 	 	\item[param] 0 100
	 	 	\item[to.theta] \verb!function(x) x!
	 	 	\item[from.theta] \verb!function(x) x!
	 	 \end{description}
	 	\item[theta6]\ 
	 	 \begin{description}
	 	 	\item[hyperid] 58106
	 	 	\item[name] beta5
	 	 	\item[short.name] beta5
	 	 	\item[initial] 0
	 	 	\item[fixed] FALSE
	 	 	\item[prior] normal
	 	 	\item[param] 0 100
	 	 	\item[to.theta] \verb!function(x) x!
	 	 	\item[from.theta] \verb!function(x) x!
	 	 \end{description}
	 	\item[theta7]\ 
	 	 \begin{description}
	 	 	\item[hyperid] 58107
	 	 	\item[name] beta6
	 	 	\item[short.name] beta6
	 	 	\item[initial] 0
	 	 	\item[fixed] FALSE
	 	 	\item[prior] normal
	 	 	\item[param] 0 100
	 	 	\item[to.theta] \verb!function(x) x!
	 	 	\item[from.theta] \verb!function(x) x!
	 	 \end{description}
	 	\item[theta8]\ 
	 	 \begin{description}
	 	 	\item[hyperid] 58108
	 	 	\item[name] beta7
	 	 	\item[short.name] beta7
	 	 	\item[initial] 0
	 	 	\item[fixed] FALSE
	 	 	\item[prior] normal
	 	 	\item[param] 0 100
	 	 	\item[to.theta] \verb!function(x) x!
	 	 	\item[from.theta] \verb!function(x) x!
	 	 \end{description}
	 	\item[theta9]\ 
	 	 \begin{description}
	 	 	\item[hyperid] 58109
	 	 	\item[name] beta8
	 	 	\item[short.name] beta8
	 	 	\item[initial] 0
	 	 	\item[fixed] FALSE
	 	 	\item[prior] normal
	 	 	\item[param] 0 100
	 	 	\item[to.theta] \verb!function(x) x!
	 	 	\item[from.theta] \verb!function(x) x!
	 	 \end{description}
	 	\item[theta10]\ 
	 	 \begin{description}
	 	 	\item[hyperid] 58110
	 	 	\item[name] beta9
	 	 	\item[short.name] beta9
	 	 	\item[initial] 0
	 	 	\item[fixed] FALSE
	 	 	\item[prior] normal
	 	 	\item[param] 0 100
	 	 	\item[to.theta] \verb!function(x) x!
	 	 	\item[from.theta] \verb!function(x) x!
	 	 \end{description}
	 	\item[theta11]\ 
	 	 \begin{description}
	 	 	\item[hyperid] 58111
	 	 	\item[name] beta10
	 	 	\item[short.name] beta10
	 	 	\item[initial] 0
	 	 	\item[fixed] FALSE
	 	 	\item[prior] normal
	 	 	\item[param] 0 100
	 	 	\item[to.theta] \verb!function(x) x!
	 	 	\item[from.theta] \verb!function(x) x!
	 	 \end{description}
	 \end{description}
	\item[survival] TRUE
	\item[discrete] FALSE
	\item[status] experimental
	\item[link] default log neglog
	\item[pdf] gammasurv
\end{description}



\subsection*{Example 1}

In the following example we estimate the parameters in a simulated
example.
\verbatiminput{example-gamma.R}

\subsection*{Example 2}

This is an example using the quantile link.
\verbatiminput{example-gamma-quantile.R}

\subsection*{Notes}

None.

\end{document}


% LocalWords:  hyperparameter overdispersion Hyperparameters nbinomial

%%% Local Variables: 
%%% TeX-master: t
%%% End: 
