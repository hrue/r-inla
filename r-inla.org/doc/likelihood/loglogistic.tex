\documentclass[a4paper,11pt]{article}
\usepackage[scale={0.8,0.9},centering,includeheadfoot]{geometry}
\usepackage{amstext}
\usepackage{amsmath}
\usepackage{verbatim}
\newcommand{\vect}[1]{\boldsymbol{#1}}
\begin{document}

\section*{LogLogistic likelihood}

\subsection*{Parametrisation}

The LogLogistic distribution has cumulative distribution function
\begin{displaymath}
    F_0(y) = \frac{1}{1 + \lambda y^{-\alpha}}, \qquad y > 0
\end{displaymath}
if \texttt{variant=0}, or
\begin{displaymath}
    F_1(y) = \frac{1}{1 + (\lambda y)^{-\alpha}}, \qquad y > 0
\end{displaymath}
if \texttt{variant=1}, where
\begin{description}
\item[$\alpha > 0$] is a shape parameter, and
\item[$\lambda > 0$] is a scale parameter.
\end{description}

\subsection*{Link-functions}

The parameter $\lambda$ is linked to the linear predictor, by default
as
\[
    \lambda = \exp(\eta)
\]

\subsection*{Hyperparameters}

The $\alpha$ parameter is represented as
\[
    \theta = \log\alpha
\]
and the prior is defined on $\theta$.

\subsection*{Specification}

\begin{itemize}
\item \texttt{family} equals \texttt{loglogistic} (regression) or
    \texttt{loglogisticsurv} (survival)
\item \texttt{variant=0} (default) or 1, chosing between
    parameterisation $F_0$ or $F_1$.
\item Required arguments: $y$ (regression) or an
    \texttt{inla.surv}-object using \texttt{inla.surv()} (for survival
    data)
\end{itemize}

\subsubsection*{Hyperparameter spesification and default values}
\textbf{Regression:} \documentclass[a4paper,11pt]{article}
\usepackage[scale={0.8,0.9},centering,includeheadfoot]{geometry}
\usepackage{amstext}
\usepackage{amsmath}
\usepackage{verbatim}
\newcommand{\vect}[1]{\boldsymbol{#1}}
\begin{document}
\section*{LogLogistic}

\subsection*{Parametrisation}

The LogLogistic distribution has cumulative distribution function
\begin{displaymath}
    F(y) = \frac{1}{1 + (y/\beta)^{-\alpha}}
\end{displaymath}
where
\begin{description}
\item[$\alpha > 0$] is the shape parameter.
\item[$\beta > 0$] is the scale parameter.
\end{description}

\subsection*{Link-function}

The parameter $\beta$ is linked to the linear predictor as:
\[
\beta = \exp(\eta)
\]
\subsection*{Hyperparameters}

The $\alpha$ parameter is represented as
\[
\theta = \log\alpha
\]
and the prior is defined on $\theta$.

\subsection*{Specification}

\begin{itemize}
\item $\text{family}=\texttt{loglogistic}$
\item Required arguments: $y$ (to be given in a format by using
    $\texttt{inla.surv()}$ function )
\end{itemize}

\subsubsection*{Hyperparameter spesification and default values}
\documentclass[a4paper,11pt]{article}
\usepackage[scale={0.8,0.9},centering,includeheadfoot]{geometry}
\usepackage{amstext}
\usepackage{amsmath}
\usepackage{verbatim}
\newcommand{\vect}[1]{\boldsymbol{#1}}
\begin{document}
\section*{LogLogistic}

\subsection*{Parametrisation}

The LogLogistic distribution has cumulative distribution function
\begin{displaymath}
    F(y) = \frac{1}{1 + (y/\beta)^{-\alpha}}
\end{displaymath}
where
\begin{description}
\item[$\alpha > 0$] is the shape parameter.
\item[$\beta > 0$] is the scale parameter.
\end{description}

\subsection*{Link-function}

The parameter $\beta$ is linked to the linear predictor as:
\[
\beta = \exp(\eta)
\]
\subsection*{Hyperparameters}

The $\alpha$ parameter is represented as
\[
\theta = \log\alpha
\]
and the prior is defined on $\theta$.

\subsection*{Specification}

\begin{itemize}
\item $\text{family}=\texttt{loglogistic}$
\item Required arguments: $y$ (to be given in a format by using
    $\texttt{inla.surv()}$ function )
\end{itemize}

\subsubsection*{Hyperparameter spesification and default values}
\documentclass[a4paper,11pt]{article}
\usepackage[scale={0.8,0.9},centering,includeheadfoot]{geometry}
\usepackage{amstext}
\usepackage{amsmath}
\usepackage{verbatim}
\newcommand{\vect}[1]{\boldsymbol{#1}}
\begin{document}
\section*{LogLogistic}

\subsection*{Parametrisation}

The LogLogistic distribution has cumulative distribution function
\begin{displaymath}
    F(y) = \frac{1}{1 + (y/\beta)^{-\alpha}}
\end{displaymath}
where
\begin{description}
\item[$\alpha > 0$] is the shape parameter.
\item[$\beta > 0$] is the scale parameter.
\end{description}

\subsection*{Link-function}

The parameter $\beta$ is linked to the linear predictor as:
\[
\beta = \exp(\eta)
\]
\subsection*{Hyperparameters}

The $\alpha$ parameter is represented as
\[
\theta = \log\alpha
\]
and the prior is defined on $\theta$.

\subsection*{Specification}

\begin{itemize}
\item $\text{family}=\texttt{loglogistic}$
\item Required arguments: $y$ (to be given in a format by using
    $\texttt{inla.surv()}$ function )
\end{itemize}

\subsubsection*{Hyperparameter spesification and default values}
\input{../hyper/likelihood/loglogistic.tex}


\subsection*{Example}

In the following example we estimate the parameters in a simulated
case \verbatiminput{example-loglogistic.R}

\subsection*{Notes}

\begin{itemize}
\item Loglogistic model can be used for right censored, left censored,
    interval censored data.
\item A general frame work to represent time is given by
    $\texttt{inla.surv}$
\item If the observed times $y$ are large/huge, then this can cause
    numerical overflow in the likelihood routines giving error
    messages like
\begin{verbatim}
        file: smtp-taucs.c  hgid: 891deb69ae0c  date: Tue Nov 09 22:34:28 2010 +0100
        Function: GMRFLib_build_sparse_matrix_TAUCS(), Line: 611, Thread: 0
        Variable evaluates to NAN/INF. This does not make sense. Abort...
\end{verbatim}
    If you encounter this problem, try to scale the observatios,
    \verb|time = time / max(time)| or similar, before running
    \verb|inla()|. 
\end{itemize}


\end{document}



%%% Local Variables: 
%%% TeX-master: t
%%% End: 




\subsection*{Example}

In the following example we estimate the parameters in a simulated
case \verbatiminput{example-loglogistic.R}

\subsection*{Notes}

\begin{itemize}
\item Loglogistic model can be used for right censored, left censored,
    interval censored data.
\item A general frame work to represent time is given by
    $\texttt{inla.surv}$
\item If the observed times $y$ are large/huge, then this can cause
    numerical overflow in the likelihood routines giving error
    messages like
\begin{verbatim}
        file: smtp-taucs.c  hgid: 891deb69ae0c  date: Tue Nov 09 22:34:28 2010 +0100
        Function: GMRFLib_build_sparse_matrix_TAUCS(), Line: 611, Thread: 0
        Variable evaluates to NAN/INF. This does not make sense. Abort...
\end{verbatim}
    If you encounter this problem, try to scale the observatios,
    \verb|time = time / max(time)| or similar, before running
    \verb|inla()|. 
\end{itemize}


\end{document}



%%% Local Variables: 
%%% TeX-master: t
%%% End: 




\subsection*{Example}

In the following example we estimate the parameters in a simulated
case \verbatiminput{example-loglogistic.R}

\subsection*{Notes}

\begin{itemize}
\item Loglogistic model can be used for right censored, left censored,
    interval censored data.
\item A general frame work to represent time is given by
    $\texttt{inla.surv}$
\item If the observed times $y$ are large/huge, then this can cause
    numerical overflow in the likelihood routines giving error
    messages like
\begin{verbatim}
        file: smtp-taucs.c  hgid: 891deb69ae0c  date: Tue Nov 09 22:34:28 2010 +0100
        Function: GMRFLib_build_sparse_matrix_TAUCS(), Line: 611, Thread: 0
        Variable evaluates to NAN/INF. This does not make sense. Abort...
\end{verbatim}
    If you encounter this problem, try to scale the observatios,
    \verb|time = time / max(time)| or similar, before running
    \verb|inla()|. 
\end{itemize}


\end{document}



%%% Local Variables: 
%%% TeX-master: t
%%% End: 



\textbf{Survival:} %% DO NOT EDIT!
%% This file is generated automatically from models.R
\begin{description}
	\item[doc] The loglogistic likelihood (survival)
	\item[hyper]\ 
	 \begin{description}
	 	\item[theta]\ 
	 	 \begin{description}
	 	 	\item[hyperid] 80011
	 	 	\item[name] log alpha
	 	 	\item[short.name] alpha
	 	 	\item[initial] 1
	 	 	\item[fixed] FALSE
	 	 	\item[prior] loggamma
	 	 	\item[param] 25 25
	 	 	\item[to.theta] \verb!function(x) log(x)!
	 	 	\item[from.theta] \verb!function(x) exp(x)!
	 	 \end{description}
	 \end{description}
	\item[status] changed:Oct.25.2017
	\item[survival] TRUE
	\item[discrete] FALSE
	\item[link] default log neglog
	\item[pdf] loglogistic
\end{description}



\subsection*{Example}

In the following example we estimate the parameters in a simulated
case \verbatiminput{example-loglogistic.R}

\subsection*{Notes}

\begin{itemize}
\item Loglogisticsurv model can be used for right censored, left
    censored, interval censored data. If the observed times $y$ are
    large/huge, then this can cause numerical overflow in the
    likelihood routine. If you encounter this problem, try to scale
    the observatios, \verb|time = time / max(time)| or similar.
\end{itemize}

\end{document}



%%% Local Variables: 
%%% TeX-master: t
%%% End: 
