\documentclass[a4paper,11pt]{article}
\usepackage[scale={0.8,0.9},centering,includeheadfoot]{geometry}
\usepackage{amstext}
\usepackage{amsmath}
\usepackage{verbatim}

\begin{document}
\section*{ME-fixed effect}

\subsection*{Parametrisation}

This family is part of the ``ME'' (measurement error) family to
emulate various measurement error models.
This ME-fixed-effect represent this model
\begin{displaymath}
    \beta(x + u)
\end{displaymath}
where $\beta$ is the fixed-effect, $x$ is the observed covariate, and
$u$ is the Gaussian error with precision $\tau$.

\subsection*{Link-function}

Not relevant.

\subsection*{Hyperparameters}

This model has two hyperparameters. 
The fixed-effect $\beta$ is $\theta_{1}$,
\begin{displaymath}
    \theta_{1} = \beta
\end{displaymath}
and the log-precision of the Gaussian error is $\theta_{2}$,
\begin{displaymath}
    \theta_{2} = \log(\tau)
\end{displaymath}
The prior is defined on $(\theta_{1},\theta_{2})$.

\subsection*{Specification}

\begin{itemize}
\item $\text{family}=\texttt{mefixedeffect}$
\item Required arguments: $x$ (as the response)
\end{itemize}
\subsubsection*{Hyperparameter spesification and default values}
\documentclass[a4paper,11pt]{article}
\usepackage[scale={0.8,0.9},centering,includeheadfoot]{geometry}
\usepackage{amstext}
\usepackage{amsmath}
\usepackage{verbatim}

\begin{document}
\section*{ME-fixed effect}

\subsection*{Parametrisation}

This family is part of the ``ME'' (measurement error) family to
emulate various measurement error models.
This ME-fixed-effect represent this model
\begin{displaymath}
    \beta(x + u)
\end{displaymath}
where $\beta$ is the fixed-effect, $x$ is the observed covariate, and
$u$ is the Gaussian error with precision $\tau$.

\subsection*{Link-function}

Not relevant.

\subsection*{Hyperparameters}

This model has two hyperparameters. 
The fixed-effect $\beta$ is $\theta_{1}$,
\begin{displaymath}
    \theta_{1} = \beta
\end{displaymath}
and the log-precision of the Gaussian error is $\theta_{2}$,
\begin{displaymath}
    \theta_{2} = \log(\tau)
\end{displaymath}
The prior is defined on $(\theta_{1},\theta_{2})$.

\subsection*{Specification}

\begin{itemize}
\item $\text{family}=\texttt{mefixedeffect}$
\item Required arguments: $x$ (as the response)
\end{itemize}
\subsubsection*{Hyperparameter spesification and default values}
\documentclass[a4paper,11pt]{article}
\usepackage[scale={0.8,0.9},centering,includeheadfoot]{geometry}
\usepackage{amstext}
\usepackage{amsmath}
\usepackage{verbatim}

\begin{document}
\section*{ME-fixed effect}

\subsection*{Parametrisation}

This family is part of the ``ME'' (measurement error) family to
emulate various measurement error models.
This ME-fixed-effect represent this model
\begin{displaymath}
    \beta(x + u)
\end{displaymath}
where $\beta$ is the fixed-effect, $x$ is the observed covariate, and
$u$ is the Gaussian error with precision $\tau$.

\subsection*{Link-function}

Not relevant.

\subsection*{Hyperparameters}

This model has two hyperparameters. 
The fixed-effect $\beta$ is $\theta_{1}$,
\begin{displaymath}
    \theta_{1} = \beta
\end{displaymath}
and the log-precision of the Gaussian error is $\theta_{2}$,
\begin{displaymath}
    \theta_{2} = \log(\tau)
\end{displaymath}
The prior is defined on $(\theta_{1},\theta_{2})$.

\subsection*{Specification}

\begin{itemize}
\item $\text{family}=\texttt{mefixedeffect}$
\item Required arguments: $x$ (as the response)
\end{itemize}
\subsubsection*{Hyperparameter spesification and default values}
\documentclass[a4paper,11pt]{article}
\usepackage[scale={0.8,0.9},centering,includeheadfoot]{geometry}
\usepackage{amstext}
\usepackage{amsmath}
\usepackage{verbatim}

\begin{document}
\section*{ME-fixed effect}

\subsection*{Parametrisation}

This family is part of the ``ME'' (measurement error) family to
emulate various measurement error models.
This ME-fixed-effect represent this model
\begin{displaymath}
    \beta(x + u)
\end{displaymath}
where $\beta$ is the fixed-effect, $x$ is the observed covariate, and
$u$ is the Gaussian error with precision $\tau$.

\subsection*{Link-function}

Not relevant.

\subsection*{Hyperparameters}

This model has two hyperparameters. 
The fixed-effect $\beta$ is $\theta_{1}$,
\begin{displaymath}
    \theta_{1} = \beta
\end{displaymath}
and the log-precision of the Gaussian error is $\theta_{2}$,
\begin{displaymath}
    \theta_{2} = \log(\tau)
\end{displaymath}
The prior is defined on $(\theta_{1},\theta_{2})$.

\subsection*{Specification}

\begin{itemize}
\item $\text{family}=\texttt{mefixedeffect}$
\item Required arguments: $x$ (as the response)
\end{itemize}
\subsubsection*{Hyperparameter spesification and default values}
\input{../hyper/likelihood/mefixedeffect.tex}

\subsection*{Example}

\verbatiminput{example-mefixedeffect.R}


\subsection*{Notes}

This model is classified as ``work in progress'' and can be changed
without further notice. In the future, it will appear as a ``f()''
model, but the internal structure in the code prevent this from happen
right now.


\end{document}


% LocalWords:  np Hyperparameters Ntrials gaussian

%%% Local Variables: 
%%% TeX-master: t
%%% End: 


\subsection*{Example}

\verbatiminput{example-mefixedeffect.R}


\subsection*{Notes}

This model is classified as ``work in progress'' and can be changed
without further notice. In the future, it will appear as a ``f()''
model, but the internal structure in the code prevent this from happen
right now.


\end{document}


% LocalWords:  np Hyperparameters Ntrials gaussian

%%% Local Variables: 
%%% TeX-master: t
%%% End: 


\subsection*{Example}

\verbatiminput{example-mefixedeffect.R}


\subsection*{Notes}

This model is classified as ``work in progress'' and can be changed
without further notice. In the future, it will appear as a ``f()''
model, but the internal structure in the code prevent this from happen
right now.


\end{document}


% LocalWords:  np Hyperparameters Ntrials gaussian

%%% Local Variables: 
%%% TeX-master: t
%%% End: 


\subsection*{Example}

\verbatiminput{example-mefixedeffect.R}


\subsection*{Notes}

This model is classified as ``work in progress'' and can be changed
without further notice. In the future, it will appear as a ``f()''
model, but the internal structure in the code prevent this from happen
right now.


\end{document}


% LocalWords:  np Hyperparameters Ntrials gaussian

%%% Local Variables: 
%%% TeX-master: t
%%% End: 
