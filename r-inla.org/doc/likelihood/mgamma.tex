\documentclass[a4paper,11pt]{article}
\usepackage[scale={0.8,0.9},centering,includeheadfoot]{geometry}
\usepackage{amstext}
\usepackage{amsmath}
\usepackage{verbatim}

\begin{document}
\section*{The modal-Gamma-distribution}

\subsection*{Parametrisation}

The modal-Gamma-distribution has the following density
\begin{displaymath}
    \pi(y) = \frac{b^{a}}{\Gamma(a)} y^{a-1} \exp(-by), \qquad a>1,
    \quad b>0, \quad y >0.
\end{displaymath}
where the mode $m=(a-1)/b$ is the target of the regression model
instead of the mean or quantile. \textbf{Note the constraint $a>1$}.

With precision
\begin{displaymath}
    \tau = (s\phi) / m^{2}
\end{displaymath}
for fixed scaling $s$ and precision parameter $\phi$, this complete
the spesification. The mapping back to the $(a,b)$-parameterisation is
then $\phi'=s\phi$,
$\delta = \left(\sqrt{\phi'(\phi'+4)} + \phi'\right)/2$, then
\begin{displaymath}
    a = 1 + \delta \quad\text{and}\quad b=\delta / m
\end{displaymath}


\subsection*{Link-function}

The linear predictor $\eta$ is linked to the mode $m$ using a
default log-link
\begin{displaymath}
    m = \exp(\eta)
\end{displaymath}

\subsection*{Hyperparameter}

The hyperparameter is the precision parameter $\phi$, which is
represented as
\begin{displaymath}
    \phi = \exp(\theta)
\end{displaymath}
and the prior is defined on $\theta$.

\subsection*{Specification}

\begin{itemize}
\item \texttt{family="mgamma"} for regression models and
    \texttt{family="mgamma.surv"} for survival models.
\item Required arguments: for \texttt{mgamma.surv}, $y$ (to be given in
    a format by using $\texttt{inla.surv()}$), and for \texttt{mgamma},
    $y$ and $s$ (default value 1).
\end{itemize}
The scalings $s$ is \textbf{not} used for \texttt{family="mgamma.surv"}.

\subsubsection*{Hyperparameter spesification and default values}
\documentclass[a4paper,11pt]{article}
\usepackage[scale={0.8,0.9},centering,includeheadfoot]{geometry}
\usepackage{amstext}
\usepackage{amsmath}
\usepackage{verbatim}

\begin{document}
\section*{The modal-Gamma-distribution}

\subsection*{Parametrisation}

The modal-Gamma-distribution has the following density
\begin{displaymath}
    \pi(y) = \frac{b^{a}}{\Gamma(a)} y^{a-1} \exp(-by), \qquad a>1,
    \quad b>0, \quad y >0.
\end{displaymath}
where the mode $m=(a-1)/b$ is the target of the regression model
instead of the mean or quantile. \textbf{Note the constraint $a>1$}.

With precision
\begin{displaymath}
    \tau = (s\phi) / m^{2}
\end{displaymath}
for fixed scaling $s$ and precision parameter $\phi$, this complete
the spesification. The mapping back to the $(a,b)$-parameterisation is
then $\phi'=s\phi$,
$\delta = \left(\sqrt{\phi'(\phi'+4)} + \phi'\right)/2$, then
\begin{displaymath}
    a = 1 + \delta \quad\text{and}\quad b=\delta / m
\end{displaymath}


\subsection*{Link-function}

The linear predictor $\eta$ is linked to the mode $m$ using a
default log-link
\begin{displaymath}
    m = \exp(\eta)
\end{displaymath}

\subsection*{Hyperparameter}

The hyperparameter is the precision parameter $\phi$, which is
represented as
\begin{displaymath}
    \phi = \exp(\theta)
\end{displaymath}
and the prior is defined on $\theta$.

\subsection*{Specification}

\begin{itemize}
\item \texttt{family="mgamma"} for regression models and
    \texttt{family="mgamma.surv"} for survival models.
\item Required arguments: for \texttt{mgamma.surv}, $y$ (to be given in
    a format by using $\texttt{inla.surv()}$), and for \texttt{mgamma},
    $y$ and $s$ (default value 1).
\end{itemize}
The scalings $s$ is \textbf{not} used for \texttt{family="mgamma.surv"}.

\subsubsection*{Hyperparameter spesification and default values}
\documentclass[a4paper,11pt]{article}
\usepackage[scale={0.8,0.9},centering,includeheadfoot]{geometry}
\usepackage{amstext}
\usepackage{amsmath}
\usepackage{verbatim}

\begin{document}
\section*{The modal-Gamma-distribution}

\subsection*{Parametrisation}

The modal-Gamma-distribution has the following density
\begin{displaymath}
    \pi(y) = \frac{b^{a}}{\Gamma(a)} y^{a-1} \exp(-by), \qquad a>1,
    \quad b>0, \quad y >0.
\end{displaymath}
where the mode $m=(a-1)/b$ is the target of the regression model
instead of the mean or quantile. \textbf{Note the constraint $a>1$}.

With precision
\begin{displaymath}
    \tau = (s\phi) / m^{2}
\end{displaymath}
for fixed scaling $s$ and precision parameter $\phi$, this complete
the spesification. The mapping back to the $(a,b)$-parameterisation is
then $\phi'=s\phi$,
$\delta = \left(\sqrt{\phi'(\phi'+4)} + \phi'\right)/2$, then
\begin{displaymath}
    a = 1 + \delta \quad\text{and}\quad b=\delta / m
\end{displaymath}


\subsection*{Link-function}

The linear predictor $\eta$ is linked to the mode $m$ using a
default log-link
\begin{displaymath}
    m = \exp(\eta)
\end{displaymath}

\subsection*{Hyperparameter}

The hyperparameter is the precision parameter $\phi$, which is
represented as
\begin{displaymath}
    \phi = \exp(\theta)
\end{displaymath}
and the prior is defined on $\theta$.

\subsection*{Specification}

\begin{itemize}
\item \texttt{family="mgamma"} for regression models and
    \texttt{family="mgamma.surv"} for survival models.
\item Required arguments: for \texttt{mgamma.surv}, $y$ (to be given in
    a format by using $\texttt{inla.surv()}$), and for \texttt{mgamma},
    $y$ and $s$ (default value 1).
\end{itemize}
The scalings $s$ is \textbf{not} used for \texttt{family="mgamma.surv"}.

\subsubsection*{Hyperparameter spesification and default values}
\documentclass[a4paper,11pt]{article}
\usepackage[scale={0.8,0.9},centering,includeheadfoot]{geometry}
\usepackage{amstext}
\usepackage{amsmath}
\usepackage{verbatim}

\begin{document}
\section*{The modal-Gamma-distribution}

\subsection*{Parametrisation}

The modal-Gamma-distribution has the following density
\begin{displaymath}
    \pi(y) = \frac{b^{a}}{\Gamma(a)} y^{a-1} \exp(-by), \qquad a>1,
    \quad b>0, \quad y >0.
\end{displaymath}
where the mode $m=(a-1)/b$ is the target of the regression model
instead of the mean or quantile. \textbf{Note the constraint $a>1$}.

With precision
\begin{displaymath}
    \tau = (s\phi) / m^{2}
\end{displaymath}
for fixed scaling $s$ and precision parameter $\phi$, this complete
the spesification. The mapping back to the $(a,b)$-parameterisation is
then $\phi'=s\phi$,
$\delta = \left(\sqrt{\phi'(\phi'+4)} + \phi'\right)/2$, then
\begin{displaymath}
    a = 1 + \delta \quad\text{and}\quad b=\delta / m
\end{displaymath}


\subsection*{Link-function}

The linear predictor $\eta$ is linked to the mode $m$ using a
default log-link
\begin{displaymath}
    m = \exp(\eta)
\end{displaymath}

\subsection*{Hyperparameter}

The hyperparameter is the precision parameter $\phi$, which is
represented as
\begin{displaymath}
    \phi = \exp(\theta)
\end{displaymath}
and the prior is defined on $\theta$.

\subsection*{Specification}

\begin{itemize}
\item \texttt{family="mgamma"} for regression models and
    \texttt{family="mgamma.surv"} for survival models.
\item Required arguments: for \texttt{mgamma.surv}, $y$ (to be given in
    a format by using $\texttt{inla.surv()}$), and for \texttt{mgamma},
    $y$ and $s$ (default value 1).
\end{itemize}
The scalings $s$ is \textbf{not} used for \texttt{family="mgamma.surv"}.

\subsubsection*{Hyperparameter spesification and default values}
\input{../hyper/likelihood/mgamma.tex}
\input{../hyper/likelihood/mgammasurv.tex}


\subsection*{Example}
In the following example we estimate the parameters in a simulated
example.
\verbatiminput{example-mgamma.R}

\subsection*{Notes}

None.

\end{document}


% LocalWords:  hyperparameter overdispersion Hyperparameters nbinomial

%%% Local Variables: 
%%% TeX-master: t
%%% End: 

%% DO NOT EDIT!
%% This file is generated automatically from models.R
\begin{description}
	\item[doc] \verb!The modal Gamma likelihood (survival)!
	\item[hyper]\ 
	 \begin{description}
	 	\item[theta1]\ 
	 	 \begin{description}
	 	 	\item[hyperid] \verb!58121!
	 	 	\item[name] \verb!precision parameter!
	 	 	\item[short.name] \verb!prec!
	 	 	\item[output.name] \verb!Precision-parameter for the modal Gamma surv observations!
	 	 	\item[output.name.intern] \verb!Intern precision-parameter for the modal Gamma surv observations!
	 	 	\item[initial] \verb!0!
	 	 	\item[fixed] \verb!FALSE!
	 	 	\item[prior] \verb!loggamma!
	 	 	\item[param] \verb!1 0.01!
	 	 	\item[to.theta] \verb!function(x) log(x)!
	 	 	\item[from.theta] \verb!function(x) exp(x)!
	 	 \end{description}
	 	\item[theta2]\ 
	 	 \begin{description}
	 	 	\item[hyperid] \verb!58122!
	 	 	\item[name] \verb!beta1!
	 	 	\item[short.name] \verb!beta1!
	 	 	\item[output.name] \verb!beta1 for modal Gamma-Cure!
	 	 	\item[output.name.intern] \verb!beta1 for modal Gamma-Cure!
	 	 	\item[initial] \verb!-7!
	 	 	\item[fixed] \verb!FALSE!
	 	 	\item[prior] \verb!normal!
	 	 	\item[param] \verb!-4 100!
	 	 	\item[to.theta] \verb!function(x) x!
	 	 	\item[from.theta] \verb!function(x) x!
	 	 \end{description}
	 	\item[theta3]\ 
	 	 \begin{description}
	 	 	\item[hyperid] \verb!58123!
	 	 	\item[name] \verb!beta2!
	 	 	\item[short.name] \verb!beta2!
	 	 	\item[output.name] \verb!beta2 for modal Gamma-Cure!
	 	 	\item[output.name.intern] \verb!beta2 for modal Gamma-Cure!
	 	 	\item[initial] \verb!0!
	 	 	\item[fixed] \verb!FALSE!
	 	 	\item[prior] \verb!normal!
	 	 	\item[param] \verb!0 100!
	 	 	\item[to.theta] \verb!function(x) x!
	 	 	\item[from.theta] \verb!function(x) x!
	 	 \end{description}
	 	\item[theta4]\ 
	 	 \begin{description}
	 	 	\item[hyperid] \verb!58124!
	 	 	\item[name] \verb!beta3!
	 	 	\item[short.name] \verb!beta3!
	 	 	\item[output.name] \verb!beta3 for modal Gamma-Cure!
	 	 	\item[output.name.intern] \verb!beta3 for modal Gamma-Cure!
	 	 	\item[initial] \verb!0!
	 	 	\item[fixed] \verb!FALSE!
	 	 	\item[prior] \verb!normal!
	 	 	\item[param] \verb!0 100!
	 	 	\item[to.theta] \verb!function(x) x!
	 	 	\item[from.theta] \verb!function(x) x!
	 	 \end{description}
	 	\item[theta5]\ 
	 	 \begin{description}
	 	 	\item[hyperid] \verb!58125!
	 	 	\item[name] \verb!beta4!
	 	 	\item[short.name] \verb!beta4!
	 	 	\item[output.name] \verb!beta4 for Ga mma-Cure!
	 	 	\item[output.name.intern] \verb!beta4 for modal Gamma-Cure!
	 	 	\item[initial] \verb!0!
	 	 	\item[fixed] \verb!FALSE!
	 	 	\item[prior] \verb!normal!
	 	 	\item[param] \verb!0 100!
	 	 	\item[to.theta] \verb!function(x) x!
	 	 	\item[from.theta] \verb!function(x) x!
	 	 \end{description}
	 	\item[theta6]\ 
	 	 \begin{description}
	 	 	\item[hyperid] \verb!58126!
	 	 	\item[name] \verb!beta5!
	 	 	\item[short.name] \verb!beta5!
	 	 	\item[output.name] \verb!beta5 for modal Gamma-Cure!
	 	 	\item[output.name.intern] \verb!beta5 for modal Gamma-Cure!
	 	 	\item[initial] \verb!0!
	 	 	\item[fixed] \verb!FALSE!
	 	 	\item[prior] \verb!normal!
	 	 	\item[param] \verb!0 100!
	 	 	\item[to.theta] \verb!function(x) x!
	 	 	\item[from.theta] \verb!function(x) x!
	 	 \end{description}
	 	\item[theta7]\ 
	 	 \begin{description}
	 	 	\item[hyperid] \verb!58127!
	 	 	\item[name] \verb!beta6!
	 	 	\item[short.name] \verb!beta6!
	 	 	\item[output.name] \verb!beta6 for modal Gamma-Cure!
	 	 	\item[output.name.intern] \verb!beta6 for modal Gamma-Cure!
	 	 	\item[initial] \verb!0!
	 	 	\item[fixed] \verb!FALSE!
	 	 	\item[prior] \verb!normal!
	 	 	\item[param] \verb!0 100!
	 	 	\item[to.theta] \verb!function(x) x!
	 	 	\item[from.theta] \verb!function(x) x!
	 	 \end{description}
	 	\item[theta8]\ 
	 	 \begin{description}
	 	 	\item[hyperid] \verb!58128!
	 	 	\item[name] \verb!beta7!
	 	 	\item[short.name] \verb!beta7!
	 	 	\item[output.name] \verb!beta7 for modal Gamma-Cure!
	 	 	\item[output.name.intern] \verb!beta7 for modal Gamma-Cure!
	 	 	\item[initial] \verb!0!
	 	 	\item[fixed] \verb!FALSE!
	 	 	\item[prior] \verb!normal!
	 	 	\item[param] \verb!0 100!
	 	 	\item[to.theta] \verb!function(x) x!
	 	 	\item[from.theta] \verb!function(x) x!
	 	 \end{description}
	 	\item[theta9]\ 
	 	 \begin{description}
	 	 	\item[hyperid] \verb!58129!
	 	 	\item[name] \verb!beta8!
	 	 	\item[short.name] \verb!beta8!
	 	 	\item[output.name] \verb!beta8 for modal Gamma-Cure!
	 	 	\item[output.name.intern] \verb!beta8 for modal Gamma-Cure!
	 	 	\item[initial] \verb!0!
	 	 	\item[fixed] \verb!FALSE!
	 	 	\item[prior] \verb!normal!
	 	 	\item[param] \verb!0 100!
	 	 	\item[to.theta] \verb!function(x) x!
	 	 	\item[from.theta] \verb!function(x) x!
	 	 \end{description}
	 	\item[theta10]\ 
	 	 \begin{description}
	 	 	\item[hyperid] \verb!58130!
	 	 	\item[name] \verb!beta9!
	 	 	\item[short.name] \verb!beta9!
	 	 	\item[output.name] \verb!beta9 for modal Gamma-Cure!
	 	 	\item[output.name.intern] \verb!beta9 for modal Gamma-Cure!
	 	 	\item[initial] \verb!0!
	 	 	\item[fixed] \verb!FALSE!
	 	 	\item[prior] \verb!normal!
	 	 	\item[param] \verb!0 100!
	 	 	\item[to.theta] \verb!function(x) x!
	 	 	\item[from.theta] \verb!function(x) x!
	 	 \end{description}
	 	\item[theta11]\ 
	 	 \begin{description}
	 	 	\item[hyperid] \verb!58131!
	 	 	\item[name] \verb!beta10!
	 	 	\item[short.name] \verb!beta10!
	 	 	\item[output.name] \verb!beta10 for modal Gamma-Cure!
	 	 	\item[output.name.intern] \verb!beta10 for modal Gamma-Cure!
	 	 	\item[initial] \verb!0!
	 	 	\item[fixed] \verb!FALSE!
	 	 	\item[prior] \verb!normal!
	 	 	\item[param] \verb!0 100!
	 	 	\item[to.theta] \verb!function(x) x!
	 	 	\item[from.theta] \verb!function(x) x!
	 	 \end{description}
	 \end{description}
	\item[survival] \verb!TRUE!
	\item[discrete] \verb!FALSE!
	\item[link] \verb!default log neglog!
	\item[pdf] \verb!agamma!
\end{description}



\subsection*{Example}
In the following example we estimate the parameters in a simulated
example.
\verbatiminput{example-mgamma.R}

\subsection*{Notes}

None.

\end{document}


% LocalWords:  hyperparameter overdispersion Hyperparameters nbinomial

%%% Local Variables: 
%%% TeX-master: t
%%% End: 

%% DO NOT EDIT!
%% This file is generated automatically from models.R
\begin{description}
	\item[doc] \verb!The modal Gamma likelihood (survival)!
	\item[hyper]\ 
	 \begin{description}
	 	\item[theta1]\ 
	 	 \begin{description}
	 	 	\item[hyperid] \verb!58121!
	 	 	\item[name] \verb!precision parameter!
	 	 	\item[short.name] \verb!prec!
	 	 	\item[output.name] \verb!Precision-parameter for the modal Gamma surv observations!
	 	 	\item[output.name.intern] \verb!Intern precision-parameter for the modal Gamma surv observations!
	 	 	\item[initial] \verb!0!
	 	 	\item[fixed] \verb!FALSE!
	 	 	\item[prior] \verb!loggamma!
	 	 	\item[param] \verb!1 0.01!
	 	 	\item[to.theta] \verb!function(x) log(x)!
	 	 	\item[from.theta] \verb!function(x) exp(x)!
	 	 \end{description}
	 	\item[theta2]\ 
	 	 \begin{description}
	 	 	\item[hyperid] \verb!58122!
	 	 	\item[name] \verb!beta1!
	 	 	\item[short.name] \verb!beta1!
	 	 	\item[output.name] \verb!beta1 for modal Gamma-Cure!
	 	 	\item[output.name.intern] \verb!beta1 for modal Gamma-Cure!
	 	 	\item[initial] \verb!-7!
	 	 	\item[fixed] \verb!FALSE!
	 	 	\item[prior] \verb!normal!
	 	 	\item[param] \verb!-4 100!
	 	 	\item[to.theta] \verb!function(x) x!
	 	 	\item[from.theta] \verb!function(x) x!
	 	 \end{description}
	 	\item[theta3]\ 
	 	 \begin{description}
	 	 	\item[hyperid] \verb!58123!
	 	 	\item[name] \verb!beta2!
	 	 	\item[short.name] \verb!beta2!
	 	 	\item[output.name] \verb!beta2 for modal Gamma-Cure!
	 	 	\item[output.name.intern] \verb!beta2 for modal Gamma-Cure!
	 	 	\item[initial] \verb!0!
	 	 	\item[fixed] \verb!FALSE!
	 	 	\item[prior] \verb!normal!
	 	 	\item[param] \verb!0 100!
	 	 	\item[to.theta] \verb!function(x) x!
	 	 	\item[from.theta] \verb!function(x) x!
	 	 \end{description}
	 	\item[theta4]\ 
	 	 \begin{description}
	 	 	\item[hyperid] \verb!58124!
	 	 	\item[name] \verb!beta3!
	 	 	\item[short.name] \verb!beta3!
	 	 	\item[output.name] \verb!beta3 for modal Gamma-Cure!
	 	 	\item[output.name.intern] \verb!beta3 for modal Gamma-Cure!
	 	 	\item[initial] \verb!0!
	 	 	\item[fixed] \verb!FALSE!
	 	 	\item[prior] \verb!normal!
	 	 	\item[param] \verb!0 100!
	 	 	\item[to.theta] \verb!function(x) x!
	 	 	\item[from.theta] \verb!function(x) x!
	 	 \end{description}
	 	\item[theta5]\ 
	 	 \begin{description}
	 	 	\item[hyperid] \verb!58125!
	 	 	\item[name] \verb!beta4!
	 	 	\item[short.name] \verb!beta4!
	 	 	\item[output.name] \verb!beta4 for Ga mma-Cure!
	 	 	\item[output.name.intern] \verb!beta4 for modal Gamma-Cure!
	 	 	\item[initial] \verb!0!
	 	 	\item[fixed] \verb!FALSE!
	 	 	\item[prior] \verb!normal!
	 	 	\item[param] \verb!0 100!
	 	 	\item[to.theta] \verb!function(x) x!
	 	 	\item[from.theta] \verb!function(x) x!
	 	 \end{description}
	 	\item[theta6]\ 
	 	 \begin{description}
	 	 	\item[hyperid] \verb!58126!
	 	 	\item[name] \verb!beta5!
	 	 	\item[short.name] \verb!beta5!
	 	 	\item[output.name] \verb!beta5 for modal Gamma-Cure!
	 	 	\item[output.name.intern] \verb!beta5 for modal Gamma-Cure!
	 	 	\item[initial] \verb!0!
	 	 	\item[fixed] \verb!FALSE!
	 	 	\item[prior] \verb!normal!
	 	 	\item[param] \verb!0 100!
	 	 	\item[to.theta] \verb!function(x) x!
	 	 	\item[from.theta] \verb!function(x) x!
	 	 \end{description}
	 	\item[theta7]\ 
	 	 \begin{description}
	 	 	\item[hyperid] \verb!58127!
	 	 	\item[name] \verb!beta6!
	 	 	\item[short.name] \verb!beta6!
	 	 	\item[output.name] \verb!beta6 for modal Gamma-Cure!
	 	 	\item[output.name.intern] \verb!beta6 for modal Gamma-Cure!
	 	 	\item[initial] \verb!0!
	 	 	\item[fixed] \verb!FALSE!
	 	 	\item[prior] \verb!normal!
	 	 	\item[param] \verb!0 100!
	 	 	\item[to.theta] \verb!function(x) x!
	 	 	\item[from.theta] \verb!function(x) x!
	 	 \end{description}
	 	\item[theta8]\ 
	 	 \begin{description}
	 	 	\item[hyperid] \verb!58128!
	 	 	\item[name] \verb!beta7!
	 	 	\item[short.name] \verb!beta7!
	 	 	\item[output.name] \verb!beta7 for modal Gamma-Cure!
	 	 	\item[output.name.intern] \verb!beta7 for modal Gamma-Cure!
	 	 	\item[initial] \verb!0!
	 	 	\item[fixed] \verb!FALSE!
	 	 	\item[prior] \verb!normal!
	 	 	\item[param] \verb!0 100!
	 	 	\item[to.theta] \verb!function(x) x!
	 	 	\item[from.theta] \verb!function(x) x!
	 	 \end{description}
	 	\item[theta9]\ 
	 	 \begin{description}
	 	 	\item[hyperid] \verb!58129!
	 	 	\item[name] \verb!beta8!
	 	 	\item[short.name] \verb!beta8!
	 	 	\item[output.name] \verb!beta8 for modal Gamma-Cure!
	 	 	\item[output.name.intern] \verb!beta8 for modal Gamma-Cure!
	 	 	\item[initial] \verb!0!
	 	 	\item[fixed] \verb!FALSE!
	 	 	\item[prior] \verb!normal!
	 	 	\item[param] \verb!0 100!
	 	 	\item[to.theta] \verb!function(x) x!
	 	 	\item[from.theta] \verb!function(x) x!
	 	 \end{description}
	 	\item[theta10]\ 
	 	 \begin{description}
	 	 	\item[hyperid] \verb!58130!
	 	 	\item[name] \verb!beta9!
	 	 	\item[short.name] \verb!beta9!
	 	 	\item[output.name] \verb!beta9 for modal Gamma-Cure!
	 	 	\item[output.name.intern] \verb!beta9 for modal Gamma-Cure!
	 	 	\item[initial] \verb!0!
	 	 	\item[fixed] \verb!FALSE!
	 	 	\item[prior] \verb!normal!
	 	 	\item[param] \verb!0 100!
	 	 	\item[to.theta] \verb!function(x) x!
	 	 	\item[from.theta] \verb!function(x) x!
	 	 \end{description}
	 	\item[theta11]\ 
	 	 \begin{description}
	 	 	\item[hyperid] \verb!58131!
	 	 	\item[name] \verb!beta10!
	 	 	\item[short.name] \verb!beta10!
	 	 	\item[output.name] \verb!beta10 for modal Gamma-Cure!
	 	 	\item[output.name.intern] \verb!beta10 for modal Gamma-Cure!
	 	 	\item[initial] \verb!0!
	 	 	\item[fixed] \verb!FALSE!
	 	 	\item[prior] \verb!normal!
	 	 	\item[param] \verb!0 100!
	 	 	\item[to.theta] \verb!function(x) x!
	 	 	\item[from.theta] \verb!function(x) x!
	 	 \end{description}
	 \end{description}
	\item[survival] \verb!TRUE!
	\item[discrete] \verb!FALSE!
	\item[link] \verb!default log neglog!
	\item[pdf] \verb!agamma!
\end{description}



\subsection*{Example}
In the following example we estimate the parameters in a simulated
example.
\verbatiminput{example-mgamma.R}

\subsection*{Notes}

None.

\end{document}


% LocalWords:  hyperparameter overdispersion Hyperparameters nbinomial

%%% Local Variables: 
%%% TeX-master: t
%%% End: 

%% DO NOT EDIT!
%% This file is generated automatically from models.R
\begin{description}
	\item[doc] \verb!The modal Gamma likelihood (survival)!
	\item[hyper]\ 
	 \begin{description}
	 	\item[theta1]\ 
	 	 \begin{description}
	 	 	\item[hyperid] \verb!58121!
	 	 	\item[name] \verb!precision parameter!
	 	 	\item[short.name] \verb!prec!
	 	 	\item[output.name] \verb!Precision-parameter for the modal Gamma surv observations!
	 	 	\item[output.name.intern] \verb!Intern precision-parameter for the modal Gamma surv observations!
	 	 	\item[initial] \verb!0!
	 	 	\item[fixed] \verb!FALSE!
	 	 	\item[prior] \verb!loggamma!
	 	 	\item[param] \verb!1 0.01!
	 	 	\item[to.theta] \verb!function(x) log(x)!
	 	 	\item[from.theta] \verb!function(x) exp(x)!
	 	 \end{description}
	 	\item[theta2]\ 
	 	 \begin{description}
	 	 	\item[hyperid] \verb!58122!
	 	 	\item[name] \verb!beta1!
	 	 	\item[short.name] \verb!beta1!
	 	 	\item[output.name] \verb!beta1 for modal Gamma-Cure!
	 	 	\item[output.name.intern] \verb!beta1 for modal Gamma-Cure!
	 	 	\item[initial] \verb!-7!
	 	 	\item[fixed] \verb!FALSE!
	 	 	\item[prior] \verb!normal!
	 	 	\item[param] \verb!-4 100!
	 	 	\item[to.theta] \verb!function(x) x!
	 	 	\item[from.theta] \verb!function(x) x!
	 	 \end{description}
	 	\item[theta3]\ 
	 	 \begin{description}
	 	 	\item[hyperid] \verb!58123!
	 	 	\item[name] \verb!beta2!
	 	 	\item[short.name] \verb!beta2!
	 	 	\item[output.name] \verb!beta2 for modal Gamma-Cure!
	 	 	\item[output.name.intern] \verb!beta2 for modal Gamma-Cure!
	 	 	\item[initial] \verb!0!
	 	 	\item[fixed] \verb!FALSE!
	 	 	\item[prior] \verb!normal!
	 	 	\item[param] \verb!0 100!
	 	 	\item[to.theta] \verb!function(x) x!
	 	 	\item[from.theta] \verb!function(x) x!
	 	 \end{description}
	 	\item[theta4]\ 
	 	 \begin{description}
	 	 	\item[hyperid] \verb!58124!
	 	 	\item[name] \verb!beta3!
	 	 	\item[short.name] \verb!beta3!
	 	 	\item[output.name] \verb!beta3 for modal Gamma-Cure!
	 	 	\item[output.name.intern] \verb!beta3 for modal Gamma-Cure!
	 	 	\item[initial] \verb!0!
	 	 	\item[fixed] \verb!FALSE!
	 	 	\item[prior] \verb!normal!
	 	 	\item[param] \verb!0 100!
	 	 	\item[to.theta] \verb!function(x) x!
	 	 	\item[from.theta] \verb!function(x) x!
	 	 \end{description}
	 	\item[theta5]\ 
	 	 \begin{description}
	 	 	\item[hyperid] \verb!58125!
	 	 	\item[name] \verb!beta4!
	 	 	\item[short.name] \verb!beta4!
	 	 	\item[output.name] \verb!beta4 for Ga mma-Cure!
	 	 	\item[output.name.intern] \verb!beta4 for modal Gamma-Cure!
	 	 	\item[initial] \verb!0!
	 	 	\item[fixed] \verb!FALSE!
	 	 	\item[prior] \verb!normal!
	 	 	\item[param] \verb!0 100!
	 	 	\item[to.theta] \verb!function(x) x!
	 	 	\item[from.theta] \verb!function(x) x!
	 	 \end{description}
	 	\item[theta6]\ 
	 	 \begin{description}
	 	 	\item[hyperid] \verb!58126!
	 	 	\item[name] \verb!beta5!
	 	 	\item[short.name] \verb!beta5!
	 	 	\item[output.name] \verb!beta5 for modal Gamma-Cure!
	 	 	\item[output.name.intern] \verb!beta5 for modal Gamma-Cure!
	 	 	\item[initial] \verb!0!
	 	 	\item[fixed] \verb!FALSE!
	 	 	\item[prior] \verb!normal!
	 	 	\item[param] \verb!0 100!
	 	 	\item[to.theta] \verb!function(x) x!
	 	 	\item[from.theta] \verb!function(x) x!
	 	 \end{description}
	 	\item[theta7]\ 
	 	 \begin{description}
	 	 	\item[hyperid] \verb!58127!
	 	 	\item[name] \verb!beta6!
	 	 	\item[short.name] \verb!beta6!
	 	 	\item[output.name] \verb!beta6 for modal Gamma-Cure!
	 	 	\item[output.name.intern] \verb!beta6 for modal Gamma-Cure!
	 	 	\item[initial] \verb!0!
	 	 	\item[fixed] \verb!FALSE!
	 	 	\item[prior] \verb!normal!
	 	 	\item[param] \verb!0 100!
	 	 	\item[to.theta] \verb!function(x) x!
	 	 	\item[from.theta] \verb!function(x) x!
	 	 \end{description}
	 	\item[theta8]\ 
	 	 \begin{description}
	 	 	\item[hyperid] \verb!58128!
	 	 	\item[name] \verb!beta7!
	 	 	\item[short.name] \verb!beta7!
	 	 	\item[output.name] \verb!beta7 for modal Gamma-Cure!
	 	 	\item[output.name.intern] \verb!beta7 for modal Gamma-Cure!
	 	 	\item[initial] \verb!0!
	 	 	\item[fixed] \verb!FALSE!
	 	 	\item[prior] \verb!normal!
	 	 	\item[param] \verb!0 100!
	 	 	\item[to.theta] \verb!function(x) x!
	 	 	\item[from.theta] \verb!function(x) x!
	 	 \end{description}
	 	\item[theta9]\ 
	 	 \begin{description}
	 	 	\item[hyperid] \verb!58129!
	 	 	\item[name] \verb!beta8!
	 	 	\item[short.name] \verb!beta8!
	 	 	\item[output.name] \verb!beta8 for modal Gamma-Cure!
	 	 	\item[output.name.intern] \verb!beta8 for modal Gamma-Cure!
	 	 	\item[initial] \verb!0!
	 	 	\item[fixed] \verb!FALSE!
	 	 	\item[prior] \verb!normal!
	 	 	\item[param] \verb!0 100!
	 	 	\item[to.theta] \verb!function(x) x!
	 	 	\item[from.theta] \verb!function(x) x!
	 	 \end{description}
	 	\item[theta10]\ 
	 	 \begin{description}
	 	 	\item[hyperid] \verb!58130!
	 	 	\item[name] \verb!beta9!
	 	 	\item[short.name] \verb!beta9!
	 	 	\item[output.name] \verb!beta9 for modal Gamma-Cure!
	 	 	\item[output.name.intern] \verb!beta9 for modal Gamma-Cure!
	 	 	\item[initial] \verb!0!
	 	 	\item[fixed] \verb!FALSE!
	 	 	\item[prior] \verb!normal!
	 	 	\item[param] \verb!0 100!
	 	 	\item[to.theta] \verb!function(x) x!
	 	 	\item[from.theta] \verb!function(x) x!
	 	 \end{description}
	 	\item[theta11]\ 
	 	 \begin{description}
	 	 	\item[hyperid] \verb!58131!
	 	 	\item[name] \verb!beta10!
	 	 	\item[short.name] \verb!beta10!
	 	 	\item[output.name] \verb!beta10 for modal Gamma-Cure!
	 	 	\item[output.name.intern] \verb!beta10 for modal Gamma-Cure!
	 	 	\item[initial] \verb!0!
	 	 	\item[fixed] \verb!FALSE!
	 	 	\item[prior] \verb!normal!
	 	 	\item[param] \verb!0 100!
	 	 	\item[to.theta] \verb!function(x) x!
	 	 	\item[from.theta] \verb!function(x) x!
	 	 \end{description}
	 \end{description}
	\item[survival] \verb!TRUE!
	\item[discrete] \verb!FALSE!
	\item[link] \verb!default log neglog!
	\item[pdf] \verb!agamma!
\end{description}



\subsection*{Example}

In the following example we estimate the parameters in a simulated
example.
\verbatiminput{example-mgamma.R}

\subsection*{Notes}

None.

\end{document}


% LocalWords:  hyperparameter overdispersion Hyperparameters nbinomial

%%% Local Variables: 
%%% TeX-master: t
%%% End: 
