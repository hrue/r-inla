\documentclass[a4paper,11pt]{article}
\usepackage[scale={0.8,0.9},centering,includeheadfoot]{geometry}
\usepackage{amstext}
\usepackage{amsmath}
\usepackage{verbatim}

\begin{document}
\section*{Skew-Normal stochastic volatility likelihood}

\subsection*{Parametrisation}

The standardised Skew-Normal distribution is
\begin{displaymath}
    f(z) = \frac{2}{\omega_{\alpha}}
    \phi\left(\frac{z -\xi_{\alpha}}{\omega_{\alpha}}\right)
    \Phi\left(\alpha \frac{z -\xi_{\alpha}}{\omega_{\alpha}}\right)
\end{displaymath}
where $\omega_\alpha$ and $\xi_\alpha$ are so that the mean is zero
and variance is one, and they a functions of the skewness parameter
$\alpha$. The skewness $s$ is the standardised skewness (standarized
third central moment), which is a function of $\alpha$

The skew-normal stochastic likelihood is defined as the density wrt
$y$, where $y$ is Skew-Normal distributed with zero mean, skewness $s$
and variance
\begin{displaymath}
    \text{variance} = \exp(\eta) + 1/\tau
\end{displaymath}
and
\begin{description}
\item[$\eta$:] is the the linear predictor
\item[$\tau$:] is an offset in the variance
\end{description}

\subsection*{Link-function}

The variance depends on the linear predictor
\begin{displaymath}
    \mu = \exp(\eta) + 1/\tau
\end{displaymath}

\subsection*{Hyperparameters}

The (standardised) skewness $s$, is represented as
\begin{displaymath}
    \gamma=0.988 (2\frac{\exp(\theta_1)}{1+\exp(\theta_1)}-1)
\end{displaymath}
and the prior is defined on $\theta_{1}$.

The offset in the variance, $1/\tau$ is represented as
\begin{displaymath}
    \tau = \exp(\theta_2)
\end{displaymath}
and the prior is defined on $\theta_2$. (By default $\theta_2$ is
fixed to a high value which makes $1/\tau \approx 0$.)

\subsection*{Specification}

\begin{itemize}
\item \texttt{family="stochvolsn"}
\item Required arguments: $y$.
\end{itemize}

\subsubsection*{Hyperparameter specification and default values}
%% DO NOT EDIT!
%% This file is generated automatically from models.R
\begin{description}
	\item[doc] \verb!The SkewNormal stochvol likelihood!
	\item[hyper]\ 
	 \begin{description}
	 	\item[theta1]\ 
	 	 \begin{description}
	 	 	\item[hyperid] \verb!82101!
	 	 	\item[name] \verb!logit skew!
	 	 	\item[short.name] \verb!skew!
	 	 	\item[output.name] \verb!Skewness for stochvol_sn observations!
	 	 	\item[output.name.intern] \verb!Intern skewness for stochvol_sn observations!
	 	 	\item[initial] \verb!0.00123456789!
	 	 	\item[fixed] \verb!FALSE!
	 	 	\item[prior] \verb!pc.sn!
	 	 	\item[param] \verb!10!
	 	 	\item[to.theta] \verb!function(x, skew.max = 0.988) log((1 + x / skew.max) / (1 - x / skew.max))!
	 	 	\item[from.theta] \verb!function(x, skew.max = 0.988) skew.max * (2 * exp(x) / (1 + exp(x)) - 1)!
	 	 \end{description}
	 	\item[theta2]\ 
	 	 \begin{description}
	 	 	\item[hyperid] \verb!82102!
	 	 	\item[name] \verb!log precision!
	 	 	\item[short.name] \verb!prec!
	 	 	\item[output.name] \verb!Offset precision for stochvol_sn!
	 	 	\item[output.name.intern] \verb!Log offset precision for stochvol_sn!
	 	 	\item[initial] \verb!500!
	 	 	\item[fixed] \verb!TRUE!
	 	 	\item[prior] \verb!loggamma!
	 	 	\item[param] \verb!1 0.005!
	 	 	\item[to.theta] \verb!function(x) log(x)!
	 	 	\item[from.theta] \verb!function(x) exp(x)!
	 	 \end{description}
	 \end{description}
	\item[status] \verb!experimental!
	\item[survival] \verb!FALSE!
	\item[discrete] \verb!FALSE!
	\item[link] \verb!default log!
	\item[pdf] \verb!stochvolsn!
\end{description}



\subsection*{Example}

\verbatiminput{example-stochvolsn.R}

\subsection*{Notes}

This implementation is similar to \texttt{family="sn"}, see also that
documentation.

\end{document}

% LocalWords:  

%%% Local Variables: 
%%% TeX-master: t
%%% End: 

