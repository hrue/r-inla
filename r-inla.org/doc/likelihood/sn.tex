\documentclass[a4paper,11pt]{article}
\usepackage[scale={0.8,0.9},centering,includeheadfoot]{geometry}
\usepackage{amstext}
\usepackage{amsmath}
\usepackage{verbatim}

\begin{document}
\section*{Skew-Normal (version 1 and 2)}

\subsection*{Parametrisation}

The Skew-Normal distribution is
\begin{displaymath}
    f(y) = 2\frac{\sqrt{w\tau}}{\sqrt{2\pi}} \exp\left( -\frac{1}{2}
      w\tau \left(y-\mu\right)^{2}\right) \;
    \Phi(a\; a_{\text{max}} \left[w\tau \left(y-\mu\right)\right])
\end{displaymath}
for continuously responses $y$ where $\Phi(\cdot)$ is the cumulative
distribution function for a standard Normal, and
\begin{description}
\item[$\mu$:] is the the location parameter
\item[$\tau$:] is the inverse scale
\item[$w$:] is a fixed weight, $w>0$,
\item[$a$:] is the shape parameter
\item[$a_{\text{max}}$:] is the (fixed) maximum value of the shape
    paramter (added for stability reasons). Default value is $5$.
\end{description}

\subsection*{Link-function}

The location parameter is linked to the linear predictor by
\begin{displaymath}
    \mu = \eta
\end{displaymath}

\subsection*{Hyperparameters}

The inverse scale is represented as
\begin{displaymath}
    \theta_{1} = \log \tau
\end{displaymath}
and the prior is defined on $\theta_{1}$. 

The shape parameter is 
\begin{displaymath}
    a = 2 \frac{\exp(\theta_{2})}{1+\exp(\theta_{2})}-1
\end{displaymath}
and the prior is defined on $\theta_{2}$. 

\subsection*{Specification}

\begin{itemize}
\item $\text{family}=\texttt{sn}$
\item Required arguments: $y$ and $w$ (keyword \texttt{scale}). The
    weights has default value 1.
\item Optional control arguments: \texttt{sn.shape.max}. Default value is
    $5.0$.
\end{itemize}

\subsubsection*{Hyperparameter spesification and default values}
\begin{description}
	\item[hyper]\ 
	 \begin{description}
	 	\item[theta1]\ 
	 	 \begin{description}
	 	 	 \item[ name ] log inverse scale 
	 	 	 \item[ short.name ] iscale 
	 	 	 \item[ initial ] 4 
	 	 	 \item[ fixed ] FALSE 
	 	 	 \item[ prior ] loggamma 
	 	 	 \item[ param ] 1 5e-05 
	 	 \end{description}
	 	\item[theta2]\ 
	 	 \begin{description}
	 	 	 \item[ name ] logit skewness 
	 	 	 \item[ short.name ] skew 
	 	 	 \item[ initial ] 4 
	 	 	 \item[ fixed ] FALSE 
	 	 	 \item[ prior ] gaussian 
	 	 	 \item[ param ] 0 10 
	 	 	 \item[ to.theta ] \verb|| 
	 	 	 \item[ from.theta ] \verb|| 
	 	 \end{description}
	 \end{description}
	 \item[ survival ] FALSE 
	 \item[ discrete ] FALSE 
	 \item[ link ] default identity 
	 \item[ pdf ] sn 
\end{description}




\subsection*{Example}

This is a simulated example requiring the package \verb@sn@.
\verbatiminput{example-sn.R}

\subsection*{Notes}

An simpler approximation to $\Phi(\cdot)$ is used to improve the
speed, which has maximum absolute error of 0.00197323; see the source
code for further details.

\clearpage


\section*{Skew-Normal (version 2)}

\subsection*{Parametrisation (version 2)}

In the family ``sn2'' we offer an alternative parametersisation of the
skew-normal with moment parameterr mean $\mu$, precision $w\tau$
(where $w$ is a fixed weight or scale) and standarized skewness
$\gamma$ (where $|\gamma|<1$
due to the skew-normal family\footnote{%
    or to be presice,
    $|\gamma| < \frac{4-\pi}{2\left(\pi/2-1\right)^{3/2}} =
    0.995271746\ldots$}). In this parameterisation, the mean
parameter is linked to the linear predictor by
\begin{displaymath}
    \mu = \eta
\end{displaymath}

\subsection*{Hyperparameters}
The precision $\tau$ is represented as
\begin{displaymath}
    \theta_{1} = \log \tau
\end{displaymath}
and the prior is defined on $\theta_{1}$. 
The (standarized) skewness $\gamma$ is 
\begin{displaymath}
    \gamma = 2 \frac{\exp(\theta_{2})}{1+\exp(\theta_{2})}-1
\end{displaymath}
and the prior is defined on $\theta_{2}$. 

The function \texttt{INLA:::inla.sn.reparam} offer the mapping between
the moments (mean, variance and skewness) and the parameters used in
the skew-normal density in the format used in the package \texttt{sn},
which are ($\xi, \omega, \alpha$), where
\begin{displaymath}
    f(x) = \frac{2}{\omega} \phi\left(\frac{x-\xi}{\omega}\right)
    \Phi\left(\alpha \left[\frac{x-\xi}{\omega}\right]\right)
\end{displaymath}
where $\phi(\cdot)$ and $\Phi(\cdot)$ is the density and cumulative
distribution function for the standard Gaussian distribution. 

\subsubsection*{Hyperparameter spesification and default values for
    \texttt{sn2}}
%% DO NOT EDIT!
%% This file is generated automatically from models.R
\begin{description}
	\item[hyper]\ 
	 \begin{description}
	 	\item[theta1]\ 
	 	 \begin{description}
	 	 	\item[name] log precision
	 	 	\item[short.name] prec
	 	 	\item[initial] 1
	 	 	\item[fixed] FALSE
	 	 	\item[prior] loggamma
	 	 	\item[param] 1 5e-05
	 	 \end{description}
	 	\item[theta2]\ 
	 	 \begin{description}
	 	 	\item[name] logit skewness
	 	 	\item[short.name] skew
	 	 	\item[initial] 0
	 	 	\item[fixed] FALSE
	 	 	\item[prior] gaussian
	 	 	\item[param] 0 10
	 	 	\item[to.theta] \verb!function(x) log((1+x)/(1-x))!
	 	 	\item[from.theta] \verb!function(x) (2*exp(x)/(1+exp(x))-1)!
	 	 \end{description}
	 \end{description}
	\item[survival] FALSE
	\item[discrete] FALSE
	\item[link] default identity
	\item[status] experimental
	\item[pdf] sn2
\end{description}


\subsection*{Example for \texttt{sn2}}

This is a simulated example requiring the package \verb@sn@.
\verbatiminput{example-sn2.R}

\subsection*{Notes for \texttt{sn2}}

In this parametersiation there is no \texttt{sn.shape.max}.

An simpler approximation to $\Phi(\cdot)$ is used to improve the
speed, which has maximum absolute error of 0.00197323; see the source
code for further details.


\end{document}


% LocalWords:  np Hyperparameters Ntrials gaussian

%%% Local Variables: 
%%% TeX-master: t
%%% End: 
