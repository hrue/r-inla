\documentclass[a4paper,11pt]{article}
\usepackage[scale={0.8,0.9},centering,includeheadfoot]{geometry}
\usepackage{amstext}
\usepackage{amsmath}
\usepackage{verbatim}

\begin{document}
\section*{Censored Poisson}

\subsection*{Parametrisation}

The Poisson distribution is
\begin{displaymath}
    \text{Prob}(y) = \frac{\lambda^{y}}{y!}\exp(-\lambda)
\end{displaymath}
for responses $y=0, 1, 2, \ldots$, where $\lambda$ is the expected
value.  The cencored version is that reponse values in the interval
$L \le y \le H$ are cencored (and reported as $y=L$, say), whereas
other values are reported as is.

\subsection*{Link-function}

The mean-parameter is $\lambda$ and is linked to the linear predictor
by
\begin{displaymath}
    \lambda(\eta) = E \exp(\eta)
\end{displaymath}
where $E>0$ is a known constant (or $\log(E)$ is the offset of $\eta$).

\subsection*{Hyperparameters}

None.

\subsection*{Specification}

\begin{itemize}
\item \texttt{family="cenpoisson"}
\item Required arguments: $y$, $E$, $L$ and $H$ (family-argument
    \texttt{cenpoisson.I=c(L,H)}).
\end{itemize}

\subsection*{Example}

In the following example we estimate the parameters in a simulated
example with Poisson responses.
{\small
\verbatiminput{example-cenpoisson.R}
}

\subsection*{Notes}

For censored values, then $y$ must be one arbitrary value in the interval; \texttt{NA} does not work!!!

\end{document}


% LocalWords:  np Hyperparameters Ntrials poisson

%%% Local Variables: 
%%% TeX-master: t
%%% End: 



% LocalWords: 

%%% Local Variables: 
%%% TeX-master: t
%%% End: 
