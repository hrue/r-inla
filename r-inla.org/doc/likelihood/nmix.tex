\documentclass[a4paper,11pt]{article}
\usepackage[scale={0.8,0.9},centering,includeheadfoot]{geometry}
\usepackage{amstext}
\usepackage{amsmath}
\usepackage{verbatim}
\def\mmax{5}
\begin{document}
\section*{NMix}

\subsection*{Parametrisation}

The N-Mixture distribution is a Poisson mixture of the Binomials, as
\begin{displaymath}
    \text{Prob}(y) = \sum_{n=y}^{\infty} {n \choose y} \ p^n
    (1-p)^{n-y} \times \frac{\lambda^{n}}{n!}\exp(-\lambda)
\end{displaymath}
for responses $y=0, 1, 2, \ldots,n$, where $n$ is Poisson number of
trials, and $p$ is probability of success.

\subsection*{Link-function}

The probability $p$ is linked to the linear predictor by
\begin{displaymath}
    p(\eta) = \frac{\exp(\eta)}{1+\exp(\eta)}
\end{displaymath}
for the default logit link, while $\lambda$ depends on fixed
covariates
\begin{displaymath}
    \log(\lambda) = \sum_{j=1}^{m} \beta_j x_j
\end{displaymath}
with one vector of covariates for each observation. $m$ can be maximum
$\mmax$ and minimum $1$.

\subsubsection*{Hyperparameters}
The parameters $\beta_1, \beta_2, \ldots, \beta_m$

\subsubsection*{Hyperparameter spesification and default values}
\documentclass[a4paper,11pt]{article}
\usepackage[scale={0.8,0.9},centering,includeheadfoot]{geometry}
\usepackage{amstext}
\usepackage{amsmath}
\usepackage{verbatim}
\def\mmax{5}
\begin{document}
\section*{NMix}

\subsection*{Parametrisation}

The N-Mixture distribution is a Poisson mixture of the Binomials, as
\begin{displaymath}
    \text{Prob}(y) = \sum_{n=y}^{\infty} {n \choose y} \ p^n
    (1-p)^{n-y} \times \frac{\lambda^{n}}{n!}\exp(-\lambda)
\end{displaymath}
for responses $y=0, 1, 2, \ldots,n$, where $n$ is Poisson number of
trials, and $p$ is probability of success.

\subsection*{Link-function}

The probability $p$ is linked to the linear predictor by
\begin{displaymath}
    p(\eta) = \frac{\exp(\eta)}{1+\exp(\eta)}
\end{displaymath}
for the default logit link, while $\lambda$ depends on fixed
covariates
\begin{displaymath}
    \log(\lambda) = \sum_{j=1}^{m} \beta_j x_j
\end{displaymath}
with one vector of covariates for each observation. $m$ can be maximum
$\mmax$ and minimum $1$.

\subsubsection*{Hyperparameters}
The parameters $\beta_1, \beta_2, \ldots, \beta_m$

\subsubsection*{Hyperparameter spesification and default values}
\documentclass[a4paper,11pt]{article}
\usepackage[scale={0.8,0.9},centering,includeheadfoot]{geometry}
\usepackage{amstext}
\usepackage{amsmath}
\usepackage{verbatim}
\def\mmax{5}
\begin{document}
\section*{NMix}

\subsection*{Parametrisation}

The N-Mixture distribution is a Poisson mixture of the Binomials, as
\begin{displaymath}
    \text{Prob}(y) = \sum_{n=y}^{\infty} {n \choose y} \ p^n
    (1-p)^{n-y} \times \frac{\lambda^{n}}{n!}\exp(-\lambda)
\end{displaymath}
for responses $y=0, 1, 2, \ldots,n$, where $n$ is Poisson number of
trials, and $p$ is probability of success.

\subsection*{Link-function}

The probability $p$ is linked to the linear predictor by
\begin{displaymath}
    p(\eta) = \frac{\exp(\eta)}{1+\exp(\eta)}
\end{displaymath}
for the default logit link, while $\lambda$ depends on fixed
covariates
\begin{displaymath}
    \log(\lambda) = \sum_{j=1}^{m} \beta_j x_j
\end{displaymath}
with one vector of covariates for each observation. $m$ can be maximum
$\mmax$ and minimum $1$.

\subsubsection*{Hyperparameters}
The parameters $\beta_1, \beta_2, \ldots, \beta_m$

\subsubsection*{Hyperparameter spesification and default values}
\documentclass[a4paper,11pt]{article}
\usepackage[scale={0.8,0.9},centering,includeheadfoot]{geometry}
\usepackage{amstext}
\usepackage{amsmath}
\usepackage{verbatim}
\def\mmax{5}
\begin{document}
\section*{NMix}

\subsection*{Parametrisation}

The N-Mixture distribution is a Poisson mixture of the Binomials, as
\begin{displaymath}
    \text{Prob}(y) = \sum_{n=y}^{\infty} {n \choose y} \ p^n
    (1-p)^{n-y} \times \frac{\lambda^{n}}{n!}\exp(-\lambda)
\end{displaymath}
for responses $y=0, 1, 2, \ldots,n$, where $n$ is Poisson number of
trials, and $p$ is probability of success.

\subsection*{Link-function}

The probability $p$ is linked to the linear predictor by
\begin{displaymath}
    p(\eta) = \frac{\exp(\eta)}{1+\exp(\eta)}
\end{displaymath}
for the default logit link, while $\lambda$ depends on fixed
covariates
\begin{displaymath}
    \log(\lambda) = \sum_{j=1}^{m} \beta_j x_j
\end{displaymath}
with one vector of covariates for each observation. $m$ can be maximum
$\mmax$ and minimum $1$.

\subsubsection*{Hyperparameters}
The parameters $\beta_1, \beta_2, \ldots, \beta_m$

\subsubsection*{Hyperparameter spesification and default values}
\input{../hyper/likelihood/nmix.tex}

\subsection*{Specification}

\begin{itemize}
\item $\text{family}=\texttt{nmix}$
\item Required arguments: the response and covariates as\\
    \verb|inla.mdata(response, covariates [, covariates])|
\end{itemize}
The response is a vector, and the covariates is a vector, matrix or
data.frame. Each row of the covariates, is
$(x_{i1}, x_{i2}, \ldots, x_{im})$, and the covariates used for the
$i$'th response. By convension, $\beta_{m+1}, \ldots, \beta_{\mmax}$
are fixed to zero.


\subsection*{Example}

In the following example we estimate the parameters in a simulated
example with binomial responses.
\verbatiminput{example-nmix.R}

\subsection*{Notes}

\end{document}


% LocalWords:  np Hyperparameters Ntrials

%%% Local Variables: 
%%% TeX-master: t
%%% End: 



% LocalWords: 

%%% Local Variables: 
%%% TeX-master: t
%%% End: 


\subsection*{Specification}

\begin{itemize}
\item $\text{family}=\texttt{nmix}$
\item Required arguments: the response and covariates as\\
    \verb|inla.mdata(response, covariates [, covariates])|
\end{itemize}
The response is a vector, and the covariates is a vector, matrix or
data.frame. Each row of the covariates, is
$(x_{i1}, x_{i2}, \ldots, x_{im})$, and the covariates used for the
$i$'th response. By convension, $\beta_{m+1}, \ldots, \beta_{\mmax}$
are fixed to zero.


\subsection*{Example}

In the following example we estimate the parameters in a simulated
example with binomial responses.
\verbatiminput{example-nmix.R}

\subsection*{Notes}

\end{document}


% LocalWords:  np Hyperparameters Ntrials

%%% Local Variables: 
%%% TeX-master: t
%%% End: 



% LocalWords: 

%%% Local Variables: 
%%% TeX-master: t
%%% End: 


\subsection*{Specification}

\begin{itemize}
\item $\text{family}=\texttt{nmix}$
\item Required arguments: the response and covariates as\\
    \verb|inla.mdata(response, covariates [, covariates])|
\end{itemize}
The response is a vector, and the covariates is a vector, matrix or
data.frame. Each row of the covariates, is
$(x_{i1}, x_{i2}, \ldots, x_{im})$, and the covariates used for the
$i$'th response. By convension, $\beta_{m+1}, \ldots, \beta_{\mmax}$
are fixed to zero.


\subsection*{Example}

In the following example we estimate the parameters in a simulated
example with binomial responses.
\verbatiminput{example-nmix.R}

\subsection*{Notes}

\end{document}


% LocalWords:  np Hyperparameters Ntrials

%%% Local Variables: 
%%% TeX-master: t
%%% End: 



% LocalWords: 

%%% Local Variables: 
%%% TeX-master: t
%%% End: 


\subsection*{Specification}

\begin{itemize}
\item $\text{family}=\texttt{nmix}$
\item Required arguments: the response and covariates as\\
    \verb|inla.mdata(response, covariates [, covariates])|
\end{itemize}
The response is a vector, and the covariates is a vector, matrix or
data.frame. Each row of the covariates, is
$(x_{i1}, x_{i2}, \ldots, x_{im})$, and the covariates used for the
$i$'th response. By convension, $\beta_{m+1}, \ldots, \beta_{\mmax}$
are fixed to zero.


\subsection*{Example}

In the following example we estimate the parameters in a simulated
example with binomial responses.
\verbatiminput{example-nmix.R}

\subsection*{Notes}

\end{document}


% LocalWords:  np Hyperparameters Ntrials

%%% Local Variables: 
%%% TeX-master: t
%%% End: 



% LocalWords: 

%%% Local Variables: 
%%% TeX-master: t
%%% End: 
