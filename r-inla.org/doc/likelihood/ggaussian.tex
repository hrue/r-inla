\documentclass[a4paper,11pt]{article}
\usepackage[scale={0.8,0.9},centering,includeheadfoot]{geometry}
\usepackage{amstext}
\usepackage{amsmath}
\usepackage{verbatim}

\begin{document}
\section*{More general Gaussian likelihoods: GGaussian and GGaussianS}

\subsection*{Parameterisation}

The Gaussian distribution is
\begin{displaymath}
    f(y) = \frac{\sqrt{\tau}}{\sqrt{2\pi}} \exp\left( -\frac{1}{2}
      \tau \left(y-\mu\right)^{2}\right)
\end{displaymath}
for continuously responses $y$ where
\begin{description}
\item[$\mu$:] is the the mean
\item[$\tau$:] is the precision
\end{description}
These likelihood families generalise the normal procedure. We allow
for two linear predictors in the model. One needs to be ``simple'',
only consists of fixed effects, while the other one is general and
defined via the formula.

For family=``ggaussian'', then
\begin{displaymath}
    \text{link}(\mu) = \text{formula} \qquad
    \text{link.simple}(\tau) = \text{simple}
\end{displaymath}
while for family=``ggaussianS'', then
\begin{displaymath}
    \text{link}(\tau) = \text{formula} \qquad
    \text{link.simple}(\mu) = \text{simple}
\end{displaymath}
The default link is the \emph{identity} for the mean and the
\emph{log} for the precision. We will describe each model separately,
as the specifications are a little different.

\subsection*{Family \emph{ggaussian}}
\subsubsection*{Link-function}

The mean is given by the linear predictor $\eta$ from the formula
\begin{displaymath}
    \text{link}(\mu) = \eta
\end{displaymath}
where the default is the \emph{identity}-link. The precision is given
as
\begin{displaymath}
    \text{link.simple}(\frac{1}{s}\tau) = \beta_1 z_1 + \beta_2
    z_2 + \ldots + \beta_{m} z_{m}, \qquad m \leq 10
\end{displaymath}
where $s$ is a \textbf{fixed} scaling (or often log-offset) and
similar to the argument \emph{scale} for \text{family=``gaussian''}.
For the log-link, the precision is expressed as
\begin{displaymath}
    \tau = s \times \exp\left(\beta_1 z_1 + \beta_2
      z_2 + \ldots + \beta_{m} z_{m}\right)
\end{displaymath}
\textbf{Note:} there is no default intercept in the simple model, so
it is common to set $z_1=1$.

\subsubsection*{Hyperparameters}
The hyperparameters in the model are
$\theta_1=\beta_1, \ldots, \theta_m=\beta_m$, and the priors are
defined on $\theta_1, \ldots, \theta_m$.

\textbf{Note:} The default prior for $\theta_1$ is similar to the
default prior for the precision for the Gaussian family, please change
if you do not want this feature.


\subsubsection*{Specification}
\begin{itemize}
\item \texttt{family="ggaussian"}
\item Required arguments: \textbf{inla.mdata}-object that defines the
    response, $s$ and covariates. The \textbf{inla.mdata} object is
    defined as
    \begin{displaymath}
        \text{inla.mdata}(y, s, z_1, \ldots, z_m)
    \end{displaymath}
    where each argument are vectors of the same length.
\item \textbf{Note:} The scaling argument $s$ \textbf{MUST} be given
    as it has no default value. Often we can just use $s=1$ as it will
    auto-expand.
\item \textbf{Note:} $m=0$ is allowed, which means that $\tau=s$ with
    with log link.simple.
\end{itemize}

\subsubsection*{Hyperparameter spesification and default values}
\textbf{\texttt{family="ggaussian"}}
%% DO NOT EDIT!
%% This file is generated automatically from models.R
\begin{description}
	\item[doc] \verb!Generalized Gaussian!
	\item[hyper]\ 
	 \begin{description}
	 	\item[theta1]\ 
	 	 \begin{description}
	 	 	\item[hyperid] \verb!66501!
	 	 	\item[name] \verb!beta1!
	 	 	\item[short.name] \verb!beta1!
	 	 	\item[output.name] \verb!beta1 for ggaussian observations!
	 	 	\item[output.name.intern] \verb!beta1 for ggaussian observations!
	 	 	\item[initial] \verb!4!
	 	 	\item[fixed] \verb!FALSE!
	 	 	\item[prior] \verb!normal!
	 	 	\item[param] \verb!9.33 0.61!
	 	 	\item[to.theta] \verb!function(x) x!
	 	 	\item[from.theta] \verb!function(x) x!
	 	 \end{description}
	 	\item[theta2]\ 
	 	 \begin{description}
	 	 	\item[hyperid] \verb!66502!
	 	 	\item[name] \verb!beta2!
	 	 	\item[short.name] \verb!beta2!
	 	 	\item[output.name] \verb!beta2 for ggaussian observations!
	 	 	\item[output.name.intern] \verb!beta2 for ggaussian observations!
	 	 	\item[initial] \verb!0!
	 	 	\item[fixed] \verb!FALSE!
	 	 	\item[prior] \verb!normal!
	 	 	\item[param] \verb!0 10!
	 	 	\item[to.theta] \verb!function(x) x!
	 	 	\item[from.theta] \verb!function(x) x!
	 	 \end{description}
	 	\item[theta3]\ 
	 	 \begin{description}
	 	 	\item[hyperid] \verb!66503!
	 	 	\item[name] \verb!beta3!
	 	 	\item[short.name] \verb!beta3!
	 	 	\item[output.name] \verb!beta3 for ggaussian observations!
	 	 	\item[output.name.intern] \verb!beta3 for ggaussian observations!
	 	 	\item[initial] \verb!0!
	 	 	\item[fixed] \verb!FALSE!
	 	 	\item[prior] \verb!normal!
	 	 	\item[param] \verb!0 10!
	 	 	\item[to.theta] \verb!function(x) x!
	 	 	\item[from.theta] \verb!function(x) x!
	 	 \end{description}
	 	\item[theta4]\ 
	 	 \begin{description}
	 	 	\item[hyperid] \verb!66504!
	 	 	\item[name] \verb!beta4!
	 	 	\item[short.name] \verb!beta4!
	 	 	\item[output.name] \verb!beta4 for ggaussian observations!
	 	 	\item[output.name.intern] \verb!beta4 for ggaussian observations!
	 	 	\item[initial] \verb!0!
	 	 	\item[fixed] \verb!FALSE!
	 	 	\item[prior] \verb!normal!
	 	 	\item[param] \verb!0 10!
	 	 	\item[to.theta] \verb!function(x) x!
	 	 	\item[from.theta] \verb!function(x) x!
	 	 \end{description}
	 	\item[theta5]\ 
	 	 \begin{description}
	 	 	\item[hyperid] \verb!66505!
	 	 	\item[name] \verb!beta5!
	 	 	\item[short.name] \verb!beta5!
	 	 	\item[output.name] \verb!beta5 for ggaussian observations!
	 	 	\item[output.name.intern] \verb!beta5 for ggaussian observations!
	 	 	\item[initial] \verb!0!
	 	 	\item[fixed] \verb!FALSE!
	 	 	\item[prior] \verb!normal!
	 	 	\item[param] \verb!0 10!
	 	 	\item[to.theta] \verb!function(x) x!
	 	 	\item[from.theta] \verb!function(x) x!
	 	 \end{description}
	 	\item[theta6]\ 
	 	 \begin{description}
	 	 	\item[hyperid] \verb!66506!
	 	 	\item[name] \verb!beta6!
	 	 	\item[short.name] \verb!beta6!
	 	 	\item[output.name] \verb!beta6 for ggaussian observations!
	 	 	\item[output.name.intern] \verb!beta6 for ggaussian observations!
	 	 	\item[initial] \verb!0!
	 	 	\item[fixed] \verb!FALSE!
	 	 	\item[prior] \verb!normal!
	 	 	\item[param] \verb!0 10!
	 	 	\item[to.theta] \verb!function(x) x!
	 	 	\item[from.theta] \verb!function(x) x!
	 	 \end{description}
	 	\item[theta7]\ 
	 	 \begin{description}
	 	 	\item[hyperid] \verb!66507!
	 	 	\item[name] \verb!beta7!
	 	 	\item[short.name] \verb!beta7!
	 	 	\item[output.name] \verb!beta7 for ggaussian observations!
	 	 	\item[output.name.intern] \verb!beta7 for ggaussian observations!
	 	 	\item[initial] \verb!0!
	 	 	\item[fixed] \verb!FALSE!
	 	 	\item[prior] \verb!normal!
	 	 	\item[param] \verb!0 10!
	 	 	\item[to.theta] \verb!function(x) x!
	 	 	\item[from.theta] \verb!function(x) x!
	 	 \end{description}
	 	\item[theta8]\ 
	 	 \begin{description}
	 	 	\item[hyperid] \verb!66508!
	 	 	\item[name] \verb!beta8!
	 	 	\item[short.name] \verb!beta8!
	 	 	\item[output.name] \verb!beta8 for ggaussian observations!
	 	 	\item[output.name.intern] \verb!beta8 for ggaussian observations!
	 	 	\item[initial] \verb!0!
	 	 	\item[fixed] \verb!FALSE!
	 	 	\item[prior] \verb!normal!
	 	 	\item[param] \verb!0 10!
	 	 	\item[to.theta] \verb!function(x) x!
	 	 	\item[from.theta] \verb!function(x) x!
	 	 \end{description}
	 	\item[theta9]\ 
	 	 \begin{description}
	 	 	\item[hyperid] \verb!66509!
	 	 	\item[name] \verb!beta9!
	 	 	\item[short.name] \verb!beta9!
	 	 	\item[output.name] \verb!beta9 for ggaussian observations!
	 	 	\item[output.name.intern] \verb!beta9 for ggaussian observations!
	 	 	\item[initial] \verb!0!
	 	 	\item[fixed] \verb!FALSE!
	 	 	\item[prior] \verb!normal!
	 	 	\item[param] \verb!0 10!
	 	 	\item[to.theta] \verb!function(x) x!
	 	 	\item[from.theta] \verb!function(x) x!
	 	 \end{description}
	 	\item[theta10]\ 
	 	 \begin{description}
	 	 	\item[hyperid] \verb!66510!
	 	 	\item[name] \verb!beta10!
	 	 	\item[short.name] \verb!beta10!
	 	 	\item[output.name] \verb!beta10 for ggaussian observations!
	 	 	\item[output.name.intern] \verb!beta10 for ggaussian observations!
	 	 	\item[initial] \verb!0!
	 	 	\item[fixed] \verb!FALSE!
	 	 	\item[prior] \verb!normal!
	 	 	\item[param] \verb!0 10!
	 	 	\item[to.theta] \verb!function(x) x!
	 	 	\item[from.theta] \verb!function(x) x!
	 	 \end{description}
	 \end{description}
	\item[status] \verb!experimental!
	\item[survival] \verb!FALSE!
	\item[discrete] \verb!FALSE!
	\item[link] \verb!default identity!
	\item[link.simple] \verb!default log!
	\item[pdf] \verb!ggaussian!
\end{description}


\subsubsection*{Example}
\verbatiminput{example-ggaussian.R}

\clearpage

\subsection*{Family \emph{ggaussianS}}
\subsubsection*{Link-function}

This is the swapped version. The mean is given by a simple linear
predictor
\begin{displaymath}
    \text{link.simple}(\mu) = \text{off} + \beta_1 z_1 + \beta_2 z_2 + \ldots + \beta_{m} z_{m}, \qquad m \leq 10
\end{displaymath}
where ``$\text{off}$'' is a fixed offset and where link.simple is the
\emph{identity}-link by default. \textbf{Note:} there is no default
intercept in the simple model, so it is common to set $z_1=1$. The
precision is specified in the formula
\begin{displaymath}
    \text{link}(\tau) = \text{formula}
\end{displaymath}
using the log-link as default.

\subsubsection*{Hyperparameters}
The hyperparameters in the model are
$\theta_1=\beta_1, \ldots, \theta_m=\beta_m$, and the priors are
defined on $\theta_1, \ldots, \theta_m$.

\subsubsection*{Specification}
\begin{itemize}
\item \texttt{family="ggaussianS"}
\item Required arguments: \textbf{inla.mdata}-object that defines the
    response, the offset ``$\text{off}$'' and covariates. The
    \textbf{inla.mdata} object is defined as
    \begin{displaymath}
        \text{inla.mdata}(y, \text{off}, z_1, \ldots, z_m)
    \end{displaymath}
    where each argument are vectors of the same length.
\item \textbf{Note:} The offset argument \textbf{MUST} be given as it
    has no default value. Often we can just use off$=0$ as it will
    auto-expand.
\item \textbf{Note:} $m=0$ is allowed, which means that
    $\mu=\text{off}$ with with identity link.simple.
\end{itemize}

\subsubsection*{Hyperparameter spesification and default values}
\textbf{\texttt{family="ggaussianS"}}
%% DO NOT EDIT!
%% This file is generated automatically from models.R
\begin{description}
	\item[doc] \verb!Generalized GaussianS!
	\item[hyper]\ 
	 \begin{description}
	 	\item[theta1]\ 
	 	 \begin{description}
	 	 	\item[hyperid] \verb!66601!
	 	 	\item[name] \verb!beta1!
	 	 	\item[short.name] \verb!beta1!
	 	 	\item[output.name] \verb!beta1 for ggaussianS observations!
	 	 	\item[output.name.intern] \verb!beta1 for ggaussianS observations!
	 	 	\item[initial] \verb!0!
	 	 	\item[fixed] \verb!FALSE!
	 	 	\item[prior] \verb!normal!
	 	 	\item[param] \verb!0 0.001!
	 	 	\item[to.theta] \verb!function(x) x!
	 	 	\item[from.theta] \verb!function(x) x!
	 	 \end{description}
	 	\item[theta2]\ 
	 	 \begin{description}
	 	 	\item[hyperid] \verb!66602!
	 	 	\item[name] \verb!beta2!
	 	 	\item[short.name] \verb!beta2!
	 	 	\item[output.name] \verb!beta2 for ggaussianS observations!
	 	 	\item[output.name.intern] \verb!beta2 for ggaussianS observations!
	 	 	\item[initial] \verb!0!
	 	 	\item[fixed] \verb!FALSE!
	 	 	\item[prior] \verb!normal!
	 	 	\item[param] \verb!0 0.001!
	 	 	\item[to.theta] \verb!function(x) x!
	 	 	\item[from.theta] \verb!function(x) x!
	 	 \end{description}
	 	\item[theta3]\ 
	 	 \begin{description}
	 	 	\item[hyperid] \verb!66603!
	 	 	\item[name] \verb!beta3!
	 	 	\item[short.name] \verb!beta3!
	 	 	\item[output.name] \verb!beta3 for ggaussianS observations!
	 	 	\item[output.name.intern] \verb!beta3 for ggaussianS observations!
	 	 	\item[initial] \verb!0!
	 	 	\item[fixed] \verb!FALSE!
	 	 	\item[prior] \verb!normal!
	 	 	\item[param] \verb!0 0.001!
	 	 	\item[to.theta] \verb!function(x) x!
	 	 	\item[from.theta] \verb!function(x) x!
	 	 \end{description}
	 	\item[theta4]\ 
	 	 \begin{description}
	 	 	\item[hyperid] \verb!66604!
	 	 	\item[name] \verb!beta4!
	 	 	\item[short.name] \verb!beta4!
	 	 	\item[output.name] \verb!beta4 for ggaussianS observations!
	 	 	\item[output.name.intern] \verb!beta4 for ggaussianS observations!
	 	 	\item[initial] \verb!0!
	 	 	\item[fixed] \verb!FALSE!
	 	 	\item[prior] \verb!normal!
	 	 	\item[param] \verb!0 0.001!
	 	 	\item[to.theta] \verb!function(x) x!
	 	 	\item[from.theta] \verb!function(x) x!
	 	 \end{description}
	 	\item[theta5]\ 
	 	 \begin{description}
	 	 	\item[hyperid] \verb!66605!
	 	 	\item[name] \verb!beta5!
	 	 	\item[short.name] \verb!beta5!
	 	 	\item[output.name] \verb!beta5 for ggaussianS observations!
	 	 	\item[output.name.intern] \verb!beta5 for ggaussianS observations!
	 	 	\item[initial] \verb!0!
	 	 	\item[fixed] \verb!FALSE!
	 	 	\item[prior] \verb!normal!
	 	 	\item[param] \verb!0 0.001!
	 	 	\item[to.theta] \verb!function(x) x!
	 	 	\item[from.theta] \verb!function(x) x!
	 	 \end{description}
	 	\item[theta6]\ 
	 	 \begin{description}
	 	 	\item[hyperid] \verb!66606!
	 	 	\item[name] \verb!beta6!
	 	 	\item[short.name] \verb!beta6!
	 	 	\item[output.name] \verb!beta6 for ggaussianS observations!
	 	 	\item[output.name.intern] \verb!beta6 for ggaussianS observations!
	 	 	\item[initial] \verb!0!
	 	 	\item[fixed] \verb!FALSE!
	 	 	\item[prior] \verb!normal!
	 	 	\item[param] \verb!0 0.001!
	 	 	\item[to.theta] \verb!function(x) x!
	 	 	\item[from.theta] \verb!function(x) x!
	 	 \end{description}
	 	\item[theta7]\ 
	 	 \begin{description}
	 	 	\item[hyperid] \verb!66607!
	 	 	\item[name] \verb!beta7!
	 	 	\item[short.name] \verb!beta7!
	 	 	\item[output.name] \verb!beta7 for ggaussianS observations!
	 	 	\item[output.name.intern] \verb!beta7 for ggaussianS observations!
	 	 	\item[initial] \verb!0!
	 	 	\item[fixed] \verb!FALSE!
	 	 	\item[prior] \verb!normal!
	 	 	\item[param] \verb!0 0.001!
	 	 	\item[to.theta] \verb!function(x) x!
	 	 	\item[from.theta] \verb!function(x) x!
	 	 \end{description}
	 	\item[theta8]\ 
	 	 \begin{description}
	 	 	\item[hyperid] \verb!66608!
	 	 	\item[name] \verb!beta8!
	 	 	\item[short.name] \verb!beta8!
	 	 	\item[output.name] \verb!beta8 for ggaussianS observations!
	 	 	\item[output.name.intern] \verb!beta8 for ggaussianS observations!
	 	 	\item[initial] \verb!0!
	 	 	\item[fixed] \verb!FALSE!
	 	 	\item[prior] \verb!normal!
	 	 	\item[param] \verb!0 0.001!
	 	 	\item[to.theta] \verb!function(x) x!
	 	 	\item[from.theta] \verb!function(x) x!
	 	 \end{description}
	 	\item[theta9]\ 
	 	 \begin{description}
	 	 	\item[hyperid] \verb!66609!
	 	 	\item[name] \verb!beta9!
	 	 	\item[short.name] \verb!beta9!
	 	 	\item[output.name] \verb!beta9 for ggaussianS observations!
	 	 	\item[output.name.intern] \verb!beta9 for ggaussianS observations!
	 	 	\item[initial] \verb!0!
	 	 	\item[fixed] \verb!FALSE!
	 	 	\item[prior] \verb!normal!
	 	 	\item[param] \verb!0 0.001!
	 	 	\item[to.theta] \verb!function(x) x!
	 	 	\item[from.theta] \verb!function(x) x!
	 	 \end{description}
	 	\item[theta10]\ 
	 	 \begin{description}
	 	 	\item[hyperid] \verb!66610!
	 	 	\item[name] \verb!beta10!
	 	 	\item[short.name] \verb!beta10!
	 	 	\item[output.name] \verb!beta10 for ggaussianS observations!
	 	 	\item[output.name.intern] \verb!beta10 for ggaussianS observations!
	 	 	\item[initial] \verb!0!
	 	 	\item[fixed] \verb!FALSE!
	 	 	\item[prior] \verb!normal!
	 	 	\item[param] \verb!0 0.001!
	 	 	\item[to.theta] \verb!function(x) x!
	 	 	\item[from.theta] \verb!function(x) x!
	 	 \end{description}
	 \end{description}
	\item[status] \verb!experimental!
	\item[survival] \verb!FALSE!
	\item[discrete] \verb!TRUE!
	\item[link] \verb!default log!
	\item[link.simple] \verb!default identity!
	\item[pdf] \verb!ggaussian!
\end{description}


\subsubsection*{Example}
\verbatiminput{example-ggaussianS.R}

\end{document}


% LocalWords:  GGaussian GGaussianS Parameterisation ggaussian inla
% LocalWords:  ggaussianS gaussian Hyperparameters hyperparameters
% LocalWords:  mdata covariates Hyperparameter spesification

%%% Local Variables: 
%%% TeX-master: t
%%% End: 
