\documentclass[a4paper,11pt]{article}
\usepackage[scale={0.8,0.9},centering,includeheadfoot]{geometry}
\usepackage{amstext}
\usepackage{amsmath}
\usepackage{verbatim}

\begin{document}
\section*{Gaussian}

\subsection*{Parametrisation}

The Gaussian distribution is
\begin{displaymath}
    f(y) = \frac{\sqrt{s\tau}}{\sqrt{2\pi}} \exp\left( -\frac{1}{2}
      s\tau \left(y-\mu\right)^{2}\right)
\end{displaymath}
for continuously responses $y$ where
\begin{description}
\item[$\mu$:] is the the mean
\item[$\tau$:] is the precision
\item[$s$:] is a fixed scaling, $s>0$.    
\end{description}

\subsection*{Link-function}

The mean and variance of $y$ are given as
\begin{displaymath}
    \mu \quad\text{and}\qquad \sigma^{2} = \frac{1}{s\tau}
\end{displaymath}
and the mean is linked to the linear predictor by
\begin{displaymath}
    \mu = \eta
\end{displaymath}

\subsection*{Hyperparameters}

The default behaviour is to represent the precision $\tau = \kappa_1$
where
\begin{displaymath}
    \theta_1 = \log \kappa_1
\end{displaymath}
and the prior is defined on $\theta_1$.

The more general formulation have a second (fixed) hyperparameter
$\theta_2$ which determines a fixed offset $1/\kappa_2$,
$\theta_2=\log\kappa_2$, for the variance (scaling not included) of
the response. In this case,
\begin{displaymath}
    1/\tau = 1/\kappa_1 + 1/\kappa_2
\end{displaymath}
or
\begin{displaymath}
    \tau = \frac{1}{1/\kappa_1 + 1/\kappa_2}
\end{displaymath}
In the case where $1/\kappa_2$ is zero, then $\tau = \kappa_1$ and we
are back to the default behaviour. We use the convension that
$1/\kappa_2$ is zero if $1/\kappa_2 <$\verb|.Machine$double.eps|,%%
which is $\theta_2 \ge 36.05$ for common machines.

\subsection*{Specification}

\begin{itemize}
\item $\text{family}=\texttt{gaussian}$
\item Required arguments: $y$ and $s$ (argument \texttt{scale})
\end{itemize}
The scalings have default value 1.

\subsubsection*{Hyperparameter spesification and default values}
%% DO NOT EDIT!
%% This file is generated automatically from models.R
\begin{description}
	\item[hyper]\ 
	 \begin{description}
	 	\item[theta]\ 
	 	 \begin{description}
	 	 	\item[name] log precision
	 	 	\item[short.name] prec
	 	 	\item[initial] 4
	 	 	\item[fixed] FALSE
	 	 	\item[prior] loggamma
	 	 	\item[param] 1 5e-05
	 	 	\item[to.theta] \verb|function(x) log(x)|
	 	 	\item[from.theta] \verb|function(x) exp(x)|
	 	 \end{description}
	 \end{description}
	\item[survival] FALSE
	\item[discrete] FALSE
	\item[link] default identity logit log
	\item[pdf] gaussian
\end{description}


\subsection*{Example}

The first example estimate the parameters in a simulated example with
Gaussian responses, giving $\tau$ a Gamma-prior with parameters
$(1, 0.01)$ and initial value (for the optimisations) of $\exp(2.0)$.
The second example shows the use of an fixed offset in the variance.
%%
\verbatiminput{example-gaussian.R}

\subsection*{Notes}

An error is given if $\theta_2$ is not fixed. 


\end{document}


% LocalWords:  np Hyperparameters Ntrials gaussian

%%% Local Variables: 
%%% TeX-master: t
%%% End: 
