
\documentclass[a4paper,11pt]{article}
\usepackage[compat2]{geometry}
\usepackage{amstext}
\usepackage{amsmath,amssymb}


\def\mm#1{\ensuremath{\boldsymbol{#1}}} % version: amsmath


\begin{document}
\bibliographystyle{apalike}

\section{Introduction}

The aim of this paper is to describe the class of models which can be implemented using the {\tt R inla} package.

The {\tt R inla} packages solves models using 
Integrated nested Laplace approximation (INLA) which is a new approach to
statistical inference for latent Gaussian Markov random field (GMRF)
models described in  \cite{art451}. 

In short, a latent GMRF model is a hierarchical model where, at the
first stage we find a distributional assumption for the observables
$\mm{y}$ usually assumed to be conditionally independent given some
latent parameters $\mm{\eta}$ and, possibly, some additional
parameters $\mm{\theta}_1$
\[
\pi(\mm{y}|\mm{\eta},\mm{\theta}_1)=\prod_{j}\pi(y_j|\eta_j,\mm{\theta}_1).
\]
The third, and last, stage of the model consists of the prior
distribution for the hyperparameters
$\mm{\theta}=(\mm{\theta}_1,\mm{\theta}_2)$.
The INLA approach provides a recipe for fast Bayesian inference using
accurate approximations to $\pi(\mm{\theta}|\mm{y})$ and
$\pi(x_i|\mm{y})$, $i=0,\dots,n-1$, i.e. the marginal posterior
density for the hyperparameters and the posterior marginal densities
for the latent variables.  Different types of approximations are
available, see \cite{art451} for details. The approximate posterior
marginals can then be used to compute summary statistics of interest,
such as posterior means, variances or quantiles.

Using the INLA approach it is also possible to challenge the model
itself. The model can be assessed through cross-validation in a
reasonable time. Moreover, Bayes factors and deviance information
criterion (DIC) can be computed in an efficient way providing tools
for model comparison.

\section{Model description}


The {\tt R inla} library supports hierarchical GMRF models of the
following type
\begin{eqnarray}\label{eq1}
    y_j|\eta_j,\mm{\theta}_1\sim\pi(y_j|\eta_j,\mm{\theta}_1)\qquad
    j\in J \\\label{eq2}
    \eta_i=\mbox{Offset}_i+
    \sum_{k=0}^{n_f-1} w_{ki}\ f_k( c_{ki}) + \mm{z}_i^T\mm{\beta}+\epsilon_i\qquad i=0,\dots,n_{\eta}-1
\end{eqnarray}
where
\begin{itemize}
\item $J$ is a subset of $\{0,1,\dots,n_{\eta}-1\}$, that is, not
    necessarily all latent parameters $\mm{\eta}$ are observed through
    the data $\mm{y}$.
\item $\pi(y_j|\eta_j,\mm{\theta}_1)$ is the likelihood of the
    observed data assumed to be conditional independent given the
    latent parameters $\mm{\eta}$, and, possibly, some additional
    parameters $\mm{\theta}_1$. The latent variable $\eta_i$ enters
    the likelihood through a known link function.
\item $\mm{\epsilon}$ is a vector of unstructured random effects of
    length $n_{\eta}$ with i.i.d Gaussian priors with precision
    $\lambda_{\eta}$:
    \begin{equation}
        \mm{\epsilon}|\lambda_{\eta}\sim\mathcal{N}(\mm{0},\lambda_{\eta}\mm{I})
    \end{equation}
\item $\mm{\eta}=(\eta_1,\eta_2,\dots)$ is a vector of predictors.
\item{$\mbox{Offset}$} is an a priori known component to be
    included in the linear predictor during fitting.
\item$\mm{w}_k$ known weights defined for each observed data point.
\item $f_k(c_{ki})$ is the effect of a generic covariate $k$ which
    assumes value $c_{ki}$ for observation $i$.  The functions $f_k$,
    $k=0,\dots,n_f-1$ comprise usual nonlinear effect of continuous
    covariates, time trends and seasonal effects, two dimensional
    surfaces, iid random intercepts and slopes and spatial random
    effects. The unknown functions, or more exactly the corresponding
    vector of function evaluations
    $\mm{f}_k=(f_{0k},\dots,f_{(m_k-1)k})^T$, are modelled as GMRFs
    given some parameters $\mm{\theta}_{f_k}$, that is
    \begin{equation}
        \mm{f}_k|\mm{\theta}_{f_k}\sim\mathcal{N}(\mm{0},\mm{Q}^{-1}_k)
    \end{equation}
\item $\mm{z}_i$ is a vector of $n_{\beta}$ covariates assumed to have
    a linear effect, and is $\mm{\beta}$ the corresponding vector of
    unknown parameters with independent zero-mean Gaussian prior with
    fixed precisions.
\end{itemize}
The full latent field, of dimension
$n=n_{\eta}+\sum_{j=0}^{n_f-1}m_j+n_{\beta}$, is then
\[
\mm{x}=(\mm{\eta}^T,\mm{f}^T_0,\dots,\mm{f}_{n_f-1}^T,\mm{\beta}^T).
\]

Note that  the latent field $\mm{x}$ is
parametrised using the predictors $\mm{\eta}$ instead of the
unstructured terms $\mm{\epsilon}$.

All elements of vector $\mm{x}$ are defined as GMRFs, hence $\mm{x}$
is itself a GMRF with density:
\begin{equation}\label{eq3}
    \pi(\mm{x}|\mm{\theta}_2)=
    \prod_{i=0}^{n_{\eta}-1}\pi(\eta_i|\mm{f}_0,\dots,\mm{f}_{n_f-1},\mm{\beta},\lambda_{\eta})
    \prod_{k=0}^{n_f-1}\pi(\mm{f}_k|\mm{\kappa}_{f_k})\prod_{m=0}^{n_{\beta}-1}\pi(\beta_m)
\end{equation}
where
\begin{equation}\label{predictor}
    \eta_i|\mm{f}_0,\dots,\mm{f}_{n_f-1},\mm{\beta}\sim\mathcal{N}(\sum_{k=0}^{n_f-1} f_k( c_{ki}) + \mm{z}_i^T\mm{\beta},\lambda_{\eta})
\end{equation}
and
$\mm{\theta}_2=\{\log\lambda_{\eta},\mm{\theta}_{f_0},\dots,\mm{\theta}_{n_f-1}\}$
is a vector of unknown hyperparameters. 

The last element in the definition of our hierarchical model is a
prior distribution for the hyperparameters
$\mm{\theta}=(\mm{\theta}_1,\mm{\theta}_2)$. 

\section{Examples}

Many well known models from the literature can be written as special
cases of (\ref{eq1}) and (\ref{eq2})
\begin{itemize}
\item {\it Time series models}

    Time series models are obtained if $c_k=t$ represents time. The
    functions $f_k$ can model nonlinear trends or seasonal effects
    \[\eta_t=f_{trend}(t)+f_{seasonal}(t)+\mm{z}_t^T\mm{\beta}\]
  
\item {\it Generalised additive models (GAM)}
    
    A GAM model is obtained if $\pi(y_i|\eta_i,\mm{\theta}_l)$ belongs
    to an exponential family, $c_k$ are univariate, continuous
    covariates and $f_k$ are smooth functions.
    
\item {\it Generalised additive mixed models (GAMM) for longitudinal
        data}

    Consider longitudinal data for individuals $i=0,\dots,n_i-1$,
    observed at time points $t_0,t_1,\dots$. A GAMM model extends a
    GAM by introducing individual specific random effects, i.e.
    \[\eta_{it}=f_0(c_{it0})+\dots,+f_{n_f-1}(c_{it(n_f-1))})+b_{0i}w_{it0}+\dots+b_{(n_b-1)i}w_{it(n_b-1)}\]
    where $\eta_{it}$ is the predictor for individual $i$ at time $t$,
    $x_{itk}$, $k=0,\dots,n_f-1$,$w_{itq}$, $q=0,\dots,n_b-1$ are
    covariate values for individual $i$ at time $t$, and
    $b_{0i},\dots,b_{(n_b-1)i}$ is a vector of $n_b$ individual
    specific random intercepts (if $w_{itq}=1$) or slopes. The above
    model can be written in the general form in equation (\ref{eq2}) by
    defining $r=(i,t)$, $c_{rj}=c_{itj}$ for $j=0,\dots,n_f-1$ and
    $c_{r,(n_f-1)+q}=w_{itq}$,
    $f_{(n_f-1)+q}(c_{r,(n_f-1)+q})=b_{qi}w_{itq}$ for
    $q=0,\dots,n_b$. In the same way GAMM's for cluster data can be
    written in the general form (\ref{eq2}).
    
\item {\it Geoadditive models}

    If geographical information for the observations in the data set
    are available, they might be included in the model as
    \[\eta_i=f_1(c_{0i})+\dots+f_{n_f-1}(c_{(n_f-1)i})+f_{spat}(s_i)+\mm{z}_i^T\mm{\beta}\]
    where $s_i$ indicates the location of observation $i$ and
    $f_{spat}$ is a spatially correlated effect. Models where one of
    the covariate represent the spatial effect have recently been
    coined geoadditive by \cite{art270}.
    
\item {\it ANOVA type interaction model}

    The effect of two continuous covariate $w$ and $v$ can be modelled
    as
    \[
    \eta_i=f_1(w_i)+f_2(v_i)+f_{1|2}(w_i,v_i)+\dots
    \]
    where $f_1$ and $f_2$ are the main effects of the two covariates
    and $f_{1|2}$ is a two dimensional interaction surface. The above
    model can be written in the general form (\ref{eq2}) simply by
    defining $c_{1i}=w_i$, $c_{2i}=v_i$, $c_{3i}=(w_i,v_i)$,

\item {\it Univariate stochastic volatility model}

    Stochastic volatility models are time series models with Gaussian
    likelihood where it is the variance, and not the mean of the
    observed data, to be part of the latent GMRF model. That is
    \[
    y_i|\eta_i\sim\mathcal{N}(0,\exp(\eta_i))
    \]
    The latent field is then typically modelled as a autoregressive
    model of order 1.
\end{itemize}

\small\bibliography{mybib} \newpage


\end{document}