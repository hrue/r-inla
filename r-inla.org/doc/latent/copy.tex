\documentclass[a4paper,11pt]{article}
\usepackage[scale={0.8,0.9},centering,includeheadfoot]{geometry}
\usepackage{amstext}
\usepackage{listings}
\usepackage{verbatim}
\def\opening\null
\usepackage{block}
\begin{document}

\section*{Copy of another model component: ``copy''}

This describes the way to copy another model component with an
optional scaling. This issue arise frequently, like with a model like
\begin{displaymath}
    \eta_{ij} = u_i + u_j
\end{displaymath}
where $u$ is the \textbf{same model component}. This is not doable
with normal use of 'formula', hence we introduce an identical (up to
numerical error)
\begin{displaymath}
    v = \texttt{copy}(u)
\end{displaymath}
so we can write
\begin{displaymath}
    \eta = u + v
\end{displaymath}
We can also enable an optional scaling, so that
\begin{displaymath}
    v = \beta\times\text{copy}(u)
\end{displaymath}
where $\beta$ is estimated as well.

Note the following.
\begin{itemize}
\item One model component can be copied many times.
\item $v$ will inherit the main properties from $u$, such as
    \texttt{replicate}, \texttt{nrep}, \texttt{group}, \texttt{ngroup}
    and \texttt{values}. So it is natural to use same kind of indexing
    for $v$ as $u$. For example
    \begin{quote}
        \texttt{y $\sim$ f(i, replicate=r, group=g) +
            f(j, copy="i", replicate=rr, group=gg)}
    \end{quote}
    Here, the argument \texttt{copy}, say that this model component
    is a copy of \texttt{"i"}.
\item If the same indexing is \textbf{not} used, then the equivalent
    one-dimensional indexing is used as defined in
    \texttt{inla.idx()}.
\end{itemize}

\subsection*{Hyperparameters}

The optional hyperparameter is $\beta$ which is \textbf{default fixed
    to $\beta=1$}.

We can estimate $\beta$ by setting \texttt{fixed=FALSE}, like
\begin{quote}
    \texttt{f(j, copy="i", hyper=list(beta=list(fixed=FALSE)))}
\end{quote}

We can control $\beta$, in two ways.
\begin{enumerate}
\item We can fix $\beta$ to be in an interval defined by argument
    \texttt{range=c(r1,r2)}, where $r_1 \leq r_2$

    \begin{itemize}
    \item If $r_1=r_2$, then there is no restrictions on $\beta$.
    \item If $-\infty < r_1 < r_2 < \infty$, then
        $\beta$ is defined to be in the interval $(r_1,
        r_2)$, and the prior, initial values are defined on
        $\tilde{\beta}$, where
        \begin{displaymath}
            \beta = r_1 + (r_2-r_1) \frac{1}{1+\exp(-\tilde{\beta})}
        \end{displaymath}
    \item If $-\infty < r_1 < r_2=\infty$, then
        $\beta$ is defined to be in the interval
        $(r_1,\infty)$ and the prior, initial values are defined on
        $\tilde{\beta}$, where
        \begin{displaymath}
            \beta = r_1 + \exp(\tilde{\beta})
        \end{displaymath}
    \end{itemize}
\item We can make $\beta$ the same as a $\beta$ in another copy, with
    argument \texttt{same.as}, so that
    \begin{displaymath}
        \eta_{ijk} = u_i + \beta u_j + \beta u_k
    \end{displaymath}
    with
    \begin{quote}
        \texttt{y $\sim$ f(i) +\\
                f(j, copy="i", hyper=list(beta=list(fixed=FALSE)))
                +\\
                f(k, copy="i", same.as="j")}
    \end{quote}
\item How cloes a the copy is, is determined by the argument
    \texttt{precision=...}, and the default value is
    \begin{quote}
        \texttt{f(j, copy="i", precision=exp(14))}
    \end{quote}
    meaning that $v = u + \epsilon$, where $\epsilon$ is iid zero mean
    Normal with precision $\exp(14)$.
\end{enumerate}

\section*{Spesification}

%% DO NOT EDIT!
%% This file is generated automatically from models.R
\begin{description}
	\item[doc] Create a copy of a model component
	\item[hyper]\ 
	 \begin{description}
	 	\item[theta]\ 
	 	 \begin{description}
	 	 	\item[hyperid] 36001
	 	 	\item[name] beta
	 	 	\item[short.name] b
	 	 	\item[initial] 0
	 	 	\item[fixed] TRUE
	 	 	\item[prior] normal
	 	 	\item[param] 1 10
	 	 	\item[to.theta] \verb!function(x, REPLACE.ME.low, REPLACE.ME.high) {! \verb!                                if (all(is.infinite(c(low, high))) || low == high) {! \verb!                                    return(x)! \verb!                                } else if (all(is.finite(c(low, high)))) {! \verb!                                    stopifnot(low < high)! \verb!                                    return(log(-(low - x) / (high - x)))! \verb!                                } else if (is.finite(low) && is.infinite(high) && high > low) {! \verb!                                    return(log(x - low))! \verb!                                } else {! \verb!                                    stop("Condition not yet implemented")! \verb!                                }! \verb!                            }!
	 	 	\item[from.theta] \verb!function(x, REPLACE.ME.low, REPLACE.ME.high) {! \verb!                                if (all(is.infinite(c(low, high))) || low == high) {! \verb!                                    return(x)! \verb!                                } else if (all(is.finite(c(low, high)))) {! \verb!                                    stopifnot(low < high)! \verb!                                    return(low + exp(x) / (1 + exp(x)) * (high - low))! \verb!                                } else if (is.finite(low) && is.infinite(high) && high > low) {! \verb!                                    return(low + exp(x))! \verb!                                } else {! \verb!                                    stop("Condition not yet implemented")! \verb!                                }! \verb!                            }!
	 	 \end{description}
	 \end{description}
	\item[constr] FALSE
	\item[nrow.ncol] FALSE
	\item[augmented] FALSE
	\item[aug.factor] 1
	\item[aug.constr] 
	\item[n.div.by] 
	\item[n.required] FALSE
	\item[set.default.values] FALSE
	\item[pdf] NA
\end{description}


\subsection*{Example}

Just simulate some data and estimate the parameters back. 

{\small\verbatiminput{example-copy.R}}

\subsection*{Notes}

\end{document}



% LocalWords: 

%%% Local Variables: 
%%% TeX-master: t
%%% End: 
