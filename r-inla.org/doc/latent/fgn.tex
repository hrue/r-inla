\documentclass[a4paper,11pt]{article}
\usepackage[scale={0.8,0.9},centering,includeheadfoot]{geometry}
\usepackage{amstext}
\usepackage{listings}
\begin{document}

\section*{Fractional Gaussian Noise (FGN)}

\subsection*{Parametrization}

The (stationary) FGN (Gaussian) process has correlation function  at
lag $k$
\begin{displaymath}
    \rho(k) = |k+1|^{2H} -2|k|^{2H} + |k-1|^{2H}
\end{displaymath}
where $H$ is the Hurst parameter or self-similarity parameter, which
we assume to be
\begin{displaymath}
    1/2 \le H < 1.
\end{displaymath}
so the process has long range properties for $H>1/2$. The locations of
the the process is fixed to $1, 2, \ldots, n$, where $n$ is the
dimension of the finite representation of the FGN process.

\subsection*{Hyperparameters}

The marginal precision, $\tau$, of the process is represented as
\begin{displaymath}
    \tau = \exp(\theta_1)
\end{displaymath}
The Hurst parameter $H$ is represented as
\[
    H = \frac{1}{2} + \frac{1}{2} \frac{\exp(\theta_2)}{1+\exp(\theta_2)}
\]
and the prior is defined on $\mathbf{\theta}=(\theta_1,\theta_2)$. 

\subsection*{Specification}

The FGN model is specified as
\begin{verbatim}
 f(<whatever>, model="fgn", order=<order>, hyper = <hyper>)
\end{verbatim}
The parmeter \texttt{order} gives the order of the Markov
approximation. Currently, only \texttt{order=3} is implemented.

\subsubsection*{Hyperparameter spesification and default values for \texttt{model="fgn"}}
%% DO NOT EDIT!
%% This file is generated automatically from models.R
\begin{description}
	\item[doc] Fractional Gaussian noise model
	\item[hyper]\ 
	 \begin{description}
	 	\item[theta1]\ 
	 	 \begin{description}
	 	 	\item[hyperid] 13101
	 	 	\item[name] log precision
	 	 	\item[short.name] prec
	 	 	\item[prior] pc.prec
	 	 	\item[param] 3 0.01
	 	 	\item[initial] 1
	 	 	\item[fixed] FALSE
	 	 	\item[to.theta] \verb!function(x) log(x)!
	 	 	\item[from.theta] \verb!function(x) exp(x)!
	 	 \end{description}
	 	\item[theta2]\ 
	 	 \begin{description}
	 	 	\item[hyperid] 13102
	 	 	\item[name] logit H
	 	 	\item[short.name] H
	 	 	\item[prior] pcfgnh
	 	 	\item[param] 0.9 0.1
	 	 	\item[initial] 2
	 	 	\item[fixed] FALSE
	 	 	\item[to.theta] \verb!function(x) log((2*x-1)/(2*(1-x)))!
	 	 	\item[from.theta] \verb!function(x) 0.5 + 0.5*exp(x)/(1+exp(x))!
	 	 \end{description}
	 \end{description}
	\item[constr] FALSE
	\item[nrow.ncol] FALSE
	\item[augmented] TRUE
	\item[aug.factor] 4
	\item[aug.constr] 1
	\item[n.div.by] 
	\item[n.required] FALSE
	\item[set.default.values] TRUE
	\item[order.default] 4
	\item[order.defined] 3 4
	\item[pdf] fgn
\end{description}


\subsubsection*{Hyperparameter spesification and default values for \texttt{model="fgn2"}}
%% DO NOT EDIT!
%% This file is generated automatically from models.R
\begin{description}
	\item[doc] Fractional Gaussian noise model (alt 2)
	\item[hyper]\ 
	 \begin{description}
	 	\item[theta1]\ 
	 	 \begin{description}
	 	 	\item[hyperid] 13101
	 	 	\item[name] log precision
	 	 	\item[short.name] prec
	 	 	\item[prior] pc.prec
	 	 	\item[param] 3 0.01
	 	 	\item[initial] 1
	 	 	\item[fixed] FALSE
	 	 	\item[to.theta] \verb!function(x) log(x)!
	 	 	\item[from.theta] \verb!function(x) exp(x)!
	 	 \end{description}
	 	\item[theta2]\ 
	 	 \begin{description}
	 	 	\item[hyperid] 13102
	 	 	\item[name] logit H
	 	 	\item[short.name] H
	 	 	\item[prior] pcfgnh
	 	 	\item[param] 0.9 0.1
	 	 	\item[initial] 2
	 	 	\item[fixed] FALSE
	 	 	\item[to.theta] \verb!function(x) log((2*x-1)/(2*(1-x)))!
	 	 	\item[from.theta] \verb!function(x) 0.5 + 0.5*exp(x)/(1+exp(x))!
	 	 \end{description}
	 \end{description}
	\item[constr] FALSE
	\item[nrow.ncol] FALSE
	\item[augmented] TRUE
	\item[aug.factor] 4
	\item[aug.constr] 1
	\item[n.div.by] 
	\item[n.required] FALSE
	\item[set.default.values] TRUE
	\item[order.default] 4
	\item[order.defined] 3 4
	\item[pdf] fgn
\end{description}


\subsection*{Example}

\begin{verbatim}
library(FGN)
n = 1000
H = 0.77
y = SimulateFGN(n, H)
y = y - mean(y)
r = inla(y ~ -1 + f(idx, model="fgn"), 
         data = data.frame(y, idx=1:n),
         control.family =list(hyper = list(prec = list(initial = 12, fixed=TRUE))))
print(c(MLE=FitFGN(y, demean=TRUE)$H,
        Post.mean=r$summary.hyperpar[2,"mean"],
        Post.mode=r$summary.hyperpar[2,"mode"]))
\end{verbatim}


\subsection*{Notes}

In the example above, then the \texttt{f(idx,model="fgn")} object will
expand into a Gaussian of length \texttt{(order + 1)*n}. The first $n$
elements is the FGN model (which is of interrest), then there are
\texttt{order} vector of AR1 processes each of length $n$, and the sum
of these AR1 processes is used to represent the FGN.

Another alternative, is \texttt{f(idx,model="fgn2")} object will
expand into a Gaussian of length \texttt{order*n}, which are the
cummulative sums of the the \texttt{order} vector of AR1 processes
each of length $n$. If \texttt{order==3}, with weighted AR1 processes
(and with the given precision), $x$, $xx$ and $xxx$, then
\texttt{model="fgn2"} return the vector $(x+xx+xxx, xx+xxx, xxx)$
where $\phi_{x} < \phi_{xx} < \phi_{xxx}$.



The PC-prior for $H$ take two arguments $(U, \alpha)$ where
$\text{Prob}(U < H < 1) = \alpha$.

\end{document}


% LocalWords: 

%%% Local Variables: 
%%% TeX-master: t
%%% End: 

