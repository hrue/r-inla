\documentclass[a4paper,11pt]{article}
\usepackage[scale={0.8,0.9},centering,includeheadfoot]{geometry}
\usepackage{amstext}
\usepackage{listings}
\begin{document}

\section*{log1exp effect of a covariate}

\subsection*{Parametrization}

This model implements a non-linear effect of a positive covariate $x$
as a part of the linear predictor,
\begin{displaymath}
    \beta \log\left(1 + \exp(\alpha -\gamma x)\right)
\end{displaymath}
where $\beta, \alpha, \gamma \in \Re$ and $x\ge 0$.

\subsection*{Hyperparameters}

This model has three hyperparameters, the scaling $\beta$, halflife
$a$ and shape $k$,
\begin{displaymath}
    \theta_1 = \beta \qquad \theta_2 = \alpha \qquad  \theta_3 = \gamma
\end{displaymath}
and the priors are given for $\theta_1, \theta_2$ and $\theta_3$.


\subsection*{Specification}

\begin{verbatim}
    f(x, model="log1exp", hyper = ..., precision = <precision>)
\end{verbatim}
where \texttt{precision} is the precision for the tiny noise used to
implement this as a latent model. 

\subsubsection*{Hyperparameter specification and default values}
\documentclass[a4paper,11pt]{article}
\usepackage[scale={0.8,0.9},centering,includeheadfoot]{geometry}
\usepackage{amstext}
\usepackage{listings}
\begin{document}

\section*{log1exp effect of a covariate}

\subsection*{Parametrization}

This model implements a non-linear effect of a positive covariate $x$
as a part of the linear predictor,
\begin{displaymath}
    \beta \log\left(1 + \exp(\alpha -\gamma x)\right)
\end{displaymath}
where $\beta, \alpha, \gamma \in \Re$ and $x\ge 0$.

\subsection*{Hyperparameters}

This model has three hyperparameters, the scaling $\beta$, halflife
$a$ and shape $k$,
\begin{displaymath}
    \theta_1 = \beta \qquad \theta_2 = \alpha \qquad  \theta_3 = \gamma
\end{displaymath}
and the priors are given for $\theta_1, \theta_2$ and $\theta_3$.


\subsection*{Specification}

\begin{verbatim}
    f(x, model="log1exp", hyper = ..., precision = <precision>)
\end{verbatim}
where \texttt{precision} is the precision for the tiny noise used to
implement this as a latent model. 

\subsubsection*{Hyperparameter specification and default values}
\documentclass[a4paper,11pt]{article}
\usepackage[scale={0.8,0.9},centering,includeheadfoot]{geometry}
\usepackage{amstext}
\usepackage{listings}
\begin{document}

\section*{log1exp effect of a covariate}

\subsection*{Parametrization}

This model implements a non-linear effect of a positive covariate $x$
as a part of the linear predictor,
\begin{displaymath}
    \beta \log\left(1 + \exp(\alpha -\gamma x)\right)
\end{displaymath}
where $\beta, \alpha, \gamma \in \Re$ and $x\ge 0$.

\subsection*{Hyperparameters}

This model has three hyperparameters, the scaling $\beta$, halflife
$a$ and shape $k$,
\begin{displaymath}
    \theta_1 = \beta \qquad \theta_2 = \alpha \qquad  \theta_3 = \gamma
\end{displaymath}
and the priors are given for $\theta_1, \theta_2$ and $\theta_3$.


\subsection*{Specification}

\begin{verbatim}
    f(x, model="log1exp", hyper = ..., precision = <precision>)
\end{verbatim}
where \texttt{precision} is the precision for the tiny noise used to
implement this as a latent model. 

\subsubsection*{Hyperparameter specification and default values}
\documentclass[a4paper,11pt]{article}
\usepackage[scale={0.8,0.9},centering,includeheadfoot]{geometry}
\usepackage{amstext}
\usepackage{listings}
\begin{document}

\section*{log1exp effect of a covariate}

\subsection*{Parametrization}

This model implements a non-linear effect of a positive covariate $x$
as a part of the linear predictor,
\begin{displaymath}
    \beta \log\left(1 + \exp(\alpha -\gamma x)\right)
\end{displaymath}
where $\beta, \alpha, \gamma \in \Re$ and $x\ge 0$.

\subsection*{Hyperparameters}

This model has three hyperparameters, the scaling $\beta$, halflife
$a$ and shape $k$,
\begin{displaymath}
    \theta_1 = \beta \qquad \theta_2 = \alpha \qquad  \theta_3 = \gamma
\end{displaymath}
and the priors are given for $\theta_1, \theta_2$ and $\theta_3$.


\subsection*{Specification}

\begin{verbatim}
    f(x, model="log1exp", hyper = ..., precision = <precision>)
\end{verbatim}
where \texttt{precision} is the precision for the tiny noise used to
implement this as a latent model. 

\subsubsection*{Hyperparameter specification and default values}
\input{../hyper/latent/log1exp.tex}

\subsection*{Example}

\begin{verbatim}
log1exp = function(x, beta,  alpha, gamma)
{
    return (beta * log(1.0 + exp(alpha - gamma * x)))
}

n = 100
lambda = 2
s=0.1
x = rpois(n, lambda = lambda)
beta = 1
alpha = 0
gamma = .5

y = log1exp(x, beta,  alpha, gamma) + rnorm(n, sd = s)
r = inla(y ~ -1 + f(x, model="log1exp"),
        data = data.frame(y, x),
        family = "gaussian",
        control.inla = list(h=0.001), 
        control.family = list(
                hyper = list(
                        prec = list(
                                initial = log(1/s^2),
                                fixed = TRUE))))
summary(r)
\end{verbatim}

\subsection*{Notes}
None

\end{document}

% LocalWords: 

%%% Local Variables: 
%%% TeX-master: t
%%% End: 


\subsection*{Example}

\begin{verbatim}
log1exp = function(x, beta,  alpha, gamma)
{
    return (beta * log(1.0 + exp(alpha - gamma * x)))
}

n = 100
lambda = 2
s=0.1
x = rpois(n, lambda = lambda)
beta = 1
alpha = 0
gamma = .5

y = log1exp(x, beta,  alpha, gamma) + rnorm(n, sd = s)
r = inla(y ~ -1 + f(x, model="log1exp"),
        data = data.frame(y, x),
        family = "gaussian",
        control.inla = list(h=0.001), 
        control.family = list(
                hyper = list(
                        prec = list(
                                initial = log(1/s^2),
                                fixed = TRUE))))
summary(r)
\end{verbatim}

\subsection*{Notes}
None

\end{document}

% LocalWords: 

%%% Local Variables: 
%%% TeX-master: t
%%% End: 


\subsection*{Example}

\begin{verbatim}
log1exp = function(x, beta,  alpha, gamma)
{
    return (beta * log(1.0 + exp(alpha - gamma * x)))
}

n = 100
lambda = 2
s=0.1
x = rpois(n, lambda = lambda)
beta = 1
alpha = 0
gamma = .5

y = log1exp(x, beta,  alpha, gamma) + rnorm(n, sd = s)
r = inla(y ~ -1 + f(x, model="log1exp"),
        data = data.frame(y, x),
        family = "gaussian",
        control.inla = list(h=0.001), 
        control.family = list(
                hyper = list(
                        prec = list(
                                initial = log(1/s^2),
                                fixed = TRUE))))
summary(r)
\end{verbatim}

\subsection*{Notes}
None

\end{document}

% LocalWords: 

%%% Local Variables: 
%%% TeX-master: t
%%% End: 


\subsection*{Example}

\begin{verbatim}
log1exp = function(x, beta,  alpha, gamma)
{
    return (beta * log(1.0 + exp(alpha - gamma * x)))
}

n = 100
lambda = 2
s=0.1
x = rpois(n, lambda = lambda)
beta = 1
alpha = 0
gamma = .5

y = log1exp(x, beta,  alpha, gamma) + rnorm(n, sd = s)
r = inla(y ~ -1 + f(x, model="log1exp"),
        data = data.frame(y, x),
        family = "gaussian",
        control.inla = list(h=0.001), 
        control.family = list(
                hyper = list(
                        prec = list(
                                initial = log(1/s^2),
                                fixed = TRUE))))
summary(r)
\end{verbatim}

\subsection*{Notes}
None

\end{document}

% LocalWords: 

%%% Local Variables: 
%%% TeX-master: t
%%% End: 
