\documentclass[a4paper,11pt]{article}
\usepackage[scale={0.8,0.9},centering,includeheadfoot]{geometry}
\usepackage{amstext}
\usepackage{listings}
\usepackage{verbatim}
\def\opening\null
\usepackage{block}
\begin{document}

\section*{Smooth-Copy of another model component: ``scopy''}

This model is a generalization of \texttt{copy}, please refer to
\texttt{inla.doc("copy")} first.

This describes the way to copy another model component with an
optional smooth/spline scaling, like with
\begin{displaymath}
    \eta = u + v
\end{displaymath}
where $v$ is a smooth copy of $u$
\begin{displaymath}
    v = \beta(z)\times\text{copy}(u)
\end{displaymath}
where $\beta(z)$, a smooth/spline function of the covariate $z$.

\subsection*{Hyperparameters}

The optional hyperparameter is the spline at $n$ fixed locations,
$(l_i, \beta_i)$, for $i=1, \ldots, n$. The function $\beta(z)$ is
defined as follows, using $z$ as the covariate
\begin{verbatim}
zr <- range(z)
l <- seq(zr[1], zr[2], len=n)
beta.z <- splinefun(l, beta, method = "natural")
\end{verbatim}

We can control $\beta$ and its prior distribution using argument
\texttt{control.scopy} within \texttt{f()},
\begin{quote}
    \texttt{control.scopy = list(\\
        covariate = ..., \\
        n = 5, \\
        model = "rw2",\\
        mean = 1.0, \\
        prec.mean =  1.0,  \\
        prec.betas =  10.0})
\end{quote}
where
\begin{description}
\item[covariate] gives the covariate that is used
\item[n] is the number of hyperparameters used in the spline ($3 \leq
    n \leq 15$).
\item[model] the prior model for $\{\beta_i\}$, either \texttt{rw1} or
    \texttt{rw2}. This model is scaled (like with
    \texttt{scale.model=TRUE}).
\item[mean] The prior mean for the mean of $\{\beta_i\}$
\item[prec.mean] The prior precision for the mean of $\{\beta_i\}$
\item[prec.betas] The prior precision for the \texttt{rw1/rw2} model
    for $\{\beta_i\}$.
\end{description}
Note that both precisions are \emph{fixed} and not \emph{random}.

The \texttt{f()}-argument \texttt{precision}, defines how close the
copy is, is similar as for model \texttt{copy}.

\section*{Spesification}

\documentclass[a4paper,11pt]{article}
\usepackage[scale={0.8,0.9},centering,includeheadfoot]{geometry}
\usepackage{amstext}
\usepackage{listings}
\usepackage{verbatim}
\def\opening\null
\usepackage{block}
\begin{document}

\section*{Smooth-Copy of another model component: ``scopy''}

This model is a generalization of \texttt{copy}, please refer to
\texttt{inla.doc("copy")} first.

This describes the way to copy another model component with an
optional smooth/spline scaling, like with
\begin{displaymath}
    \eta = u + v
\end{displaymath}
where $v$ is a smooth copy of $u$
\begin{displaymath}
    v = \beta(z)\times\text{copy}(u)
\end{displaymath}
where $\beta(z)$, a smooth/spline function of the covariate $z$.

\subsection*{Hyperparameters}

The optional hyperparameter is the spline at $n$ fixed locations,
$(l_i, \beta_i)$, for $i=1, \ldots, n$. The function $\beta(z)$ is
defined as follows, using $z$ as the covariate
\begin{verbatim}
zr <- range(z)
l <- seq(zr[1], zr[2], len=n)
beta.z <- splinefun(l, beta, method = "natural")
\end{verbatim}

We can control $\beta$ and its prior distribution using argument
\texttt{control.scopy} within \texttt{f()},
\begin{quote}
    \texttt{control.scopy = list(\\
        covariate = ..., \\
        n = 5, \\
        model = "rw2",\\
        mean = 1.0, \\
        prec.mean =  1.0,  \\
        prec.betas =  10.0})
\end{quote}
where
\begin{description}
\item[covariate] gives the covariate that is used
\item[n] is the number of hyperparameters used in the spline ($3 \leq
    n \leq 15$).
\item[model] the prior model for $\{\beta_i\}$, either \texttt{rw1} or
    \texttt{rw2}. This model is scaled (like with
    \texttt{scale.model=TRUE}).
\item[mean] The prior mean for the mean of $\{\beta_i\}$
\item[prec.mean] The prior precision for the mean of $\{\beta_i\}$
\item[prec.betas] The prior precision for the \texttt{rw1/rw2} model
    for $\{\beta_i\}$.
\end{description}
Note that both precisions are \emph{fixed} and not \emph{random}.

The \texttt{f()}-argument \texttt{precision}, defines how close the
copy is, is similar as for model \texttt{copy}.

\section*{Spesification}

\documentclass[a4paper,11pt]{article}
\usepackage[scale={0.8,0.9},centering,includeheadfoot]{geometry}
\usepackage{amstext}
\usepackage{listings}
\usepackage{verbatim}
\def\opening\null
\usepackage{block}
\begin{document}

\section*{Smooth-Copy of another model component: ``scopy''}

This model is a generalization of \texttt{copy}, please refer to
\texttt{inla.doc("copy")} first.

This describes the way to copy another model component with an
optional smooth/spline scaling, like with
\begin{displaymath}
    \eta = u + v
\end{displaymath}
where $v$ is a smooth copy of $u$
\begin{displaymath}
    v = \beta(z)\times\text{copy}(u)
\end{displaymath}
where $\beta(z)$, a smooth/spline function of the covariate $z$.

\subsection*{Hyperparameters}

The optional hyperparameter is the spline at $n$ fixed locations,
$(l_i, \beta_i)$, for $i=1, \ldots, n$. The function $\beta(z)$ is
defined as follows, using $z$ as the covariate
\begin{verbatim}
zr <- range(z)
l <- seq(zr[1], zr[2], len=n)
beta.z <- splinefun(l, beta, method = "natural")
\end{verbatim}

We can control $\beta$ and its prior distribution using argument
\texttt{control.scopy} within \texttt{f()},
\begin{quote}
    \texttt{control.scopy = list(\\
        covariate = ..., \\
        n = 5, \\
        model = "rw2",\\
        mean = 1.0, \\
        prec.mean =  1.0,  \\
        prec.betas =  10.0})
\end{quote}
where
\begin{description}
\item[covariate] gives the covariate that is used
\item[n] is the number of hyperparameters used in the spline ($3 \leq
    n \leq 15$).
\item[model] the prior model for $\{\beta_i\}$, either \texttt{rw1} or
    \texttt{rw2}. This model is scaled (like with
    \texttt{scale.model=TRUE}).
\item[mean] The prior mean for the mean of $\{\beta_i\}$
\item[prec.mean] The prior precision for the mean of $\{\beta_i\}$
\item[prec.betas] The prior precision for the \texttt{rw1/rw2} model
    for $\{\beta_i\}$.
\end{description}
Note that both precisions are \emph{fixed} and not \emph{random}.

The \texttt{f()}-argument \texttt{precision}, defines how close the
copy is, is similar as for model \texttt{copy}.

\section*{Spesification}

\documentclass[a4paper,11pt]{article}
\usepackage[scale={0.8,0.9},centering,includeheadfoot]{geometry}
\usepackage{amstext}
\usepackage{listings}
\usepackage{verbatim}
\def\opening\null
\usepackage{block}
\begin{document}

\section*{Smooth-Copy of another model component: ``scopy''}

This model is a generalization of \texttt{copy}, please refer to
\texttt{inla.doc("copy")} first.

This describes the way to copy another model component with an
optional smooth/spline scaling, like with
\begin{displaymath}
    \eta = u + v
\end{displaymath}
where $v$ is a smooth copy of $u$
\begin{displaymath}
    v = \beta(z)\times\text{copy}(u)
\end{displaymath}
where $\beta(z)$, a smooth/spline function of the covariate $z$.

\subsection*{Hyperparameters}

The optional hyperparameter is the spline at $n$ fixed locations,
$(l_i, \beta_i)$, for $i=1, \ldots, n$. The function $\beta(z)$ is
defined as follows, using $z$ as the covariate
\begin{verbatim}
zr <- range(z)
l <- seq(zr[1], zr[2], len=n)
beta.z <- splinefun(l, beta, method = "natural")
\end{verbatim}

We can control $\beta$ and its prior distribution using argument
\texttt{control.scopy} within \texttt{f()},
\begin{quote}
    \texttt{control.scopy = list(\\
        covariate = ..., \\
        n = 5, \\
        model = "rw2",\\
        mean = 1.0, \\
        prec.mean =  1.0,  \\
        prec.betas =  10.0})
\end{quote}
where
\begin{description}
\item[covariate] gives the covariate that is used
\item[n] is the number of hyperparameters used in the spline ($3 \leq
    n \leq 15$).
\item[model] the prior model for $\{\beta_i\}$, either \texttt{rw1} or
    \texttt{rw2}. This model is scaled (like with
    \texttt{scale.model=TRUE}).
\item[mean] The prior mean for the mean of $\{\beta_i\}$
\item[prec.mean] The prior precision for the mean of $\{\beta_i\}$
\item[prec.betas] The prior precision for the \texttt{rw1/rw2} model
    for $\{\beta_i\}$.
\end{description}
Note that both precisions are \emph{fixed} and not \emph{random}.

The \texttt{f()}-argument \texttt{precision}, defines how close the
copy is, is similar as for model \texttt{copy}.

\section*{Spesification}

\input{../hyper/latent/scopy.tex}

\subsection*{Example}

Just simulate some data and estimate the parameters back. 

{\small\verbatiminput{example-scopy.R}}

\subsection*{Notes}

\end{document}



% LocalWords: 

%%% Local Variables: 
%%% TeX-master: t
%%% End: 


\subsection*{Example}

Just simulate some data and estimate the parameters back. 

{\small\verbatiminput{example-scopy.R}}

\subsection*{Notes}

\end{document}



% LocalWords: 

%%% Local Variables: 
%%% TeX-master: t
%%% End: 


\subsection*{Example}

Just simulate some data and estimate the parameters back. 

{\small\verbatiminput{example-scopy.R}}

\subsection*{Notes}

\end{document}



% LocalWords: 

%%% Local Variables: 
%%% TeX-master: t
%%% End: 


\subsection*{Example}

Just simulate some data and estimate the parameters back. 

{\small\verbatiminput{example-scopy.R}}

\subsection*{Notes}

\end{document}



% LocalWords: 

%%% Local Variables: 
%%% TeX-master: t
%%% End: 
