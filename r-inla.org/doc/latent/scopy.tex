\documentclass[a4paper,11pt]{article}
\usepackage[scale={0.8,0.9},centering,includeheadfoot]{geometry}
\usepackage{amstext}
\usepackage{listings}
\usepackage{verbatim}
\usepackage{block}

\begin{document}

\section*{Smooth-Copy of another model component: "scopy"}

This model is a generalization of \texttt{copy}, please refer to
\texttt{inla.doc("copy")} first.

This describes the way to copy another model component with an
optional smooth/spline scaling, like with
\begin{displaymath}
    \eta = u + v
\end{displaymath}
where $v$ is a smooth copy of $u$ (component-wise)
\begin{displaymath}
    v = \beta(z)\times\text{copy}(u)
\end{displaymath}
where $\beta(z)$, a smooth/spline function of the covariate $z$. The
smooth scaling is done \textbf{component-wise} for $u$, so if $u$ are
defined with domain $(1, 2, \ldots, m)$, i.e.\
$u=(u_1, u_2, \ldots, u_m)$, then $z$ must be
$z=(z_1, z_2, \ldots, z_m)$, so that
\begin{displaymath}
    v_i = \beta(z_i) u_i, \qquad i=1, 2, \ldots, m.
\end{displaymath}

\subsection*{Hyperparameters}

The hyperparameters are the value of the spline at $n$ fixed
locations, $(l_i, \beta_i)$, for $i=1, \ldots, n$. The function
$\beta(z)$, is defined as (using $z$ as the covariate)
\begin{verbatim}
zr <- range(z)
l <- seq(zr[1], zr[2], len=n)
beta.z <- splinefun(l, beta, method = "natural")
\end{verbatim}

We can control $\beta$ and its prior distribution using argument
\texttt{control.scopy} within \texttt{f()},
\begin{quote}
    \texttt{control.scopy = list(\\
        covariate = ..., \\
        n = 5, \\
        model = "rw2",\\
        mean = 1.0, \\
        prec.mean =  1.0,  \\
        prec.betas =  10.0})
\end{quote}
where
\begin{description}
\item[covariate] gives the covariate that is used
\item[n] is the number of hyperparameters used in the spline
    ($3 \leq n \leq 15$).
\item[model] the prior model for $\{\beta_i\}$, either \texttt{rw1} or
    \texttt{rw2}. This model is scaled (like with
    \texttt{scale.model=TRUE}).
\item[mean] The prior mean for the (weighted-)mean\footnote{The mean
        of $\{\beta_i\}$ is defined to approximate the integral of the
        RW, hence its
        $\left(\frac{1}{2}(\beta_1 + \beta_n) +
          \sum_{j=2}^{n-1}\beta_j\right)/(n-1)$.} of $\{\beta_i\}$
\item[prec.mean] The prior precision for the (weighted-)mean of
    $\{\beta_i\}$
\item[prec.betas] The prior precision for the \texttt{rw1/rw2} model
    for $\{\beta_i\}$
\end{description}
Note that the prior mean and both prior precisions, are \emph{fixed}
and not \emph{random}.

The \texttt{f()}-argument \texttt{precision}, defines how close the
copy is, is similar as for model \texttt{copy}.

\clearpage
\section*{Spesification}
%% DO NOT EDIT!
%% This file is generated automatically from models.R
\begin{description}
	\item[doc] \verb!Create a scopy of a model component!
	\item[hyper]\ 
	 \begin{description}
	 	\item[theta1]\ 
	 	 \begin{description}
	 	 	\item[hyperid] \verb!36101!
	 	 	\item[name] \verb!mean!
	 	 	\item[short.name] \verb!mean!
	 	 	\item[initial] \verb!1!
	 	 	\item[fixed] \verb!FALSE!
	 	 	\item[prior] \verb!normal!
	 	 	\item[param] \verb!1 10!
	 	 	\item[to.theta] \verb!function(x) x!
	 	 	\item[from.theta] \verb!function(x) x!
	 	 \end{description}
	 	\item[theta2]\ 
	 	 \begin{description}
	 	 	\item[hyperid] \verb!36102!
	 	 	\item[name] \verb!slope!
	 	 	\item[short.name] \verb!slope!
	 	 	\item[initial] \verb!0!
	 	 	\item[fixed] \verb!FALSE!
	 	 	\item[prior] \verb!normal!
	 	 	\item[param] \verb!0 10!
	 	 	\item[to.theta] \verb!function(x) x!
	 	 	\item[from.theta] \verb!function(x) x!
	 	 \end{description}
	 	\item[theta3]\ 
	 	 \begin{description}
	 	 	\item[hyperid] \verb!36103!
	 	 	\item[name] \verb!spline.theta1!
	 	 	\item[short.name] \verb!spline!
	 	 	\item[initial] \verb!0!
	 	 	\item[fixed] \verb!FALSE!
	 	 	\item[prior] \verb!laplace!
	 	 	\item[param] \verb!0 10!
	 	 	\item[to.theta] \verb!function(x) x!
	 	 	\item[from.theta] \verb!function(x) x!
	 	 \end{description}
	 	\item[theta4]\ 
	 	 \begin{description}
	 	 	\item[hyperid] \verb!36104!
	 	 	\item[name] \verb!spline.theta2!
	 	 	\item[short.name] \verb!spline2!
	 	 	\item[initial] \verb!0!
	 	 	\item[fixed] \verb!FALSE!
	 	 	\item[prior] \verb!none!
	 	 	\item[param] \verb!!
	 	 	\item[to.theta] \verb!function(x) x!
	 	 	\item[from.theta] \verb!function(x) x!
	 	 \end{description}
	 	\item[theta5]\ 
	 	 \begin{description}
	 	 	\item[hyperid] \verb!36105!
	 	 	\item[name] \verb!spline.theta3!
	 	 	\item[short.name] \verb!spline3!
	 	 	\item[initial] \verb!0!
	 	 	\item[fixed] \verb!FALSE!
	 	 	\item[prior] \verb!none!
	 	 	\item[param] \verb!!
	 	 	\item[to.theta] \verb!function(x) x!
	 	 	\item[from.theta] \verb!function(x) x!
	 	 \end{description}
	 	\item[theta6]\ 
	 	 \begin{description}
	 	 	\item[hyperid] \verb!36106!
	 	 	\item[name] \verb!spline.theta4!
	 	 	\item[short.name] \verb!spline4!
	 	 	\item[initial] \verb!0!
	 	 	\item[fixed] \verb!FALSE!
	 	 	\item[prior] \verb!none!
	 	 	\item[param] \verb!!
	 	 	\item[to.theta] \verb!function(x) x!
	 	 	\item[from.theta] \verb!function(x) x!
	 	 \end{description}
	 	\item[theta7]\ 
	 	 \begin{description}
	 	 	\item[hyperid] \verb!36107!
	 	 	\item[name] \verb!spline.theta5!
	 	 	\item[short.name] \verb!spline5!
	 	 	\item[initial] \verb!0!
	 	 	\item[fixed] \verb!FALSE!
	 	 	\item[prior] \verb!none!
	 	 	\item[param] \verb!!
	 	 	\item[to.theta] \verb!function(x) x!
	 	 	\item[from.theta] \verb!function(x) x!
	 	 \end{description}
	 	\item[theta8]\ 
	 	 \begin{description}
	 	 	\item[hyperid] \verb!36108!
	 	 	\item[name] \verb!spline.theta6!
	 	 	\item[short.name] \verb!spline6!
	 	 	\item[initial] \verb!0!
	 	 	\item[fixed] \verb!FALSE!
	 	 	\item[prior] \verb!none!
	 	 	\item[param] \verb!!
	 	 	\item[to.theta] \verb!function(x) x!
	 	 	\item[from.theta] \verb!function(x) x!
	 	 \end{description}
	 	\item[theta9]\ 
	 	 \begin{description}
	 	 	\item[hyperid] \verb!36109!
	 	 	\item[name] \verb!spline.theta7!
	 	 	\item[short.name] \verb!spline7!
	 	 	\item[initial] \verb!0!
	 	 	\item[fixed] \verb!FALSE!
	 	 	\item[prior] \verb!none!
	 	 	\item[param] \verb!!
	 	 	\item[to.theta] \verb!function(x) x!
	 	 	\item[from.theta] \verb!function(x) x!
	 	 \end{description}
	 	\item[theta10]\ 
	 	 \begin{description}
	 	 	\item[hyperid] \verb!36110!
	 	 	\item[name] \verb!spline.theta8!
	 	 	\item[short.name] \verb!spline8!
	 	 	\item[initial] \verb!0!
	 	 	\item[fixed] \verb!FALSE!
	 	 	\item[prior] \verb!none!
	 	 	\item[param] \verb!!
	 	 	\item[to.theta] \verb!function(x) x!
	 	 	\item[from.theta] \verb!function(x) x!
	 	 \end{description}
	 	\item[theta11]\ 
	 	 \begin{description}
	 	 	\item[hyperid] \verb!36111!
	 	 	\item[name] \verb!spline.theta9!
	 	 	\item[short.name] \verb!spline9!
	 	 	\item[initial] \verb!0!
	 	 	\item[fixed] \verb!FALSE!
	 	 	\item[prior] \verb!none!
	 	 	\item[param] \verb!!
	 	 	\item[to.theta] \verb!function(x) x!
	 	 	\item[from.theta] \verb!function(x) x!
	 	 \end{description}
	 	\item[theta12]\ 
	 	 \begin{description}
	 	 	\item[hyperid] \verb!36112!
	 	 	\item[name] \verb!spline.theta10!
	 	 	\item[short.name] \verb!spline10!
	 	 	\item[initial] \verb!0!
	 	 	\item[fixed] \verb!FALSE!
	 	 	\item[prior] \verb!none!
	 	 	\item[param] \verb!!
	 	 	\item[to.theta] \verb!function(x) x!
	 	 	\item[from.theta] \verb!function(x) x!
	 	 \end{description}
	 	\item[theta13]\ 
	 	 \begin{description}
	 	 	\item[hyperid] \verb!36113!
	 	 	\item[name] \verb!spline.theta11!
	 	 	\item[short.name] \verb!spline11!
	 	 	\item[initial] \verb!0!
	 	 	\item[fixed] \verb!FALSE!
	 	 	\item[prior] \verb!none!
	 	 	\item[param] \verb!!
	 	 	\item[to.theta] \verb!function(x) x!
	 	 	\item[from.theta] \verb!function(x) x!
	 	 \end{description}
	 	\item[theta14]\ 
	 	 \begin{description}
	 	 	\item[hyperid] \verb!36114!
	 	 	\item[name] \verb!spline.theta12!
	 	 	\item[short.name] \verb!spline12!
	 	 	\item[initial] \verb!0!
	 	 	\item[fixed] \verb!FALSE!
	 	 	\item[prior] \verb!none!
	 	 	\item[param] \verb!!
	 	 	\item[to.theta] \verb!function(x) x!
	 	 	\item[from.theta] \verb!function(x) x!
	 	 \end{description}
	 	\item[theta15]\ 
	 	 \begin{description}
	 	 	\item[hyperid] \verb!36115!
	 	 	\item[name] \verb!spline.theta13!
	 	 	\item[short.name] \verb!spline13!
	 	 	\item[initial] \verb!0!
	 	 	\item[fixed] \verb!FALSE!
	 	 	\item[prior] \verb!none!
	 	 	\item[param] \verb!!
	 	 	\item[to.theta] \verb!function(x) x!
	 	 	\item[from.theta] \verb!function(x) x!
	 	 \end{description}
	 \end{description}
	\item[constr] \verb!FALSE!
	\item[nrow.ncol] \verb!FALSE!
	\item[augmented] \verb!FALSE!
	\item[aug.factor] \verb!1!
	\item[aug.constr] \verb!!
	\item[n.div.by] \verb!!
	\item[n.required] \verb!FALSE!
	\item[set.default.values] \verb!FALSE!
	\item[status] \verb!experimental!
	\item[pdf] \verb!scopy!
\end{description}


\clearpage
\subsection*{Example}
Just simulate some data and estimate the parameters back. 
{\small\verbatiminput{example-scopy.R}}

\subsection*{Notes}

This model is experimental.

\end{document}



% LocalWords: 

%%% Local Variables: 
%%% TeX-master: t
%%% End: 
