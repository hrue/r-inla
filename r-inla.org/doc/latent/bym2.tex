\documentclass[a4paper,11pt]{article}
\usepackage[scale={0.8,0.9},centering,includeheadfoot]{geometry}
\usepackage{amstext}
\usepackage{amsmath}
\usepackage{listings}
\begin{document}

\section*{Bym2 model for spatial effects}

\subsection*{Parametrization}

This model is a reparameterisation of the BYM-model, which is a union
of the \lstinline$besag$ model $u^{*}$ and a \lstinline$iid$ model
$v^{*}$, so that
\begin{displaymath}
    x =
    \begin{pmatrix}
        v^{*} + u^{*}\\
        u^{*}
    \end{pmatrix}
\end{displaymath}
where both $u^{*}$ and $v^{*}$ has a precision (hyper-)parameter.  The
length of $x$ is $2n$ if the length of $u^{*}$ (and $v^{*}$) is
$n$. The BYM2 model uses a different parameterisation of the
hyperparameters where
\begin{displaymath}
    x =
    \begin{pmatrix}
        \frac{1}{\sqrt{\tau}}\left(\sqrt{1-\phi} \;v +
          \sqrt{\phi} \;u\right)\\
        u
    \end{pmatrix}
\end{displaymath}
where both $u$ and $v$ are \emph{standardised} to have (generalised)
variance equal to one.  The \emph{marginal} precision is then $\tau$
and the proportion of the marginal variance explained by the spatial
effect ($u$) is $\phi$.

\subsection*{Hyperparameters}
The hyperparameters are the margainal precision $\tau$ and the mixing
parameter $\phi$.  The marginal precision $\tau$ is represented as
\begin{displaymath}
    \theta_{1} = \log(\tau)
\end{displaymath}
and the mixing parameter as
\begin{displaymath}
    \theta_{2} = \log\left(\frac{\phi}{1-\phi}\right)
\end{displaymath}
and the prior is defined on $\mathbf{\theta} = (\theta_{1}, \theta_{2})$.

\subsection*{Specification}

The bym2 model is specified inside the {\tt f()} function as
\begin{verbatim}
 f(<whatever>, model="bym2", graph=<graph>,
   hyper=<hyper>, adjust.for.con.comp = TRUE)
\end{verbatim}
The neighbourhood structure of $\mathbf{x}$ is passed to the program
through the {\tt graph} argument.

The option \verb|adjust.for.con.comp| adjust the model if the graph
has more than one connected compoment, and this adjustment can be
disabled setting this option to \texttt{FALSE}. This means that
\texttt{constr=TRUE} is interpreted as a sum-to-zero constraint on
\emph{each} connected component and the \texttt{rankdef} parameter is
set accordingly. 

\subsubsection*{Hyperparameter spesification and default values}
%% DO NOT EDIT!
%% This file is generated automatically from models.R
\begin{description}
	\item[hyper]\ 
	 \begin{description}
	 	\item[theta1]\ 
	 	 \begin{description}
	 	 	\item[name] log precision
	 	 	\item[short.name] prec
	 	 	\item[prior] pc.prec
	 	 	\item[param] 1 0.01
	 	 	\item[initial] 4
	 	 	\item[fixed] FALSE
	 	 	\item[to.theta] \verb|function(x) log(x)|
	 	 	\item[from.theta] \verb|function(x) exp(x)|
	 	 \end{description}
	 	\item[theta2]\ 
	 	 \begin{description}
	 	 	\item[name] logit phi
	 	 	\item[short.name] phi
	 	 	\item[prior] pc
	 	 	\item[param] 0.5 -1
	 	 	\item[initial] -3
	 	 	\item[fixed] FALSE
	 	 	\item[to.theta] \verb|function(x) log(x/(1-x))|
	 	 	\item[from.theta] \verb|function(x) exp(x)/(1+exp(x))|
	 	 \end{description}
	 \end{description}
	\item[constr] TRUE
	\item[nrow.ncol] FALSE
	\item[augmented] TRUE
	\item[aug.factor] 2
	\item[aug.constr] 2
	\item[n.div.by] 
	\item[n.required] TRUE
	\item[set.default.values] TRUE
	\item[status] experimental
	\item[pdf] bym2
\end{description}



\subsection*{Example}

%%\documentclass{article}
%%\usepackage{amsmath}
%%\begin{document}

\section*{Details on the implementation}

This gives some details of the implementation, which depends on the
following variables
\begin{description}
\item[nc1] Number of connected components in the graph with size 1.
    These nodes, \emph{singletons}, have no neigbours.
\item[nc2] Number of connected components in the graph with size
    $\ge2$.
\item[scale.model] The value of the logical flag, if the model should
    be scaled or not. (Default FALSE)
\item[adjust.for.con.comp] The value of the logical flag if the
    \texttt{constr=TRUE} option should be reinterpreted.
\end{description}

\subsubsection*{The case \texttt{(scale.model==FALSE \&\&
        adjust.for.con.comp == FALSE)}}

The option \texttt{constr=TRUE} is interpreted as a sum-to-zero
constraint over the whole graph. Singletons are given a uniform
distribution on $(-\infty,\infty)$ before the constraint, which may
give a singular posterior.

\subsubsection*{The case \texttt{(scale.model==TRUE \&\&
        adjust.for.con.comp == FALSE)}}

The option \texttt{constr=TRUE} is interpreted as a sum-to-zero
constraint over the whole graph. Let $Q = \tau R$ be the standard
precision matrix from the \texttt{besag}-model with precision
parameter $\tau$. Then $R$, except the singletons, are scaled so that
the geometric mean of the marginal variances is 1, and $R$ is modified
so that singletons have a standard Normal distribution.

\subsubsection*{The case \texttt{(scale.model==FALSE \&\&
        adjust.for.con.comp == TRUE)}}

The option \texttt{constr=TRUE} is interpreted as one sum-to-zero
constraint over each of the \texttt{nc2} connected components of size
$\ge2$. Singletons are given a uniform distribution on
$(-\infty,\infty)$, which may give a singular posterior.

\subsubsection*{The case \texttt{(scale.model==TRUE \&\&
        adjust.for.con.comp == TRUE)}}

The option \texttt{constr=TRUE} is interpreted as \texttt{nc2}
sum-to-zero constraints for each of the connected components of size
$\ge2$. Let $Q = \tau R$ be the standard precision matrix from the
\texttt{besag}-model with precision parameter $\tau$. Then $R$, are
scaled so that the geometric mean of the marginal variances in each
connected component of size $\ge2$ is 1, and modified so that
singletons have a standard Normal distribution.

%%\end{document}


\subsection*{Notes}

The term $\frac{1}{2}\log(|R|^{*})$ of the normalisation constant is
not computed, hence you need to add this part to the log marginal
likelihood estimate, if you need it. Here $R$ is the precision matrix
for the standardised Besag part of the model.

The generic PC-prior for $\phi$ is available as \texttt{prior="pc"}
and parameters \texttt{param="c(u, alpha)"}, where $\text{Prob}(\phi
\le u) = \alpha$. If $\alpha < 0$ or $\alpha>1$, then it is set to a value
close to the minimum value of $\alpha$ allowed. This prior depends on
the graph and its computational cost is ${\mathcal O}(n^{3})$.


\end{document}


% LocalWords: 

%%% Local Variables: 
%%% TeX-master: t
%%% End: 
