\documentclass[a4paper,11pt]{article}
\usepackage[scale={0.8,0.9},centering,includeheadfoot]{geometry}
\usepackage{amstext}
\usepackage{listings}
\usepackage{verbatim}
\def\o{_{\text{obs}}}

\begin{document}


\section*{The ME model: details}

This note gives the missing details in the ME model.

The model is
\begin{displaymath}
    y = \beta x + \epsilon{}
\end{displaymath}
where $y$ is the response, $\beta$ the effect of the true covariate
$x$ with zero mean Gaussian noise $\epsilon$. The issue is that $x$ is
not observed directly, but only through $x_{\text{obs}}$, where
\begin{displaymath}
    x\o = x + \nu{}
\end{displaymath}
where $\nu$ is zero mean Gaussian noise.  The parameters are: $\beta$
has prior $\pi(\beta)$, $x$ is apriori ${\mathcal N}(\mu_{x} {I},
\tau_{x} {I})$, and $\tau\o$ is the observation precision for $x$ (ie
$\text{Prec}(x\o|x)$)\footnote{Note: The second argument in ${\mathcal
        N}(,)$ is the precision not the variance.}.

Assume that the precision of the observations $y$, $\tau_{y}$, is
known as it does not influence the calculations. Let $\theta =
(\beta, \tau_{x}, \tau\o, \mu_{x})$.  The full posterior is
\begin{displaymath}
    \pi(x, \theta | y, x\o) \;\propto\; \pi(\theta)
    \; \pi(x |\theta) \; \pi(x\o | x, \theta) \;
    \pi(y | x, \theta)
\end{displaymath}
Using that
\begin{displaymath}
    \pi(x | \theta) \; \pi(x\o | x, \theta) = 
    \pi(x | x\o, \theta) \; \pi(x\o | \theta)
\end{displaymath}
we get
\begin{displaymath}
    \pi(x, \theta | y, x\o) \;\propto\; \pi(\theta)\; 
    \pi(x | x\o, \theta) \; \pi(x\o | \theta) \;
    \pi(y | x, \theta).
\end{displaymath}
This means that $x$ only enters in \emph{one term} (apart from the
likelihood) hence can be used as an ordinary latent model
\texttt{f()}. Its easy to derive that
\begin{displaymath}
    x | x\o, \theta \;\sim\;{\mathcal N}\left(
      \frac{\tau_{x} \mu_{x} I + \tau\o x\o}{
          \tau_{x}+\tau\o}, 
      ({\tau_{x} + \tau\o}) I \right).
\end{displaymath}
and 
\begin{displaymath}
    x\o | \theta \;\sim\; {\mathcal N}\left(
      \mu_{x}I, \frac{1}{1/\tau_{x} + 1/\tau\o} I\right).
\end{displaymath}
Note that $x\o | \theta$ does not depend on $x$, hence conditionally
on $\theta$, its a constant. But it do need to be included in the
model, as its log-density is
\begin{displaymath}
    -\frac{n}{2}\log(2\pi) + \frac{n}{2} \log(\frac{1}{1/\tau_{x} +
        1/\tau\o})
    - \frac{1}{2} \frac{1}{1/\tau_{x} + 1/\tau\o} (x\o -\mu_{x}
    I)^{T}(x\o -\mu_{x}I)
\end{displaymath}
and do depend on $\theta$.

The last tweak, is that we do the change of variable from $(x, \beta)$
to $(u, \beta)$, where $u = \beta x$, so that
\begin{displaymath}
    y = u + \epsilon.
\end{displaymath}
This makes the implementation more convenient.
Then we get
\begin{eqnarray}\label{eq1}%
    \pi(u, \theta | y, x\o) &\propto& \pi(\theta)\\
    && \pi(u | x\o, \theta) \; \pi(x\o | \theta) \\
    && \pi(y | u, \theta).
\end{eqnarray}
where
\begin{displaymath}
    u | \theta, x\o \;\sim\;{\mathcal N}\left(
      \beta\frac{\tau_{x} \mu_{x} I + \tau\o x\o}{
          \tau_{x}+\tau\o}, 
      \frac{{\tau_{x} + \tau\o}}{\beta^{2}} I \right).
\end{displaymath}

\end{document}
