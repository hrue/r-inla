\documentclass[a4paper,11pt]{article}
\usepackage[scale={0.8,0.9},centering,includeheadfoot]{geometry}
\usepackage{amstext}
\usepackage{listings}
\usepackage{verbatim}
\begin{document}

\section*{Proper/Non-intrinsic Besag model for spatial effects
    (variant 2)}

\subsection*{Parametrization}

The (2nd) proper version of the Besag model for random vector
$\mathbf{x}=(x_1,\dots,x_n)$ is defined with precision
matrix\footnote{ Brian G Leroux, Xingye Lei, and Norman
    Breslow. Estimation of disease rates in small areas: A new mixed
    model for spatial dependence. In Statistical Models in
    Epidemiology, the Environment, and Clinical Trials,
    Springer, 2000}
\begin{equation}\label{eq.besag}
    \tau ((1-\lambda) I + \lambda R)
\end{equation}
where $R$ is the (unit precision) precision matrix for the Besag
model, $\tau$ is a precision parameter and $0<\lambda<1$.

\subsection*{Hyperparameters}

The precision parameter $\tau$ is represented as
\begin{displaymath}
    \theta_{1} =\log \tau
\end{displaymath}
and the prior is defined on $\theta_{1}$. The $\lambda$ parameter is
represented as
\begin{displaymath}
    \theta_{2} = \log\left( \lambda/(1-\lambda)\right)
\end{displaymath}
and the prior is defined on $\theta_{2}$. 

\subsection*{Specification}

The model is specified inside the {\tt f()} function as
\begin{verbatim}
 f(<whatever>, model="besagproper2", graph=<graph>,
   hyper=<hyper>)
\end{verbatim}

The neighbourhood structure of $\mathbf{x}$ is passed to the program
through the {\tt graph} argument.  The structure of this file is
described below.

\subsubsection*{Hyperparameter spesification and default values}
%% DO NOT EDIT!
%% This file is generated automatically from models.R
\begin{description}
	\item[doc] \verb!An alternative proper version of the Besag model!
	\item[hyper]\ 
	 \begin{description}
	 	\item[theta1]\ 
	 	 \begin{description}
	 	 	\item[hyperid] \verb!13001!
	 	 	\item[name] \verb!log precision!
	 	 	\item[short.name] \verb!prec!
	 	 	\item[prior] \verb!loggamma!
	 	 	\item[param] \verb!1 5e-04!
	 	 	\item[initial] \verb!2!
	 	 	\item[fixed] \verb!FALSE!
	 	 	\item[to.theta] \verb!function(x) log(x)!
	 	 	\item[from.theta] \verb!function(x) exp(x)!
	 	 \end{description}
	 	\item[theta2]\ 
	 	 \begin{description}
	 	 	\item[hyperid] \verb!13002!
	 	 	\item[name] \verb!logit lambda!
	 	 	\item[short.name] \verb!lambda!
	 	 	\item[prior] \verb!gaussian!
	 	 	\item[param] \verb!0 0.45!
	 	 	\item[initial] \verb!3!
	 	 	\item[fixed] \verb!FALSE!
	 	 	\item[to.theta] \verb!function(x) log(x / (1 - x))!
	 	 	\item[from.theta] \verb!function(x) exp(x) / (1 + exp(x))!
	 	 \end{description}
	 \end{description}
	\item[constr] \verb!FALSE!
	\item[nrow.ncol] \verb!FALSE!
	\item[augmented] \verb!FALSE!
	\item[aug.factor] \verb!1!
	\item[aug.constr] \verb!!
	\item[n.div.by] \verb!!
	\item[n.required] \verb!TRUE!
	\item[set.default.values] \verb!TRUE!
	\item[pdf] \verb!besagproper2!
\end{description}



\subsection*{Example}

\verbatiminput{besagproper2-example.R}

\subsection*{Notes}

None

\end{document}


% LocalWords: 

%%% Local Variables: 
%%% TeX-master: t
%%% End: 



