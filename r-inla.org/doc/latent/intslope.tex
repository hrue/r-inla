\documentclass[a4paper,11pt]{article}
\usepackage[scale={0.8,0.9},centering,includeheadfoot]{geometry}
\usepackage{amstext}
\usepackage{listings}
\usepackage{verbatim}
\begin{document}

\section*{Intercept-slope model}

\subsection*{Parametrization}

The intercept-slope model is a convenient reimplementation of a
commonly used construct, where
\begin{displaymath}
    (a,b)
\end{displaymath}
is bivariate Gaussian with a Wishart prior for the precision
matrix\footnote{The documentation for the model ``iid2d'' gives the
    details of the definition of the parameterisation of the precision
    matrix and the Wishart-prior.}, and various forms of
\begin{equation}\label{eq1}%
    \gamma(a + bz),
\end{equation}
where $z$ is a covariate and $\gamma$ is a (random) scaling, goes into
the linear predictor. Replicates of $(a,b)$ is indexed by
\emph{subject}, $i=1, \ldots, n$, and the various scaling of
Eq.~\ref{eq1} by \emph{strata} $j=1, \ldots, m$, leading to a model
for (a subset of)
\begin{displaymath}
    \left\{\gamma_j(a_i + b_i z_{ij}), \quad i=1, \ldots,n, \quad j=1,\ldots,m\right\},
\end{displaymath}
as not all combinations need to be present.

\subsection*{Hyperparameters}

The hyperparameters are $(\theta_1,\theta_2,\theta_3)$ as in the model
``iid2d'' (related to the precisions of $a$ and $b$, and their
correlation), and
$\theta_4=\gamma_1, \ldots, \theta_{13}=\gamma_{10}$. Since $m$ is
defined in the input, only $\gamma_1, \ldots, \gamma_m$ are used. $m$
is limited to $m \le 10$. Note that $\gamma_1$ is by default
\textbf{fixed} to $1$, due to confounding issues.


\subsection*{Specification}

The is specified as
\begin{verbatim}
    f(idx, model="intslope", hyper = ...,
      precision = exp(14),
      args.intslope = list(subject=i, strata=j, covariate = z))
\end{verbatim}
The defintion of the model is through the \verb|args.intslope|
argument, where \verb|i| and \verb|j| are factors/integers and
\verb|z| is numerical, all with same length $N$, say. The argument
\verb|idx|, index which row that is used, hence values of \verb|idx|
must take integer values in the interval $1$ to $N$. The precision
argument, defines the tiny small noise added to each $\gamma(a+bz)$ to
avoid a singular joint model. The \verb|subject| and \verb|strata|
argument, is converted internally as
\begin{verbatim}
    subject = as.numerical(as.factor(subject))
    strata  = as.numerical(as.factor(strata))
\end{verbatim}
and the results is shown after this conversion.


\subsubsection*{Hyperparameter spesification and default values}
{\small
%% DO NOT EDIT!
%% This file is generated automatically from models.R
\begin{description}
	\item[doc] \verb!Intecept-slope model with Wishart-prior!
	\item[hyper]\ 
	 \begin{description}
	 	\item[theta1]\ 
	 	 \begin{description}
	 	 	\item[hyperid] \verb!16101!
	 	 	\item[name] \verb!log precision1!
	 	 	\item[short.name] \verb!prec1!
	 	 	\item[initial] \verb!4!
	 	 	\item[fixed] \verb!FALSE!
	 	 	\item[prior] \verb!wishart2d!
	 	 	\item[param] \verb!4 1 1 0!
	 	 	\item[to.theta] \verb!function(x) log(x)!
	 	 	\item[from.theta] \verb!function(x) exp(x)!
	 	 \end{description}
	 	\item[theta2]\ 
	 	 \begin{description}
	 	 	\item[hyperid] \verb!16102!
	 	 	\item[name] \verb!log precision2!
	 	 	\item[short.name] \verb!prec2!
	 	 	\item[initial] \verb!4!
	 	 	\item[fixed] \verb!FALSE!
	 	 	\item[prior] \verb!none!
	 	 	\item[param] \verb!!
	 	 	\item[to.theta] \verb!function(x) log(x)!
	 	 	\item[from.theta] \verb!function(x) exp(x)!
	 	 \end{description}
	 	\item[theta3]\ 
	 	 \begin{description}
	 	 	\item[hyperid] \verb!16103!
	 	 	\item[name] \verb!logit correlation!
	 	 	\item[short.name] \verb!cor!
	 	 	\item[initial] \verb!4!
	 	 	\item[fixed] \verb!FALSE!
	 	 	\item[prior] \verb!none!
	 	 	\item[param] \verb!!
	 	 	\item[to.theta] \verb!function(x) log((1 + x) / (1 - x))!
	 	 	\item[from.theta] \verb!function(x) 2 * exp(x) / (1 + exp(x)) - 1!
	 	 \end{description}
	 	\item[theta4]\ 
	 	 \begin{description}
	 	 	\item[hyperid] \verb!16104!
	 	 	\item[name] \verb!gamma1!
	 	 	\item[short.name] \verb!g1!
	 	 	\item[initial] \verb!1!
	 	 	\item[fixed] \verb!TRUE!
	 	 	\item[prior] \verb!normal!
	 	 	\item[param] \verb!1 36!
	 	 	\item[to.theta] \verb!function(x) x!
	 	 	\item[from.theta] \verb!function(x) x!
	 	 \end{description}
	 	\item[theta5]\ 
	 	 \begin{description}
	 	 	\item[hyperid] \verb!16105!
	 	 	\item[name] \verb!gamma2!
	 	 	\item[short.name] \verb!g2!
	 	 	\item[initial] \verb!1!
	 	 	\item[fixed] \verb!TRUE!
	 	 	\item[prior] \verb!normal!
	 	 	\item[param] \verb!1 36!
	 	 	\item[to.theta] \verb!function(x) x!
	 	 	\item[from.theta] \verb!function(x) x!
	 	 \end{description}
	 	\item[theta6]\ 
	 	 \begin{description}
	 	 	\item[hyperid] \verb!16106!
	 	 	\item[name] \verb!gamma3!
	 	 	\item[short.name] \verb!g3!
	 	 	\item[initial] \verb!1!
	 	 	\item[fixed] \verb!TRUE!
	 	 	\item[prior] \verb!normal!
	 	 	\item[param] \verb!1 36!
	 	 	\item[to.theta] \verb!function(x) x!
	 	 	\item[from.theta] \verb!function(x) x!
	 	 \end{description}
	 	\item[theta7]\ 
	 	 \begin{description}
	 	 	\item[hyperid] \verb!16107!
	 	 	\item[name] \verb!gamma4!
	 	 	\item[short.name] \verb!g4!
	 	 	\item[initial] \verb!1!
	 	 	\item[fixed] \verb!TRUE!
	 	 	\item[prior] \verb!normal!
	 	 	\item[param] \verb!1 36!
	 	 	\item[to.theta] \verb!function(x) x!
	 	 	\item[from.theta] \verb!function(x) x!
	 	 \end{description}
	 	\item[theta8]\ 
	 	 \begin{description}
	 	 	\item[hyperid] \verb!16108!
	 	 	\item[name] \verb!gamma5!
	 	 	\item[short.name] \verb!g5!
	 	 	\item[initial] \verb!1!
	 	 	\item[fixed] \verb!TRUE!
	 	 	\item[prior] \verb!normal!
	 	 	\item[param] \verb!1 36!
	 	 	\item[to.theta] \verb!function(x) x!
	 	 	\item[from.theta] \verb!function(x) x!
	 	 \end{description}
	 	\item[theta9]\ 
	 	 \begin{description}
	 	 	\item[hyperid] \verb!16109!
	 	 	\item[name] \verb!gamma6!
	 	 	\item[short.name] \verb!g6!
	 	 	\item[initial] \verb!1!
	 	 	\item[fixed] \verb!TRUE!
	 	 	\item[prior] \verb!normal!
	 	 	\item[param] \verb!1 36!
	 	 	\item[to.theta] \verb!function(x) x!
	 	 	\item[from.theta] \verb!function(x) x!
	 	 \end{description}
	 	\item[theta10]\ 
	 	 \begin{description}
	 	 	\item[hyperid] \verb!16110!
	 	 	\item[name] \verb!gamma7!
	 	 	\item[short.name] \verb!g7!
	 	 	\item[initial] \verb!1!
	 	 	\item[fixed] \verb!TRUE!
	 	 	\item[prior] \verb!normal!
	 	 	\item[param] \verb!1 36!
	 	 	\item[to.theta] \verb!function(x) x!
	 	 	\item[from.theta] \verb!function(x) x!
	 	 \end{description}
	 	\item[theta11]\ 
	 	 \begin{description}
	 	 	\item[hyperid] \verb!16111!
	 	 	\item[name] \verb!gamma8!
	 	 	\item[short.name] \verb!g8!
	 	 	\item[initial] \verb!1!
	 	 	\item[fixed] \verb!TRUE!
	 	 	\item[prior] \verb!normal!
	 	 	\item[param] \verb!1 36!
	 	 	\item[to.theta] \verb!function(x) x!
	 	 	\item[from.theta] \verb!function(x) x!
	 	 \end{description}
	 	\item[theta12]\ 
	 	 \begin{description}
	 	 	\item[hyperid] \verb!16112!
	 	 	\item[name] \verb!gamma9!
	 	 	\item[short.name] \verb!g9!
	 	 	\item[initial] \verb!1!
	 	 	\item[fixed] \verb!TRUE!
	 	 	\item[prior] \verb!normal!
	 	 	\item[param] \verb!1 36!
	 	 	\item[to.theta] \verb!function(x) x!
	 	 	\item[from.theta] \verb!function(x) x!
	 	 \end{description}
	 	\item[theta13]\ 
	 	 \begin{description}
	 	 	\item[hyperid] \verb!16113!
	 	 	\item[name] \verb!gamma10!
	 	 	\item[short.name] \verb!g10!
	 	 	\item[initial] \verb!1!
	 	 	\item[fixed] \verb!TRUE!
	 	 	\item[prior] \verb!normal!
	 	 	\item[param] \verb!1 36!
	 	 	\item[to.theta] \verb!function(x) x!
	 	 	\item[from.theta] \verb!function(x) x!
	 	 \end{description}
	 \end{description}
	\item[constr] \verb!FALSE!
	\item[nrow.ncol] \verb!FALSE!
	\item[augmented] \verb!FALSE!
	\item[aug.factor] \verb!1!
	\item[aug.constr] \verb!!
	\item[n.div.by] \verb!!
	\item[n.required] \verb!FALSE!
	\item[set.default.values] \verb!TRUE!
	\item[status] \verb!experimental!
	\item[pdf] \verb!intslope!
\end{description}

}

\clearpage
\subsection*{Example}
{\small\verbatiminput{example-intslope.R}}

\subsection*{Notes}
\begin{itemize}
\item With $n_s=\max($subject$)$, the internal storage of this model
    is
    \begin{displaymath}
        \left(\gamma_{j_1} (a_{i_1} + z_{1}b_{i_{1}}), \ldots, 
        \gamma_{j_N} (a_{i_N} + z_{N}b_{i_N}), 
        a_1, \ldots, a_{n_s}, b_1, \ldots, b_{n_s}\right),
    \end{displaymath}
    i.e.\ a vector of length $N+2n_s$.
\end{itemize}

\end{document}


% LocalWords: 

%%% Local Variables: 
%%% TeX-master: t
%%% End: 
