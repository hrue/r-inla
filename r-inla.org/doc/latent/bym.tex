\documentclass[a4paper,11pt]{article}
\usepackage[scale={0.8,0.9},centering,includeheadfoot]{geometry}
\usepackage{amstext}
\usepackage{listings}
\begin{document}

\section*{Bym model for spatial effects}

\subsection*{Parametrization}

This model is simply the sum of a \lstinline$besag$ model and a
\lstinline$iid$ model.
 
The benefite is that this allows to get the posterior marginals of the
sum of the spatial and iid model; otherwise it offers no advantages.

\subsection*{Hyperparameters}
The hyperparameters are the precision $\tau_1$ of the \lstinline$iid$
model and the precision $\tau_2$ of the \lstinline$besag$ model.  The
precision parameters are represented as
\begin{displaymath}
    \theta=(\theta_1,\theta_2) =(\log \tau_1,\log \tau_2)
\end{displaymath}
and the prior is defined on $\mathbf{\theta}$.

\subsection*{Specification}

The bym model is specified inside the {\tt f()} function as
\begin{verbatim}
 f(<whatever>,model="bym",graph.file=<graph file name>,
   hyper=<hyper>, adjust.for.con.comp = TRUE)
\end{verbatim}

The neighbourhood structure of $\mathbf{x}$ is passed to the program
through the {\tt graph.file} argument.  The structure of this file is
described below.

The option \verb|adjust.for.con.comp| adjust the model if the graph
has more than one connected compoment, and this adjustment can be
disabled setting this option to FALSE.

\subsubsection*{Hyperparameter spesification and default values}
\documentclass[a4paper,11pt]{article}
\usepackage[scale={0.8,0.9},centering,includeheadfoot]{geometry}
\usepackage{amstext}
\usepackage{listings}
\usepackage{amsmath,amssymb}

\def\mm#1{\ensuremath{\boldsymbol{#1}}} % version: amsmath


\begin{document}
\bibliographystyle{apalike}

\section*{Bym: An example of disease mapping with covariate }

Larynx cancer mortality counts are observed in the 544 district of
    Germany from 1986 to 1990. We assume
    the data to be conditionally independent Poisson random variables
    with mean $E_i \exp(\eta_i)$, where $E_i$ is fixed and accounts
    for demographic variation, and $\eta_i$ is the log-relative
    risk. Together with the counts, for each district, the level of
    smoking consumption $c$ is registered.

    The model for $\eta_i$ takes the following form
    \begin{equation}
        \label{eq10}
        \eta_i=\mu + f_s(s_i)+f(c_i)+u_i
    \end{equation}
    where  $f_s(\cdot)$ is the spatial effect and $u_i$ is the
    unstructured random effect. 
    
    The prior model for
    $\mm{f}_s=(f(0),\dots,f(s),\dots,f(S-1))$ implemented in the {\tt
      inla} program is a simple (but most often used) intrinsic GMRF
    model, see \cite[Ch. 3]{book80}, defined as:
    \begin{equation}\label{eq.besag}
      f_s(s)|f_s(s'),s\neq s',\lambda_s\sim\mathcal{N}(\frac{1}{n_s}\sum_{s\sim s'}f_s(s'),\frac{1}{n_s\lambda_s})
    \end{equation}
    where $n_s$ is the number of neighbours of site $s$, $s\sim s'$
    indicates that the two sites $s$ and $s'$ are neighbours.  $\lambda_s$
    is the unknown precision parameter.
    
    The remaining term in (\ref{eq10}),
    $f(c_i)$, is the unknown effect of of the exposure covariate which
    assumes value $c_i$ for observation $i$. The effect of covariate
    $c$ is modelled as a smooth function $f(\cdot)$ parametrised as
    unknown values $\mm{f}=(f_0,\dots,f_{m-1})^T$ at $m=100$
    equidistant values of $c_i$.  We have scaled the covariate values
    so that they belong to the interval $[0,10]$. The vector $\mm{f}$
    is modelled with a second-order random walk (RW2) prior with
    unknown precision $\lambda_f$. A sum-to-zero constraint is imposed
    on $\mm{f}_s$ and $\mm{f}$ separate out the spatial effect and the
    effect of the covariate from the common mean $\mu$.

    The model has three hyperparameters
    $\mm{\theta}=(\log\lambda_s,\log\lambda_f,\log\lambda_{\eta})$. Following
    \cite{tech80} we assign a vague LogGamma prior to each element of
    $\mm{\theta}$.

\subsection*{Linear effect for the covariate}
An alternative model is to assume a linear effect for the covariate $c$:
\begin{equation}\label{eq10-lin}
    \eta_i=\mu+f_s(s_i)+\beta c_i+u_i
\end{equation}
In this case the numer of hyperparameters is reduced to  tw, namely 
 $\mm{\theta}=(\log\lambda_s,\log\lambda_{\eta})$

\small\bibliography{../mybib} \newpage


\end{document}


% LocalWords: 

%%% Local Variables: 
%%% TeX-master: t
%%% End: 



\subsubsection*{Structure of the graph file}

We describe the required format for the graph file using a small
example. Let the file {\tt gra.dat}, relative to a small graph of only
5 elements, be
\begin{lstlisting}[basicstyle=\footnotesize]
    5
    1 1 2
    2 2 1 3
    3 3 2 4 5 
    4 1 3
    5 1 3
\end{lstlisting}
Line 1 declares the total number of nodes in the graph (5), then, in
lines 2-6 each node is described. For example, line 4 states that node
3 has 4 neighbours and these are nodes 2, 4 and 5.

The graph file can either have nodes indexed from 1 to $n$, or from 0
to $n-1$.  Note that in the latter case, node $i$ seen from R
corresponds to node $i-1$ in the 0-indexed graph.

\subsection*{Example}

For examples of application of this model see the {\tt Bym} example in
Volume I.

\subsection*{Notes}

None

\end{document}


% LocalWords: 

%%% Local Variables: 
%%% TeX-master: t
%%% End: 
