\documentclass[a4paper,11pt]{article}
\usepackage[scale={0.8,0.9},centering,includeheadfoot]{geometry}
\usepackage{amstext}
\usepackage{amsmath,amssymb}
\usepackage{listings}
\def\mm#1{\ensuremath{\boldsymbol{#1}}} % version: amsmath
\begin{document}

\section*{Generic3 model}

\subsection*{Parametrization}

The generic3 model implements the following precision matrix
\begin{equation}\label{eq1}%
    \mathbf{Q}= \tau_{\text{common}} \sum_{i=1}^{m} \tau_i \mm{R}_i,
    \qquad 1 < m \le 10,
\end{equation}
where each $\mm{R}_i \ge 0$, and $\tau_i$ is the specific precision
parameter and $\tau_{\text{common}}$ is a shared one.

\subsection*{Hyperparameters}

The hyperparameters are defined as 
\begin{eqnarray*}
    \theta_i = \log(\tau_i), \qquad i=1, 10
\end{eqnarray*}
and
\begin{displaymath}
    \theta_{11} = \log(\tau_{\text{common}})
\end{displaymath}
and priors are assigned to $(\theta_1,\theta_2, \ldots)$.

\subsection*{Specification}

The generic3 model is specified inside the {\tt f()} function as
\begin{verbatim}
 f(<whatever>, model="generic3", Cmatrix = <list.of.Cmat>, hyper = <hyper>)
\end{verbatim}
where {\tt <list.of.Cmat>} a list of length $m$ (maximum 10) of
$\mm{R}_i$-matrices. By default, $\theta_j$ for $j=m+1, \ldots, 11$ is
set to fixed (this includes $\tau_{\text{common}}$).


\subsubsection*{Hyperparameter spesification and default values}
%% DO NOT EDIT!
%% This file is generated automatically from models.R
\begin{description}
	\item[doc] \verb!A generic model (type 3)!
	\item[hyper]\ 
	 \begin{description}
	 	\item[theta1]\ 
	 	 \begin{description}
	 	 	\item[hyperid] \verb!21001!
	 	 	\item[name] \verb!log precision1!
	 	 	\item[short.name] \verb!prec1!
	 	 	\item[initial] \verb!4!
	 	 	\item[fixed] \verb!FALSE!
	 	 	\item[prior] \verb!loggamma!
	 	 	\item[param] \verb!1 5e-05!
	 	 	\item[to.theta] \verb!function(x) log(x)!
	 	 	\item[from.theta] \verb!function(x) exp(x)!
	 	 \end{description}
	 	\item[theta2]\ 
	 	 \begin{description}
	 	 	\item[hyperid] \verb!21002!
	 	 	\item[name] \verb!log precision2!
	 	 	\item[short.name] \verb!prec2!
	 	 	\item[initial] \verb!4!
	 	 	\item[fixed] \verb!FALSE!
	 	 	\item[prior] \verb!loggamma!
	 	 	\item[param] \verb!1 5e-05!
	 	 	\item[to.theta] \verb!function(x) log(x)!
	 	 	\item[from.theta] \verb!function(x) exp(x)!
	 	 \end{description}
	 	\item[theta3]\ 
	 	 \begin{description}
	 	 	\item[hyperid] \verb!21003!
	 	 	\item[name] \verb!log precision3!
	 	 	\item[short.name] \verb!prec3!
	 	 	\item[initial] \verb!4!
	 	 	\item[fixed] \verb!FALSE!
	 	 	\item[prior] \verb!loggamma!
	 	 	\item[param] \verb!1 5e-05!
	 	 	\item[to.theta] \verb!function(x) log(x)!
	 	 	\item[from.theta] \verb!function(x) exp(x)!
	 	 \end{description}
	 	\item[theta4]\ 
	 	 \begin{description}
	 	 	\item[hyperid] \verb!21004!
	 	 	\item[name] \verb!log precision4!
	 	 	\item[short.name] \verb!prec4!
	 	 	\item[initial] \verb!4!
	 	 	\item[fixed] \verb!FALSE!
	 	 	\item[prior] \verb!loggamma!
	 	 	\item[param] \verb!1 5e-05!
	 	 	\item[to.theta] \verb!function(x) log(x)!
	 	 	\item[from.theta] \verb!function(x) exp(x)!
	 	 \end{description}
	 	\item[theta5]\ 
	 	 \begin{description}
	 	 	\item[hyperid] \verb!21005!
	 	 	\item[name] \verb!log precision5!
	 	 	\item[short.name] \verb!prec5!
	 	 	\item[initial] \verb!4!
	 	 	\item[fixed] \verb!FALSE!
	 	 	\item[prior] \verb!loggamma!
	 	 	\item[param] \verb!1 5e-05!
	 	 	\item[to.theta] \verb!function(x) log(x)!
	 	 	\item[from.theta] \verb!function(x) exp(x)!
	 	 \end{description}
	 	\item[theta6]\ 
	 	 \begin{description}
	 	 	\item[hyperid] \verb!21006!
	 	 	\item[name] \verb!log precision6!
	 	 	\item[short.name] \verb!prec6!
	 	 	\item[initial] \verb!4!
	 	 	\item[fixed] \verb!FALSE!
	 	 	\item[prior] \verb!loggamma!
	 	 	\item[param] \verb!1 5e-05!
	 	 	\item[to.theta] \verb!function(x) log(x)!
	 	 	\item[from.theta] \verb!function(x) exp(x)!
	 	 \end{description}
	 	\item[theta7]\ 
	 	 \begin{description}
	 	 	\item[hyperid] \verb!21007!
	 	 	\item[name] \verb!log precision7!
	 	 	\item[short.name] \verb!prec7!
	 	 	\item[initial] \verb!4!
	 	 	\item[fixed] \verb!FALSE!
	 	 	\item[prior] \verb!loggamma!
	 	 	\item[param] \verb!1 5e-05!
	 	 	\item[to.theta] \verb!function(x) log(x)!
	 	 	\item[from.theta] \verb!function(x) exp(x)!
	 	 \end{description}
	 	\item[theta8]\ 
	 	 \begin{description}
	 	 	\item[hyperid] \verb!21008!
	 	 	\item[name] \verb!log precision8!
	 	 	\item[short.name] \verb!prec8!
	 	 	\item[initial] \verb!4!
	 	 	\item[fixed] \verb!FALSE!
	 	 	\item[prior] \verb!loggamma!
	 	 	\item[param] \verb!1 5e-05!
	 	 	\item[to.theta] \verb!function(x) log(x)!
	 	 	\item[from.theta] \verb!function(x) exp(x)!
	 	 \end{description}
	 	\item[theta9]\ 
	 	 \begin{description}
	 	 	\item[hyperid] \verb!21009!
	 	 	\item[name] \verb!log precision9!
	 	 	\item[short.name] \verb!prec9!
	 	 	\item[initial] \verb!4!
	 	 	\item[fixed] \verb!FALSE!
	 	 	\item[prior] \verb!loggamma!
	 	 	\item[param] \verb!1 5e-05!
	 	 	\item[to.theta] \verb!function(x) log(x)!
	 	 	\item[from.theta] \verb!function(x) exp(x)!
	 	 \end{description}
	 	\item[theta10]\ 
	 	 \begin{description}
	 	 	\item[hyperid] \verb!21010!
	 	 	\item[name] \verb!log precision10!
	 	 	\item[short.name] \verb!prec10!
	 	 	\item[initial] \verb!4!
	 	 	\item[fixed] \verb!FALSE!
	 	 	\item[prior] \verb!loggamma!
	 	 	\item[param] \verb!1 5e-05!
	 	 	\item[to.theta] \verb!function(x) log(x)!
	 	 	\item[from.theta] \verb!function(x) exp(x)!
	 	 \end{description}
	 	\item[theta11]\ 
	 	 \begin{description}
	 	 	\item[hyperid] \verb!21011!
	 	 	\item[name] \verb!log precision common!
	 	 	\item[short.name] \verb!prec.common!
	 	 	\item[initial] \verb!0!
	 	 	\item[fixed] \verb!TRUE!
	 	 	\item[prior] \verb!loggamma!
	 	 	\item[param] \verb!1 5e-05!
	 	 	\item[to.theta] \verb!function(x) log(x)!
	 	 	\item[from.theta] \verb!function(x) exp(x)!
	 	 \end{description}
	 \end{description}
	\item[constr] \verb!FALSE!
	\item[nrow.ncol] \verb!FALSE!
	\item[augmented] \verb!FALSE!
	\item[aug.factor] \verb!1!
	\item[aug.constr] \verb!!
	\item[n.div.by] \verb!!
	\item[n.required] \verb!TRUE!
	\item[set.default.values] \verb!TRUE!
	\item[status] \verb!experimental!
	\item[pdf] \verb!generic3!
\end{description}


\subsection*{Example}
{\small\begin{verbatim}
make.sm = function(n, prob = 0.1)
{
    A = matrix(runif(n*n), n, n)
    Patt = matrix(runif(n*n), n, n)
    Patt[Patt > prob] = 0
    A = A * Patt
    diag(A) = runif(n)
    A = A %*% t(A)
    A = A / max(diag(A))
    return (inla.as.sparse(A))
}
    
nsim = 100
n = 5
m = 3
Cmat = list()
Q = inla.as.sparse(matrix(0, n, n))
for(i in 1:m) {
    Cmat[[i]] = make.sm(n)
    Q = Q + i*Cmat[[i]]
}

yy = inla.qsample(nsim, Q)
y = c(yy)
idx = rep(1:n, nsim)
r = rep(1:nsim, each = n)

r = inla(y ~ -1 +
    f(idx, model="generic3",
      Cmatrix = Cmat, replicate = r, 
      hyper = list(
          prec1 = list(initial = log(1)), 
          prec2 = list(initial = log(2)),
          prec3 = list(initial = log(3)))), 
    data = list(y=y, Cmat=Cmat, r=r),
    verbose=TRUE, 
    control.family = list(
        hyper = list(
            prec = list(
                initial = 10,
                fixed = TRUE))))
summary(r)
\end{verbatim}}

\subsection*{Notes}

\end{document}

%%% Local Variables: 
%%% TeX-master: t
%%% End: 
