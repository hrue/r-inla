\documentclass[a4paper,11pt]{article}
\usepackage[scale={0.8,0.9},centering,includeheadfoot]{geometry}
\usepackage{amstext}
\usepackage{listings}
\begin{document}

\section*{Autoregressive model of order $1$ (AR1)}

\subsection*{Parametrization}

The autoregressive model of order $1$ (AR1) for the Gaussian vector
$\mathbf{x}=(x_1,\dots,x_n)$ is defined as:
\begin{eqnarray}\nonumber
    x_1&\sim&\mathcal{N}(0,(\tau(1-\rho^2))^{-1}) \\\nonumber
    x_i&=&\rho\ x_{i-1}+\epsilon_i; \qquad \epsilon_i\sim\mathcal{N}(0,\tau^{-1}) \qquad  i=2,\dots,n
\end{eqnarray}
where
\[
|\rho|<1
\]

\subsection*{Hyperparameters}

The precision parameter $\kappa$ is represented as
\begin{displaymath}
    \theta_1 =\log(\kappa) 
\end{displaymath}
where $\kappa$ is the \emph{marginal} precision,
\begin{displaymath}
    \kappa = \tau (1-\rho^{2}).
\end{displaymath}
The parameter $\rho$ is represented as
\[
\theta_2 = \log\left(\frac{1+\rho}{1-\rho}\right)
\]
and the prior is defined on $\mathbf{\theta}=(\theta_1,\theta_2)$. 

\subsection*{Specification}

The AR1 model is specified inside the {\tt f()} function as
\begin{verbatim}
 f(<whatever>, model="ar1", values=<values>, hyper = <hyper>)
\end{verbatim}
The (optional) argument {\tt values } is a numeric or factor vector
giving the values assumed by the covariate for which we want the
effect to be estimated. See the example for RW1 for an application.

\subsubsection*{Hyperparameter spesification and default values}
%% DO NOT EDIT!
%% This file is generated automatically from models.R
\begin{description}
	\item[doc] \verb!Auto-regressive model of order 1 (AR(1))!
	\item[hyper]\ 
	 \begin{description}
	 	\item[theta1]\ 
	 	 \begin{description}
	 	 	\item[hyperid] \verb!14001!
	 	 	\item[name] \verb!log precision!
	 	 	\item[short.name] \verb!prec!
	 	 	\item[prior] \verb!loggamma!
	 	 	\item[param] \verb!1 5e-05!
	 	 	\item[initial] \verb!4!
	 	 	\item[fixed] \verb!FALSE!
	 	 	\item[to.theta] \verb!function(x) log(x)!
	 	 	\item[from.theta] \verb!function(x) exp(x)!
	 	 \end{description}
	 	\item[theta2]\ 
	 	 \begin{description}
	 	 	\item[hyperid] \verb!14002!
	 	 	\item[name] \verb!logit lag one correlation!
	 	 	\item[short.name] \verb!rho!
	 	 	\item[prior] \verb!normal!
	 	 	\item[param] \verb!0 0.15!
	 	 	\item[initial] \verb!2!
	 	 	\item[fixed] \verb!FALSE!
	 	 	\item[to.theta] \verb!function(x) log((1 + x) / (1 - x))!
	 	 	\item[from.theta] \verb!function(x) 2 * exp(x) / (1 + exp(x)) - 1!
	 	 \end{description}
	 	\item[theta3]\ 
	 	 \begin{description}
	 	 	\item[hyperid] \verb!14003!
	 	 	\item[name] \verb!mean!
	 	 	\item[short.name] \verb!mean!
	 	 	\item[prior] \verb!normal!
	 	 	\item[param] \verb!0 1!
	 	 	\item[initial] \verb!0!
	 	 	\item[fixed] \verb!TRUE!
	 	 	\item[to.theta] \verb!function(x) x!
	 	 	\item[from.theta] \verb!function(x) x!
	 	 \end{description}
	 \end{description}
	\item[constr] \verb!FALSE!
	\item[nrow.ncol] \verb!FALSE!
	\item[augmented] \verb!FALSE!
	\item[aug.factor] \verb!1!
	\item[aug.constr] \verb!!
	\item[n.div.by] \verb!!
	\item[n.required] \verb!FALSE!
	\item[set.default.values] \verb!FALSE!
	\item[pdf] \verb!ar1!
\end{description}


\subsection*{Example}

In this exaple we implement an ar1 model observed with Poisson counts
\begin{verbatim}
#simulate data
n = 100
rho = 0.8
prec = 10
## note that the marginal precision would be
marg.prec = prec * (1-rho^2)

E=sample(c(5,4,10,12),size=n,replace=T)
eta = as.vector(arima.sim(list(order = c(1,0,0), ar = rho), n = n,sd=sqrt(1/prec)))
y=rpois(n,E*exp(eta))
data = list(y=y, z=1:n, E=E)

## fit the model
formula = y~f(z,model="ar1")
result = inla(formula,family="poisson", data = data, E=E)
\end{verbatim}


\subsection*{Notes}

A third hyperparameter $\theta_3$ is \textbf{experimental}, and the
\emph{mean} of the AR1 process. By default this parameter is fixed to
the be zero.

\end{document}


% LocalWords: 

%%% Local Variables: 
%%% TeX-master: t
%%% End: 
