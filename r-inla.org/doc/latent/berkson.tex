\documentclass[a4paper,11pt]{article}
\usepackage[scale={0.8,0.9},centering,includeheadfoot]{geometry}
\usepackage{amstext}
\usepackage{listings}
\usepackage{verbatim}
\begin{document}

\section*{The Berkson model}

\subsection*{Parametrization}

This is an implementation of the Berkson error model for a fixed
effect. It is best described by an example, let  the model be
\begin{displaymath}
    y = \beta x + \epsilon{}
\end{displaymath}
where $y$ is the response, $\beta$ the effect of the true covariate
$x$ with zero mean Gaussian noise $\epsilon$. The issue is that $x$ is not observed
directly, but only through $x_{\text{obs}}$, where
\begin{displaymath}
    x_{\text{obs}} = x + \nu{}
\end{displaymath}
where $\nu$ is zero mean Gaussian noise. Even though this setup is
possible to implement using basic features ("copy" and multiple
likelihoods), we provide the following model which replaces the above,
\begin{displaymath}
    y = u + \epsilon{}
\end{displaymath}
where $u$ has the correct distribution depending on various parameters:
%% \begin{displaymath}
%%     u \;\sim\;{\mathcal N}\left(
%%       \beta \frac{\tau_{x} \mu_{x} I + \tau_{\text{obs}} x_{\text{obs}}}{
%%           \tau_{x}+\tau_{\text{obs}}}, 
%%      \frac{\tau_{x} + \tau_{\text{obs}}}{\beta^{2}} I \right).
%% \end{displaymath}
$\beta$ has prior $\pi(\beta)$, $x$ is apriori ${\mathcal N}(\mu_{x}
{I}, \tau_{x} {I})$, and $\tau_{\text{obs}}$ is the observation
precision for $x$ (ie $\text{Prec}(x_{\text{obs}}|x)$).\footnote{Note:
    The second argument in ${\mathcal N}(,)$ is the precision not the
    variance.}



\subsection*{Hyperparameters}

This model has 4 hyperparameters, $\theta = (\theta_{1}, \theta_{2},
\theta_{3}, \theta_{4})$ where $\theta_{2}$, $\theta_{3}$ and
$\theta_{4}$ are default set to be fixed (ie defined to be known). The
values of $\theta_{2}, \theta_{3}$ and $\theta_{4}$ are set to mimic a
classical fixed effect, so they will always make sense. To achieve the
Berkson measurement model, please use the appropriate choices for
(some of) these parameters!

The hyperparameter specification is as follows:
\begin{displaymath}
    \theta_{1} = \beta
\end{displaymath}
and the prior is defined on $\theta_{1}$,
\begin{displaymath}
    \theta_{2} = \log(\tau_{\text{obs}})
p\end{displaymath}
and the prior is defined on $\theta_{2}$,
\begin{displaymath}
    \theta_{3} = \mu_{x}
\end{displaymath}
and the prior is defined on $\theta_{3}$,
\begin{displaymath}
    \theta_{4} = \log(\tau_{\text{x}})
\end{displaymath}
and the prior is defined on $\theta_{4}$.


\subsection*{Specification}

The Berkson is specified inside the {\tt f()}
function as
\begin{verbatim}
 f(x.obs, [<weights>,] model="berkson", hyper = <hyper>)
\end{verbatim}
The \texttt{x.obs} are the observed values of the unknown covariates
$x$, with the \emph{assumption}, that if two or more elements of
\texttt{x.obs} are \emph{identical}, then they refer to the
\emph{same} element in the true covariate $x$.

\subsubsection*{Hyperparameter specification and default values}
\documentclass[a4paper,11pt]{article}
\usepackage[scale={0.8,0.9},centering,includeheadfoot]{geometry}
\usepackage{amstext}
\usepackage{listings}
\usepackage{verbatim}
\begin{document}

\section*{The Berkson model}

\subsection*{Parametrization}

This is an implementation of the Berkson error model for a fixed
effect. It is best described by an example, let  the model be
\begin{displaymath}
    y = \beta x + \epsilon{}
\end{displaymath}
where $y$ is the response, $\beta$ the effect of the true covariate
$x$ with zero mean Gaussian noise $\epsilon$. The issue is that $x$ is not observed
directly, but only through $x_{\text{obs}}$, where
\begin{displaymath}
    x_{\text{obs}} = x + \nu{}
\end{displaymath}
where $\nu$ is zero mean Gaussian noise. Even though this setup is
possible to implement using basic features ("copy" and multiple
likelihoods), we provide the following model which replaces the above,
\begin{displaymath}
    y = u + \epsilon{}
\end{displaymath}
where $u$ has the correct distribution depending on various parameters:
%% \begin{displaymath}
%%     u \;\sim\;{\mathcal N}\left(
%%       \beta \frac{\tau_{x} \mu_{x} I + \tau_{\text{obs}} x_{\text{obs}}}{
%%           \tau_{x}+\tau_{\text{obs}}}, 
%%      \frac{\tau_{x} + \tau_{\text{obs}}}{\beta^{2}} I \right).
%% \end{displaymath}
$\beta$ has prior $\pi(\beta)$, $x$ is apriori ${\mathcal N}(\mu_{x}
{I}, \tau_{x} {I})$, and $\tau_{\text{obs}}$ is the observation
precision for $x$ (ie $\text{Prec}(x_{\text{obs}}|x)$).\footnote{Note:
    The second argument in ${\mathcal N}(,)$ is the precision not the
    variance.}



\subsection*{Hyperparameters}

This model has 4 hyperparameters, $\theta = (\theta_{1}, \theta_{2},
\theta_{3}, \theta_{4})$ where $\theta_{2}$, $\theta_{3}$ and
$\theta_{4}$ are default set to be fixed (ie defined to be known). The
values of $\theta_{2}, \theta_{3}$ and $\theta_{4}$ are set to mimic a
classical fixed effect, so they will always make sense. To achieve the
Berkson measurement model, please use the appropriate choices for
(some of) these parameters!

The hyperparameter specification is as follows:
\begin{displaymath}
    \theta_{1} = \beta
\end{displaymath}
and the prior is defined on $\theta_{1}$,
\begin{displaymath}
    \theta_{2} = \log(\tau_{\text{obs}})
p\end{displaymath}
and the prior is defined on $\theta_{2}$,
\begin{displaymath}
    \theta_{3} = \mu_{x}
\end{displaymath}
and the prior is defined on $\theta_{3}$,
\begin{displaymath}
    \theta_{4} = \log(\tau_{\text{x}})
\end{displaymath}
and the prior is defined on $\theta_{4}$.


\subsection*{Specification}

The Berkson is specified inside the {\tt f()}
function as
\begin{verbatim}
 f(x.obs, [<weights>,] model="berkson", hyper = <hyper>)
\end{verbatim}
The \texttt{x.obs} are the observed values of the unknown covariates
$x$, with the \emph{assumption}, that if two or more elements of
\texttt{x.obs} are \emph{identical}, then they refer to the
\emph{same} element in the true covariate $x$.

\subsubsection*{Hyperparameter specification and default values}
\documentclass[a4paper,11pt]{article}
\usepackage[scale={0.8,0.9},centering,includeheadfoot]{geometry}
\usepackage{amstext}
\usepackage{listings}
\usepackage{verbatim}
\begin{document}

\section*{The Berkson model}

\subsection*{Parametrization}

This is an implementation of the Berkson error model for a fixed
effect. It is best described by an example, let  the model be
\begin{displaymath}
    y = \beta x + \epsilon{}
\end{displaymath}
where $y$ is the response, $\beta$ the effect of the true covariate
$x$ with zero mean Gaussian noise $\epsilon$. The issue is that $x$ is not observed
directly, but only through $x_{\text{obs}}$, where
\begin{displaymath}
    x_{\text{obs}} = x + \nu{}
\end{displaymath}
where $\nu$ is zero mean Gaussian noise. Even though this setup is
possible to implement using basic features ("copy" and multiple
likelihoods), we provide the following model which replaces the above,
\begin{displaymath}
    y = u + \epsilon{}
\end{displaymath}
where $u$ has the correct distribution depending on various parameters:
%% \begin{displaymath}
%%     u \;\sim\;{\mathcal N}\left(
%%       \beta \frac{\tau_{x} \mu_{x} I + \tau_{\text{obs}} x_{\text{obs}}}{
%%           \tau_{x}+\tau_{\text{obs}}}, 
%%      \frac{\tau_{x} + \tau_{\text{obs}}}{\beta^{2}} I \right).
%% \end{displaymath}
$\beta$ has prior $\pi(\beta)$, $x$ is apriori ${\mathcal N}(\mu_{x}
{I}, \tau_{x} {I})$, and $\tau_{\text{obs}}$ is the observation
precision for $x$ (ie $\text{Prec}(x_{\text{obs}}|x)$).\footnote{Note:
    The second argument in ${\mathcal N}(,)$ is the precision not the
    variance.}



\subsection*{Hyperparameters}

This model has 4 hyperparameters, $\theta = (\theta_{1}, \theta_{2},
\theta_{3}, \theta_{4})$ where $\theta_{2}$, $\theta_{3}$ and
$\theta_{4}$ are default set to be fixed (ie defined to be known). The
values of $\theta_{2}, \theta_{3}$ and $\theta_{4}$ are set to mimic a
classical fixed effect, so they will always make sense. To achieve the
Berkson measurement model, please use the appropriate choices for
(some of) these parameters!

The hyperparameter specification is as follows:
\begin{displaymath}
    \theta_{1} = \beta
\end{displaymath}
and the prior is defined on $\theta_{1}$,
\begin{displaymath}
    \theta_{2} = \log(\tau_{\text{obs}})
p\end{displaymath}
and the prior is defined on $\theta_{2}$,
\begin{displaymath}
    \theta_{3} = \mu_{x}
\end{displaymath}
and the prior is defined on $\theta_{3}$,
\begin{displaymath}
    \theta_{4} = \log(\tau_{\text{x}})
\end{displaymath}
and the prior is defined on $\theta_{4}$.


\subsection*{Specification}

The Berkson is specified inside the {\tt f()}
function as
\begin{verbatim}
 f(x.obs, [<weights>,] model="berkson", hyper = <hyper>)
\end{verbatim}
The \texttt{x.obs} are the observed values of the unknown covariates
$x$, with the \emph{assumption}, that if two or more elements of
\texttt{x.obs} are \emph{identical}, then they refer to the
\emph{same} element in the true covariate $x$.

\subsubsection*{Hyperparameter specification and default values}
\documentclass[a4paper,11pt]{article}
\usepackage[scale={0.8,0.9},centering,includeheadfoot]{geometry}
\usepackage{amstext}
\usepackage{listings}
\usepackage{verbatim}
\begin{document}

\section*{The Berkson model}

\subsection*{Parametrization}

This is an implementation of the Berkson error model for a fixed
effect. It is best described by an example, let  the model be
\begin{displaymath}
    y = \beta x + \epsilon{}
\end{displaymath}
where $y$ is the response, $\beta$ the effect of the true covariate
$x$ with zero mean Gaussian noise $\epsilon$. The issue is that $x$ is not observed
directly, but only through $x_{\text{obs}}$, where
\begin{displaymath}
    x_{\text{obs}} = x + \nu{}
\end{displaymath}
where $\nu$ is zero mean Gaussian noise. Even though this setup is
possible to implement using basic features ("copy" and multiple
likelihoods), we provide the following model which replaces the above,
\begin{displaymath}
    y = u + \epsilon{}
\end{displaymath}
where $u$ has the correct distribution depending on various parameters:
%% \begin{displaymath}
%%     u \;\sim\;{\mathcal N}\left(
%%       \beta \frac{\tau_{x} \mu_{x} I + \tau_{\text{obs}} x_{\text{obs}}}{
%%           \tau_{x}+\tau_{\text{obs}}}, 
%%      \frac{\tau_{x} + \tau_{\text{obs}}}{\beta^{2}} I \right).
%% \end{displaymath}
$\beta$ has prior $\pi(\beta)$, $x$ is apriori ${\mathcal N}(\mu_{x}
{I}, \tau_{x} {I})$, and $\tau_{\text{obs}}$ is the observation
precision for $x$ (ie $\text{Prec}(x_{\text{obs}}|x)$).\footnote{Note:
    The second argument in ${\mathcal N}(,)$ is the precision not the
    variance.}



\subsection*{Hyperparameters}

This model has 4 hyperparameters, $\theta = (\theta_{1}, \theta_{2},
\theta_{3}, \theta_{4})$ where $\theta_{2}$, $\theta_{3}$ and
$\theta_{4}$ are default set to be fixed (ie defined to be known). The
values of $\theta_{2}, \theta_{3}$ and $\theta_{4}$ are set to mimic a
classical fixed effect, so they will always make sense. To achieve the
Berkson measurement model, please use the appropriate choices for
(some of) these parameters!

The hyperparameter specification is as follows:
\begin{displaymath}
    \theta_{1} = \beta
\end{displaymath}
and the prior is defined on $\theta_{1}$,
\begin{displaymath}
    \theta_{2} = \log(\tau_{\text{obs}})
p\end{displaymath}
and the prior is defined on $\theta_{2}$,
\begin{displaymath}
    \theta_{3} = \mu_{x}
\end{displaymath}
and the prior is defined on $\theta_{3}$,
\begin{displaymath}
    \theta_{4} = \log(\tau_{\text{x}})
\end{displaymath}
and the prior is defined on $\theta_{4}$.


\subsection*{Specification}

The Berkson is specified inside the {\tt f()}
function as
\begin{verbatim}
 f(x.obs, [<weights>,] model="berkson", hyper = <hyper>)
\end{verbatim}
The \texttt{x.obs} are the observed values of the unknown covariates
$x$, with the \emph{assumption}, that if two or more elements of
\texttt{x.obs} are \emph{identical}, then they refer to the
\emph{same} element in the true covariate $x$.

\subsubsection*{Hyperparameter specification and default values}
\input{../hyper/latent/berkson.tex}

\subsection*{Example}

\verbatiminput{berkson-example.R}

\subsection*{Notes}

\begin{itemize}
\item \texttt{INLA} provide the posterior of $u$ and NOT $x$.
\item The posterior of $u$ comes in the order given by the sorted
    (from low to high) values of \texttt{x.obs}. The entry \verb|$ID|
    gives the mapping.
\end{itemize}


\end{document}



%%% Local Variables: 
%%% TeX-master: t
%%% End: 

% LocalWords:  Berkson Parametrization covariate apriori ie Prec hyperparameter
% LocalWords:  Hyperparameters hyperparameters covariates berkson INLA


\subsection*{Example}

\verbatiminput{berkson-example.R}

\subsection*{Notes}

\begin{itemize}
\item \texttt{INLA} provide the posterior of $u$ and NOT $x$.
\item The posterior of $u$ comes in the order given by the sorted
    (from low to high) values of \texttt{x.obs}. The entry \verb|$ID|
    gives the mapping.
\end{itemize}


\end{document}



%%% Local Variables: 
%%% TeX-master: t
%%% End: 

% LocalWords:  Berkson Parametrization covariate apriori ie Prec hyperparameter
% LocalWords:  Hyperparameters hyperparameters covariates berkson INLA


\subsection*{Example}

\verbatiminput{berkson-example.R}

\subsection*{Notes}

\begin{itemize}
\item \texttt{INLA} provide the posterior of $u$ and NOT $x$.
\item The posterior of $u$ comes in the order given by the sorted
    (from low to high) values of \texttt{x.obs}. The entry \verb|$ID|
    gives the mapping.
\end{itemize}


\end{document}



%%% Local Variables: 
%%% TeX-master: t
%%% End: 

% LocalWords:  Berkson Parametrization covariate apriori ie Prec hyperparameter
% LocalWords:  Hyperparameters hyperparameters covariates berkson INLA


\subsection*{Example}

\verbatiminput{berkson-example.R}

\subsection*{Notes}

\begin{itemize}
\item \texttt{INLA} provide the posterior of $u$ and NOT $x$.
\item The posterior of $u$ comes in the order given by the sorted
    (from low to high) values of \texttt{x.obs}. The entry \verb|$ID|
    gives the mapping.
\end{itemize}


\end{document}



%%% Local Variables: 
%%% TeX-master: t
%%% End: 

% LocalWords:  Berkson Parametrization covariate apriori ie Prec hyperparameter
% LocalWords:  Hyperparameters hyperparameters covariates berkson INLA
