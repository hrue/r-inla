\documentclass[a4paper,11pt]{article}
\usepackage[scale={0.8,0.9},centering,includeheadfoot]{geometry}
\usepackage{amstext}
\usepackage{amsmath}
\usepackage{listings}
\usepackage{verbatim}
\begin{document}


\section*{Model ``rgeneric''}


This is a class of generic models allows the user to define latent
(sub-)models in \texttt{R}, for cases where the requested model is not
yet implemented in \texttt{INLA}, and do the Bayesian inference using
\texttt{INLA}. 
%% If you think the (missing) model would be of benefite for others as
%% well, please request it to be added into INLA.
The actual implementation of this feature is somewhat technical as all
the computations are done in a standalone C-program and not within
\texttt{R}, so therefore this option is \emph{only} available for
Linux and MacOSX, and \emph{not} available for Windows. Also you need
to run it on a dual-core (or more cores) computer and have installed
the \texttt{multicore}-library. It will run more slowly for two
reasons:
\begin{enumerate}
\item The C-program doing the calculations has to communicate with
    \texttt{R} to get model-properties such as the precision matrix,
    the graph, etc, and this communication is not for free.
\item Due the same reason, the C-program doing the calculations has to
    run in serial mode and cannot make use of multi-cores.
\end{enumerate}
Anyway, we still hope it would be useful.

\section*{Defining a latent model in \texttt{R}}

The use of this feature, is in short the following. First we pass our
definition of the model \texttt{model.def}, to define a
\texttt{inla-rgeneric} object,
\begin{verbatim}
    my.rgeneric.model = inla.rgeneric.define(my.model.def, <args>)
\end{verbatim}
Here \texttt{args} could be parameters to \texttt{my.model.def}
such as dimension, prior-settings, etc, so \texttt{my.model.def} can
be written more generically.  Then we can use this object as a latent
model
\begin{verbatim}
    y ~ ... + f(i, model="rgeneric", rgeneric = my.rgeneric.model)
\end{verbatim}

\section*{Example: the AR1 model}

The \texttt{my.model.def} needs to follow some rules and provide the
required features. It is easier with an example, so let us implement
the AR1-model (see also \texttt{inla.doc("ar1")}), which is defined
as\footnote{The second argument in ${\mathcal N}(,)$ is the precision
    matrix not the covariance matrix.}
\begin{displaymath}
    x_{1} \;\sim\; {\mathcal N}(0, \tau)
\end{displaymath}
and
\begin{displaymath}
    x_{t} \mid x_{1}, \ldots, x_{t-1} \;\sim\; {\mathcal N}(\rho
    x_{t-1}, \tau_{I}), \qquad t=2, \ldots,n.
\end{displaymath}
Although our scale-parameter is the marginal precision $\tau$ (and
there are good good reasons for this!), the joint is more naturally
expressed using the innovation precision $\tau_{I} =
\tau/(1-\rho^{2})$. The joint density of $x$ is Gaussian
\begin{displaymath}
    \pi(x|\rho,\tau) = \left(\frac{1}{\sqrt{2\pi}}\right)^{n}
    \tau_{I}^{n/2} (1-\rho^{2})^{1/2}
    \exp\left(-\frac{\tau_{I}}{2} x^{T} Q x\right)
\end{displaymath}
where the precision matrix is
\begin{displaymath}
    Q =
    \begin{bmatrix}
        1 & -\rho &&&& \\
        -\rho & 1+\rho^{2}& -\rho &&& \\
        &-\rho & 1+\rho^{2}& -\rho && \\
        && \ddots& \ddots& \ddots& \\
        &       &       &       -\rho & 1 + \rho^{2} & -\rho\\
        &       &       &       & -\rho & 1
    \end{bmatrix}
\end{displaymath}
There are two (hyper-)parameters for this model, it is the marginal
precision $\tau$ and the lag-one correlation $\rho$.  We
reparameterise these as
\begin{displaymath}
    \tau = \exp(\theta_1) 
\end{displaymath}
and
\begin{displaymath}
    \rho = 2\frac{\exp(\theta_{2})}{1+\exp(\theta_{2})} - 1
\end{displaymath}
It is required that the parameters $\theta = (\theta_{1}, \theta_{2})$
have support on $\Re$ and the priors for $\tau$ and $\rho$ are given
as the corresponding priors for $\theta_{1}$ and $\theta_{2}$. We
assign a (Gamma) $\Gamma(.; a,b)$ prior (with mean $a/b$ and variance
$a/b^{2}$) for $\tau$ and a Gaussian prior ${\mathcal N}(\mu,\kappa)$
for $\theta_{2}$, so the joint prior for $\theta$ becomes
\begin{displaymath}
    \pi(\theta) = \Gamma(\exp(\theta_1); a,b) \exp(\theta_1) \times
    {\mathcal N}(\theta_{2}; \mu, \kappa).
\end{displaymath}
We will use $a=b=1$, $\mu=0$ and $\kappa=1$.

In order to define the AR1-model, we need to make functions that
returns
\begin{itemize}
\item the precision matrix $Q(\theta)$,
\item the graph,
\item the initial values of $\theta$,
\item the (log-)normalising constant, and
\item the (log-)prior
\end{itemize}
which except for the graph, depends on the current value of
$\theta$. We need to wrap this into a common function, which process
the request from the C-program. Note the 
list of commands and its names
\begin{verbatim}
        cmd = c("Q", "graph", "initial", "log.norm.const", "log.prior", "quit"),
\end{verbatim}
are fixed and predefined (in the file \texttt{inla.h}).  Our skeleton
for defining a model is as follows.
\begin{verbatim}
`inla.rgeneric.ar1.model` = function(
        cmd = c("graph", "Q", "initial", "log.norm.const", "log.prior", "quit"),
        theta = NULL,
        args = NULL)
{
    graph = function(n, ntheta, theta){ <to be completed> }
    Q = function(n, ntheta, theta) { <to be completed> }
    log.norm.const = function(n, ntheta, theta) { <to be completed> }
    log.prior = function(n, ntheta, theta) { <to be completed> }
    initial = function(n, ntheta, theta) { <to be completed> }
    quit = function(n, ntheta, theta) { <to be completed> }

    cmd = match.arg(cmd)
    val = do.call(cmd, args = list(
                               n = as.integer(args$n),
                               ntheta = as.integer(args$ntheta), 
                               theta = theta))
    return (val)
}
\end{verbatim}
The input parameters are
\begin{description}
\item[cmd] What to return or process
\item[theta] The values of the $\theta$-parameters to evaluate at
\item[args] User-defined argument(s) from the
    \texttt{inla.rgeneric.define()}-call. In this example we use
    \texttt{n} and \texttt{ntheta} as the user-defined arguments and
    pass them to every function, so we need to pass \texttt{n} and
    \texttt{ntheta} when defining the model
\begin{verbatim}
        model = inla.rgeneric.define(inla.rgeneric.ar1.model,
                                     n = 100, ntheta = 2)
\end{verbatim}
    for example, but you may chose to do this differently.
\end{description}
Our next task, is the ``fill in the blanks'', ie to define the
functions \texttt{graph, Q, initial, log.norm.const, log.prior} and
\texttt{quit} for our AR1-model.

\subsection*{Function \texttt{graph}}

This function must return a matrix, (must) preferably a sparseMatrix
(for efficiency reasons), with non-zero entries defining the
graph. Only the lower-triangular part of the matrix is used.
\begin{verbatim}
    graph = function(n, ntheta, theta)
    {
        ## return the graph of the model. the values of Q is only
        ## interpreted as zero or non-zero.
        if (FALSE) {
            ## slow and easy: dense-matrices
            G = toeplitz(c(1, 1, rep(0, n-2L)))
            ## you *can* do this to avoid transfering zeros, but then
            ## its better to use the sparseMatrix()-approach below.
            G = inla.as.dgTMatrix(G)
        } else {
            ## fast and little more complicated: sparse matrices. we
            ## only need to define the lower-triangular of G!
            i = c(
                    ## diagonal
                    1L, n, 2L:(n-1L),
                    ## off-diagonal
                    1L:(n-1L))
            j = c(
                    ## diagonal
                    1L, n, 2L:(n-1L),
                    ## off-diagonal
                    2L:n)
            x = 1 ## meaning that all are 1
            G = sparseMatrix(i=i, j=j, x=x, giveCsparse = FALSE)
        }            
        return (G)
    }
\end{verbatim}

\subsection*{Function \texttt{Q}}

This function must return the precision matrix $Q(\theta)$, (must)
preferably a sparseMatrix (for efficiency reasons). Only the
lower-triangular part of the matrix is used. We will make use of the
helper function
\begin{verbatim}
    interpret.theta = function(n, ntheta, theta)
    {
        ## internal helper-function to map the parameters from the
        ## internal-scale to the user-scale
        return (list(prec = exp(theta[1L]),
                     rho = 2*exp(theta[2L])/(1+exp(theta[2L])) - 1.0))
    }
\end{verbatim}
to convert from $\theta_{1}$ to $\tau$, and from $\theta_{2}$ to
$\rho$.  The \texttt{Q}-function can then be implemented as follows.
\begin{verbatim}
    Q = function(n, ntheta, theta)
    {
        ## returns the precision matrix for given parameters
        param = interpret.theta(n, ntheta, theta)
        if (FALSE) {
            ## slow and easy: dense-matrices
            Q = param$prec/(1-param$rho^2) *
                toeplitz(c(1+param$rho^2, -param$rho, rep(0, n-2L)))
            ## first and last diagonal term is 1*marg.prec
            Q[1, 1] = Q[n, n] = param$prec/(1-param$rho^2)
            ## you *can* do this to avoid transfering zeros, but then
            ## its better to use the sparseMatrix()-approach below.
            Q = inla.as.dgTMatrix(Q)
        } else {
            ## fast and a little more complicated: sparse matrices. we
            ## only need to define the lower-triangular G!
            i = c(
                    ## diagonal
                    1L, n, 2L:(n-1L),
                    ## off-diagonal
                    1L:(n-1L))
            j = c(
                    ## diagonal
                    1L, n, 2L:(n-1L),
                    ## off-diagonal
                    2L:n)
            x = param$prec/(1-param$rho^2) * c(
                    ## diagonal
                    1L, 1L, rep(1+param$rho^2, n-2L),
                    ## off-diagonal
                    rep(-param$rho, n-1L))
            Q = sparseMatrix(i=i, j=j, x=x, giveCsparse=FALSE)
        }            
        return (Q)
    }
\end{verbatim}

\subsection*{Function \texttt{log.norm.const}}

This function must return the log of the normalising constant. For the 
 AR1-model the normalising constant is 
\begin{displaymath}
    \left(\frac{1}{\sqrt{2\pi}}\right)^{n}
    \tau_{I}^{n/2} (1-\rho^{2})^{1/2}
\end{displaymath}
where
\begin{displaymath}
    \tau_{I} = \tau/(1-\rho^{2}).
\end{displaymath}
The function can then be implemented as
\begin{verbatim}
    log.norm.const = function(n, ntheta, theta)
    {
        ## return the log(normalising constant) for the model.
        param = interpret.theta(n, ntheta, theta)
        prec.innovation  = param$prec / (1.0 - param$rho^2)
        val = n * (- 0.5 * log(2*pi) + 0.5 * log(prec.innovation)) +
              0.5 * log(1.0 - param$rho^2)
        return (val)
    }
\end{verbatim}

\subsection*{Function \texttt{log.prior}}

This function must return the (log-)prior of the prior density for
$\theta$. For the AR1-model, we have chosen this prior
\begin{displaymath}
    \pi(\theta) = \Gamma(\exp(\theta_1); a,b) \exp(\theta_1) \times
    {\mathcal N}(\theta_{2}; \mu, \kappa)
\end{displaymath}
so we can implement this as with our choices $a=b=1$, $\mu=0$ and
$\kappa=1$:
\begin{verbatim}
    log.prior = function(n, ntheta, theta)
    {
        ## return the log-prior for the hyperparameters. the
        ## '+theta[1L]' is the log(Jacobian) for having a gamma prior
        ## on the precision and convert it into the prior for the
        ## log(precision).
        param = interpret.theta(n, ntheta, theta)
        val = (dgamma(param$prec, shape = 1, rate = 1, log=TRUE) + theta[1L] + 
               dnorm(theta[2L], mean = 0, sd = 1, log=TRUE))
        return (val)
    }
\end{verbatim}

\subsection*{Function \texttt{initial}}

This function returns the initial values for $\theta$.
\begin{verbatim}
    initial = function(n, ntheta, theta)
    {
        ## return the initial values
        return (rep(1, ntheta))
    }
\end{verbatim}

\subsection*{Function \texttt{quit}}

This function is called when all the computations are done and before
exit-ing the C-program. Usually, there is nothing in particular to do,
but if there is something that should be done, you can do this here.
\begin{verbatim}
    quit = function(n, ntheta, theta)
    {
        return ()
    }
\end{verbatim}

\section*{The complete definition of the AR1-model}

For completeness, we include here the complete code for the AR1-model,
collecting all the functions already defined. The function is
predefined in the \texttt{INLA}-library.
\begin{verbatim}
`inla.rgeneric.ar1.model` = function(
        cmd = c("graph", "Q", "initial", "log.norm.const", "log.prior", "quit"),
        theta = NULL,
        args = NULL)
{
    ## this is an example of the 'rgeneric' model. here we implement
    ## the AR-1 model as described in inla.doc("ar1"), where 'rho' is
    ## the lag-1 correlation and 'prec' is the *marginal* (not
    ## conditional) precision.

    interpret.theta = function(n, ntheta, theta)
    {
        ## internal helper-function to map the parameters from the
        ## internal-scale to the user-scale
        return (list(prec = exp(theta[1L]),
                     rho = 2*exp(theta[2L])/(1+exp(theta[2L])) - 1.0))
    }

    graph = function(n, ntheta, theta)
    {
        ## return the graph of the model. the values of Q is only
        ## interpreted as zero or non-zero.
        if (FALSE) {
            ## slow and easy: dense-matrices
            G = toeplitz(c(1, 1, rep(0, n-2L)))
            ## you *can* do this to avoid transfering zeros, but then
            ## its better to use the sparseMatrix()-approach below.
            G = inla.as.dgTMatrix(G)
        } else {
            ## fast and little more complicated: sparse matrices. we
            ## only need to define the lower-triangular of G!
            i = c(
                    ## diagonal
                    1L, n, 2L:(n-1L),
                    ## off-diagonal
                    1L:(n-1L))
            j = c(
                    ## diagonal
                    1L, n, 2L:(n-1L),
                    ## off-diagonal
                    2L:n)
            x = 1 ## meaning that all are 1
            G = sparseMatrix(i=i, j=j, x=x, giveCsparse = FALSE)
        }            
        return (G)
    }

    Q = function(n, ntheta, theta)
    {
        ## returns the precision matrix for given parameters
        param = interpret.theta(n, ntheta, theta)
        if (FALSE) {
            ## slow and easy: dense-matrices
            Q = param$prec/(1-param$rho^2) *
                toeplitz(c(1+param$rho^2, -param$rho, rep(0, n-2L)))
            ## first and last diagonal term is 1*marg.prec
            Q[1, 1] = Q[n, n] = param$prec/(1-param$rho^2)
            ## you *can* do this to avoid transfering zeros, but then
            ## its better to use the sparseMatrix()-approach below.
            Q = inla.as.dgTMatrix(Q)
        } else {
            ## fast and a little more complicated: sparse matrices. we
            ## only need to define the lower-triangular G!
            i = c(
                    ## diagonal
                    1L, n, 2L:(n-1L),
                    ## off-diagonal
                    1L:(n-1L))
            j = c(
                    ## diagonal
                    1L, n, 2L:(n-1L),
                    ## off-diagonal
                    2L:n)
            x = param$prec/(1-param$rho^2) * c(
                    ## diagonal
                    1L, 1L, rep(1+param$rho^2, n-2L),
                    ## off-diagonal
                    rep(-param$rho, n-1L))
            Q = sparseMatrix(i=i, j=j, x=x, giveCsparse=FALSE)
        }            
        return (Q)
    }

    log.norm.const = function(n, ntheta, theta)
    {
        ## return the log(normalising constant) for the model.
        param = interpret.theta(n, ntheta, theta)
        prec.innovation  = param$prec / (1.0 - param$rho^2)
        val = n * (- 0.5 * log(2*pi) + 0.5 * log(prec.innovation)) +
              0.5 * log(1.0 - param$rho^2)
        return (val)
    }

    log.prior = function(n, ntheta, theta)
    {
        ## return the log-prior for the hyperparameters. the
        ## '+theta[1L]' is the log(Jacobian) for having a gamma prior
        ## on the precision and convert it into the prior for the
        ## log(precision).
        param = interpret.theta(n, ntheta, theta)
        val = (dgamma(param$prec, shape = 1, rate = 1, log=TRUE) + theta[1L] + 
               dnorm(theta[2L], mean = 0, sd = 1, log=TRUE))
        return (val)
    }

    initial = function(n, ntheta, theta)
    {
        ## return the initial values
        return (rep(1, ntheta))
    }

    quit = function(n, ntheta, theta)
    {
        return (invisible())
    }

    cmd = match.arg(cmd)
    val = do.call(cmd, args = list(
                               n = as.integer(args$n),
                               ntheta = as.integer(args$ntheta), 
                               theta = theta))
    return (val)
}
\end{verbatim}

\section*{Example of usage}

\verbatiminput{rgeneric-example.R}


\end{document}

% LocalWords:  rgeneric INLA MacOSX multi inla args ar covariance cmd ntheta ie

%%% Local Variables: 
%%% TeX-master: t
%%% End: 
% LocalWords:  reparameterise sparseMatrix ing
