\documentclass[a4paper,11pt]{article}
\usepackage[scale={0.8,0.9},centering,includeheadfoot]{geometry}
\usepackage{amstext}
\usepackage{amsmath}
\usepackage{listings}
\usepackage{verbatim}
\begin{document}


\section*{Model \textbf{rgeneric}}

This is a class of generic models allows the user to define latent
model-component in \texttt{R}, for cases where the requested model is
not yet implemented in \texttt{INLA}, and do the Bayesian inference
using \texttt{INLA}. It will run slower as the model properties has to
be evaluated in \texttt{R} within a \texttt{C}-program.

\section*{Defining a latent model in \texttt{R}}

The use of this feature, is in short the following. First we pass our
definition of the model \texttt{rmodel}, to define a
\texttt{inla-rgeneric} object,
{\small
\begin{verbatim}
    model = inla.rgeneric.define(rmodel, debug, ...)
\end{verbatim}
} Here, \texttt{rmodel} is model definition encoded as an
\texttt{R}-function, \texttt{debug} is a logical parameter if debug
information should be printed, and \texttt{...} are further named
variables that goes into the environment of \texttt{rmodel} (like
dimension, prior-settings, etc). Then the model can used as {\small
\begin{verbatim}
    y ~ ... + f(i, model=model)
\end{verbatim}
}

\section*{Example: the AR1 model}

The \texttt{rmodel} needs to follow some rules to provide the required
features. As an example, we will show how to implement the AR1-model,
see \texttt{inla.doc("ar1")}). This model is defined as\footnote{The
    second argument in ${\mathcal N}(,)$ is the precision not the
    covariance.}
\begin{displaymath}
    x_{1} \;\sim\; {\mathcal N}(0, \tau)
\end{displaymath}
and
\begin{displaymath}
    x_{t} \mid x_{1}, \ldots, x_{t-1} \;\sim\; {\mathcal N}(\rho
    x_{t-1}, \tau_{I}), \qquad t=2, \ldots,n.
\end{displaymath}
The scale-parameter is the marginal precision $\tau$, but the joint
density is more naturally expressed using the innovation precision
$\tau_{I} = \tau/(1-\rho^{2})$. The joint density of $x$ is Gaussian
\begin{displaymath}
    \pi(x|\rho,\tau) = \left(\frac{1}{\sqrt{2\pi}}\right)^{n}
    \tau_{I}^{n/2} (1-\rho^{2})^{1/2}
    \exp\left(-\frac{\tau_{I}}{2} x^{T} R x\right)
\end{displaymath}
where the precision-matrix is
\begin{displaymath}
    Q = \tau_I R = 
    \tau_I
    \begin{bmatrix}
        1 & -\rho &&&& \\
        -\rho & 1+\rho^{2}& -\rho &&& \\
        &-\rho & 1+\rho^{2}& -\rho && \\
        && \ddots& \ddots& \ddots& \\
        &       &       &       -\rho & 1 + \rho^{2} & -\rho\\
        & & & & -\rho & 1
    \end{bmatrix}
\end{displaymath}
There are two (hyper-)parameters for this model, it is the marginal
precision $\tau$ and the lag-one correlation $\rho$.  We
reparameterise these as
\begin{displaymath}
    \tau = \exp(\theta_1) 
\end{displaymath}
and
\begin{displaymath}
    \rho = 2\frac{\exp(\theta_{2})}{1+\exp(\theta_{2})} - 1
\end{displaymath}
It is required that the parameters $\theta = (\theta_{1}, \theta_{2})$
have support on $\Re$ and the priors for $\tau$ and $\rho$ are given
as the corresponding priors for $\theta_{1}$ and $\theta_{2}$.
\textbf{IMPORTANT: A good re-parametersation is required for INLA to
    work well. A good parmeterisation makes, ideally, the ``Fisher
    information matrix'' of $\theta$ constant (wrt to $\theta$). It is
    sufficient to check this in a frequentistic setting with data
    directly from the AR$(1)$ model, in this case.} Note that
\texttt{INLA} only provide the marginal posteriors for $\theta$, but
you can use \texttt{inla.tmarginal} to convert it to the appropriate
marginals for $\rho$ and $\tau$.

We assign a (Gamma) $\Gamma(.; a,b)$ prior (with mean $a/b$ and
variance $a/b^{2}$) for $\tau$ and a Gaussian prior
${\mathcal N}(\mu,\kappa)$ for $\theta_{2}$, so the joint prior for
$\theta$ becomes
\begin{displaymath}
    \pi(\theta) = \Gamma(\exp(\theta_1); a,b) \exp(\theta_1) \times
    {\mathcal N}(\theta_{2}; \mu, \kappa).
\end{displaymath}
We will use $a=b=1$, $\mu=0$ and $\kappa=1$.

In order to define the AR1-model, we need to make functions that
returns
\begin{itemize}
\item the graph,
\item the precision matrix $Q(\theta)$,
\item the zero mean,
\item the initial values of $\theta$,
\item the log-normalising constant, and
\item the log-prior
\end{itemize}
which except for the graph, depends on the current value of
$\theta$. We need to wrap this into a common function, which process
the request from the C-program. The list of commands and its names
{\small
\begin{verbatim}
        cmd = c("Q", "graph", "mu", "initial", "log.norm.const",
                "log.prior", "quit"),
\end{verbatim}}
are fixed.  The skeleton-function for defining a model is 
{\small
\begin{verbatim}
`inla.rgeneric.ar1.model` = function(
        cmd = c("graph", "Q", "mu", "initial", "log.norm.const",
                "log.prior", "quit"),
        theta = NULL)
{
    graph = function(){ <to be completed> }
    Q = function() { <to be completed> }
    mu = function() { <to be completed> }
    log.norm.const = function() { <to be completed> }
    log.prior = function() { <to be completed> }
    initial = function() { <to be completed> }
    quit = function() { <to be completed> }

    val = do.call(match.arg(cmd), args = list())
    return (val)
}
\end{verbatim}
}
The input parameters are
\begin{description}
\item[cmd] What to return
\item[theta] The values of the $\theta$-parameters
\end{description}
Other parameters in the model definition, like $n$ and possibly the
parameters of the prior, goes into the ``...'' part of
\texttt{inla.rgeneric.define}, like
\begin{verbatim}
        model = inla.rgeneric.define(inla.rgeneric.ar1.model, n = 100)
\end{verbatim}
and is assigned in the environment to
\texttt{inla.rgeneric.ar1.model}. This is because several instances of
\texttt{rgeneric} models will share the same \texttt{.GlobalEnv}.

Our next task, is the ``fill in the blanks'' in this function.

\subsection*{Function \texttt{graph}}

This function must return a sparseMatrix, with the non-zero elements
of the precision matrix. Only the lower-triangular part of the matrix
is used. 
{\small
\begin{verbatim}
    graph = function()
    {
        ## return the graph of the model. the values of Q is only interpreted as zero or
        ## non-zero. return a sparse.matrix
        if (FALSE) {
            ## slow and easy: dense-matrices
            G = toeplitz(c(1, 1, rep(0, n-2L)))
            G = inla.as.sparse(G)
        } else {
            ## faster. we only need to define the lower-triangular of G
            i = c(
                ## diagonal
                1L, n, 2L:(n-1L),
                ## off-diagonal
                1L:(n-1L))
            j = c(
                ## diagonal
                1L, n, 2L:(n-1L),
                ## off-diagonal
                2L:n)
            x = 1 ## meaning that all are 1
            G = sparseMatrix(i=i, j=j, x=x, giveCsparse = FALSE)
        }            
        return (G)
    }
\end{verbatim}
}

\subsection*{Function \texttt{Q}}

This function must return the precision matrix $Q(\theta)$, and 
must be a sparseMatrix. Only the
lower-triangular part of the matrix is used. We will make use of the
helper function
{\small
\begin{verbatim}
    interpret.theta = function()
    {
        ## internal helper-function to map the parameters from the internal-scale to the
        ## user-scale
        return (list(prec = exp(theta[1L]),
                     rho = 2*exp(theta[2L])/(1+exp(theta[2L])) - 1.0))
    }
\end{verbatim}
}
to convert from $\theta_{1}$ to $\tau$, and from $\theta_{2}$ to
$\rho$.  The \texttt{Q}-function can then be implemented as follows.
{\small
\begin{verbatim}
    Q = function()
    {
        ## returns the precision matrix for given parameters
        param = interpret.theta()
        if (FALSE) {
            ## slow and easy: dense-matrices
            Q = param$prec/(1-param$rho^2) * toeplitz(c(1+param$rho^2, -param$rho, rep(0, n-2L)))
            Q[1, 1] = Q[n, n] = param$prec/(1-param$rho^2)
            Q = inla.as.sparse(Q)
        } else {
            ## faster. we only need to define the lower-triangular Q!
            i = c(
                ## diagonal
                1L, n, 2L:(n-1L),
                ## off-diagonal
                1L:(n-1L))
            j = c(
                ## diagonal
                1L, n, 2L:(n-1L),
                ## off-diagonal
                2L:n)
            x = param$prec/(1-param$rho^2) *
                c(  ## diagonal
                    1L, 1L, rep(1+param$rho^2, n-2L),
                    ## off-diagonal
                    rep(-param$rho, n-1L))
            Q = sparseMatrix(i=i, j=j, x=x, giveCsparse=FALSE)
        }            
        return (Q)
    }
\end{verbatim}
}

\subsection*{Function \texttt{mu}}

This function must return the mean of the model. Often, the mean is
zero, but sometimes it might depend on the hyperparameters as well. If
\texttt{numeric(0)} is returned, then this is equivalent that the mean
is zero. An alternative in this example, would be to return
\texttt{rep(0,n)}.

{\small
\begin{verbatim}
    mu = function()
    {
        return(numeric(0))
    }
\end{verbatim}
}


\subsection*{Function \texttt{log.norm.const}}

This function must return the log of the normalising constant. For the
AR1-model the normalising constant is
\begin{displaymath}
    \left(\frac{1}{\sqrt{2\pi}}\right)^{n} \tau_{I}^{n/2} (1-\rho^{2})^{1/2}
\end{displaymath}
where
\begin{displaymath}
    \tau_{I} = \tau/(1-\rho^{2}).
\end{displaymath}
The function can then be implemented as
{\small
\begin{verbatim}
    log.norm.const = function()
    {
        ## return the log(normalising constant) for the model
        param = interpret.theta()
        prec.innovation  = param$prec / (1.0 - param$rho^2)
        val = n * (- 0.5 * log(2*pi) + 0.5 * log(prec.innovation)) + 0.5 * log(1.0 - param$rho^2)
        return (val)
    }
\end{verbatim}
}
\noindent
\textbf{NOTE:} If the log-normalizing constant in any case need to be
computed from scratch as $$-\frac{n}{2}\log(2\pi) +
\frac{1}{2} \log(|{Q}|),$$ then \texttt{INLA} will compute this if
\texttt{numeric(0)} is returned, like {\small
\begin{verbatim}
    log.norm.const = function()
    {
        ## let INLA compute it
        return (numeric(0))
    }
\end{verbatim}
}


\subsection*{Function \texttt{log.prior}}

This function must return the (log-)prior of the prior density for
$\theta$. For the AR1-model, we have for simplicity chosen this prior
\begin{displaymath}
    \pi(\theta) = \Gamma(\exp(\theta_1); a,b) \exp(\theta_1) \times
    {\mathcal N}(\theta_{2}; \mu, \kappa)
\end{displaymath}
so we can implement this as with our choices $a=b=1$, $\mu=0$ and
$\kappa=1$ as
{\small
\begin{verbatim}
   log.prior = function()
    {
        ## return the log-prior for the hyperparameters. the '+theta[1L]' is the log(Jacobian)
        ## for having a gamma prior on the precision and convert it into the prior for the
        ## log(precision).
        param = interpret.theta()
        val = (dgamma(param$prec, shape = 1, rate = 1, log=TRUE) + theta[1L] + 
                   dnorm(theta[2L], mean = 0, sd = 1, log=TRUE))
        return (val)
    }
\end{verbatim}
}
An alternative, is to pass the parameters of the these priors as when
defining the model using \texttt{inla.rgeneric.define} in the
\texttt{...} argument.

\subsection*{Function \texttt{initial}}

This function returns the initial values for $\theta$.
{\small
\begin{verbatim}
    initial = function()
    {
        ## return initial values
        ntheta = 2
        return (rep(1, ntheta))
    }
\end{verbatim}
}

\subsection*{Function \texttt{quit}}

This function is called when all the computations are done and before
exit-ing the \texttt{C}-program. Usually, there is nothing in
particular to do, but if there is something that should be done, you
can do this here.
{\small
\begin{verbatim}
    quit = function()
    {
        return (invisible())
    }
\end{verbatim}
}


\section*{The complete definition of the AR1-model}

For completeness, we include here the complete code for the AR1-model,
collecting all the functions already defined. The function is
predefined in the \texttt{INLA}-library.
{\small
\begin{verbatim}
`inla.rgeneric.ar1.model` = function(cmd = c("graph", "Q", "mu", "initial",
                                             "log.norm.const", "log.prior", "quit"),
                                     theta = NULL)
{
    ## this is an example of the 'rgeneric' model. here we implement
    ## the AR-1 model as described in inla.doc("ar1"), where 'rho' is
    ## the lag-1 correlation and 'prec' is the *marginal* (not
    ## conditional) precision.
    
    interpret.theta = function()
    {
        ## internal helper-function to map the parameters from the internal-scale to the
        ## user-scale
        return (list(prec = exp(theta[1L]),
                     rho = 2*exp(theta[2L])/(1+exp(theta[2L])) - 1.0))
    }

    graph = function()
    {
        ## return the graph of the model. the values of Q is only interpreted as zero or
        ## non-zero. return a sparse.matrix
        if (FALSE) {
            ## slow and easy: dense-matrices
            G = toeplitz(c(1, 1, rep(0, n-2L)))
            G = inla.as.sparse(G)
        } else {
            ## faster. we only need to define the lower-triangular of G
            i = c(
                ## diagonal
                1L, n, 2L:(n-1L),
                ## off-diagonal
                1L:(n-1L))
            j = c(
                ## diagonal
                1L, n, 2L:(n-1L),
                ## off-diagonal
                2L:n)
            x = 1 ## meaning that all are 1
            G = sparseMatrix(i=i, j=j, x=x, giveCsparse = FALSE)
        }            
        return (G)
    }

    Q = function()
    {
        ## returns the precision matrix for given parameters
        param = interpret.theta()
        if (FALSE) {
            ## slow and easy: dense-matrices
            Q = param$prec/(1-param$rho^2) * toeplitz(c(1+param$rho^2, -param$rho, rep(0, n-2L)))
            Q[1, 1] = Q[n, n] = param$prec/(1-param$rho^2)
            Q = inla.as.sparse(Q)
        } else {
            ## faster. we only need to define the lower-triangular Q!
            i = c(
                ## diagonal
                1L, n, 2L:(n-1L),
                ## off-diagonal
                1L:(n-1L))
            j = c(
                ## diagonal
                1L, n, 2L:(n-1L),
                ## off-diagonal
                2L:n)
            x = param$prec/(1-param$rho^2) *
                c(  ## diagonal
                    1L, 1L, rep(1+param$rho^2, n-2L),
                    ## off-diagonal
                    rep(-param$rho, n-1L))
            Q = sparseMatrix(i=i, j=j, x=x, giveCsparse=FALSE)
        }            
        return (Q)
    }

    mu = function()
    {
        return(numeric(0))
    }

    log.norm.const = function()
    {
        ## return the log(normalising constant) for the model
        param = interpret.theta()
        prec.innovation  = param$prec / (1.0 - param$rho^2)
        val = n * (- 0.5 * log(2*pi) + 0.5 * log(prec.innovation)) + 0.5 * log(1.0 - param$rho^2)
        return (val)
    }

    log.prior = function()
    {
        ## return the log-prior for the hyperparameters. the '+theta[1L]' is the log(Jacobian)
        ## for having a gamma prior on the precision and convert it into the prior for the
        ## log(precision).
        param = interpret.theta()
        val = (dgamma(param$prec, shape = 1, rate = 1, log=TRUE) + theta[1L] + 
                   dnorm(theta[2L], mean = 0, sd = 1, log=TRUE))
        return (val)
    }

    initial = function()
    {
        ## return initial values
        ntheta = 2
        return (rep(1, ntheta))
    }

    quit = function()
    {
        return (invisible())
    }

    val = do.call(match.arg(cmd), args = list())
    return (val)
}
\end{verbatim}
}

\section*{Example of usage}

{\small
\verbatiminput{rgeneric-example.R}
}

\clearpage

\section*{Example: the iid-model}

The following function defines the \texttt{iid}-model, see
\texttt{inla.doc("iid")}, which we give without further comments. To
run this model in \texttt{R}, you may do \texttt{demo(rgeneric)}.

{\small
\begin{verbatim}
`inla.rgeneric.iid.model` = function(cmd = c("graph", "Q", "mu", "initial",
                                             "log.norm.const", "log.prior", "quit"),
                                     theta = NULL)
{
    ## this is an example of the 'rgeneric' model. here we implement the iid model as described
    ## in inla.doc("iid"), without the scaling-option

    interpret.theta = function()
    {
        return (list(prec = exp(theta[1L])))
    }

    graph = function()
    {
        G = Diagonal(n, x= rep(1, n))
        return (G)
    }

    Q = function()
    {
        prec = interpret.theta()$prec
        Q = Diagonal(n, x= rep(prec, n))
        return (Q)
    }

    mu = function()
    {
        return(numeric(0))
    }

    log.norm.const = function()
    {
        prec = interpret.theta()$prec
        val = sum(dnorm(rep(0, n), sd = 1/sqrt(prec), log=TRUE))
        return (val)
    }

    log.prior = function()
    {
        prec = interpret.theta()$prec
        val = dgamma(prec, shape = 1, rate = 1, log=TRUE) + theta[1L]
        return (val)
    }

    initial = function()
    {
        ntheta = 1
        return (rep(1, ntheta))
    }

    quit = function()
    {
        return (invisible())
    }

    val = do.call(match.arg(cmd), args = list())
    return (val)
}
\end{verbatim}
}

\clearpage
\section*{Example: a model for the mean structure}

{\small
\verbatiminput{rgeneric-example-3.R}
}


\end{document}

% LocalWords:  rgeneric INLA MacOSX multi inla args ar covariance cmd ntheta ie
% LocalWords:  reparameterise sparseMatrix ing rmodel init tmarginal iid

%%% Local Variables: 
%%% TeX-master: t
%%% End: 
