\documentclass[a4paper,11pt]{article}
\usepackage[scale={0.8,0.9},centering,includeheadfoot]{geometry}
\usepackage{amstext}
\usepackage{listings}
\usepackage{verbatim}
\def\opening\null
\usepackage{block}
\begin{document}

\section*{Correlated random effects: \texttt{iidkd}}

This model is available for dimensions $k=2$, to $24$. We describe in
detail the case for $k=3$ as other ones are similar. This model do the
same as models \texttt{iid2d}, \texttt{iid3d}, \texttt{iid4d},
\texttt{iid5d}, but uses a different and more efficient parameterisation. 

\subsection*{Parametrization}

The $(k=3)$-dimensional Normal-Wishard model is used if one want to define
three vectors of ``random effects'', $u$ and $v$ and $w$, say, for which
$(u_{i}, v_{i}, w_i)$ are iid bivariate Normals
\begin{displaymath}
    \left(
      \begin{array}{c}
        u_{i}\\
        v_{i}\\
        w_{i}
      \end{array}\right)
    \sim \mathcal{N}\left(\mathbf{0}, \mathbf{W}^{-1}\right)
\end{displaymath}
where the  covariance matrix $\mathbf{W}^{-1}$ is parameterised as
$\mathbf{W}=\mathbf{L}\mathbf{L}^{T}$, where
\begin{equation}
    \label{precision}
    \mathbf{L} = \left(\begin{array}{ccc}
                         \exp(\theta_1) & & \\
                         \theta_4 & \exp(\theta_2) & \\
                         \theta_5 & \theta_6 & \exp(\theta_3)
      \end{array}\right)
\end{equation}
and $\theta_1, \theta_2, \theta_3, \theta_4, \theta_5, \theta_6$ can
take any value. The number of hyperparameters are $k(k+1)/2$, which is
$3$, $6$, $10$, $15$, $21$, $28$, $36$, $45$, $55, \ldots$, for
$k=2, 3, 4, 5, 6, 7, 8, 9, 10, \ldots$.

For these models the precision matrix $\mathbf{W}$ is Wishart
distributed
\begin{displaymath}
    \mathbf{W}
    \;\sim\;\text{Wishart}_{k}(r, \mathbf{R}^{-1}), 
\end{displaymath}
with density
\begin{displaymath}
    \pi(\mathbf{W}) = c^{-1} |\mathbf{W}|^{(r-(k+1))/2} \exp\left\{
      -\frac{1}{2}\text{Trace}(\mathbf{W}\mathbf{R})\right\}, \quad r > k+1
\end{displaymath}
and
\begin{displaymath}
    c = 2^{(rk)/2} |\mathbf{R}|^{-r/2} \pi^{(k(k-1))/4}\prod_{j=1}^{k}
    \Gamma((r+1-j)/2).
\end{displaymath}
Then,
\begin{displaymath}
    \text{E}(\mathbf{W}) = r\mathbf{R}^{-1}, \quad\text{and}\quad
    \text{E}(\mathbf{W}^{-1}) = \mathbf{R}/(r-(k+1)).
\end{displaymath}

\subsection*{Hyperparameters}

The hyperparameters are
$\theta_1, \theta_2, \theta_3, \theta_4, \theta_5, \theta_6$.

The prior-parameters are
\begin{displaymath}
    (r,R_{1}, R_{2}, R_{3}, R_{4}, R_{5}, R_{6})
\end{displaymath}
where 
\begin{displaymath}
    \mathbf{R}= \left(
      \begin{array}{ccc}
        R_{1} & R_{4} & R_{5}\\
        R_{4} & R_{2} & R_{6} \\
        R_{5} & R_{6} & R_{3}
      \end{array}\right)
\end{displaymath}

The {\tt inla} function reports posterior distribution for the
hyperparameters $\{\theta_i\}$, and the conversion into interpretable
quantities can be done using simulation as described below.

The prior for $\theta$ is {\bf fixed} to be {\tt wishartkd}, and
number of prior parameters required are $1 + k(k+1)/2$. By default the
prior-parameters are
\begin{displaymath}
    (r=100, \underbrace{1, \ldots, 1}_{k\;\text{times}}, 0, \ldots, 0)
\end{displaymath}


\subsection*{Specification}

The model \texttt{iidkd}
is specified as
\begin{verbatim}
    y ~ f(i, model="iidkd", order=3, n = <length>) + ...
\end{verbatim}
where $\text{order}=k=3$, and the \texttt{iidkd} model is represented
internally as one vector of length $n$,
\begin{displaymath}
    (u_{1}, u_{2} \ldots, u_{m}, v_{1}, v_{2}, \ldots, v_{m}, w_{1},
    w_{2}, \ldots, w_{m})
\end{displaymath}
where $n = 3m$, and $n$ is the (required) argument in
\texttt{f()}.

For this model the argument \texttt{constr=TRUE} is interpreted as $3$
sum-to-zero constraints
\begin{displaymath}
    \sum u_{i} = 0, \quad \sum v_{i} = 0 \quad\text{and}\quad \sum w_{i} = 0.
\end{displaymath}

\subsubsection*{Hyperparameter spesification and default values}

(\textbf{Note:} The value ``$1048576$'' is just a code for ``replace
this by the default value''. As the default value depends on
\texttt{order}, the was the easy way out for the moment.)

%% DO NOT EDIT!
%% This file is generated automatically from models.R
\begin{description}
	\item[doc] \verb!Gaussian random effect in dim=k with Wishart prior!
	\item[hyper]\ 
	 \begin{description}
	 	\item[theta1]\ 
	 	 \begin{description}
	 	 	\item[hyperid] \verb!29101!
	 	 	\item[name] \verb!theta1!
	 	 	\item[short.name] \verb!theta1!
	 	 	\item[initial] \verb!1048576!
	 	 	\item[fixed] \verb!FALSE!
	 	 	\item[prior] \verb!wishartkd!
	 	 	\item[param] \verb!30 1048576 1048576 1048576 1048576 1048576 1048576 1048576 1048576 1048576 1048576 1048576 1048576 1048576 1048576 1048576 1048576 1048576 1048576 1048576 1048576 1048576 1048576 1048576 1048576 1048576 1048576 1048576 1048576 1048576 1048576 1048576 1048576 1048576 1048576 1048576 1048576 1048576 1048576 1048576 1048576 1048576 1048576 1048576 1048576 1048576 1048576 1048576 1048576 1048576 1048576 1048576 1048576 1048576 1048576 1048576 1048576 1048576 1048576 1048576 1048576 1048576 1048576 1048576 1048576 1048576 1048576 1048576 1048576 1048576 1048576 1048576 1048576 1048576 1048576 1048576 1048576 1048576 1048576 1048576 1048576 1048576 1048576 1048576 1048576 1048576 1048576 1048576 1048576 1048576 1048576 1048576 1048576 1048576 1048576 1048576 1048576 1048576 1048576 1048576 1048576 1048576 1048576 1048576 1048576 1048576 1048576 1048576 1048576 1048576 1048576 1048576 1048576 1048576 1048576 1048576 1048576 1048576 1048576 1048576 1048576 1048576 1048576 1048576 1048576 1048576 1048576 1048576 1048576 1048576 1048576 1048576 1048576 1048576 1048576 1048576 1048576 1048576 1048576 1048576 1048576 1048576 1048576 1048576 1048576 1048576 1048576 1048576 1048576 1048576 1048576 1048576 1048576 1048576 1048576 1048576 1048576 1048576 1048576 1048576 1048576 1048576 1048576 1048576 1048576 1048576 1048576 1048576 1048576 1048576 1048576 1048576 1048576 1048576 1048576 1048576 1048576 1048576 1048576 1048576 1048576 1048576 1048576 1048576 1048576 1048576 1048576 1048576 1048576 1048576 1048576 1048576 1048576 1048576 1048576 1048576 1048576 1048576 1048576 1048576 1048576 1048576 1048576 1048576 1048576 1048576 1048576 1048576 1048576 1048576 1048576 1048576 1048576 1048576 1048576 1048576 1048576 1048576 1048576 1048576 1048576 1048576 1048576 1048576 1048576 1048576 1048576 1048576 1048576 1048576 1048576 1048576 1048576 1048576 1048576 1048576 1048576 1048576 1048576 1048576 1048576 1048576 1048576 1048576 1048576 1048576 1048576 1048576 1048576 1048576 1048576 1048576 1048576 1048576 1048576 1048576 1048576 1048576 1048576 1048576 1048576 1048576 1048576 1048576 1048576 1048576 1048576 1048576 1048576 1048576 1048576 1048576 1048576 1048576 1048576 1048576 1048576 1048576 1048576 1048576 1048576 1048576 1048576 1048576 1048576 1048576 1048576 1048576 1048576 1048576 1048576 1048576 1048576 1048576 1048576 1048576 1048576 1048576 1048576 1048576 1048576!
	 	 	\item[to.theta] \verb!function(x) x!
	 	 	\item[from.theta] \verb!function(x) x!
	 	 \end{description}
	 	\item[theta2]\ 
	 	 \begin{description}
	 	 	\item[hyperid] \verb!29102!
	 	 	\item[name] \verb!theta2!
	 	 	\item[short.name] \verb!theta2!
	 	 	\item[initial] \verb!1048576!
	 	 	\item[fixed] \verb!FALSE!
	 	 	\item[prior] \verb!none!
	 	 	\item[param] \verb!!
	 	 	\item[to.theta] \verb!function(x) x!
	 	 	\item[from.theta] \verb!function(x) x!
	 	 \end{description}
	 	\item[theta3]\ 
	 	 \begin{description}
	 	 	\item[hyperid] \verb!29103!
	 	 	\item[name] \verb!theta3!
	 	 	\item[short.name] \verb!theta3!
	 	 	\item[initial] \verb!1048576!
	 	 	\item[fixed] \verb!FALSE!
	 	 	\item[prior] \verb!none!
	 	 	\item[param] \verb!!
	 	 	\item[to.theta] \verb!function(x) x!
	 	 	\item[from.theta] \verb!function(x) x!
	 	 \end{description}
	 	\item[theta4]\ 
	 	 \begin{description}
	 	 	\item[hyperid] \verb!29104!
	 	 	\item[name] \verb!theta4!
	 	 	\item[short.name] \verb!theta4!
	 	 	\item[initial] \verb!1048576!
	 	 	\item[fixed] \verb!FALSE!
	 	 	\item[prior] \verb!none!
	 	 	\item[param] \verb!!
	 	 	\item[to.theta] \verb!function(x) x!
	 	 	\item[from.theta] \verb!function(x) x!
	 	 \end{description}
	 	\item[theta5]\ 
	 	 \begin{description}
	 	 	\item[hyperid] \verb!29105!
	 	 	\item[name] \verb!theta5!
	 	 	\item[short.name] \verb!theta5!
	 	 	\item[initial] \verb!1048576!
	 	 	\item[fixed] \verb!FALSE!
	 	 	\item[prior] \verb!none!
	 	 	\item[param] \verb!!
	 	 	\item[to.theta] \verb!function(x) x!
	 	 	\item[from.theta] \verb!function(x) x!
	 	 \end{description}
	 	\item[theta6]\ 
	 	 \begin{description}
	 	 	\item[hyperid] \verb!29106!
	 	 	\item[name] \verb!theta6!
	 	 	\item[short.name] \verb!theta6!
	 	 	\item[initial] \verb!1048576!
	 	 	\item[fixed] \verb!FALSE!
	 	 	\item[prior] \verb!none!
	 	 	\item[param] \verb!!
	 	 	\item[to.theta] \verb!function(x) x!
	 	 	\item[from.theta] \verb!function(x) x!
	 	 \end{description}
	 	\item[theta7]\ 
	 	 \begin{description}
	 	 	\item[hyperid] \verb!29107!
	 	 	\item[name] \verb!theta7!
	 	 	\item[short.name] \verb!theta7!
	 	 	\item[initial] \verb!1048576!
	 	 	\item[fixed] \verb!FALSE!
	 	 	\item[prior] \verb!none!
	 	 	\item[param] \verb!!
	 	 	\item[to.theta] \verb!function(x) x!
	 	 	\item[from.theta] \verb!function(x) x!
	 	 \end{description}
	 	\item[theta8]\ 
	 	 \begin{description}
	 	 	\item[hyperid] \verb!29108!
	 	 	\item[name] \verb!theta8!
	 	 	\item[short.name] \verb!theta8!
	 	 	\item[initial] \verb!1048576!
	 	 	\item[fixed] \verb!FALSE!
	 	 	\item[prior] \verb!none!
	 	 	\item[param] \verb!!
	 	 	\item[to.theta] \verb!function(x) x!
	 	 	\item[from.theta] \verb!function(x) x!
	 	 \end{description}
	 	\item[theta9]\ 
	 	 \begin{description}
	 	 	\item[hyperid] \verb!29109!
	 	 	\item[name] \verb!theta9!
	 	 	\item[short.name] \verb!theta9!
	 	 	\item[initial] \verb!1048576!
	 	 	\item[fixed] \verb!FALSE!
	 	 	\item[prior] \verb!none!
	 	 	\item[param] \verb!!
	 	 	\item[to.theta] \verb!function(x) x!
	 	 	\item[from.theta] \verb!function(x) x!
	 	 \end{description}
	 	\item[theta10]\ 
	 	 \begin{description}
	 	 	\item[hyperid] \verb!29110!
	 	 	\item[name] \verb!theta10!
	 	 	\item[short.name] \verb!theta10!
	 	 	\item[initial] \verb!1048576!
	 	 	\item[fixed] \verb!FALSE!
	 	 	\item[prior] \verb!none!
	 	 	\item[param] \verb!!
	 	 	\item[to.theta] \verb!function(x) x!
	 	 	\item[from.theta] \verb!function(x) x!
	 	 \end{description}
	 	\item[theta11]\ 
	 	 \begin{description}
	 	 	\item[hyperid] \verb!29111!
	 	 	\item[name] \verb!theta11!
	 	 	\item[short.name] \verb!theta11!
	 	 	\item[initial] \verb!1048576!
	 	 	\item[fixed] \verb!FALSE!
	 	 	\item[prior] \verb!none!
	 	 	\item[param] \verb!!
	 	 	\item[to.theta] \verb!function(x) x!
	 	 	\item[from.theta] \verb!function(x) x!
	 	 \end{description}
	 	\item[theta12]\ 
	 	 \begin{description}
	 	 	\item[hyperid] \verb!29112!
	 	 	\item[name] \verb!theta12!
	 	 	\item[short.name] \verb!theta12!
	 	 	\item[initial] \verb!1048576!
	 	 	\item[fixed] \verb!FALSE!
	 	 	\item[prior] \verb!none!
	 	 	\item[param] \verb!!
	 	 	\item[to.theta] \verb!function(x) x!
	 	 	\item[from.theta] \verb!function(x) x!
	 	 \end{description}
	 	\item[theta13]\ 
	 	 \begin{description}
	 	 	\item[hyperid] \verb!29113!
	 	 	\item[name] \verb!theta13!
	 	 	\item[short.name] \verb!theta13!
	 	 	\item[initial] \verb!1048576!
	 	 	\item[fixed] \verb!FALSE!
	 	 	\item[prior] \verb!none!
	 	 	\item[param] \verb!!
	 	 	\item[to.theta] \verb!function(x) x!
	 	 	\item[from.theta] \verb!function(x) x!
	 	 \end{description}
	 	\item[theta14]\ 
	 	 \begin{description}
	 	 	\item[hyperid] \verb!29114!
	 	 	\item[name] \verb!theta14!
	 	 	\item[short.name] \verb!theta14!
	 	 	\item[initial] \verb!1048576!
	 	 	\item[fixed] \verb!FALSE!
	 	 	\item[prior] \verb!none!
	 	 	\item[param] \verb!!
	 	 	\item[to.theta] \verb!function(x) x!
	 	 	\item[from.theta] \verb!function(x) x!
	 	 \end{description}
	 	\item[theta15]\ 
	 	 \begin{description}
	 	 	\item[hyperid] \verb!29115!
	 	 	\item[name] \verb!theta15!
	 	 	\item[short.name] \verb!theta15!
	 	 	\item[initial] \verb!1048576!
	 	 	\item[fixed] \verb!FALSE!
	 	 	\item[prior] \verb!none!
	 	 	\item[param] \verb!!
	 	 	\item[to.theta] \verb!function(x) x!
	 	 	\item[from.theta] \verb!function(x) x!
	 	 \end{description}
	 	\item[theta16]\ 
	 	 \begin{description}
	 	 	\item[hyperid] \verb!29116!
	 	 	\item[name] \verb!theta16!
	 	 	\item[short.name] \verb!theta16!
	 	 	\item[initial] \verb!1048576!
	 	 	\item[fixed] \verb!FALSE!
	 	 	\item[prior] \verb!none!
	 	 	\item[param] \verb!!
	 	 	\item[to.theta] \verb!function(x) x!
	 	 	\item[from.theta] \verb!function(x) x!
	 	 \end{description}
	 	\item[theta17]\ 
	 	 \begin{description}
	 	 	\item[hyperid] \verb!29117!
	 	 	\item[name] \verb!theta17!
	 	 	\item[short.name] \verb!theta17!
	 	 	\item[initial] \verb!1048576!
	 	 	\item[fixed] \verb!FALSE!
	 	 	\item[prior] \verb!none!
	 	 	\item[param] \verb!!
	 	 	\item[to.theta] \verb!function(x) x!
	 	 	\item[from.theta] \verb!function(x) x!
	 	 \end{description}
	 	\item[theta18]\ 
	 	 \begin{description}
	 	 	\item[hyperid] \verb!29118!
	 	 	\item[name] \verb!theta18!
	 	 	\item[short.name] \verb!theta18!
	 	 	\item[initial] \verb!1048576!
	 	 	\item[fixed] \verb!FALSE!
	 	 	\item[prior] \verb!none!
	 	 	\item[param] \verb!!
	 	 	\item[to.theta] \verb!function(x) x!
	 	 	\item[from.theta] \verb!function(x) x!
	 	 \end{description}
	 	\item[theta19]\ 
	 	 \begin{description}
	 	 	\item[hyperid] \verb!29119!
	 	 	\item[name] \verb!theta19!
	 	 	\item[short.name] \verb!theta19!
	 	 	\item[initial] \verb!1048576!
	 	 	\item[fixed] \verb!FALSE!
	 	 	\item[prior] \verb!none!
	 	 	\item[param] \verb!!
	 	 	\item[to.theta] \verb!function(x) x!
	 	 	\item[from.theta] \verb!function(x) x!
	 	 \end{description}
	 	\item[theta20]\ 
	 	 \begin{description}
	 	 	\item[hyperid] \verb!29120!
	 	 	\item[name] \verb!theta20!
	 	 	\item[short.name] \verb!theta20!
	 	 	\item[initial] \verb!1048576!
	 	 	\item[fixed] \verb!FALSE!
	 	 	\item[prior] \verb!none!
	 	 	\item[param] \verb!!
	 	 	\item[to.theta] \verb!function(x) x!
	 	 	\item[from.theta] \verb!function(x) x!
	 	 \end{description}
	 	\item[theta21]\ 
	 	 \begin{description}
	 	 	\item[hyperid] \verb!29121!
	 	 	\item[name] \verb!theta21!
	 	 	\item[short.name] \verb!theta21!
	 	 	\item[initial] \verb!1048576!
	 	 	\item[fixed] \verb!FALSE!
	 	 	\item[prior] \verb!none!
	 	 	\item[param] \verb!!
	 	 	\item[to.theta] \verb!function(x) x!
	 	 	\item[from.theta] \verb!function(x) x!
	 	 \end{description}
	 	\item[theta22]\ 
	 	 \begin{description}
	 	 	\item[hyperid] \verb!29122!
	 	 	\item[name] \verb!theta22!
	 	 	\item[short.name] \verb!theta22!
	 	 	\item[initial] \verb!1048576!
	 	 	\item[fixed] \verb!FALSE!
	 	 	\item[prior] \verb!none!
	 	 	\item[param] \verb!!
	 	 	\item[to.theta] \verb!function(x) x!
	 	 	\item[from.theta] \verb!function(x) x!
	 	 \end{description}
	 	\item[theta23]\ 
	 	 \begin{description}
	 	 	\item[hyperid] \verb!29123!
	 	 	\item[name] \verb!theta23!
	 	 	\item[short.name] \verb!theta23!
	 	 	\item[initial] \verb!1048576!
	 	 	\item[fixed] \verb!FALSE!
	 	 	\item[prior] \verb!none!
	 	 	\item[param] \verb!!
	 	 	\item[to.theta] \verb!function(x) x!
	 	 	\item[from.theta] \verb!function(x) x!
	 	 \end{description}
	 	\item[theta24]\ 
	 	 \begin{description}
	 	 	\item[hyperid] \verb!29124!
	 	 	\item[name] \verb!theta24!
	 	 	\item[short.name] \verb!theta24!
	 	 	\item[initial] \verb!1048576!
	 	 	\item[fixed] \verb!FALSE!
	 	 	\item[prior] \verb!none!
	 	 	\item[param] \verb!!
	 	 	\item[to.theta] \verb!function(x) x!
	 	 	\item[from.theta] \verb!function(x) x!
	 	 \end{description}
	 	\item[theta25]\ 
	 	 \begin{description}
	 	 	\item[hyperid] \verb!29125!
	 	 	\item[name] \verb!theta25!
	 	 	\item[short.name] \verb!theta25!
	 	 	\item[initial] \verb!1048576!
	 	 	\item[fixed] \verb!FALSE!
	 	 	\item[prior] \verb!none!
	 	 	\item[param] \verb!!
	 	 	\item[to.theta] \verb!function(x) x!
	 	 	\item[from.theta] \verb!function(x) x!
	 	 \end{description}
	 	\item[theta26]\ 
	 	 \begin{description}
	 	 	\item[hyperid] \verb!29126!
	 	 	\item[name] \verb!theta26!
	 	 	\item[short.name] \verb!theta26!
	 	 	\item[initial] \verb!1048576!
	 	 	\item[fixed] \verb!FALSE!
	 	 	\item[prior] \verb!none!
	 	 	\item[param] \verb!!
	 	 	\item[to.theta] \verb!function(x) x!
	 	 	\item[from.theta] \verb!function(x) x!
	 	 \end{description}
	 	\item[theta27]\ 
	 	 \begin{description}
	 	 	\item[hyperid] \verb!29127!
	 	 	\item[name] \verb!theta27!
	 	 	\item[short.name] \verb!theta27!
	 	 	\item[initial] \verb!1048576!
	 	 	\item[fixed] \verb!FALSE!
	 	 	\item[prior] \verb!none!
	 	 	\item[param] \verb!!
	 	 	\item[to.theta] \verb!function(x) x!
	 	 	\item[from.theta] \verb!function(x) x!
	 	 \end{description}
	 	\item[theta28]\ 
	 	 \begin{description}
	 	 	\item[hyperid] \verb!29128!
	 	 	\item[name] \verb!theta28!
	 	 	\item[short.name] \verb!theta28!
	 	 	\item[initial] \verb!1048576!
	 	 	\item[fixed] \verb!FALSE!
	 	 	\item[prior] \verb!none!
	 	 	\item[param] \verb!!
	 	 	\item[to.theta] \verb!function(x) x!
	 	 	\item[from.theta] \verb!function(x) x!
	 	 \end{description}
	 	\item[theta29]\ 
	 	 \begin{description}
	 	 	\item[hyperid] \verb!29129!
	 	 	\item[name] \verb!theta29!
	 	 	\item[short.name] \verb!theta29!
	 	 	\item[initial] \verb!1048576!
	 	 	\item[fixed] \verb!FALSE!
	 	 	\item[prior] \verb!none!
	 	 	\item[param] \verb!!
	 	 	\item[to.theta] \verb!function(x) x!
	 	 	\item[from.theta] \verb!function(x) x!
	 	 \end{description}
	 	\item[theta30]\ 
	 	 \begin{description}
	 	 	\item[hyperid] \verb!29130!
	 	 	\item[name] \verb!theta30!
	 	 	\item[short.name] \verb!theta30!
	 	 	\item[initial] \verb!1048576!
	 	 	\item[fixed] \verb!FALSE!
	 	 	\item[prior] \verb!none!
	 	 	\item[param] \verb!!
	 	 	\item[to.theta] \verb!function(x) x!
	 	 	\item[from.theta] \verb!function(x) x!
	 	 \end{description}
	 	\item[theta31]\ 
	 	 \begin{description}
	 	 	\item[hyperid] \verb!29131!
	 	 	\item[name] \verb!theta31!
	 	 	\item[short.name] \verb!theta31!
	 	 	\item[initial] \verb!1048576!
	 	 	\item[fixed] \verb!FALSE!
	 	 	\item[prior] \verb!none!
	 	 	\item[param] \verb!!
	 	 	\item[to.theta] \verb!function(x) x!
	 	 	\item[from.theta] \verb!function(x) x!
	 	 \end{description}
	 	\item[theta32]\ 
	 	 \begin{description}
	 	 	\item[hyperid] \verb!29132!
	 	 	\item[name] \verb!theta32!
	 	 	\item[short.name] \verb!theta32!
	 	 	\item[initial] \verb!1048576!
	 	 	\item[fixed] \verb!FALSE!
	 	 	\item[prior] \verb!none!
	 	 	\item[param] \verb!!
	 	 	\item[to.theta] \verb!function(x) x!
	 	 	\item[from.theta] \verb!function(x) x!
	 	 \end{description}
	 	\item[theta33]\ 
	 	 \begin{description}
	 	 	\item[hyperid] \verb!29133!
	 	 	\item[name] \verb!theta33!
	 	 	\item[short.name] \verb!theta33!
	 	 	\item[initial] \verb!1048576!
	 	 	\item[fixed] \verb!FALSE!
	 	 	\item[prior] \verb!none!
	 	 	\item[param] \verb!!
	 	 	\item[to.theta] \verb!function(x) x!
	 	 	\item[from.theta] \verb!function(x) x!
	 	 \end{description}
	 	\item[theta34]\ 
	 	 \begin{description}
	 	 	\item[hyperid] \verb!29134!
	 	 	\item[name] \verb!theta34!
	 	 	\item[short.name] \verb!theta34!
	 	 	\item[initial] \verb!1048576!
	 	 	\item[fixed] \verb!FALSE!
	 	 	\item[prior] \verb!none!
	 	 	\item[param] \verb!!
	 	 	\item[to.theta] \verb!function(x) x!
	 	 	\item[from.theta] \verb!function(x) x!
	 	 \end{description}
	 	\item[theta35]\ 
	 	 \begin{description}
	 	 	\item[hyperid] \verb!29135!
	 	 	\item[name] \verb!theta35!
	 	 	\item[short.name] \verb!theta35!
	 	 	\item[initial] \verb!1048576!
	 	 	\item[fixed] \verb!FALSE!
	 	 	\item[prior] \verb!none!
	 	 	\item[param] \verb!!
	 	 	\item[to.theta] \verb!function(x) x!
	 	 	\item[from.theta] \verb!function(x) x!
	 	 \end{description}
	 	\item[theta36]\ 
	 	 \begin{description}
	 	 	\item[hyperid] \verb!29136!
	 	 	\item[name] \verb!theta36!
	 	 	\item[short.name] \verb!theta36!
	 	 	\item[initial] \verb!1048576!
	 	 	\item[fixed] \verb!FALSE!
	 	 	\item[prior] \verb!none!
	 	 	\item[param] \verb!!
	 	 	\item[to.theta] \verb!function(x) x!
	 	 	\item[from.theta] \verb!function(x) x!
	 	 \end{description}
	 	\item[theta37]\ 
	 	 \begin{description}
	 	 	\item[hyperid] \verb!29137!
	 	 	\item[name] \verb!theta37!
	 	 	\item[short.name] \verb!theta37!
	 	 	\item[initial] \verb!1048576!
	 	 	\item[fixed] \verb!FALSE!
	 	 	\item[prior] \verb!none!
	 	 	\item[param] \verb!!
	 	 	\item[to.theta] \verb!function(x) x!
	 	 	\item[from.theta] \verb!function(x) x!
	 	 \end{description}
	 	\item[theta38]\ 
	 	 \begin{description}
	 	 	\item[hyperid] \verb!29138!
	 	 	\item[name] \verb!theta38!
	 	 	\item[short.name] \verb!theta38!
	 	 	\item[initial] \verb!1048576!
	 	 	\item[fixed] \verb!FALSE!
	 	 	\item[prior] \verb!none!
	 	 	\item[param] \verb!!
	 	 	\item[to.theta] \verb!function(x) x!
	 	 	\item[from.theta] \verb!function(x) x!
	 	 \end{description}
	 	\item[theta39]\ 
	 	 \begin{description}
	 	 	\item[hyperid] \verb!29139!
	 	 	\item[name] \verb!theta39!
	 	 	\item[short.name] \verb!theta39!
	 	 	\item[initial] \verb!1048576!
	 	 	\item[fixed] \verb!FALSE!
	 	 	\item[prior] \verb!none!
	 	 	\item[param] \verb!!
	 	 	\item[to.theta] \verb!function(x) x!
	 	 	\item[from.theta] \verb!function(x) x!
	 	 \end{description}
	 	\item[theta40]\ 
	 	 \begin{description}
	 	 	\item[hyperid] \verb!29140!
	 	 	\item[name] \verb!theta40!
	 	 	\item[short.name] \verb!theta40!
	 	 	\item[initial] \verb!1048576!
	 	 	\item[fixed] \verb!FALSE!
	 	 	\item[prior] \verb!none!
	 	 	\item[param] \verb!!
	 	 	\item[to.theta] \verb!function(x) x!
	 	 	\item[from.theta] \verb!function(x) x!
	 	 \end{description}
	 	\item[theta41]\ 
	 	 \begin{description}
	 	 	\item[hyperid] \verb!29141!
	 	 	\item[name] \verb!theta41!
	 	 	\item[short.name] \verb!theta41!
	 	 	\item[initial] \verb!1048576!
	 	 	\item[fixed] \verb!FALSE!
	 	 	\item[prior] \verb!none!
	 	 	\item[param] \verb!!
	 	 	\item[to.theta] \verb!function(x) x!
	 	 	\item[from.theta] \verb!function(x) x!
	 	 \end{description}
	 	\item[theta42]\ 
	 	 \begin{description}
	 	 	\item[hyperid] \verb!29142!
	 	 	\item[name] \verb!theta42!
	 	 	\item[short.name] \verb!theta42!
	 	 	\item[initial] \verb!1048576!
	 	 	\item[fixed] \verb!FALSE!
	 	 	\item[prior] \verb!none!
	 	 	\item[param] \verb!!
	 	 	\item[to.theta] \verb!function(x) x!
	 	 	\item[from.theta] \verb!function(x) x!
	 	 \end{description}
	 	\item[theta43]\ 
	 	 \begin{description}
	 	 	\item[hyperid] \verb!29143!
	 	 	\item[name] \verb!theta43!
	 	 	\item[short.name] \verb!theta43!
	 	 	\item[initial] \verb!1048576!
	 	 	\item[fixed] \verb!FALSE!
	 	 	\item[prior] \verb!none!
	 	 	\item[param] \verb!!
	 	 	\item[to.theta] \verb!function(x) x!
	 	 	\item[from.theta] \verb!function(x) x!
	 	 \end{description}
	 	\item[theta44]\ 
	 	 \begin{description}
	 	 	\item[hyperid] \verb!29144!
	 	 	\item[name] \verb!theta44!
	 	 	\item[short.name] \verb!theta44!
	 	 	\item[initial] \verb!1048576!
	 	 	\item[fixed] \verb!FALSE!
	 	 	\item[prior] \verb!none!
	 	 	\item[param] \verb!!
	 	 	\item[to.theta] \verb!function(x) x!
	 	 	\item[from.theta] \verb!function(x) x!
	 	 \end{description}
	 	\item[theta45]\ 
	 	 \begin{description}
	 	 	\item[hyperid] \verb!29145!
	 	 	\item[name] \verb!theta45!
	 	 	\item[short.name] \verb!theta45!
	 	 	\item[initial] \verb!1048576!
	 	 	\item[fixed] \verb!FALSE!
	 	 	\item[prior] \verb!none!
	 	 	\item[param] \verb!!
	 	 	\item[to.theta] \verb!function(x) x!
	 	 	\item[from.theta] \verb!function(x) x!
	 	 \end{description}
	 	\item[theta46]\ 
	 	 \begin{description}
	 	 	\item[hyperid] \verb!29146!
	 	 	\item[name] \verb!theta46!
	 	 	\item[short.name] \verb!theta46!
	 	 	\item[initial] \verb!1048576!
	 	 	\item[fixed] \verb!FALSE!
	 	 	\item[prior] \verb!none!
	 	 	\item[param] \verb!!
	 	 	\item[to.theta] \verb!function(x) x!
	 	 	\item[from.theta] \verb!function(x) x!
	 	 \end{description}
	 	\item[theta47]\ 
	 	 \begin{description}
	 	 	\item[hyperid] \verb!29147!
	 	 	\item[name] \verb!theta47!
	 	 	\item[short.name] \verb!theta47!
	 	 	\item[initial] \verb!1048576!
	 	 	\item[fixed] \verb!FALSE!
	 	 	\item[prior] \verb!none!
	 	 	\item[param] \verb!!
	 	 	\item[to.theta] \verb!function(x) x!
	 	 	\item[from.theta] \verb!function(x) x!
	 	 \end{description}
	 	\item[theta48]\ 
	 	 \begin{description}
	 	 	\item[hyperid] \verb!29148!
	 	 	\item[name] \verb!theta48!
	 	 	\item[short.name] \verb!theta48!
	 	 	\item[initial] \verb!1048576!
	 	 	\item[fixed] \verb!FALSE!
	 	 	\item[prior] \verb!none!
	 	 	\item[param] \verb!!
	 	 	\item[to.theta] \verb!function(x) x!
	 	 	\item[from.theta] \verb!function(x) x!
	 	 \end{description}
	 	\item[theta49]\ 
	 	 \begin{description}
	 	 	\item[hyperid] \verb!29149!
	 	 	\item[name] \verb!theta49!
	 	 	\item[short.name] \verb!theta49!
	 	 	\item[initial] \verb!1048576!
	 	 	\item[fixed] \verb!FALSE!
	 	 	\item[prior] \verb!none!
	 	 	\item[param] \verb!!
	 	 	\item[to.theta] \verb!function(x) x!
	 	 	\item[from.theta] \verb!function(x) x!
	 	 \end{description}
	 	\item[theta50]\ 
	 	 \begin{description}
	 	 	\item[hyperid] \verb!29150!
	 	 	\item[name] \verb!theta50!
	 	 	\item[short.name] \verb!theta50!
	 	 	\item[initial] \verb!1048576!
	 	 	\item[fixed] \verb!FALSE!
	 	 	\item[prior] \verb!none!
	 	 	\item[param] \verb!!
	 	 	\item[to.theta] \verb!function(x) x!
	 	 	\item[from.theta] \verb!function(x) x!
	 	 \end{description}
	 	\item[theta51]\ 
	 	 \begin{description}
	 	 	\item[hyperid] \verb!29151!
	 	 	\item[name] \verb!theta51!
	 	 	\item[short.name] \verb!theta51!
	 	 	\item[initial] \verb!1048576!
	 	 	\item[fixed] \verb!FALSE!
	 	 	\item[prior] \verb!none!
	 	 	\item[param] \verb!!
	 	 	\item[to.theta] \verb!function(x) x!
	 	 	\item[from.theta] \verb!function(x) x!
	 	 \end{description}
	 	\item[theta52]\ 
	 	 \begin{description}
	 	 	\item[hyperid] \verb!29152!
	 	 	\item[name] \verb!theta52!
	 	 	\item[short.name] \verb!theta52!
	 	 	\item[initial] \verb!1048576!
	 	 	\item[fixed] \verb!FALSE!
	 	 	\item[prior] \verb!none!
	 	 	\item[param] \verb!!
	 	 	\item[to.theta] \verb!function(x) x!
	 	 	\item[from.theta] \verb!function(x) x!
	 	 \end{description}
	 	\item[theta53]\ 
	 	 \begin{description}
	 	 	\item[hyperid] \verb!29153!
	 	 	\item[name] \verb!theta53!
	 	 	\item[short.name] \verb!theta53!
	 	 	\item[initial] \verb!1048576!
	 	 	\item[fixed] \verb!FALSE!
	 	 	\item[prior] \verb!none!
	 	 	\item[param] \verb!!
	 	 	\item[to.theta] \verb!function(x) x!
	 	 	\item[from.theta] \verb!function(x) x!
	 	 \end{description}
	 	\item[theta54]\ 
	 	 \begin{description}
	 	 	\item[hyperid] \verb!29154!
	 	 	\item[name] \verb!theta54!
	 	 	\item[short.name] \verb!theta54!
	 	 	\item[initial] \verb!1048576!
	 	 	\item[fixed] \verb!FALSE!
	 	 	\item[prior] \verb!none!
	 	 	\item[param] \verb!!
	 	 	\item[to.theta] \verb!function(x) x!
	 	 	\item[from.theta] \verb!function(x) x!
	 	 \end{description}
	 	\item[theta55]\ 
	 	 \begin{description}
	 	 	\item[hyperid] \verb!29155!
	 	 	\item[name] \verb!theta55!
	 	 	\item[short.name] \verb!theta55!
	 	 	\item[initial] \verb!1048576!
	 	 	\item[fixed] \verb!FALSE!
	 	 	\item[prior] \verb!none!
	 	 	\item[param] \verb!!
	 	 	\item[to.theta] \verb!function(x) x!
	 	 	\item[from.theta] \verb!function(x) x!
	 	 \end{description}
	 	\item[theta56]\ 
	 	 \begin{description}
	 	 	\item[hyperid] \verb!29156!
	 	 	\item[name] \verb!theta56!
	 	 	\item[short.name] \verb!theta56!
	 	 	\item[initial] \verb!1048576!
	 	 	\item[fixed] \verb!FALSE!
	 	 	\item[prior] \verb!none!
	 	 	\item[param] \verb!!
	 	 	\item[to.theta] \verb!function(x) x!
	 	 	\item[from.theta] \verb!function(x) x!
	 	 \end{description}
	 	\item[theta57]\ 
	 	 \begin{description}
	 	 	\item[hyperid] \verb!29157!
	 	 	\item[name] \verb!theta57!
	 	 	\item[short.name] \verb!theta57!
	 	 	\item[initial] \verb!1048576!
	 	 	\item[fixed] \verb!FALSE!
	 	 	\item[prior] \verb!none!
	 	 	\item[param] \verb!!
	 	 	\item[to.theta] \verb!function(x) x!
	 	 	\item[from.theta] \verb!function(x) x!
	 	 \end{description}
	 	\item[theta58]\ 
	 	 \begin{description}
	 	 	\item[hyperid] \verb!29158!
	 	 	\item[name] \verb!theta58!
	 	 	\item[short.name] \verb!theta58!
	 	 	\item[initial] \verb!1048576!
	 	 	\item[fixed] \verb!FALSE!
	 	 	\item[prior] \verb!none!
	 	 	\item[param] \verb!!
	 	 	\item[to.theta] \verb!function(x) x!
	 	 	\item[from.theta] \verb!function(x) x!
	 	 \end{description}
	 	\item[theta59]\ 
	 	 \begin{description}
	 	 	\item[hyperid] \verb!29159!
	 	 	\item[name] \verb!theta59!
	 	 	\item[short.name] \verb!theta59!
	 	 	\item[initial] \verb!1048576!
	 	 	\item[fixed] \verb!FALSE!
	 	 	\item[prior] \verb!none!
	 	 	\item[param] \verb!!
	 	 	\item[to.theta] \verb!function(x) x!
	 	 	\item[from.theta] \verb!function(x) x!
	 	 \end{description}
	 	\item[theta60]\ 
	 	 \begin{description}
	 	 	\item[hyperid] \verb!29160!
	 	 	\item[name] \verb!theta60!
	 	 	\item[short.name] \verb!theta60!
	 	 	\item[initial] \verb!1048576!
	 	 	\item[fixed] \verb!FALSE!
	 	 	\item[prior] \verb!none!
	 	 	\item[param] \verb!!
	 	 	\item[to.theta] \verb!function(x) x!
	 	 	\item[from.theta] \verb!function(x) x!
	 	 \end{description}
	 	\item[theta61]\ 
	 	 \begin{description}
	 	 	\item[hyperid] \verb!29161!
	 	 	\item[name] \verb!theta61!
	 	 	\item[short.name] \verb!theta61!
	 	 	\item[initial] \verb!1048576!
	 	 	\item[fixed] \verb!FALSE!
	 	 	\item[prior] \verb!none!
	 	 	\item[param] \verb!!
	 	 	\item[to.theta] \verb!function(x) x!
	 	 	\item[from.theta] \verb!function(x) x!
	 	 \end{description}
	 	\item[theta62]\ 
	 	 \begin{description}
	 	 	\item[hyperid] \verb!29162!
	 	 	\item[name] \verb!theta62!
	 	 	\item[short.name] \verb!theta62!
	 	 	\item[initial] \verb!1048576!
	 	 	\item[fixed] \verb!FALSE!
	 	 	\item[prior] \verb!none!
	 	 	\item[param] \verb!!
	 	 	\item[to.theta] \verb!function(x) x!
	 	 	\item[from.theta] \verb!function(x) x!
	 	 \end{description}
	 	\item[theta63]\ 
	 	 \begin{description}
	 	 	\item[hyperid] \verb!29163!
	 	 	\item[name] \verb!theta63!
	 	 	\item[short.name] \verb!theta63!
	 	 	\item[initial] \verb!1048576!
	 	 	\item[fixed] \verb!FALSE!
	 	 	\item[prior] \verb!none!
	 	 	\item[param] \verb!!
	 	 	\item[to.theta] \verb!function(x) x!
	 	 	\item[from.theta] \verb!function(x) x!
	 	 \end{description}
	 	\item[theta64]\ 
	 	 \begin{description}
	 	 	\item[hyperid] \verb!29164!
	 	 	\item[name] \verb!theta64!
	 	 	\item[short.name] \verb!theta64!
	 	 	\item[initial] \verb!1048576!
	 	 	\item[fixed] \verb!FALSE!
	 	 	\item[prior] \verb!none!
	 	 	\item[param] \verb!!
	 	 	\item[to.theta] \verb!function(x) x!
	 	 	\item[from.theta] \verb!function(x) x!
	 	 \end{description}
	 	\item[theta65]\ 
	 	 \begin{description}
	 	 	\item[hyperid] \verb!29165!
	 	 	\item[name] \verb!theta65!
	 	 	\item[short.name] \verb!theta65!
	 	 	\item[initial] \verb!1048576!
	 	 	\item[fixed] \verb!FALSE!
	 	 	\item[prior] \verb!none!
	 	 	\item[param] \verb!!
	 	 	\item[to.theta] \verb!function(x) x!
	 	 	\item[from.theta] \verb!function(x) x!
	 	 \end{description}
	 	\item[theta66]\ 
	 	 \begin{description}
	 	 	\item[hyperid] \verb!29166!
	 	 	\item[name] \verb!theta66!
	 	 	\item[short.name] \verb!theta66!
	 	 	\item[initial] \verb!1048576!
	 	 	\item[fixed] \verb!FALSE!
	 	 	\item[prior] \verb!none!
	 	 	\item[param] \verb!!
	 	 	\item[to.theta] \verb!function(x) x!
	 	 	\item[from.theta] \verb!function(x) x!
	 	 \end{description}
	 	\item[theta67]\ 
	 	 \begin{description}
	 	 	\item[hyperid] \verb!29167!
	 	 	\item[name] \verb!theta67!
	 	 	\item[short.name] \verb!theta67!
	 	 	\item[initial] \verb!1048576!
	 	 	\item[fixed] \verb!FALSE!
	 	 	\item[prior] \verb!none!
	 	 	\item[param] \verb!!
	 	 	\item[to.theta] \verb!function(x) x!
	 	 	\item[from.theta] \verb!function(x) x!
	 	 \end{description}
	 	\item[theta68]\ 
	 	 \begin{description}
	 	 	\item[hyperid] \verb!29168!
	 	 	\item[name] \verb!theta68!
	 	 	\item[short.name] \verb!theta68!
	 	 	\item[initial] \verb!1048576!
	 	 	\item[fixed] \verb!FALSE!
	 	 	\item[prior] \verb!none!
	 	 	\item[param] \verb!!
	 	 	\item[to.theta] \verb!function(x) x!
	 	 	\item[from.theta] \verb!function(x) x!
	 	 \end{description}
	 	\item[theta69]\ 
	 	 \begin{description}
	 	 	\item[hyperid] \verb!29169!
	 	 	\item[name] \verb!theta69!
	 	 	\item[short.name] \verb!theta69!
	 	 	\item[initial] \verb!1048576!
	 	 	\item[fixed] \verb!FALSE!
	 	 	\item[prior] \verb!none!
	 	 	\item[param] \verb!!
	 	 	\item[to.theta] \verb!function(x) x!
	 	 	\item[from.theta] \verb!function(x) x!
	 	 \end{description}
	 	\item[theta70]\ 
	 	 \begin{description}
	 	 	\item[hyperid] \verb!29170!
	 	 	\item[name] \verb!theta70!
	 	 	\item[short.name] \verb!theta70!
	 	 	\item[initial] \verb!1048576!
	 	 	\item[fixed] \verb!FALSE!
	 	 	\item[prior] \verb!none!
	 	 	\item[param] \verb!!
	 	 	\item[to.theta] \verb!function(x) x!
	 	 	\item[from.theta] \verb!function(x) x!
	 	 \end{description}
	 	\item[theta71]\ 
	 	 \begin{description}
	 	 	\item[hyperid] \verb!29171!
	 	 	\item[name] \verb!theta71!
	 	 	\item[short.name] \verb!theta71!
	 	 	\item[initial] \verb!1048576!
	 	 	\item[fixed] \verb!FALSE!
	 	 	\item[prior] \verb!none!
	 	 	\item[param] \verb!!
	 	 	\item[to.theta] \verb!function(x) x!
	 	 	\item[from.theta] \verb!function(x) x!
	 	 \end{description}
	 	\item[theta72]\ 
	 	 \begin{description}
	 	 	\item[hyperid] \verb!29172!
	 	 	\item[name] \verb!theta72!
	 	 	\item[short.name] \verb!theta72!
	 	 	\item[initial] \verb!1048576!
	 	 	\item[fixed] \verb!FALSE!
	 	 	\item[prior] \verb!none!
	 	 	\item[param] \verb!!
	 	 	\item[to.theta] \verb!function(x) x!
	 	 	\item[from.theta] \verb!function(x) x!
	 	 \end{description}
	 	\item[theta73]\ 
	 	 \begin{description}
	 	 	\item[hyperid] \verb!29173!
	 	 	\item[name] \verb!theta73!
	 	 	\item[short.name] \verb!theta73!
	 	 	\item[initial] \verb!1048576!
	 	 	\item[fixed] \verb!FALSE!
	 	 	\item[prior] \verb!none!
	 	 	\item[param] \verb!!
	 	 	\item[to.theta] \verb!function(x) x!
	 	 	\item[from.theta] \verb!function(x) x!
	 	 \end{description}
	 	\item[theta74]\ 
	 	 \begin{description}
	 	 	\item[hyperid] \verb!29174!
	 	 	\item[name] \verb!theta74!
	 	 	\item[short.name] \verb!theta74!
	 	 	\item[initial] \verb!1048576!
	 	 	\item[fixed] \verb!FALSE!
	 	 	\item[prior] \verb!none!
	 	 	\item[param] \verb!!
	 	 	\item[to.theta] \verb!function(x) x!
	 	 	\item[from.theta] \verb!function(x) x!
	 	 \end{description}
	 	\item[theta75]\ 
	 	 \begin{description}
	 	 	\item[hyperid] \verb!29175!
	 	 	\item[name] \verb!theta75!
	 	 	\item[short.name] \verb!theta75!
	 	 	\item[initial] \verb!1048576!
	 	 	\item[fixed] \verb!FALSE!
	 	 	\item[prior] \verb!none!
	 	 	\item[param] \verb!!
	 	 	\item[to.theta] \verb!function(x) x!
	 	 	\item[from.theta] \verb!function(x) x!
	 	 \end{description}
	 	\item[theta76]\ 
	 	 \begin{description}
	 	 	\item[hyperid] \verb!29176!
	 	 	\item[name] \verb!theta76!
	 	 	\item[short.name] \verb!theta76!
	 	 	\item[initial] \verb!1048576!
	 	 	\item[fixed] \verb!FALSE!
	 	 	\item[prior] \verb!none!
	 	 	\item[param] \verb!!
	 	 	\item[to.theta] \verb!function(x) x!
	 	 	\item[from.theta] \verb!function(x) x!
	 	 \end{description}
	 	\item[theta77]\ 
	 	 \begin{description}
	 	 	\item[hyperid] \verb!29177!
	 	 	\item[name] \verb!theta77!
	 	 	\item[short.name] \verb!theta77!
	 	 	\item[initial] \verb!1048576!
	 	 	\item[fixed] \verb!FALSE!
	 	 	\item[prior] \verb!none!
	 	 	\item[param] \verb!!
	 	 	\item[to.theta] \verb!function(x) x!
	 	 	\item[from.theta] \verb!function(x) x!
	 	 \end{description}
	 	\item[theta78]\ 
	 	 \begin{description}
	 	 	\item[hyperid] \verb!29178!
	 	 	\item[name] \verb!theta78!
	 	 	\item[short.name] \verb!theta78!
	 	 	\item[initial] \verb!1048576!
	 	 	\item[fixed] \verb!FALSE!
	 	 	\item[prior] \verb!none!
	 	 	\item[param] \verb!!
	 	 	\item[to.theta] \verb!function(x) x!
	 	 	\item[from.theta] \verb!function(x) x!
	 	 \end{description}
	 	\item[theta79]\ 
	 	 \begin{description}
	 	 	\item[hyperid] \verb!29179!
	 	 	\item[name] \verb!theta79!
	 	 	\item[short.name] \verb!theta79!
	 	 	\item[initial] \verb!1048576!
	 	 	\item[fixed] \verb!FALSE!
	 	 	\item[prior] \verb!none!
	 	 	\item[param] \verb!!
	 	 	\item[to.theta] \verb!function(x) x!
	 	 	\item[from.theta] \verb!function(x) x!
	 	 \end{description}
	 	\item[theta80]\ 
	 	 \begin{description}
	 	 	\item[hyperid] \verb!29180!
	 	 	\item[name] \verb!theta80!
	 	 	\item[short.name] \verb!theta80!
	 	 	\item[initial] \verb!1048576!
	 	 	\item[fixed] \verb!FALSE!
	 	 	\item[prior] \verb!none!
	 	 	\item[param] \verb!!
	 	 	\item[to.theta] \verb!function(x) x!
	 	 	\item[from.theta] \verb!function(x) x!
	 	 \end{description}
	 	\item[theta81]\ 
	 	 \begin{description}
	 	 	\item[hyperid] \verb!29181!
	 	 	\item[name] \verb!theta81!
	 	 	\item[short.name] \verb!theta81!
	 	 	\item[initial] \verb!1048576!
	 	 	\item[fixed] \verb!FALSE!
	 	 	\item[prior] \verb!none!
	 	 	\item[param] \verb!!
	 	 	\item[to.theta] \verb!function(x) x!
	 	 	\item[from.theta] \verb!function(x) x!
	 	 \end{description}
	 	\item[theta82]\ 
	 	 \begin{description}
	 	 	\item[hyperid] \verb!29182!
	 	 	\item[name] \verb!theta82!
	 	 	\item[short.name] \verb!theta82!
	 	 	\item[initial] \verb!1048576!
	 	 	\item[fixed] \verb!FALSE!
	 	 	\item[prior] \verb!none!
	 	 	\item[param] \verb!!
	 	 	\item[to.theta] \verb!function(x) x!
	 	 	\item[from.theta] \verb!function(x) x!
	 	 \end{description}
	 	\item[theta83]\ 
	 	 \begin{description}
	 	 	\item[hyperid] \verb!29183!
	 	 	\item[name] \verb!theta83!
	 	 	\item[short.name] \verb!theta83!
	 	 	\item[initial] \verb!1048576!
	 	 	\item[fixed] \verb!FALSE!
	 	 	\item[prior] \verb!none!
	 	 	\item[param] \verb!!
	 	 	\item[to.theta] \verb!function(x) x!
	 	 	\item[from.theta] \verb!function(x) x!
	 	 \end{description}
	 	\item[theta84]\ 
	 	 \begin{description}
	 	 	\item[hyperid] \verb!29184!
	 	 	\item[name] \verb!theta84!
	 	 	\item[short.name] \verb!theta84!
	 	 	\item[initial] \verb!1048576!
	 	 	\item[fixed] \verb!FALSE!
	 	 	\item[prior] \verb!none!
	 	 	\item[param] \verb!!
	 	 	\item[to.theta] \verb!function(x) x!
	 	 	\item[from.theta] \verb!function(x) x!
	 	 \end{description}
	 	\item[theta85]\ 
	 	 \begin{description}
	 	 	\item[hyperid] \verb!29185!
	 	 	\item[name] \verb!theta85!
	 	 	\item[short.name] \verb!theta85!
	 	 	\item[initial] \verb!1048576!
	 	 	\item[fixed] \verb!FALSE!
	 	 	\item[prior] \verb!none!
	 	 	\item[param] \verb!!
	 	 	\item[to.theta] \verb!function(x) x!
	 	 	\item[from.theta] \verb!function(x) x!
	 	 \end{description}
	 	\item[theta86]\ 
	 	 \begin{description}
	 	 	\item[hyperid] \verb!29186!
	 	 	\item[name] \verb!theta86!
	 	 	\item[short.name] \verb!theta86!
	 	 	\item[initial] \verb!1048576!
	 	 	\item[fixed] \verb!FALSE!
	 	 	\item[prior] \verb!none!
	 	 	\item[param] \verb!!
	 	 	\item[to.theta] \verb!function(x) x!
	 	 	\item[from.theta] \verb!function(x) x!
	 	 \end{description}
	 	\item[theta87]\ 
	 	 \begin{description}
	 	 	\item[hyperid] \verb!29187!
	 	 	\item[name] \verb!theta87!
	 	 	\item[short.name] \verb!theta87!
	 	 	\item[initial] \verb!1048576!
	 	 	\item[fixed] \verb!FALSE!
	 	 	\item[prior] \verb!none!
	 	 	\item[param] \verb!!
	 	 	\item[to.theta] \verb!function(x) x!
	 	 	\item[from.theta] \verb!function(x) x!
	 	 \end{description}
	 	\item[theta88]\ 
	 	 \begin{description}
	 	 	\item[hyperid] \verb!29188!
	 	 	\item[name] \verb!theta88!
	 	 	\item[short.name] \verb!theta88!
	 	 	\item[initial] \verb!1048576!
	 	 	\item[fixed] \verb!FALSE!
	 	 	\item[prior] \verb!none!
	 	 	\item[param] \verb!!
	 	 	\item[to.theta] \verb!function(x) x!
	 	 	\item[from.theta] \verb!function(x) x!
	 	 \end{description}
	 	\item[theta89]\ 
	 	 \begin{description}
	 	 	\item[hyperid] \verb!29189!
	 	 	\item[name] \verb!theta89!
	 	 	\item[short.name] \verb!theta89!
	 	 	\item[initial] \verb!1048576!
	 	 	\item[fixed] \verb!FALSE!
	 	 	\item[prior] \verb!none!
	 	 	\item[param] \verb!!
	 	 	\item[to.theta] \verb!function(x) x!
	 	 	\item[from.theta] \verb!function(x) x!
	 	 \end{description}
	 	\item[theta90]\ 
	 	 \begin{description}
	 	 	\item[hyperid] \verb!29190!
	 	 	\item[name] \verb!theta90!
	 	 	\item[short.name] \verb!theta90!
	 	 	\item[initial] \verb!1048576!
	 	 	\item[fixed] \verb!FALSE!
	 	 	\item[prior] \verb!none!
	 	 	\item[param] \verb!!
	 	 	\item[to.theta] \verb!function(x) x!
	 	 	\item[from.theta] \verb!function(x) x!
	 	 \end{description}
	 	\item[theta91]\ 
	 	 \begin{description}
	 	 	\item[hyperid] \verb!29191!
	 	 	\item[name] \verb!theta91!
	 	 	\item[short.name] \verb!theta91!
	 	 	\item[initial] \verb!1048576!
	 	 	\item[fixed] \verb!FALSE!
	 	 	\item[prior] \verb!none!
	 	 	\item[param] \verb!!
	 	 	\item[to.theta] \verb!function(x) x!
	 	 	\item[from.theta] \verb!function(x) x!
	 	 \end{description}
	 	\item[theta92]\ 
	 	 \begin{description}
	 	 	\item[hyperid] \verb!29192!
	 	 	\item[name] \verb!theta92!
	 	 	\item[short.name] \verb!theta92!
	 	 	\item[initial] \verb!1048576!
	 	 	\item[fixed] \verb!FALSE!
	 	 	\item[prior] \verb!none!
	 	 	\item[param] \verb!!
	 	 	\item[to.theta] \verb!function(x) x!
	 	 	\item[from.theta] \verb!function(x) x!
	 	 \end{description}
	 	\item[theta93]\ 
	 	 \begin{description}
	 	 	\item[hyperid] \verb!29193!
	 	 	\item[name] \verb!theta93!
	 	 	\item[short.name] \verb!theta93!
	 	 	\item[initial] \verb!1048576!
	 	 	\item[fixed] \verb!FALSE!
	 	 	\item[prior] \verb!none!
	 	 	\item[param] \verb!!
	 	 	\item[to.theta] \verb!function(x) x!
	 	 	\item[from.theta] \verb!function(x) x!
	 	 \end{description}
	 	\item[theta94]\ 
	 	 \begin{description}
	 	 	\item[hyperid] \verb!29194!
	 	 	\item[name] \verb!theta94!
	 	 	\item[short.name] \verb!theta94!
	 	 	\item[initial] \verb!1048576!
	 	 	\item[fixed] \verb!FALSE!
	 	 	\item[prior] \verb!none!
	 	 	\item[param] \verb!!
	 	 	\item[to.theta] \verb!function(x) x!
	 	 	\item[from.theta] \verb!function(x) x!
	 	 \end{description}
	 	\item[theta95]\ 
	 	 \begin{description}
	 	 	\item[hyperid] \verb!29195!
	 	 	\item[name] \verb!theta95!
	 	 	\item[short.name] \verb!theta95!
	 	 	\item[initial] \verb!1048576!
	 	 	\item[fixed] \verb!FALSE!
	 	 	\item[prior] \verb!none!
	 	 	\item[param] \verb!!
	 	 	\item[to.theta] \verb!function(x) x!
	 	 	\item[from.theta] \verb!function(x) x!
	 	 \end{description}
	 	\item[theta96]\ 
	 	 \begin{description}
	 	 	\item[hyperid] \verb!29196!
	 	 	\item[name] \verb!theta96!
	 	 	\item[short.name] \verb!theta96!
	 	 	\item[initial] \verb!1048576!
	 	 	\item[fixed] \verb!FALSE!
	 	 	\item[prior] \verb!none!
	 	 	\item[param] \verb!!
	 	 	\item[to.theta] \verb!function(x) x!
	 	 	\item[from.theta] \verb!function(x) x!
	 	 \end{description}
	 	\item[theta97]\ 
	 	 \begin{description}
	 	 	\item[hyperid] \verb!29197!
	 	 	\item[name] \verb!theta97!
	 	 	\item[short.name] \verb!theta97!
	 	 	\item[initial] \verb!1048576!
	 	 	\item[fixed] \verb!FALSE!
	 	 	\item[prior] \verb!none!
	 	 	\item[param] \verb!!
	 	 	\item[to.theta] \verb!function(x) x!
	 	 	\item[from.theta] \verb!function(x) x!
	 	 \end{description}
	 	\item[theta98]\ 
	 	 \begin{description}
	 	 	\item[hyperid] \verb!29198!
	 	 	\item[name] \verb!theta98!
	 	 	\item[short.name] \verb!theta98!
	 	 	\item[initial] \verb!1048576!
	 	 	\item[fixed] \verb!FALSE!
	 	 	\item[prior] \verb!none!
	 	 	\item[param] \verb!!
	 	 	\item[to.theta] \verb!function(x) x!
	 	 	\item[from.theta] \verb!function(x) x!
	 	 \end{description}
	 	\item[theta99]\ 
	 	 \begin{description}
	 	 	\item[hyperid] \verb!29199!
	 	 	\item[name] \verb!theta99!
	 	 	\item[short.name] \verb!theta99!
	 	 	\item[initial] \verb!1048576!
	 	 	\item[fixed] \verb!FALSE!
	 	 	\item[prior] \verb!none!
	 	 	\item[param] \verb!!
	 	 	\item[to.theta] \verb!function(x) x!
	 	 	\item[from.theta] \verb!function(x) x!
	 	 \end{description}
	 	\item[theta100]\ 
	 	 \begin{description}
	 	 	\item[hyperid] \verb!29200!
	 	 	\item[name] \verb!theta100!
	 	 	\item[short.name] \verb!theta100!
	 	 	\item[initial] \verb!1048576!
	 	 	\item[fixed] \verb!FALSE!
	 	 	\item[prior] \verb!none!
	 	 	\item[param] \verb!!
	 	 	\item[to.theta] \verb!function(x) x!
	 	 	\item[from.theta] \verb!function(x) x!
	 	 \end{description}
	 	\item[theta101]\ 
	 	 \begin{description}
	 	 	\item[hyperid] \verb!29201!
	 	 	\item[name] \verb!theta101!
	 	 	\item[short.name] \verb!theta101!
	 	 	\item[initial] \verb!1048576!
	 	 	\item[fixed] \verb!FALSE!
	 	 	\item[prior] \verb!none!
	 	 	\item[param] \verb!!
	 	 	\item[to.theta] \verb!function(x) x!
	 	 	\item[from.theta] \verb!function(x) x!
	 	 \end{description}
	 	\item[theta102]\ 
	 	 \begin{description}
	 	 	\item[hyperid] \verb!29202!
	 	 	\item[name] \verb!theta102!
	 	 	\item[short.name] \verb!theta102!
	 	 	\item[initial] \verb!1048576!
	 	 	\item[fixed] \verb!FALSE!
	 	 	\item[prior] \verb!none!
	 	 	\item[param] \verb!!
	 	 	\item[to.theta] \verb!function(x) x!
	 	 	\item[from.theta] \verb!function(x) x!
	 	 \end{description}
	 	\item[theta103]\ 
	 	 \begin{description}
	 	 	\item[hyperid] \verb!29203!
	 	 	\item[name] \verb!theta103!
	 	 	\item[short.name] \verb!theta103!
	 	 	\item[initial] \verb!1048576!
	 	 	\item[fixed] \verb!FALSE!
	 	 	\item[prior] \verb!none!
	 	 	\item[param] \verb!!
	 	 	\item[to.theta] \verb!function(x) x!
	 	 	\item[from.theta] \verb!function(x) x!
	 	 \end{description}
	 	\item[theta104]\ 
	 	 \begin{description}
	 	 	\item[hyperid] \verb!29204!
	 	 	\item[name] \verb!theta104!
	 	 	\item[short.name] \verb!theta104!
	 	 	\item[initial] \verb!1048576!
	 	 	\item[fixed] \verb!FALSE!
	 	 	\item[prior] \verb!none!
	 	 	\item[param] \verb!!
	 	 	\item[to.theta] \verb!function(x) x!
	 	 	\item[from.theta] \verb!function(x) x!
	 	 \end{description}
	 	\item[theta105]\ 
	 	 \begin{description}
	 	 	\item[hyperid] \verb!29205!
	 	 	\item[name] \verb!theta105!
	 	 	\item[short.name] \verb!theta105!
	 	 	\item[initial] \verb!1048576!
	 	 	\item[fixed] \verb!FALSE!
	 	 	\item[prior] \verb!none!
	 	 	\item[param] \verb!!
	 	 	\item[to.theta] \verb!function(x) x!
	 	 	\item[from.theta] \verb!function(x) x!
	 	 \end{description}
	 	\item[theta106]\ 
	 	 \begin{description}
	 	 	\item[hyperid] \verb!29206!
	 	 	\item[name] \verb!theta106!
	 	 	\item[short.name] \verb!theta106!
	 	 	\item[initial] \verb!1048576!
	 	 	\item[fixed] \verb!FALSE!
	 	 	\item[prior] \verb!none!
	 	 	\item[param] \verb!!
	 	 	\item[to.theta] \verb!function(x) x!
	 	 	\item[from.theta] \verb!function(x) x!
	 	 \end{description}
	 	\item[theta107]\ 
	 	 \begin{description}
	 	 	\item[hyperid] \verb!29207!
	 	 	\item[name] \verb!theta107!
	 	 	\item[short.name] \verb!theta107!
	 	 	\item[initial] \verb!1048576!
	 	 	\item[fixed] \verb!FALSE!
	 	 	\item[prior] \verb!none!
	 	 	\item[param] \verb!!
	 	 	\item[to.theta] \verb!function(x) x!
	 	 	\item[from.theta] \verb!function(x) x!
	 	 \end{description}
	 	\item[theta108]\ 
	 	 \begin{description}
	 	 	\item[hyperid] \verb!29208!
	 	 	\item[name] \verb!theta108!
	 	 	\item[short.name] \verb!theta108!
	 	 	\item[initial] \verb!1048576!
	 	 	\item[fixed] \verb!FALSE!
	 	 	\item[prior] \verb!none!
	 	 	\item[param] \verb!!
	 	 	\item[to.theta] \verb!function(x) x!
	 	 	\item[from.theta] \verb!function(x) x!
	 	 \end{description}
	 	\item[theta109]\ 
	 	 \begin{description}
	 	 	\item[hyperid] \verb!29209!
	 	 	\item[name] \verb!theta109!
	 	 	\item[short.name] \verb!theta109!
	 	 	\item[initial] \verb!1048576!
	 	 	\item[fixed] \verb!FALSE!
	 	 	\item[prior] \verb!none!
	 	 	\item[param] \verb!!
	 	 	\item[to.theta] \verb!function(x) x!
	 	 	\item[from.theta] \verb!function(x) x!
	 	 \end{description}
	 	\item[theta110]\ 
	 	 \begin{description}
	 	 	\item[hyperid] \verb!29210!
	 	 	\item[name] \verb!theta110!
	 	 	\item[short.name] \verb!theta110!
	 	 	\item[initial] \verb!1048576!
	 	 	\item[fixed] \verb!FALSE!
	 	 	\item[prior] \verb!none!
	 	 	\item[param] \verb!!
	 	 	\item[to.theta] \verb!function(x) x!
	 	 	\item[from.theta] \verb!function(x) x!
	 	 \end{description}
	 	\item[theta111]\ 
	 	 \begin{description}
	 	 	\item[hyperid] \verb!29211!
	 	 	\item[name] \verb!theta111!
	 	 	\item[short.name] \verb!theta111!
	 	 	\item[initial] \verb!1048576!
	 	 	\item[fixed] \verb!FALSE!
	 	 	\item[prior] \verb!none!
	 	 	\item[param] \verb!!
	 	 	\item[to.theta] \verb!function(x) x!
	 	 	\item[from.theta] \verb!function(x) x!
	 	 \end{description}
	 	\item[theta112]\ 
	 	 \begin{description}
	 	 	\item[hyperid] \verb!29212!
	 	 	\item[name] \verb!theta112!
	 	 	\item[short.name] \verb!theta112!
	 	 	\item[initial] \verb!1048576!
	 	 	\item[fixed] \verb!FALSE!
	 	 	\item[prior] \verb!none!
	 	 	\item[param] \verb!!
	 	 	\item[to.theta] \verb!function(x) x!
	 	 	\item[from.theta] \verb!function(x) x!
	 	 \end{description}
	 	\item[theta113]\ 
	 	 \begin{description}
	 	 	\item[hyperid] \verb!29213!
	 	 	\item[name] \verb!theta113!
	 	 	\item[short.name] \verb!theta113!
	 	 	\item[initial] \verb!1048576!
	 	 	\item[fixed] \verb!FALSE!
	 	 	\item[prior] \verb!none!
	 	 	\item[param] \verb!!
	 	 	\item[to.theta] \verb!function(x) x!
	 	 	\item[from.theta] \verb!function(x) x!
	 	 \end{description}
	 	\item[theta114]\ 
	 	 \begin{description}
	 	 	\item[hyperid] \verb!29214!
	 	 	\item[name] \verb!theta114!
	 	 	\item[short.name] \verb!theta114!
	 	 	\item[initial] \verb!1048576!
	 	 	\item[fixed] \verb!FALSE!
	 	 	\item[prior] \verb!none!
	 	 	\item[param] \verb!!
	 	 	\item[to.theta] \verb!function(x) x!
	 	 	\item[from.theta] \verb!function(x) x!
	 	 \end{description}
	 	\item[theta115]\ 
	 	 \begin{description}
	 	 	\item[hyperid] \verb!29215!
	 	 	\item[name] \verb!theta115!
	 	 	\item[short.name] \verb!theta115!
	 	 	\item[initial] \verb!1048576!
	 	 	\item[fixed] \verb!FALSE!
	 	 	\item[prior] \verb!none!
	 	 	\item[param] \verb!!
	 	 	\item[to.theta] \verb!function(x) x!
	 	 	\item[from.theta] \verb!function(x) x!
	 	 \end{description}
	 	\item[theta116]\ 
	 	 \begin{description}
	 	 	\item[hyperid] \verb!29216!
	 	 	\item[name] \verb!theta116!
	 	 	\item[short.name] \verb!theta116!
	 	 	\item[initial] \verb!1048576!
	 	 	\item[fixed] \verb!FALSE!
	 	 	\item[prior] \verb!none!
	 	 	\item[param] \verb!!
	 	 	\item[to.theta] \verb!function(x) x!
	 	 	\item[from.theta] \verb!function(x) x!
	 	 \end{description}
	 	\item[theta117]\ 
	 	 \begin{description}
	 	 	\item[hyperid] \verb!29217!
	 	 	\item[name] \verb!theta117!
	 	 	\item[short.name] \verb!theta117!
	 	 	\item[initial] \verb!1048576!
	 	 	\item[fixed] \verb!FALSE!
	 	 	\item[prior] \verb!none!
	 	 	\item[param] \verb!!
	 	 	\item[to.theta] \verb!function(x) x!
	 	 	\item[from.theta] \verb!function(x) x!
	 	 \end{description}
	 	\item[theta118]\ 
	 	 \begin{description}
	 	 	\item[hyperid] \verb!29218!
	 	 	\item[name] \verb!theta118!
	 	 	\item[short.name] \verb!theta118!
	 	 	\item[initial] \verb!1048576!
	 	 	\item[fixed] \verb!FALSE!
	 	 	\item[prior] \verb!none!
	 	 	\item[param] \verb!!
	 	 	\item[to.theta] \verb!function(x) x!
	 	 	\item[from.theta] \verb!function(x) x!
	 	 \end{description}
	 	\item[theta119]\ 
	 	 \begin{description}
	 	 	\item[hyperid] \verb!29219!
	 	 	\item[name] \verb!theta119!
	 	 	\item[short.name] \verb!theta119!
	 	 	\item[initial] \verb!1048576!
	 	 	\item[fixed] \verb!FALSE!
	 	 	\item[prior] \verb!none!
	 	 	\item[param] \verb!!
	 	 	\item[to.theta] \verb!function(x) x!
	 	 	\item[from.theta] \verb!function(x) x!
	 	 \end{description}
	 	\item[theta120]\ 
	 	 \begin{description}
	 	 	\item[hyperid] \verb!29220!
	 	 	\item[name] \verb!theta120!
	 	 	\item[short.name] \verb!theta120!
	 	 	\item[initial] \verb!1048576!
	 	 	\item[fixed] \verb!FALSE!
	 	 	\item[prior] \verb!none!
	 	 	\item[param] \verb!!
	 	 	\item[to.theta] \verb!function(x) x!
	 	 	\item[from.theta] \verb!function(x) x!
	 	 \end{description}
	 	\item[theta121]\ 
	 	 \begin{description}
	 	 	\item[hyperid] \verb!29221!
	 	 	\item[name] \verb!theta121!
	 	 	\item[short.name] \verb!theta121!
	 	 	\item[initial] \verb!1048576!
	 	 	\item[fixed] \verb!FALSE!
	 	 	\item[prior] \verb!none!
	 	 	\item[param] \verb!!
	 	 	\item[to.theta] \verb!function(x) x!
	 	 	\item[from.theta] \verb!function(x) x!
	 	 \end{description}
	 	\item[theta122]\ 
	 	 \begin{description}
	 	 	\item[hyperid] \verb!29222!
	 	 	\item[name] \verb!theta122!
	 	 	\item[short.name] \verb!theta122!
	 	 	\item[initial] \verb!1048576!
	 	 	\item[fixed] \verb!FALSE!
	 	 	\item[prior] \verb!none!
	 	 	\item[param] \verb!!
	 	 	\item[to.theta] \verb!function(x) x!
	 	 	\item[from.theta] \verb!function(x) x!
	 	 \end{description}
	 	\item[theta123]\ 
	 	 \begin{description}
	 	 	\item[hyperid] \verb!29223!
	 	 	\item[name] \verb!theta123!
	 	 	\item[short.name] \verb!theta123!
	 	 	\item[initial] \verb!1048576!
	 	 	\item[fixed] \verb!FALSE!
	 	 	\item[prior] \verb!none!
	 	 	\item[param] \verb!!
	 	 	\item[to.theta] \verb!function(x) x!
	 	 	\item[from.theta] \verb!function(x) x!
	 	 \end{description}
	 	\item[theta124]\ 
	 	 \begin{description}
	 	 	\item[hyperid] \verb!29224!
	 	 	\item[name] \verb!theta124!
	 	 	\item[short.name] \verb!theta124!
	 	 	\item[initial] \verb!1048576!
	 	 	\item[fixed] \verb!FALSE!
	 	 	\item[prior] \verb!none!
	 	 	\item[param] \verb!!
	 	 	\item[to.theta] \verb!function(x) x!
	 	 	\item[from.theta] \verb!function(x) x!
	 	 \end{description}
	 	\item[theta125]\ 
	 	 \begin{description}
	 	 	\item[hyperid] \verb!29225!
	 	 	\item[name] \verb!theta125!
	 	 	\item[short.name] \verb!theta125!
	 	 	\item[initial] \verb!1048576!
	 	 	\item[fixed] \verb!FALSE!
	 	 	\item[prior] \verb!none!
	 	 	\item[param] \verb!!
	 	 	\item[to.theta] \verb!function(x) x!
	 	 	\item[from.theta] \verb!function(x) x!
	 	 \end{description}
	 	\item[theta126]\ 
	 	 \begin{description}
	 	 	\item[hyperid] \verb!29226!
	 	 	\item[name] \verb!theta126!
	 	 	\item[short.name] \verb!theta126!
	 	 	\item[initial] \verb!1048576!
	 	 	\item[fixed] \verb!FALSE!
	 	 	\item[prior] \verb!none!
	 	 	\item[param] \verb!!
	 	 	\item[to.theta] \verb!function(x) x!
	 	 	\item[from.theta] \verb!function(x) x!
	 	 \end{description}
	 	\item[theta127]\ 
	 	 \begin{description}
	 	 	\item[hyperid] \verb!29227!
	 	 	\item[name] \verb!theta127!
	 	 	\item[short.name] \verb!theta127!
	 	 	\item[initial] \verb!1048576!
	 	 	\item[fixed] \verb!FALSE!
	 	 	\item[prior] \verb!none!
	 	 	\item[param] \verb!!
	 	 	\item[to.theta] \verb!function(x) x!
	 	 	\item[from.theta] \verb!function(x) x!
	 	 \end{description}
	 	\item[theta128]\ 
	 	 \begin{description}
	 	 	\item[hyperid] \verb!29228!
	 	 	\item[name] \verb!theta128!
	 	 	\item[short.name] \verb!theta128!
	 	 	\item[initial] \verb!1048576!
	 	 	\item[fixed] \verb!FALSE!
	 	 	\item[prior] \verb!none!
	 	 	\item[param] \verb!!
	 	 	\item[to.theta] \verb!function(x) x!
	 	 	\item[from.theta] \verb!function(x) x!
	 	 \end{description}
	 	\item[theta129]\ 
	 	 \begin{description}
	 	 	\item[hyperid] \verb!29229!
	 	 	\item[name] \verb!theta129!
	 	 	\item[short.name] \verb!theta129!
	 	 	\item[initial] \verb!1048576!
	 	 	\item[fixed] \verb!FALSE!
	 	 	\item[prior] \verb!none!
	 	 	\item[param] \verb!!
	 	 	\item[to.theta] \verb!function(x) x!
	 	 	\item[from.theta] \verb!function(x) x!
	 	 \end{description}
	 	\item[theta130]\ 
	 	 \begin{description}
	 	 	\item[hyperid] \verb!29230!
	 	 	\item[name] \verb!theta130!
	 	 	\item[short.name] \verb!theta130!
	 	 	\item[initial] \verb!1048576!
	 	 	\item[fixed] \verb!FALSE!
	 	 	\item[prior] \verb!none!
	 	 	\item[param] \verb!!
	 	 	\item[to.theta] \verb!function(x) x!
	 	 	\item[from.theta] \verb!function(x) x!
	 	 \end{description}
	 	\item[theta131]\ 
	 	 \begin{description}
	 	 	\item[hyperid] \verb!29231!
	 	 	\item[name] \verb!theta131!
	 	 	\item[short.name] \verb!theta131!
	 	 	\item[initial] \verb!1048576!
	 	 	\item[fixed] \verb!FALSE!
	 	 	\item[prior] \verb!none!
	 	 	\item[param] \verb!!
	 	 	\item[to.theta] \verb!function(x) x!
	 	 	\item[from.theta] \verb!function(x) x!
	 	 \end{description}
	 	\item[theta132]\ 
	 	 \begin{description}
	 	 	\item[hyperid] \verb!29232!
	 	 	\item[name] \verb!theta132!
	 	 	\item[short.name] \verb!theta132!
	 	 	\item[initial] \verb!1048576!
	 	 	\item[fixed] \verb!FALSE!
	 	 	\item[prior] \verb!none!
	 	 	\item[param] \verb!!
	 	 	\item[to.theta] \verb!function(x) x!
	 	 	\item[from.theta] \verb!function(x) x!
	 	 \end{description}
	 	\item[theta133]\ 
	 	 \begin{description}
	 	 	\item[hyperid] \verb!29233!
	 	 	\item[name] \verb!theta133!
	 	 	\item[short.name] \verb!theta133!
	 	 	\item[initial] \verb!1048576!
	 	 	\item[fixed] \verb!FALSE!
	 	 	\item[prior] \verb!none!
	 	 	\item[param] \verb!!
	 	 	\item[to.theta] \verb!function(x) x!
	 	 	\item[from.theta] \verb!function(x) x!
	 	 \end{description}
	 	\item[theta134]\ 
	 	 \begin{description}
	 	 	\item[hyperid] \verb!29234!
	 	 	\item[name] \verb!theta134!
	 	 	\item[short.name] \verb!theta134!
	 	 	\item[initial] \verb!1048576!
	 	 	\item[fixed] \verb!FALSE!
	 	 	\item[prior] \verb!none!
	 	 	\item[param] \verb!!
	 	 	\item[to.theta] \verb!function(x) x!
	 	 	\item[from.theta] \verb!function(x) x!
	 	 \end{description}
	 	\item[theta135]\ 
	 	 \begin{description}
	 	 	\item[hyperid] \verb!29235!
	 	 	\item[name] \verb!theta135!
	 	 	\item[short.name] \verb!theta135!
	 	 	\item[initial] \verb!1048576!
	 	 	\item[fixed] \verb!FALSE!
	 	 	\item[prior] \verb!none!
	 	 	\item[param] \verb!!
	 	 	\item[to.theta] \verb!function(x) x!
	 	 	\item[from.theta] \verb!function(x) x!
	 	 \end{description}
	 	\item[theta136]\ 
	 	 \begin{description}
	 	 	\item[hyperid] \verb!29236!
	 	 	\item[name] \verb!theta136!
	 	 	\item[short.name] \verb!theta136!
	 	 	\item[initial] \verb!1048576!
	 	 	\item[fixed] \verb!FALSE!
	 	 	\item[prior] \verb!none!
	 	 	\item[param] \verb!!
	 	 	\item[to.theta] \verb!function(x) x!
	 	 	\item[from.theta] \verb!function(x) x!
	 	 \end{description}
	 	\item[theta137]\ 
	 	 \begin{description}
	 	 	\item[hyperid] \verb!29237!
	 	 	\item[name] \verb!theta137!
	 	 	\item[short.name] \verb!theta137!
	 	 	\item[initial] \verb!1048576!
	 	 	\item[fixed] \verb!FALSE!
	 	 	\item[prior] \verb!none!
	 	 	\item[param] \verb!!
	 	 	\item[to.theta] \verb!function(x) x!
	 	 	\item[from.theta] \verb!function(x) x!
	 	 \end{description}
	 	\item[theta138]\ 
	 	 \begin{description}
	 	 	\item[hyperid] \verb!29238!
	 	 	\item[name] \verb!theta138!
	 	 	\item[short.name] \verb!theta138!
	 	 	\item[initial] \verb!1048576!
	 	 	\item[fixed] \verb!FALSE!
	 	 	\item[prior] \verb!none!
	 	 	\item[param] \verb!!
	 	 	\item[to.theta] \verb!function(x) x!
	 	 	\item[from.theta] \verb!function(x) x!
	 	 \end{description}
	 	\item[theta139]\ 
	 	 \begin{description}
	 	 	\item[hyperid] \verb!29239!
	 	 	\item[name] \verb!theta139!
	 	 	\item[short.name] \verb!theta139!
	 	 	\item[initial] \verb!1048576!
	 	 	\item[fixed] \verb!FALSE!
	 	 	\item[prior] \verb!none!
	 	 	\item[param] \verb!!
	 	 	\item[to.theta] \verb!function(x) x!
	 	 	\item[from.theta] \verb!function(x) x!
	 	 \end{description}
	 	\item[theta140]\ 
	 	 \begin{description}
	 	 	\item[hyperid] \verb!29240!
	 	 	\item[name] \verb!theta140!
	 	 	\item[short.name] \verb!theta140!
	 	 	\item[initial] \verb!1048576!
	 	 	\item[fixed] \verb!FALSE!
	 	 	\item[prior] \verb!none!
	 	 	\item[param] \verb!!
	 	 	\item[to.theta] \verb!function(x) x!
	 	 	\item[from.theta] \verb!function(x) x!
	 	 \end{description}
	 	\item[theta141]\ 
	 	 \begin{description}
	 	 	\item[hyperid] \verb!29241!
	 	 	\item[name] \verb!theta141!
	 	 	\item[short.name] \verb!theta141!
	 	 	\item[initial] \verb!1048576!
	 	 	\item[fixed] \verb!FALSE!
	 	 	\item[prior] \verb!none!
	 	 	\item[param] \verb!!
	 	 	\item[to.theta] \verb!function(x) x!
	 	 	\item[from.theta] \verb!function(x) x!
	 	 \end{description}
	 	\item[theta142]\ 
	 	 \begin{description}
	 	 	\item[hyperid] \verb!29242!
	 	 	\item[name] \verb!theta142!
	 	 	\item[short.name] \verb!theta142!
	 	 	\item[initial] \verb!1048576!
	 	 	\item[fixed] \verb!FALSE!
	 	 	\item[prior] \verb!none!
	 	 	\item[param] \verb!!
	 	 	\item[to.theta] \verb!function(x) x!
	 	 	\item[from.theta] \verb!function(x) x!
	 	 \end{description}
	 	\item[theta143]\ 
	 	 \begin{description}
	 	 	\item[hyperid] \verb!29243!
	 	 	\item[name] \verb!theta143!
	 	 	\item[short.name] \verb!theta143!
	 	 	\item[initial] \verb!1048576!
	 	 	\item[fixed] \verb!FALSE!
	 	 	\item[prior] \verb!none!
	 	 	\item[param] \verb!!
	 	 	\item[to.theta] \verb!function(x) x!
	 	 	\item[from.theta] \verb!function(x) x!
	 	 \end{description}
	 	\item[theta144]\ 
	 	 \begin{description}
	 	 	\item[hyperid] \verb!29244!
	 	 	\item[name] \verb!theta144!
	 	 	\item[short.name] \verb!theta144!
	 	 	\item[initial] \verb!1048576!
	 	 	\item[fixed] \verb!FALSE!
	 	 	\item[prior] \verb!none!
	 	 	\item[param] \verb!!
	 	 	\item[to.theta] \verb!function(x) x!
	 	 	\item[from.theta] \verb!function(x) x!
	 	 \end{description}
	 	\item[theta145]\ 
	 	 \begin{description}
	 	 	\item[hyperid] \verb!29245!
	 	 	\item[name] \verb!theta145!
	 	 	\item[short.name] \verb!theta145!
	 	 	\item[initial] \verb!1048576!
	 	 	\item[fixed] \verb!FALSE!
	 	 	\item[prior] \verb!none!
	 	 	\item[param] \verb!!
	 	 	\item[to.theta] \verb!function(x) x!
	 	 	\item[from.theta] \verb!function(x) x!
	 	 \end{description}
	 	\item[theta146]\ 
	 	 \begin{description}
	 	 	\item[hyperid] \verb!29246!
	 	 	\item[name] \verb!theta146!
	 	 	\item[short.name] \verb!theta146!
	 	 	\item[initial] \verb!1048576!
	 	 	\item[fixed] \verb!FALSE!
	 	 	\item[prior] \verb!none!
	 	 	\item[param] \verb!!
	 	 	\item[to.theta] \verb!function(x) x!
	 	 	\item[from.theta] \verb!function(x) x!
	 	 \end{description}
	 	\item[theta147]\ 
	 	 \begin{description}
	 	 	\item[hyperid] \verb!29247!
	 	 	\item[name] \verb!theta147!
	 	 	\item[short.name] \verb!theta147!
	 	 	\item[initial] \verb!1048576!
	 	 	\item[fixed] \verb!FALSE!
	 	 	\item[prior] \verb!none!
	 	 	\item[param] \verb!!
	 	 	\item[to.theta] \verb!function(x) x!
	 	 	\item[from.theta] \verb!function(x) x!
	 	 \end{description}
	 	\item[theta148]\ 
	 	 \begin{description}
	 	 	\item[hyperid] \verb!29248!
	 	 	\item[name] \verb!theta148!
	 	 	\item[short.name] \verb!theta148!
	 	 	\item[initial] \verb!1048576!
	 	 	\item[fixed] \verb!FALSE!
	 	 	\item[prior] \verb!none!
	 	 	\item[param] \verb!!
	 	 	\item[to.theta] \verb!function(x) x!
	 	 	\item[from.theta] \verb!function(x) x!
	 	 \end{description}
	 	\item[theta149]\ 
	 	 \begin{description}
	 	 	\item[hyperid] \verb!29249!
	 	 	\item[name] \verb!theta149!
	 	 	\item[short.name] \verb!theta149!
	 	 	\item[initial] \verb!1048576!
	 	 	\item[fixed] \verb!FALSE!
	 	 	\item[prior] \verb!none!
	 	 	\item[param] \verb!!
	 	 	\item[to.theta] \verb!function(x) x!
	 	 	\item[from.theta] \verb!function(x) x!
	 	 \end{description}
	 	\item[theta150]\ 
	 	 \begin{description}
	 	 	\item[hyperid] \verb!29250!
	 	 	\item[name] \verb!theta150!
	 	 	\item[short.name] \verb!theta150!
	 	 	\item[initial] \verb!1048576!
	 	 	\item[fixed] \verb!FALSE!
	 	 	\item[prior] \verb!none!
	 	 	\item[param] \verb!!
	 	 	\item[to.theta] \verb!function(x) x!
	 	 	\item[from.theta] \verb!function(x) x!
	 	 \end{description}
	 	\item[theta151]\ 
	 	 \begin{description}
	 	 	\item[hyperid] \verb!29251!
	 	 	\item[name] \verb!theta151!
	 	 	\item[short.name] \verb!theta151!
	 	 	\item[initial] \verb!1048576!
	 	 	\item[fixed] \verb!FALSE!
	 	 	\item[prior] \verb!none!
	 	 	\item[param] \verb!!
	 	 	\item[to.theta] \verb!function(x) x!
	 	 	\item[from.theta] \verb!function(x) x!
	 	 \end{description}
	 	\item[theta152]\ 
	 	 \begin{description}
	 	 	\item[hyperid] \verb!29252!
	 	 	\item[name] \verb!theta152!
	 	 	\item[short.name] \verb!theta152!
	 	 	\item[initial] \verb!1048576!
	 	 	\item[fixed] \verb!FALSE!
	 	 	\item[prior] \verb!none!
	 	 	\item[param] \verb!!
	 	 	\item[to.theta] \verb!function(x) x!
	 	 	\item[from.theta] \verb!function(x) x!
	 	 \end{description}
	 	\item[theta153]\ 
	 	 \begin{description}
	 	 	\item[hyperid] \verb!29253!
	 	 	\item[name] \verb!theta153!
	 	 	\item[short.name] \verb!theta153!
	 	 	\item[initial] \verb!1048576!
	 	 	\item[fixed] \verb!FALSE!
	 	 	\item[prior] \verb!none!
	 	 	\item[param] \verb!!
	 	 	\item[to.theta] \verb!function(x) x!
	 	 	\item[from.theta] \verb!function(x) x!
	 	 \end{description}
	 	\item[theta154]\ 
	 	 \begin{description}
	 	 	\item[hyperid] \verb!29254!
	 	 	\item[name] \verb!theta154!
	 	 	\item[short.name] \verb!theta154!
	 	 	\item[initial] \verb!1048576!
	 	 	\item[fixed] \verb!FALSE!
	 	 	\item[prior] \verb!none!
	 	 	\item[param] \verb!!
	 	 	\item[to.theta] \verb!function(x) x!
	 	 	\item[from.theta] \verb!function(x) x!
	 	 \end{description}
	 	\item[theta155]\ 
	 	 \begin{description}
	 	 	\item[hyperid] \verb!29255!
	 	 	\item[name] \verb!theta155!
	 	 	\item[short.name] \verb!theta155!
	 	 	\item[initial] \verb!1048576!
	 	 	\item[fixed] \verb!FALSE!
	 	 	\item[prior] \verb!none!
	 	 	\item[param] \verb!!
	 	 	\item[to.theta] \verb!function(x) x!
	 	 	\item[from.theta] \verb!function(x) x!
	 	 \end{description}
	 	\item[theta156]\ 
	 	 \begin{description}
	 	 	\item[hyperid] \verb!29256!
	 	 	\item[name] \verb!theta156!
	 	 	\item[short.name] \verb!theta156!
	 	 	\item[initial] \verb!1048576!
	 	 	\item[fixed] \verb!FALSE!
	 	 	\item[prior] \verb!none!
	 	 	\item[param] \verb!!
	 	 	\item[to.theta] \verb!function(x) x!
	 	 	\item[from.theta] \verb!function(x) x!
	 	 \end{description}
	 	\item[theta157]\ 
	 	 \begin{description}
	 	 	\item[hyperid] \verb!29257!
	 	 	\item[name] \verb!theta157!
	 	 	\item[short.name] \verb!theta157!
	 	 	\item[initial] \verb!1048576!
	 	 	\item[fixed] \verb!FALSE!
	 	 	\item[prior] \verb!none!
	 	 	\item[param] \verb!!
	 	 	\item[to.theta] \verb!function(x) x!
	 	 	\item[from.theta] \verb!function(x) x!
	 	 \end{description}
	 	\item[theta158]\ 
	 	 \begin{description}
	 	 	\item[hyperid] \verb!29258!
	 	 	\item[name] \verb!theta158!
	 	 	\item[short.name] \verb!theta158!
	 	 	\item[initial] \verb!1048576!
	 	 	\item[fixed] \verb!FALSE!
	 	 	\item[prior] \verb!none!
	 	 	\item[param] \verb!!
	 	 	\item[to.theta] \verb!function(x) x!
	 	 	\item[from.theta] \verb!function(x) x!
	 	 \end{description}
	 	\item[theta159]\ 
	 	 \begin{description}
	 	 	\item[hyperid] \verb!29259!
	 	 	\item[name] \verb!theta159!
	 	 	\item[short.name] \verb!theta159!
	 	 	\item[initial] \verb!1048576!
	 	 	\item[fixed] \verb!FALSE!
	 	 	\item[prior] \verb!none!
	 	 	\item[param] \verb!!
	 	 	\item[to.theta] \verb!function(x) x!
	 	 	\item[from.theta] \verb!function(x) x!
	 	 \end{description}
	 	\item[theta160]\ 
	 	 \begin{description}
	 	 	\item[hyperid] \verb!29260!
	 	 	\item[name] \verb!theta160!
	 	 	\item[short.name] \verb!theta160!
	 	 	\item[initial] \verb!1048576!
	 	 	\item[fixed] \verb!FALSE!
	 	 	\item[prior] \verb!none!
	 	 	\item[param] \verb!!
	 	 	\item[to.theta] \verb!function(x) x!
	 	 	\item[from.theta] \verb!function(x) x!
	 	 \end{description}
	 	\item[theta161]\ 
	 	 \begin{description}
	 	 	\item[hyperid] \verb!29261!
	 	 	\item[name] \verb!theta161!
	 	 	\item[short.name] \verb!theta161!
	 	 	\item[initial] \verb!1048576!
	 	 	\item[fixed] \verb!FALSE!
	 	 	\item[prior] \verb!none!
	 	 	\item[param] \verb!!
	 	 	\item[to.theta] \verb!function(x) x!
	 	 	\item[from.theta] \verb!function(x) x!
	 	 \end{description}
	 	\item[theta162]\ 
	 	 \begin{description}
	 	 	\item[hyperid] \verb!29262!
	 	 	\item[name] \verb!theta162!
	 	 	\item[short.name] \verb!theta162!
	 	 	\item[initial] \verb!1048576!
	 	 	\item[fixed] \verb!FALSE!
	 	 	\item[prior] \verb!none!
	 	 	\item[param] \verb!!
	 	 	\item[to.theta] \verb!function(x) x!
	 	 	\item[from.theta] \verb!function(x) x!
	 	 \end{description}
	 	\item[theta163]\ 
	 	 \begin{description}
	 	 	\item[hyperid] \verb!29263!
	 	 	\item[name] \verb!theta163!
	 	 	\item[short.name] \verb!theta163!
	 	 	\item[initial] \verb!1048576!
	 	 	\item[fixed] \verb!FALSE!
	 	 	\item[prior] \verb!none!
	 	 	\item[param] \verb!!
	 	 	\item[to.theta] \verb!function(x) x!
	 	 	\item[from.theta] \verb!function(x) x!
	 	 \end{description}
	 	\item[theta164]\ 
	 	 \begin{description}
	 	 	\item[hyperid] \verb!29264!
	 	 	\item[name] \verb!theta164!
	 	 	\item[short.name] \verb!theta164!
	 	 	\item[initial] \verb!1048576!
	 	 	\item[fixed] \verb!FALSE!
	 	 	\item[prior] \verb!none!
	 	 	\item[param] \verb!!
	 	 	\item[to.theta] \verb!function(x) x!
	 	 	\item[from.theta] \verb!function(x) x!
	 	 \end{description}
	 	\item[theta165]\ 
	 	 \begin{description}
	 	 	\item[hyperid] \verb!29265!
	 	 	\item[name] \verb!theta165!
	 	 	\item[short.name] \verb!theta165!
	 	 	\item[initial] \verb!1048576!
	 	 	\item[fixed] \verb!FALSE!
	 	 	\item[prior] \verb!none!
	 	 	\item[param] \verb!!
	 	 	\item[to.theta] \verb!function(x) x!
	 	 	\item[from.theta] \verb!function(x) x!
	 	 \end{description}
	 	\item[theta166]\ 
	 	 \begin{description}
	 	 	\item[hyperid] \verb!29266!
	 	 	\item[name] \verb!theta166!
	 	 	\item[short.name] \verb!theta166!
	 	 	\item[initial] \verb!1048576!
	 	 	\item[fixed] \verb!FALSE!
	 	 	\item[prior] \verb!none!
	 	 	\item[param] \verb!!
	 	 	\item[to.theta] \verb!function(x) x!
	 	 	\item[from.theta] \verb!function(x) x!
	 	 \end{description}
	 	\item[theta167]\ 
	 	 \begin{description}
	 	 	\item[hyperid] \verb!29267!
	 	 	\item[name] \verb!theta167!
	 	 	\item[short.name] \verb!theta167!
	 	 	\item[initial] \verb!1048576!
	 	 	\item[fixed] \verb!FALSE!
	 	 	\item[prior] \verb!none!
	 	 	\item[param] \verb!!
	 	 	\item[to.theta] \verb!function(x) x!
	 	 	\item[from.theta] \verb!function(x) x!
	 	 \end{description}
	 	\item[theta168]\ 
	 	 \begin{description}
	 	 	\item[hyperid] \verb!29268!
	 	 	\item[name] \verb!theta168!
	 	 	\item[short.name] \verb!theta168!
	 	 	\item[initial] \verb!1048576!
	 	 	\item[fixed] \verb!FALSE!
	 	 	\item[prior] \verb!none!
	 	 	\item[param] \verb!!
	 	 	\item[to.theta] \verb!function(x) x!
	 	 	\item[from.theta] \verb!function(x) x!
	 	 \end{description}
	 	\item[theta169]\ 
	 	 \begin{description}
	 	 	\item[hyperid] \verb!29269!
	 	 	\item[name] \verb!theta169!
	 	 	\item[short.name] \verb!theta169!
	 	 	\item[initial] \verb!1048576!
	 	 	\item[fixed] \verb!FALSE!
	 	 	\item[prior] \verb!none!
	 	 	\item[param] \verb!!
	 	 	\item[to.theta] \verb!function(x) x!
	 	 	\item[from.theta] \verb!function(x) x!
	 	 \end{description}
	 	\item[theta170]\ 
	 	 \begin{description}
	 	 	\item[hyperid] \verb!29270!
	 	 	\item[name] \verb!theta170!
	 	 	\item[short.name] \verb!theta170!
	 	 	\item[initial] \verb!1048576!
	 	 	\item[fixed] \verb!FALSE!
	 	 	\item[prior] \verb!none!
	 	 	\item[param] \verb!!
	 	 	\item[to.theta] \verb!function(x) x!
	 	 	\item[from.theta] \verb!function(x) x!
	 	 \end{description}
	 	\item[theta171]\ 
	 	 \begin{description}
	 	 	\item[hyperid] \verb!29271!
	 	 	\item[name] \verb!theta171!
	 	 	\item[short.name] \verb!theta171!
	 	 	\item[initial] \verb!1048576!
	 	 	\item[fixed] \verb!FALSE!
	 	 	\item[prior] \verb!none!
	 	 	\item[param] \verb!!
	 	 	\item[to.theta] \verb!function(x) x!
	 	 	\item[from.theta] \verb!function(x) x!
	 	 \end{description}
	 	\item[theta172]\ 
	 	 \begin{description}
	 	 	\item[hyperid] \verb!29272!
	 	 	\item[name] \verb!theta172!
	 	 	\item[short.name] \verb!theta172!
	 	 	\item[initial] \verb!1048576!
	 	 	\item[fixed] \verb!FALSE!
	 	 	\item[prior] \verb!none!
	 	 	\item[param] \verb!!
	 	 	\item[to.theta] \verb!function(x) x!
	 	 	\item[from.theta] \verb!function(x) x!
	 	 \end{description}
	 	\item[theta173]\ 
	 	 \begin{description}
	 	 	\item[hyperid] \verb!29273!
	 	 	\item[name] \verb!theta173!
	 	 	\item[short.name] \verb!theta173!
	 	 	\item[initial] \verb!1048576!
	 	 	\item[fixed] \verb!FALSE!
	 	 	\item[prior] \verb!none!
	 	 	\item[param] \verb!!
	 	 	\item[to.theta] \verb!function(x) x!
	 	 	\item[from.theta] \verb!function(x) x!
	 	 \end{description}
	 	\item[theta174]\ 
	 	 \begin{description}
	 	 	\item[hyperid] \verb!29274!
	 	 	\item[name] \verb!theta174!
	 	 	\item[short.name] \verb!theta174!
	 	 	\item[initial] \verb!1048576!
	 	 	\item[fixed] \verb!FALSE!
	 	 	\item[prior] \verb!none!
	 	 	\item[param] \verb!!
	 	 	\item[to.theta] \verb!function(x) x!
	 	 	\item[from.theta] \verb!function(x) x!
	 	 \end{description}
	 	\item[theta175]\ 
	 	 \begin{description}
	 	 	\item[hyperid] \verb!29275!
	 	 	\item[name] \verb!theta175!
	 	 	\item[short.name] \verb!theta175!
	 	 	\item[initial] \verb!1048576!
	 	 	\item[fixed] \verb!FALSE!
	 	 	\item[prior] \verb!none!
	 	 	\item[param] \verb!!
	 	 	\item[to.theta] \verb!function(x) x!
	 	 	\item[from.theta] \verb!function(x) x!
	 	 \end{description}
	 	\item[theta176]\ 
	 	 \begin{description}
	 	 	\item[hyperid] \verb!29276!
	 	 	\item[name] \verb!theta176!
	 	 	\item[short.name] \verb!theta176!
	 	 	\item[initial] \verb!1048576!
	 	 	\item[fixed] \verb!FALSE!
	 	 	\item[prior] \verb!none!
	 	 	\item[param] \verb!!
	 	 	\item[to.theta] \verb!function(x) x!
	 	 	\item[from.theta] \verb!function(x) x!
	 	 \end{description}
	 	\item[theta177]\ 
	 	 \begin{description}
	 	 	\item[hyperid] \verb!29277!
	 	 	\item[name] \verb!theta177!
	 	 	\item[short.name] \verb!theta177!
	 	 	\item[initial] \verb!1048576!
	 	 	\item[fixed] \verb!FALSE!
	 	 	\item[prior] \verb!none!
	 	 	\item[param] \verb!!
	 	 	\item[to.theta] \verb!function(x) x!
	 	 	\item[from.theta] \verb!function(x) x!
	 	 \end{description}
	 	\item[theta178]\ 
	 	 \begin{description}
	 	 	\item[hyperid] \verb!29278!
	 	 	\item[name] \verb!theta178!
	 	 	\item[short.name] \verb!theta178!
	 	 	\item[initial] \verb!1048576!
	 	 	\item[fixed] \verb!FALSE!
	 	 	\item[prior] \verb!none!
	 	 	\item[param] \verb!!
	 	 	\item[to.theta] \verb!function(x) x!
	 	 	\item[from.theta] \verb!function(x) x!
	 	 \end{description}
	 	\item[theta179]\ 
	 	 \begin{description}
	 	 	\item[hyperid] \verb!29279!
	 	 	\item[name] \verb!theta179!
	 	 	\item[short.name] \verb!theta179!
	 	 	\item[initial] \verb!1048576!
	 	 	\item[fixed] \verb!FALSE!
	 	 	\item[prior] \verb!none!
	 	 	\item[param] \verb!!
	 	 	\item[to.theta] \verb!function(x) x!
	 	 	\item[from.theta] \verb!function(x) x!
	 	 \end{description}
	 	\item[theta180]\ 
	 	 \begin{description}
	 	 	\item[hyperid] \verb!29280!
	 	 	\item[name] \verb!theta180!
	 	 	\item[short.name] \verb!theta180!
	 	 	\item[initial] \verb!1048576!
	 	 	\item[fixed] \verb!FALSE!
	 	 	\item[prior] \verb!none!
	 	 	\item[param] \verb!!
	 	 	\item[to.theta] \verb!function(x) x!
	 	 	\item[from.theta] \verb!function(x) x!
	 	 \end{description}
	 	\item[theta181]\ 
	 	 \begin{description}
	 	 	\item[hyperid] \verb!29281!
	 	 	\item[name] \verb!theta181!
	 	 	\item[short.name] \verb!theta181!
	 	 	\item[initial] \verb!1048576!
	 	 	\item[fixed] \verb!FALSE!
	 	 	\item[prior] \verb!none!
	 	 	\item[param] \verb!!
	 	 	\item[to.theta] \verb!function(x) x!
	 	 	\item[from.theta] \verb!function(x) x!
	 	 \end{description}
	 	\item[theta182]\ 
	 	 \begin{description}
	 	 	\item[hyperid] \verb!29282!
	 	 	\item[name] \verb!theta182!
	 	 	\item[short.name] \verb!theta182!
	 	 	\item[initial] \verb!1048576!
	 	 	\item[fixed] \verb!FALSE!
	 	 	\item[prior] \verb!none!
	 	 	\item[param] \verb!!
	 	 	\item[to.theta] \verb!function(x) x!
	 	 	\item[from.theta] \verb!function(x) x!
	 	 \end{description}
	 	\item[theta183]\ 
	 	 \begin{description}
	 	 	\item[hyperid] \verb!29283!
	 	 	\item[name] \verb!theta183!
	 	 	\item[short.name] \verb!theta183!
	 	 	\item[initial] \verb!1048576!
	 	 	\item[fixed] \verb!FALSE!
	 	 	\item[prior] \verb!none!
	 	 	\item[param] \verb!!
	 	 	\item[to.theta] \verb!function(x) x!
	 	 	\item[from.theta] \verb!function(x) x!
	 	 \end{description}
	 	\item[theta184]\ 
	 	 \begin{description}
	 	 	\item[hyperid] \verb!29284!
	 	 	\item[name] \verb!theta184!
	 	 	\item[short.name] \verb!theta184!
	 	 	\item[initial] \verb!1048576!
	 	 	\item[fixed] \verb!FALSE!
	 	 	\item[prior] \verb!none!
	 	 	\item[param] \verb!!
	 	 	\item[to.theta] \verb!function(x) x!
	 	 	\item[from.theta] \verb!function(x) x!
	 	 \end{description}
	 	\item[theta185]\ 
	 	 \begin{description}
	 	 	\item[hyperid] \verb!29285!
	 	 	\item[name] \verb!theta185!
	 	 	\item[short.name] \verb!theta185!
	 	 	\item[initial] \verb!1048576!
	 	 	\item[fixed] \verb!FALSE!
	 	 	\item[prior] \verb!none!
	 	 	\item[param] \verb!!
	 	 	\item[to.theta] \verb!function(x) x!
	 	 	\item[from.theta] \verb!function(x) x!
	 	 \end{description}
	 	\item[theta186]\ 
	 	 \begin{description}
	 	 	\item[hyperid] \verb!29286!
	 	 	\item[name] \verb!theta186!
	 	 	\item[short.name] \verb!theta186!
	 	 	\item[initial] \verb!1048576!
	 	 	\item[fixed] \verb!FALSE!
	 	 	\item[prior] \verb!none!
	 	 	\item[param] \verb!!
	 	 	\item[to.theta] \verb!function(x) x!
	 	 	\item[from.theta] \verb!function(x) x!
	 	 \end{description}
	 	\item[theta187]\ 
	 	 \begin{description}
	 	 	\item[hyperid] \verb!29287!
	 	 	\item[name] \verb!theta187!
	 	 	\item[short.name] \verb!theta187!
	 	 	\item[initial] \verb!1048576!
	 	 	\item[fixed] \verb!FALSE!
	 	 	\item[prior] \verb!none!
	 	 	\item[param] \verb!!
	 	 	\item[to.theta] \verb!function(x) x!
	 	 	\item[from.theta] \verb!function(x) x!
	 	 \end{description}
	 	\item[theta188]\ 
	 	 \begin{description}
	 	 	\item[hyperid] \verb!29288!
	 	 	\item[name] \verb!theta188!
	 	 	\item[short.name] \verb!theta188!
	 	 	\item[initial] \verb!1048576!
	 	 	\item[fixed] \verb!FALSE!
	 	 	\item[prior] \verb!none!
	 	 	\item[param] \verb!!
	 	 	\item[to.theta] \verb!function(x) x!
	 	 	\item[from.theta] \verb!function(x) x!
	 	 \end{description}
	 	\item[theta189]\ 
	 	 \begin{description}
	 	 	\item[hyperid] \verb!29289!
	 	 	\item[name] \verb!theta189!
	 	 	\item[short.name] \verb!theta189!
	 	 	\item[initial] \verb!1048576!
	 	 	\item[fixed] \verb!FALSE!
	 	 	\item[prior] \verb!none!
	 	 	\item[param] \verb!!
	 	 	\item[to.theta] \verb!function(x) x!
	 	 	\item[from.theta] \verb!function(x) x!
	 	 \end{description}
	 	\item[theta190]\ 
	 	 \begin{description}
	 	 	\item[hyperid] \verb!29290!
	 	 	\item[name] \verb!theta190!
	 	 	\item[short.name] \verb!theta190!
	 	 	\item[initial] \verb!1048576!
	 	 	\item[fixed] \verb!FALSE!
	 	 	\item[prior] \verb!none!
	 	 	\item[param] \verb!!
	 	 	\item[to.theta] \verb!function(x) x!
	 	 	\item[from.theta] \verb!function(x) x!
	 	 \end{description}
	 	\item[theta191]\ 
	 	 \begin{description}
	 	 	\item[hyperid] \verb!29291!
	 	 	\item[name] \verb!theta191!
	 	 	\item[short.name] \verb!theta191!
	 	 	\item[initial] \verb!1048576!
	 	 	\item[fixed] \verb!FALSE!
	 	 	\item[prior] \verb!none!
	 	 	\item[param] \verb!!
	 	 	\item[to.theta] \verb!function(x) x!
	 	 	\item[from.theta] \verb!function(x) x!
	 	 \end{description}
	 	\item[theta192]\ 
	 	 \begin{description}
	 	 	\item[hyperid] \verb!29292!
	 	 	\item[name] \verb!theta192!
	 	 	\item[short.name] \verb!theta192!
	 	 	\item[initial] \verb!1048576!
	 	 	\item[fixed] \verb!FALSE!
	 	 	\item[prior] \verb!none!
	 	 	\item[param] \verb!!
	 	 	\item[to.theta] \verb!function(x) x!
	 	 	\item[from.theta] \verb!function(x) x!
	 	 \end{description}
	 	\item[theta193]\ 
	 	 \begin{description}
	 	 	\item[hyperid] \verb!29293!
	 	 	\item[name] \verb!theta193!
	 	 	\item[short.name] \verb!theta193!
	 	 	\item[initial] \verb!1048576!
	 	 	\item[fixed] \verb!FALSE!
	 	 	\item[prior] \verb!none!
	 	 	\item[param] \verb!!
	 	 	\item[to.theta] \verb!function(x) x!
	 	 	\item[from.theta] \verb!function(x) x!
	 	 \end{description}
	 	\item[theta194]\ 
	 	 \begin{description}
	 	 	\item[hyperid] \verb!29294!
	 	 	\item[name] \verb!theta194!
	 	 	\item[short.name] \verb!theta194!
	 	 	\item[initial] \verb!1048576!
	 	 	\item[fixed] \verb!FALSE!
	 	 	\item[prior] \verb!none!
	 	 	\item[param] \verb!!
	 	 	\item[to.theta] \verb!function(x) x!
	 	 	\item[from.theta] \verb!function(x) x!
	 	 \end{description}
	 	\item[theta195]\ 
	 	 \begin{description}
	 	 	\item[hyperid] \verb!29295!
	 	 	\item[name] \verb!theta195!
	 	 	\item[short.name] \verb!theta195!
	 	 	\item[initial] \verb!1048576!
	 	 	\item[fixed] \verb!FALSE!
	 	 	\item[prior] \verb!none!
	 	 	\item[param] \verb!!
	 	 	\item[to.theta] \verb!function(x) x!
	 	 	\item[from.theta] \verb!function(x) x!
	 	 \end{description}
	 	\item[theta196]\ 
	 	 \begin{description}
	 	 	\item[hyperid] \verb!29296!
	 	 	\item[name] \verb!theta196!
	 	 	\item[short.name] \verb!theta196!
	 	 	\item[initial] \verb!1048576!
	 	 	\item[fixed] \verb!FALSE!
	 	 	\item[prior] \verb!none!
	 	 	\item[param] \verb!!
	 	 	\item[to.theta] \verb!function(x) x!
	 	 	\item[from.theta] \verb!function(x) x!
	 	 \end{description}
	 	\item[theta197]\ 
	 	 \begin{description}
	 	 	\item[hyperid] \verb!29297!
	 	 	\item[name] \verb!theta197!
	 	 	\item[short.name] \verb!theta197!
	 	 	\item[initial] \verb!1048576!
	 	 	\item[fixed] \verb!FALSE!
	 	 	\item[prior] \verb!none!
	 	 	\item[param] \verb!!
	 	 	\item[to.theta] \verb!function(x) x!
	 	 	\item[from.theta] \verb!function(x) x!
	 	 \end{description}
	 	\item[theta198]\ 
	 	 \begin{description}
	 	 	\item[hyperid] \verb!29298!
	 	 	\item[name] \verb!theta198!
	 	 	\item[short.name] \verb!theta198!
	 	 	\item[initial] \verb!1048576!
	 	 	\item[fixed] \verb!FALSE!
	 	 	\item[prior] \verb!none!
	 	 	\item[param] \verb!!
	 	 	\item[to.theta] \verb!function(x) x!
	 	 	\item[from.theta] \verb!function(x) x!
	 	 \end{description}
	 	\item[theta199]\ 
	 	 \begin{description}
	 	 	\item[hyperid] \verb!29299!
	 	 	\item[name] \verb!theta199!
	 	 	\item[short.name] \verb!theta199!
	 	 	\item[initial] \verb!1048576!
	 	 	\item[fixed] \verb!FALSE!
	 	 	\item[prior] \verb!none!
	 	 	\item[param] \verb!!
	 	 	\item[to.theta] \verb!function(x) x!
	 	 	\item[from.theta] \verb!function(x) x!
	 	 \end{description}
	 	\item[theta200]\ 
	 	 \begin{description}
	 	 	\item[hyperid] \verb!29300!
	 	 	\item[name] \verb!theta200!
	 	 	\item[short.name] \verb!theta200!
	 	 	\item[initial] \verb!1048576!
	 	 	\item[fixed] \verb!FALSE!
	 	 	\item[prior] \verb!none!
	 	 	\item[param] \verb!!
	 	 	\item[to.theta] \verb!function(x) x!
	 	 	\item[from.theta] \verb!function(x) x!
	 	 \end{description}
	 	\item[theta201]\ 
	 	 \begin{description}
	 	 	\item[hyperid] \verb!29301!
	 	 	\item[name] \verb!theta201!
	 	 	\item[short.name] \verb!theta201!
	 	 	\item[initial] \verb!1048576!
	 	 	\item[fixed] \verb!FALSE!
	 	 	\item[prior] \verb!none!
	 	 	\item[param] \verb!!
	 	 	\item[to.theta] \verb!function(x) x!
	 	 	\item[from.theta] \verb!function(x) x!
	 	 \end{description}
	 	\item[theta202]\ 
	 	 \begin{description}
	 	 	\item[hyperid] \verb!29302!
	 	 	\item[name] \verb!theta202!
	 	 	\item[short.name] \verb!theta202!
	 	 	\item[initial] \verb!1048576!
	 	 	\item[fixed] \verb!FALSE!
	 	 	\item[prior] \verb!none!
	 	 	\item[param] \verb!!
	 	 	\item[to.theta] \verb!function(x) x!
	 	 	\item[from.theta] \verb!function(x) x!
	 	 \end{description}
	 	\item[theta203]\ 
	 	 \begin{description}
	 	 	\item[hyperid] \verb!29303!
	 	 	\item[name] \verb!theta203!
	 	 	\item[short.name] \verb!theta203!
	 	 	\item[initial] \verb!1048576!
	 	 	\item[fixed] \verb!FALSE!
	 	 	\item[prior] \verb!none!
	 	 	\item[param] \verb!!
	 	 	\item[to.theta] \verb!function(x) x!
	 	 	\item[from.theta] \verb!function(x) x!
	 	 \end{description}
	 	\item[theta204]\ 
	 	 \begin{description}
	 	 	\item[hyperid] \verb!29304!
	 	 	\item[name] \verb!theta204!
	 	 	\item[short.name] \verb!theta204!
	 	 	\item[initial] \verb!1048576!
	 	 	\item[fixed] \verb!FALSE!
	 	 	\item[prior] \verb!none!
	 	 	\item[param] \verb!!
	 	 	\item[to.theta] \verb!function(x) x!
	 	 	\item[from.theta] \verb!function(x) x!
	 	 \end{description}
	 	\item[theta205]\ 
	 	 \begin{description}
	 	 	\item[hyperid] \verb!29305!
	 	 	\item[name] \verb!theta205!
	 	 	\item[short.name] \verb!theta205!
	 	 	\item[initial] \verb!1048576!
	 	 	\item[fixed] \verb!FALSE!
	 	 	\item[prior] \verb!none!
	 	 	\item[param] \verb!!
	 	 	\item[to.theta] \verb!function(x) x!
	 	 	\item[from.theta] \verb!function(x) x!
	 	 \end{description}
	 	\item[theta206]\ 
	 	 \begin{description}
	 	 	\item[hyperid] \verb!29306!
	 	 	\item[name] \verb!theta206!
	 	 	\item[short.name] \verb!theta206!
	 	 	\item[initial] \verb!1048576!
	 	 	\item[fixed] \verb!FALSE!
	 	 	\item[prior] \verb!none!
	 	 	\item[param] \verb!!
	 	 	\item[to.theta] \verb!function(x) x!
	 	 	\item[from.theta] \verb!function(x) x!
	 	 \end{description}
	 	\item[theta207]\ 
	 	 \begin{description}
	 	 	\item[hyperid] \verb!29307!
	 	 	\item[name] \verb!theta207!
	 	 	\item[short.name] \verb!theta207!
	 	 	\item[initial] \verb!1048576!
	 	 	\item[fixed] \verb!FALSE!
	 	 	\item[prior] \verb!none!
	 	 	\item[param] \verb!!
	 	 	\item[to.theta] \verb!function(x) x!
	 	 	\item[from.theta] \verb!function(x) x!
	 	 \end{description}
	 	\item[theta208]\ 
	 	 \begin{description}
	 	 	\item[hyperid] \verb!29308!
	 	 	\item[name] \verb!theta208!
	 	 	\item[short.name] \verb!theta208!
	 	 	\item[initial] \verb!1048576!
	 	 	\item[fixed] \verb!FALSE!
	 	 	\item[prior] \verb!none!
	 	 	\item[param] \verb!!
	 	 	\item[to.theta] \verb!function(x) x!
	 	 	\item[from.theta] \verb!function(x) x!
	 	 \end{description}
	 	\item[theta209]\ 
	 	 \begin{description}
	 	 	\item[hyperid] \verb!29309!
	 	 	\item[name] \verb!theta209!
	 	 	\item[short.name] \verb!theta209!
	 	 	\item[initial] \verb!1048576!
	 	 	\item[fixed] \verb!FALSE!
	 	 	\item[prior] \verb!none!
	 	 	\item[param] \verb!!
	 	 	\item[to.theta] \verb!function(x) x!
	 	 	\item[from.theta] \verb!function(x) x!
	 	 \end{description}
	 	\item[theta210]\ 
	 	 \begin{description}
	 	 	\item[hyperid] \verb!29310!
	 	 	\item[name] \verb!theta210!
	 	 	\item[short.name] \verb!theta210!
	 	 	\item[initial] \verb!1048576!
	 	 	\item[fixed] \verb!FALSE!
	 	 	\item[prior] \verb!none!
	 	 	\item[param] \verb!!
	 	 	\item[to.theta] \verb!function(x) x!
	 	 	\item[from.theta] \verb!function(x) x!
	 	 \end{description}
	 	\item[theta211]\ 
	 	 \begin{description}
	 	 	\item[hyperid] \verb!29311!
	 	 	\item[name] \verb!theta211!
	 	 	\item[short.name] \verb!theta211!
	 	 	\item[initial] \verb!1048576!
	 	 	\item[fixed] \verb!FALSE!
	 	 	\item[prior] \verb!none!
	 	 	\item[param] \verb!!
	 	 	\item[to.theta] \verb!function(x) x!
	 	 	\item[from.theta] \verb!function(x) x!
	 	 \end{description}
	 	\item[theta212]\ 
	 	 \begin{description}
	 	 	\item[hyperid] \verb!29312!
	 	 	\item[name] \verb!theta212!
	 	 	\item[short.name] \verb!theta212!
	 	 	\item[initial] \verb!1048576!
	 	 	\item[fixed] \verb!FALSE!
	 	 	\item[prior] \verb!none!
	 	 	\item[param] \verb!!
	 	 	\item[to.theta] \verb!function(x) x!
	 	 	\item[from.theta] \verb!function(x) x!
	 	 \end{description}
	 	\item[theta213]\ 
	 	 \begin{description}
	 	 	\item[hyperid] \verb!29313!
	 	 	\item[name] \verb!theta213!
	 	 	\item[short.name] \verb!theta213!
	 	 	\item[initial] \verb!1048576!
	 	 	\item[fixed] \verb!FALSE!
	 	 	\item[prior] \verb!none!
	 	 	\item[param] \verb!!
	 	 	\item[to.theta] \verb!function(x) x!
	 	 	\item[from.theta] \verb!function(x) x!
	 	 \end{description}
	 	\item[theta214]\ 
	 	 \begin{description}
	 	 	\item[hyperid] \verb!29314!
	 	 	\item[name] \verb!theta214!
	 	 	\item[short.name] \verb!theta214!
	 	 	\item[initial] \verb!1048576!
	 	 	\item[fixed] \verb!FALSE!
	 	 	\item[prior] \verb!none!
	 	 	\item[param] \verb!!
	 	 	\item[to.theta] \verb!function(x) x!
	 	 	\item[from.theta] \verb!function(x) x!
	 	 \end{description}
	 	\item[theta215]\ 
	 	 \begin{description}
	 	 	\item[hyperid] \verb!29315!
	 	 	\item[name] \verb!theta215!
	 	 	\item[short.name] \verb!theta215!
	 	 	\item[initial] \verb!1048576!
	 	 	\item[fixed] \verb!FALSE!
	 	 	\item[prior] \verb!none!
	 	 	\item[param] \verb!!
	 	 	\item[to.theta] \verb!function(x) x!
	 	 	\item[from.theta] \verb!function(x) x!
	 	 \end{description}
	 	\item[theta216]\ 
	 	 \begin{description}
	 	 	\item[hyperid] \verb!29316!
	 	 	\item[name] \verb!theta216!
	 	 	\item[short.name] \verb!theta216!
	 	 	\item[initial] \verb!1048576!
	 	 	\item[fixed] \verb!FALSE!
	 	 	\item[prior] \verb!none!
	 	 	\item[param] \verb!!
	 	 	\item[to.theta] \verb!function(x) x!
	 	 	\item[from.theta] \verb!function(x) x!
	 	 \end{description}
	 	\item[theta217]\ 
	 	 \begin{description}
	 	 	\item[hyperid] \verb!29317!
	 	 	\item[name] \verb!theta217!
	 	 	\item[short.name] \verb!theta217!
	 	 	\item[initial] \verb!1048576!
	 	 	\item[fixed] \verb!FALSE!
	 	 	\item[prior] \verb!none!
	 	 	\item[param] \verb!!
	 	 	\item[to.theta] \verb!function(x) x!
	 	 	\item[from.theta] \verb!function(x) x!
	 	 \end{description}
	 	\item[theta218]\ 
	 	 \begin{description}
	 	 	\item[hyperid] \verb!29318!
	 	 	\item[name] \verb!theta218!
	 	 	\item[short.name] \verb!theta218!
	 	 	\item[initial] \verb!1048576!
	 	 	\item[fixed] \verb!FALSE!
	 	 	\item[prior] \verb!none!
	 	 	\item[param] \verb!!
	 	 	\item[to.theta] \verb!function(x) x!
	 	 	\item[from.theta] \verb!function(x) x!
	 	 \end{description}
	 	\item[theta219]\ 
	 	 \begin{description}
	 	 	\item[hyperid] \verb!29319!
	 	 	\item[name] \verb!theta219!
	 	 	\item[short.name] \verb!theta219!
	 	 	\item[initial] \verb!1048576!
	 	 	\item[fixed] \verb!FALSE!
	 	 	\item[prior] \verb!none!
	 	 	\item[param] \verb!!
	 	 	\item[to.theta] \verb!function(x) x!
	 	 	\item[from.theta] \verb!function(x) x!
	 	 \end{description}
	 	\item[theta220]\ 
	 	 \begin{description}
	 	 	\item[hyperid] \verb!29320!
	 	 	\item[name] \verb!theta220!
	 	 	\item[short.name] \verb!theta220!
	 	 	\item[initial] \verb!1048576!
	 	 	\item[fixed] \verb!FALSE!
	 	 	\item[prior] \verb!none!
	 	 	\item[param] \verb!!
	 	 	\item[to.theta] \verb!function(x) x!
	 	 	\item[from.theta] \verb!function(x) x!
	 	 \end{description}
	 	\item[theta221]\ 
	 	 \begin{description}
	 	 	\item[hyperid] \verb!29321!
	 	 	\item[name] \verb!theta221!
	 	 	\item[short.name] \verb!theta221!
	 	 	\item[initial] \verb!1048576!
	 	 	\item[fixed] \verb!FALSE!
	 	 	\item[prior] \verb!none!
	 	 	\item[param] \verb!!
	 	 	\item[to.theta] \verb!function(x) x!
	 	 	\item[from.theta] \verb!function(x) x!
	 	 \end{description}
	 	\item[theta222]\ 
	 	 \begin{description}
	 	 	\item[hyperid] \verb!29322!
	 	 	\item[name] \verb!theta222!
	 	 	\item[short.name] \verb!theta222!
	 	 	\item[initial] \verb!1048576!
	 	 	\item[fixed] \verb!FALSE!
	 	 	\item[prior] \verb!none!
	 	 	\item[param] \verb!!
	 	 	\item[to.theta] \verb!function(x) x!
	 	 	\item[from.theta] \verb!function(x) x!
	 	 \end{description}
	 	\item[theta223]\ 
	 	 \begin{description}
	 	 	\item[hyperid] \verb!29323!
	 	 	\item[name] \verb!theta223!
	 	 	\item[short.name] \verb!theta223!
	 	 	\item[initial] \verb!1048576!
	 	 	\item[fixed] \verb!FALSE!
	 	 	\item[prior] \verb!none!
	 	 	\item[param] \verb!!
	 	 	\item[to.theta] \verb!function(x) x!
	 	 	\item[from.theta] \verb!function(x) x!
	 	 \end{description}
	 	\item[theta224]\ 
	 	 \begin{description}
	 	 	\item[hyperid] \verb!29324!
	 	 	\item[name] \verb!theta224!
	 	 	\item[short.name] \verb!theta224!
	 	 	\item[initial] \verb!1048576!
	 	 	\item[fixed] \verb!FALSE!
	 	 	\item[prior] \verb!none!
	 	 	\item[param] \verb!!
	 	 	\item[to.theta] \verb!function(x) x!
	 	 	\item[from.theta] \verb!function(x) x!
	 	 \end{description}
	 	\item[theta225]\ 
	 	 \begin{description}
	 	 	\item[hyperid] \verb!29325!
	 	 	\item[name] \verb!theta225!
	 	 	\item[short.name] \verb!theta225!
	 	 	\item[initial] \verb!1048576!
	 	 	\item[fixed] \verb!FALSE!
	 	 	\item[prior] \verb!none!
	 	 	\item[param] \verb!!
	 	 	\item[to.theta] \verb!function(x) x!
	 	 	\item[from.theta] \verb!function(x) x!
	 	 \end{description}
	 	\item[theta226]\ 
	 	 \begin{description}
	 	 	\item[hyperid] \verb!29326!
	 	 	\item[name] \verb!theta226!
	 	 	\item[short.name] \verb!theta226!
	 	 	\item[initial] \verb!1048576!
	 	 	\item[fixed] \verb!FALSE!
	 	 	\item[prior] \verb!none!
	 	 	\item[param] \verb!!
	 	 	\item[to.theta] \verb!function(x) x!
	 	 	\item[from.theta] \verb!function(x) x!
	 	 \end{description}
	 	\item[theta227]\ 
	 	 \begin{description}
	 	 	\item[hyperid] \verb!29327!
	 	 	\item[name] \verb!theta227!
	 	 	\item[short.name] \verb!theta227!
	 	 	\item[initial] \verb!1048576!
	 	 	\item[fixed] \verb!FALSE!
	 	 	\item[prior] \verb!none!
	 	 	\item[param] \verb!!
	 	 	\item[to.theta] \verb!function(x) x!
	 	 	\item[from.theta] \verb!function(x) x!
	 	 \end{description}
	 	\item[theta228]\ 
	 	 \begin{description}
	 	 	\item[hyperid] \verb!29328!
	 	 	\item[name] \verb!theta228!
	 	 	\item[short.name] \verb!theta228!
	 	 	\item[initial] \verb!1048576!
	 	 	\item[fixed] \verb!FALSE!
	 	 	\item[prior] \verb!none!
	 	 	\item[param] \verb!!
	 	 	\item[to.theta] \verb!function(x) x!
	 	 	\item[from.theta] \verb!function(x) x!
	 	 \end{description}
	 	\item[theta229]\ 
	 	 \begin{description}
	 	 	\item[hyperid] \verb!29329!
	 	 	\item[name] \verb!theta229!
	 	 	\item[short.name] \verb!theta229!
	 	 	\item[initial] \verb!1048576!
	 	 	\item[fixed] \verb!FALSE!
	 	 	\item[prior] \verb!none!
	 	 	\item[param] \verb!!
	 	 	\item[to.theta] \verb!function(x) x!
	 	 	\item[from.theta] \verb!function(x) x!
	 	 \end{description}
	 	\item[theta230]\ 
	 	 \begin{description}
	 	 	\item[hyperid] \verb!29330!
	 	 	\item[name] \verb!theta230!
	 	 	\item[short.name] \verb!theta230!
	 	 	\item[initial] \verb!1048576!
	 	 	\item[fixed] \verb!FALSE!
	 	 	\item[prior] \verb!none!
	 	 	\item[param] \verb!!
	 	 	\item[to.theta] \verb!function(x) x!
	 	 	\item[from.theta] \verb!function(x) x!
	 	 \end{description}
	 	\item[theta231]\ 
	 	 \begin{description}
	 	 	\item[hyperid] \verb!29331!
	 	 	\item[name] \verb!theta231!
	 	 	\item[short.name] \verb!theta231!
	 	 	\item[initial] \verb!1048576!
	 	 	\item[fixed] \verb!FALSE!
	 	 	\item[prior] \verb!none!
	 	 	\item[param] \verb!!
	 	 	\item[to.theta] \verb!function(x) x!
	 	 	\item[from.theta] \verb!function(x) x!
	 	 \end{description}
	 	\item[theta232]\ 
	 	 \begin{description}
	 	 	\item[hyperid] \verb!29332!
	 	 	\item[name] \verb!theta232!
	 	 	\item[short.name] \verb!theta232!
	 	 	\item[initial] \verb!1048576!
	 	 	\item[fixed] \verb!FALSE!
	 	 	\item[prior] \verb!none!
	 	 	\item[param] \verb!!
	 	 	\item[to.theta] \verb!function(x) x!
	 	 	\item[from.theta] \verb!function(x) x!
	 	 \end{description}
	 	\item[theta233]\ 
	 	 \begin{description}
	 	 	\item[hyperid] \verb!29333!
	 	 	\item[name] \verb!theta233!
	 	 	\item[short.name] \verb!theta233!
	 	 	\item[initial] \verb!1048576!
	 	 	\item[fixed] \verb!FALSE!
	 	 	\item[prior] \verb!none!
	 	 	\item[param] \verb!!
	 	 	\item[to.theta] \verb!function(x) x!
	 	 	\item[from.theta] \verb!function(x) x!
	 	 \end{description}
	 	\item[theta234]\ 
	 	 \begin{description}
	 	 	\item[hyperid] \verb!29334!
	 	 	\item[name] \verb!theta234!
	 	 	\item[short.name] \verb!theta234!
	 	 	\item[initial] \verb!1048576!
	 	 	\item[fixed] \verb!FALSE!
	 	 	\item[prior] \verb!none!
	 	 	\item[param] \verb!!
	 	 	\item[to.theta] \verb!function(x) x!
	 	 	\item[from.theta] \verb!function(x) x!
	 	 \end{description}
	 	\item[theta235]\ 
	 	 \begin{description}
	 	 	\item[hyperid] \verb!29335!
	 	 	\item[name] \verb!theta235!
	 	 	\item[short.name] \verb!theta235!
	 	 	\item[initial] \verb!1048576!
	 	 	\item[fixed] \verb!FALSE!
	 	 	\item[prior] \verb!none!
	 	 	\item[param] \verb!!
	 	 	\item[to.theta] \verb!function(x) x!
	 	 	\item[from.theta] \verb!function(x) x!
	 	 \end{description}
	 	\item[theta236]\ 
	 	 \begin{description}
	 	 	\item[hyperid] \verb!29336!
	 	 	\item[name] \verb!theta236!
	 	 	\item[short.name] \verb!theta236!
	 	 	\item[initial] \verb!1048576!
	 	 	\item[fixed] \verb!FALSE!
	 	 	\item[prior] \verb!none!
	 	 	\item[param] \verb!!
	 	 	\item[to.theta] \verb!function(x) x!
	 	 	\item[from.theta] \verb!function(x) x!
	 	 \end{description}
	 	\item[theta237]\ 
	 	 \begin{description}
	 	 	\item[hyperid] \verb!29337!
	 	 	\item[name] \verb!theta237!
	 	 	\item[short.name] \verb!theta237!
	 	 	\item[initial] \verb!1048576!
	 	 	\item[fixed] \verb!FALSE!
	 	 	\item[prior] \verb!none!
	 	 	\item[param] \verb!!
	 	 	\item[to.theta] \verb!function(x) x!
	 	 	\item[from.theta] \verb!function(x) x!
	 	 \end{description}
	 	\item[theta238]\ 
	 	 \begin{description}
	 	 	\item[hyperid] \verb!29338!
	 	 	\item[name] \verb!theta238!
	 	 	\item[short.name] \verb!theta238!
	 	 	\item[initial] \verb!1048576!
	 	 	\item[fixed] \verb!FALSE!
	 	 	\item[prior] \verb!none!
	 	 	\item[param] \verb!!
	 	 	\item[to.theta] \verb!function(x) x!
	 	 	\item[from.theta] \verb!function(x) x!
	 	 \end{description}
	 	\item[theta239]\ 
	 	 \begin{description}
	 	 	\item[hyperid] \verb!29339!
	 	 	\item[name] \verb!theta239!
	 	 	\item[short.name] \verb!theta239!
	 	 	\item[initial] \verb!1048576!
	 	 	\item[fixed] \verb!FALSE!
	 	 	\item[prior] \verb!none!
	 	 	\item[param] \verb!!
	 	 	\item[to.theta] \verb!function(x) x!
	 	 	\item[from.theta] \verb!function(x) x!
	 	 \end{description}
	 	\item[theta240]\ 
	 	 \begin{description}
	 	 	\item[hyperid] \verb!29340!
	 	 	\item[name] \verb!theta240!
	 	 	\item[short.name] \verb!theta240!
	 	 	\item[initial] \verb!1048576!
	 	 	\item[fixed] \verb!FALSE!
	 	 	\item[prior] \verb!none!
	 	 	\item[param] \verb!!
	 	 	\item[to.theta] \verb!function(x) x!
	 	 	\item[from.theta] \verb!function(x) x!
	 	 \end{description}
	 	\item[theta241]\ 
	 	 \begin{description}
	 	 	\item[hyperid] \verb!29341!
	 	 	\item[name] \verb!theta241!
	 	 	\item[short.name] \verb!theta241!
	 	 	\item[initial] \verb!1048576!
	 	 	\item[fixed] \verb!FALSE!
	 	 	\item[prior] \verb!none!
	 	 	\item[param] \verb!!
	 	 	\item[to.theta] \verb!function(x) x!
	 	 	\item[from.theta] \verb!function(x) x!
	 	 \end{description}
	 	\item[theta242]\ 
	 	 \begin{description}
	 	 	\item[hyperid] \verb!29342!
	 	 	\item[name] \verb!theta242!
	 	 	\item[short.name] \verb!theta242!
	 	 	\item[initial] \verb!1048576!
	 	 	\item[fixed] \verb!FALSE!
	 	 	\item[prior] \verb!none!
	 	 	\item[param] \verb!!
	 	 	\item[to.theta] \verb!function(x) x!
	 	 	\item[from.theta] \verb!function(x) x!
	 	 \end{description}
	 	\item[theta243]\ 
	 	 \begin{description}
	 	 	\item[hyperid] \verb!29343!
	 	 	\item[name] \verb!theta243!
	 	 	\item[short.name] \verb!theta243!
	 	 	\item[initial] \verb!1048576!
	 	 	\item[fixed] \verb!FALSE!
	 	 	\item[prior] \verb!none!
	 	 	\item[param] \verb!!
	 	 	\item[to.theta] \verb!function(x) x!
	 	 	\item[from.theta] \verb!function(x) x!
	 	 \end{description}
	 	\item[theta244]\ 
	 	 \begin{description}
	 	 	\item[hyperid] \verb!29344!
	 	 	\item[name] \verb!theta244!
	 	 	\item[short.name] \verb!theta244!
	 	 	\item[initial] \verb!1048576!
	 	 	\item[fixed] \verb!FALSE!
	 	 	\item[prior] \verb!none!
	 	 	\item[param] \verb!!
	 	 	\item[to.theta] \verb!function(x) x!
	 	 	\item[from.theta] \verb!function(x) x!
	 	 \end{description}
	 	\item[theta245]\ 
	 	 \begin{description}
	 	 	\item[hyperid] \verb!29345!
	 	 	\item[name] \verb!theta245!
	 	 	\item[short.name] \verb!theta245!
	 	 	\item[initial] \verb!1048576!
	 	 	\item[fixed] \verb!FALSE!
	 	 	\item[prior] \verb!none!
	 	 	\item[param] \verb!!
	 	 	\item[to.theta] \verb!function(x) x!
	 	 	\item[from.theta] \verb!function(x) x!
	 	 \end{description}
	 	\item[theta246]\ 
	 	 \begin{description}
	 	 	\item[hyperid] \verb!29346!
	 	 	\item[name] \verb!theta246!
	 	 	\item[short.name] \verb!theta246!
	 	 	\item[initial] \verb!1048576!
	 	 	\item[fixed] \verb!FALSE!
	 	 	\item[prior] \verb!none!
	 	 	\item[param] \verb!!
	 	 	\item[to.theta] \verb!function(x) x!
	 	 	\item[from.theta] \verb!function(x) x!
	 	 \end{description}
	 	\item[theta247]\ 
	 	 \begin{description}
	 	 	\item[hyperid] \verb!29347!
	 	 	\item[name] \verb!theta247!
	 	 	\item[short.name] \verb!theta247!
	 	 	\item[initial] \verb!1048576!
	 	 	\item[fixed] \verb!FALSE!
	 	 	\item[prior] \verb!none!
	 	 	\item[param] \verb!!
	 	 	\item[to.theta] \verb!function(x) x!
	 	 	\item[from.theta] \verb!function(x) x!
	 	 \end{description}
	 	\item[theta248]\ 
	 	 \begin{description}
	 	 	\item[hyperid] \verb!29348!
	 	 	\item[name] \verb!theta248!
	 	 	\item[short.name] \verb!theta248!
	 	 	\item[initial] \verb!1048576!
	 	 	\item[fixed] \verb!FALSE!
	 	 	\item[prior] \verb!none!
	 	 	\item[param] \verb!!
	 	 	\item[to.theta] \verb!function(x) x!
	 	 	\item[from.theta] \verb!function(x) x!
	 	 \end{description}
	 	\item[theta249]\ 
	 	 \begin{description}
	 	 	\item[hyperid] \verb!29349!
	 	 	\item[name] \verb!theta249!
	 	 	\item[short.name] \verb!theta249!
	 	 	\item[initial] \verb!1048576!
	 	 	\item[fixed] \verb!FALSE!
	 	 	\item[prior] \verb!none!
	 	 	\item[param] \verb!!
	 	 	\item[to.theta] \verb!function(x) x!
	 	 	\item[from.theta] \verb!function(x) x!
	 	 \end{description}
	 	\item[theta250]\ 
	 	 \begin{description}
	 	 	\item[hyperid] \verb!29350!
	 	 	\item[name] \verb!theta250!
	 	 	\item[short.name] \verb!theta250!
	 	 	\item[initial] \verb!1048576!
	 	 	\item[fixed] \verb!FALSE!
	 	 	\item[prior] \verb!none!
	 	 	\item[param] \verb!!
	 	 	\item[to.theta] \verb!function(x) x!
	 	 	\item[from.theta] \verb!function(x) x!
	 	 \end{description}
	 	\item[theta251]\ 
	 	 \begin{description}
	 	 	\item[hyperid] \verb!29351!
	 	 	\item[name] \verb!theta251!
	 	 	\item[short.name] \verb!theta251!
	 	 	\item[initial] \verb!1048576!
	 	 	\item[fixed] \verb!FALSE!
	 	 	\item[prior] \verb!none!
	 	 	\item[param] \verb!!
	 	 	\item[to.theta] \verb!function(x) x!
	 	 	\item[from.theta] \verb!function(x) x!
	 	 \end{description}
	 	\item[theta252]\ 
	 	 \begin{description}
	 	 	\item[hyperid] \verb!29352!
	 	 	\item[name] \verb!theta252!
	 	 	\item[short.name] \verb!theta252!
	 	 	\item[initial] \verb!1048576!
	 	 	\item[fixed] \verb!FALSE!
	 	 	\item[prior] \verb!none!
	 	 	\item[param] \verb!!
	 	 	\item[to.theta] \verb!function(x) x!
	 	 	\item[from.theta] \verb!function(x) x!
	 	 \end{description}
	 	\item[theta253]\ 
	 	 \begin{description}
	 	 	\item[hyperid] \verb!29353!
	 	 	\item[name] \verb!theta253!
	 	 	\item[short.name] \verb!theta253!
	 	 	\item[initial] \verb!1048576!
	 	 	\item[fixed] \verb!FALSE!
	 	 	\item[prior] \verb!none!
	 	 	\item[param] \verb!!
	 	 	\item[to.theta] \verb!function(x) x!
	 	 	\item[from.theta] \verb!function(x) x!
	 	 \end{description}
	 	\item[theta254]\ 
	 	 \begin{description}
	 	 	\item[hyperid] \verb!29354!
	 	 	\item[name] \verb!theta254!
	 	 	\item[short.name] \verb!theta254!
	 	 	\item[initial] \verb!1048576!
	 	 	\item[fixed] \verb!FALSE!
	 	 	\item[prior] \verb!none!
	 	 	\item[param] \verb!!
	 	 	\item[to.theta] \verb!function(x) x!
	 	 	\item[from.theta] \verb!function(x) x!
	 	 \end{description}
	 	\item[theta255]\ 
	 	 \begin{description}
	 	 	\item[hyperid] \verb!29355!
	 	 	\item[name] \verb!theta255!
	 	 	\item[short.name] \verb!theta255!
	 	 	\item[initial] \verb!1048576!
	 	 	\item[fixed] \verb!FALSE!
	 	 	\item[prior] \verb!none!
	 	 	\item[param] \verb!!
	 	 	\item[to.theta] \verb!function(x) x!
	 	 	\item[from.theta] \verb!function(x) x!
	 	 \end{description}
	 	\item[theta256]\ 
	 	 \begin{description}
	 	 	\item[hyperid] \verb!29356!
	 	 	\item[name] \verb!theta256!
	 	 	\item[short.name] \verb!theta256!
	 	 	\item[initial] \verb!1048576!
	 	 	\item[fixed] \verb!FALSE!
	 	 	\item[prior] \verb!none!
	 	 	\item[param] \verb!!
	 	 	\item[to.theta] \verb!function(x) x!
	 	 	\item[from.theta] \verb!function(x) x!
	 	 \end{description}
	 	\item[theta257]\ 
	 	 \begin{description}
	 	 	\item[hyperid] \verb!29357!
	 	 	\item[name] \verb!theta257!
	 	 	\item[short.name] \verb!theta257!
	 	 	\item[initial] \verb!1048576!
	 	 	\item[fixed] \verb!FALSE!
	 	 	\item[prior] \verb!none!
	 	 	\item[param] \verb!!
	 	 	\item[to.theta] \verb!function(x) x!
	 	 	\item[from.theta] \verb!function(x) x!
	 	 \end{description}
	 	\item[theta258]\ 
	 	 \begin{description}
	 	 	\item[hyperid] \verb!29358!
	 	 	\item[name] \verb!theta258!
	 	 	\item[short.name] \verb!theta258!
	 	 	\item[initial] \verb!1048576!
	 	 	\item[fixed] \verb!FALSE!
	 	 	\item[prior] \verb!none!
	 	 	\item[param] \verb!!
	 	 	\item[to.theta] \verb!function(x) x!
	 	 	\item[from.theta] \verb!function(x) x!
	 	 \end{description}
	 	\item[theta259]\ 
	 	 \begin{description}
	 	 	\item[hyperid] \verb!29359!
	 	 	\item[name] \verb!theta259!
	 	 	\item[short.name] \verb!theta259!
	 	 	\item[initial] \verb!1048576!
	 	 	\item[fixed] \verb!FALSE!
	 	 	\item[prior] \verb!none!
	 	 	\item[param] \verb!!
	 	 	\item[to.theta] \verb!function(x) x!
	 	 	\item[from.theta] \verb!function(x) x!
	 	 \end{description}
	 	\item[theta260]\ 
	 	 \begin{description}
	 	 	\item[hyperid] \verb!29360!
	 	 	\item[name] \verb!theta260!
	 	 	\item[short.name] \verb!theta260!
	 	 	\item[initial] \verb!1048576!
	 	 	\item[fixed] \verb!FALSE!
	 	 	\item[prior] \verb!none!
	 	 	\item[param] \verb!!
	 	 	\item[to.theta] \verb!function(x) x!
	 	 	\item[from.theta] \verb!function(x) x!
	 	 \end{description}
	 	\item[theta261]\ 
	 	 \begin{description}
	 	 	\item[hyperid] \verb!29361!
	 	 	\item[name] \verb!theta261!
	 	 	\item[short.name] \verb!theta261!
	 	 	\item[initial] \verb!1048576!
	 	 	\item[fixed] \verb!FALSE!
	 	 	\item[prior] \verb!none!
	 	 	\item[param] \verb!!
	 	 	\item[to.theta] \verb!function(x) x!
	 	 	\item[from.theta] \verb!function(x) x!
	 	 \end{description}
	 	\item[theta262]\ 
	 	 \begin{description}
	 	 	\item[hyperid] \verb!29362!
	 	 	\item[name] \verb!theta262!
	 	 	\item[short.name] \verb!theta262!
	 	 	\item[initial] \verb!1048576!
	 	 	\item[fixed] \verb!FALSE!
	 	 	\item[prior] \verb!none!
	 	 	\item[param] \verb!!
	 	 	\item[to.theta] \verb!function(x) x!
	 	 	\item[from.theta] \verb!function(x) x!
	 	 \end{description}
	 	\item[theta263]\ 
	 	 \begin{description}
	 	 	\item[hyperid] \verb!29363!
	 	 	\item[name] \verb!theta263!
	 	 	\item[short.name] \verb!theta263!
	 	 	\item[initial] \verb!1048576!
	 	 	\item[fixed] \verb!FALSE!
	 	 	\item[prior] \verb!none!
	 	 	\item[param] \verb!!
	 	 	\item[to.theta] \verb!function(x) x!
	 	 	\item[from.theta] \verb!function(x) x!
	 	 \end{description}
	 	\item[theta264]\ 
	 	 \begin{description}
	 	 	\item[hyperid] \verb!29364!
	 	 	\item[name] \verb!theta264!
	 	 	\item[short.name] \verb!theta264!
	 	 	\item[initial] \verb!1048576!
	 	 	\item[fixed] \verb!FALSE!
	 	 	\item[prior] \verb!none!
	 	 	\item[param] \verb!!
	 	 	\item[to.theta] \verb!function(x) x!
	 	 	\item[from.theta] \verb!function(x) x!
	 	 \end{description}
	 	\item[theta265]\ 
	 	 \begin{description}
	 	 	\item[hyperid] \verb!29365!
	 	 	\item[name] \verb!theta265!
	 	 	\item[short.name] \verb!theta265!
	 	 	\item[initial] \verb!1048576!
	 	 	\item[fixed] \verb!FALSE!
	 	 	\item[prior] \verb!none!
	 	 	\item[param] \verb!!
	 	 	\item[to.theta] \verb!function(x) x!
	 	 	\item[from.theta] \verb!function(x) x!
	 	 \end{description}
	 	\item[theta266]\ 
	 	 \begin{description}
	 	 	\item[hyperid] \verb!29366!
	 	 	\item[name] \verb!theta266!
	 	 	\item[short.name] \verb!theta266!
	 	 	\item[initial] \verb!1048576!
	 	 	\item[fixed] \verb!FALSE!
	 	 	\item[prior] \verb!none!
	 	 	\item[param] \verb!!
	 	 	\item[to.theta] \verb!function(x) x!
	 	 	\item[from.theta] \verb!function(x) x!
	 	 \end{description}
	 	\item[theta267]\ 
	 	 \begin{description}
	 	 	\item[hyperid] \verb!29367!
	 	 	\item[name] \verb!theta267!
	 	 	\item[short.name] \verb!theta267!
	 	 	\item[initial] \verb!1048576!
	 	 	\item[fixed] \verb!FALSE!
	 	 	\item[prior] \verb!none!
	 	 	\item[param] \verb!!
	 	 	\item[to.theta] \verb!function(x) x!
	 	 	\item[from.theta] \verb!function(x) x!
	 	 \end{description}
	 	\item[theta268]\ 
	 	 \begin{description}
	 	 	\item[hyperid] \verb!29368!
	 	 	\item[name] \verb!theta268!
	 	 	\item[short.name] \verb!theta268!
	 	 	\item[initial] \verb!1048576!
	 	 	\item[fixed] \verb!FALSE!
	 	 	\item[prior] \verb!none!
	 	 	\item[param] \verb!!
	 	 	\item[to.theta] \verb!function(x) x!
	 	 	\item[from.theta] \verb!function(x) x!
	 	 \end{description}
	 	\item[theta269]\ 
	 	 \begin{description}
	 	 	\item[hyperid] \verb!29369!
	 	 	\item[name] \verb!theta269!
	 	 	\item[short.name] \verb!theta269!
	 	 	\item[initial] \verb!1048576!
	 	 	\item[fixed] \verb!FALSE!
	 	 	\item[prior] \verb!none!
	 	 	\item[param] \verb!!
	 	 	\item[to.theta] \verb!function(x) x!
	 	 	\item[from.theta] \verb!function(x) x!
	 	 \end{description}
	 	\item[theta270]\ 
	 	 \begin{description}
	 	 	\item[hyperid] \verb!29370!
	 	 	\item[name] \verb!theta270!
	 	 	\item[short.name] \verb!theta270!
	 	 	\item[initial] \verb!1048576!
	 	 	\item[fixed] \verb!FALSE!
	 	 	\item[prior] \verb!none!
	 	 	\item[param] \verb!!
	 	 	\item[to.theta] \verb!function(x) x!
	 	 	\item[from.theta] \verb!function(x) x!
	 	 \end{description}
	 	\item[theta271]\ 
	 	 \begin{description}
	 	 	\item[hyperid] \verb!29371!
	 	 	\item[name] \verb!theta271!
	 	 	\item[short.name] \verb!theta271!
	 	 	\item[initial] \verb!1048576!
	 	 	\item[fixed] \verb!FALSE!
	 	 	\item[prior] \verb!none!
	 	 	\item[param] \verb!!
	 	 	\item[to.theta] \verb!function(x) x!
	 	 	\item[from.theta] \verb!function(x) x!
	 	 \end{description}
	 	\item[theta272]\ 
	 	 \begin{description}
	 	 	\item[hyperid] \verb!29372!
	 	 	\item[name] \verb!theta272!
	 	 	\item[short.name] \verb!theta272!
	 	 	\item[initial] \verb!1048576!
	 	 	\item[fixed] \verb!FALSE!
	 	 	\item[prior] \verb!none!
	 	 	\item[param] \verb!!
	 	 	\item[to.theta] \verb!function(x) x!
	 	 	\item[from.theta] \verb!function(x) x!
	 	 \end{description}
	 	\item[theta273]\ 
	 	 \begin{description}
	 	 	\item[hyperid] \verb!29373!
	 	 	\item[name] \verb!theta273!
	 	 	\item[short.name] \verb!theta273!
	 	 	\item[initial] \verb!1048576!
	 	 	\item[fixed] \verb!FALSE!
	 	 	\item[prior] \verb!none!
	 	 	\item[param] \verb!!
	 	 	\item[to.theta] \verb!function(x) x!
	 	 	\item[from.theta] \verb!function(x) x!
	 	 \end{description}
	 	\item[theta274]\ 
	 	 \begin{description}
	 	 	\item[hyperid] \verb!29374!
	 	 	\item[name] \verb!theta274!
	 	 	\item[short.name] \verb!theta274!
	 	 	\item[initial] \verb!1048576!
	 	 	\item[fixed] \verb!FALSE!
	 	 	\item[prior] \verb!none!
	 	 	\item[param] \verb!!
	 	 	\item[to.theta] \verb!function(x) x!
	 	 	\item[from.theta] \verb!function(x) x!
	 	 \end{description}
	 	\item[theta275]\ 
	 	 \begin{description}
	 	 	\item[hyperid] \verb!29375!
	 	 	\item[name] \verb!theta275!
	 	 	\item[short.name] \verb!theta275!
	 	 	\item[initial] \verb!1048576!
	 	 	\item[fixed] \verb!FALSE!
	 	 	\item[prior] \verb!none!
	 	 	\item[param] \verb!!
	 	 	\item[to.theta] \verb!function(x) x!
	 	 	\item[from.theta] \verb!function(x) x!
	 	 \end{description}
	 	\item[theta276]\ 
	 	 \begin{description}
	 	 	\item[hyperid] \verb!29376!
	 	 	\item[name] \verb!theta276!
	 	 	\item[short.name] \verb!theta276!
	 	 	\item[initial] \verb!1048576!
	 	 	\item[fixed] \verb!FALSE!
	 	 	\item[prior] \verb!none!
	 	 	\item[param] \verb!!
	 	 	\item[to.theta] \verb!function(x) x!
	 	 	\item[from.theta] \verb!function(x) x!
	 	 \end{description}
	 	\item[theta277]\ 
	 	 \begin{description}
	 	 	\item[hyperid] \verb!29377!
	 	 	\item[name] \verb!theta277!
	 	 	\item[short.name] \verb!theta277!
	 	 	\item[initial] \verb!1048576!
	 	 	\item[fixed] \verb!FALSE!
	 	 	\item[prior] \verb!none!
	 	 	\item[param] \verb!!
	 	 	\item[to.theta] \verb!function(x) x!
	 	 	\item[from.theta] \verb!function(x) x!
	 	 \end{description}
	 	\item[theta278]\ 
	 	 \begin{description}
	 	 	\item[hyperid] \verb!29378!
	 	 	\item[name] \verb!theta278!
	 	 	\item[short.name] \verb!theta278!
	 	 	\item[initial] \verb!1048576!
	 	 	\item[fixed] \verb!FALSE!
	 	 	\item[prior] \verb!none!
	 	 	\item[param] \verb!!
	 	 	\item[to.theta] \verb!function(x) x!
	 	 	\item[from.theta] \verb!function(x) x!
	 	 \end{description}
	 	\item[theta279]\ 
	 	 \begin{description}
	 	 	\item[hyperid] \verb!29379!
	 	 	\item[name] \verb!theta279!
	 	 	\item[short.name] \verb!theta279!
	 	 	\item[initial] \verb!1048576!
	 	 	\item[fixed] \verb!FALSE!
	 	 	\item[prior] \verb!none!
	 	 	\item[param] \verb!!
	 	 	\item[to.theta] \verb!function(x) x!
	 	 	\item[from.theta] \verb!function(x) x!
	 	 \end{description}
	 	\item[theta280]\ 
	 	 \begin{description}
	 	 	\item[hyperid] \verb!29380!
	 	 	\item[name] \verb!theta280!
	 	 	\item[short.name] \verb!theta280!
	 	 	\item[initial] \verb!1048576!
	 	 	\item[fixed] \verb!FALSE!
	 	 	\item[prior] \verb!none!
	 	 	\item[param] \verb!!
	 	 	\item[to.theta] \verb!function(x) x!
	 	 	\item[from.theta] \verb!function(x) x!
	 	 \end{description}
	 	\item[theta281]\ 
	 	 \begin{description}
	 	 	\item[hyperid] \verb!29381!
	 	 	\item[name] \verb!theta281!
	 	 	\item[short.name] \verb!theta281!
	 	 	\item[initial] \verb!1048576!
	 	 	\item[fixed] \verb!FALSE!
	 	 	\item[prior] \verb!none!
	 	 	\item[param] \verb!!
	 	 	\item[to.theta] \verb!function(x) x!
	 	 	\item[from.theta] \verb!function(x) x!
	 	 \end{description}
	 	\item[theta282]\ 
	 	 \begin{description}
	 	 	\item[hyperid] \verb!29382!
	 	 	\item[name] \verb!theta282!
	 	 	\item[short.name] \verb!theta282!
	 	 	\item[initial] \verb!1048576!
	 	 	\item[fixed] \verb!FALSE!
	 	 	\item[prior] \verb!none!
	 	 	\item[param] \verb!!
	 	 	\item[to.theta] \verb!function(x) x!
	 	 	\item[from.theta] \verb!function(x) x!
	 	 \end{description}
	 	\item[theta283]\ 
	 	 \begin{description}
	 	 	\item[hyperid] \verb!29383!
	 	 	\item[name] \verb!theta283!
	 	 	\item[short.name] \verb!theta283!
	 	 	\item[initial] \verb!1048576!
	 	 	\item[fixed] \verb!FALSE!
	 	 	\item[prior] \verb!none!
	 	 	\item[param] \verb!!
	 	 	\item[to.theta] \verb!function(x) x!
	 	 	\item[from.theta] \verb!function(x) x!
	 	 \end{description}
	 	\item[theta284]\ 
	 	 \begin{description}
	 	 	\item[hyperid] \verb!29384!
	 	 	\item[name] \verb!theta284!
	 	 	\item[short.name] \verb!theta284!
	 	 	\item[initial] \verb!1048576!
	 	 	\item[fixed] \verb!FALSE!
	 	 	\item[prior] \verb!none!
	 	 	\item[param] \verb!!
	 	 	\item[to.theta] \verb!function(x) x!
	 	 	\item[from.theta] \verb!function(x) x!
	 	 \end{description}
	 	\item[theta285]\ 
	 	 \begin{description}
	 	 	\item[hyperid] \verb!29385!
	 	 	\item[name] \verb!theta285!
	 	 	\item[short.name] \verb!theta285!
	 	 	\item[initial] \verb!1048576!
	 	 	\item[fixed] \verb!FALSE!
	 	 	\item[prior] \verb!none!
	 	 	\item[param] \verb!!
	 	 	\item[to.theta] \verb!function(x) x!
	 	 	\item[from.theta] \verb!function(x) x!
	 	 \end{description}
	 	\item[theta286]\ 
	 	 \begin{description}
	 	 	\item[hyperid] \verb!29386!
	 	 	\item[name] \verb!theta286!
	 	 	\item[short.name] \verb!theta286!
	 	 	\item[initial] \verb!1048576!
	 	 	\item[fixed] \verb!FALSE!
	 	 	\item[prior] \verb!none!
	 	 	\item[param] \verb!!
	 	 	\item[to.theta] \verb!function(x) x!
	 	 	\item[from.theta] \verb!function(x) x!
	 	 \end{description}
	 	\item[theta287]\ 
	 	 \begin{description}
	 	 	\item[hyperid] \verb!29387!
	 	 	\item[name] \verb!theta287!
	 	 	\item[short.name] \verb!theta287!
	 	 	\item[initial] \verb!1048576!
	 	 	\item[fixed] \verb!FALSE!
	 	 	\item[prior] \verb!none!
	 	 	\item[param] \verb!!
	 	 	\item[to.theta] \verb!function(x) x!
	 	 	\item[from.theta] \verb!function(x) x!
	 	 \end{description}
	 	\item[theta288]\ 
	 	 \begin{description}
	 	 	\item[hyperid] \verb!29388!
	 	 	\item[name] \verb!theta288!
	 	 	\item[short.name] \verb!theta288!
	 	 	\item[initial] \verb!1048576!
	 	 	\item[fixed] \verb!FALSE!
	 	 	\item[prior] \verb!none!
	 	 	\item[param] \verb!!
	 	 	\item[to.theta] \verb!function(x) x!
	 	 	\item[from.theta] \verb!function(x) x!
	 	 \end{description}
	 	\item[theta289]\ 
	 	 \begin{description}
	 	 	\item[hyperid] \verb!29389!
	 	 	\item[name] \verb!theta289!
	 	 	\item[short.name] \verb!theta289!
	 	 	\item[initial] \verb!1048576!
	 	 	\item[fixed] \verb!FALSE!
	 	 	\item[prior] \verb!none!
	 	 	\item[param] \verb!!
	 	 	\item[to.theta] \verb!function(x) x!
	 	 	\item[from.theta] \verb!function(x) x!
	 	 \end{description}
	 	\item[theta290]\ 
	 	 \begin{description}
	 	 	\item[hyperid] \verb!29390!
	 	 	\item[name] \verb!theta290!
	 	 	\item[short.name] \verb!theta290!
	 	 	\item[initial] \verb!1048576!
	 	 	\item[fixed] \verb!FALSE!
	 	 	\item[prior] \verb!none!
	 	 	\item[param] \verb!!
	 	 	\item[to.theta] \verb!function(x) x!
	 	 	\item[from.theta] \verb!function(x) x!
	 	 \end{description}
	 	\item[theta291]\ 
	 	 \begin{description}
	 	 	\item[hyperid] \verb!29391!
	 	 	\item[name] \verb!theta291!
	 	 	\item[short.name] \verb!theta291!
	 	 	\item[initial] \verb!1048576!
	 	 	\item[fixed] \verb!FALSE!
	 	 	\item[prior] \verb!none!
	 	 	\item[param] \verb!!
	 	 	\item[to.theta] \verb!function(x) x!
	 	 	\item[from.theta] \verb!function(x) x!
	 	 \end{description}
	 	\item[theta292]\ 
	 	 \begin{description}
	 	 	\item[hyperid] \verb!29392!
	 	 	\item[name] \verb!theta292!
	 	 	\item[short.name] \verb!theta292!
	 	 	\item[initial] \verb!1048576!
	 	 	\item[fixed] \verb!FALSE!
	 	 	\item[prior] \verb!none!
	 	 	\item[param] \verb!!
	 	 	\item[to.theta] \verb!function(x) x!
	 	 	\item[from.theta] \verb!function(x) x!
	 	 \end{description}
	 	\item[theta293]\ 
	 	 \begin{description}
	 	 	\item[hyperid] \verb!29393!
	 	 	\item[name] \verb!theta293!
	 	 	\item[short.name] \verb!theta293!
	 	 	\item[initial] \verb!1048576!
	 	 	\item[fixed] \verb!FALSE!
	 	 	\item[prior] \verb!none!
	 	 	\item[param] \verb!!
	 	 	\item[to.theta] \verb!function(x) x!
	 	 	\item[from.theta] \verb!function(x) x!
	 	 \end{description}
	 	\item[theta294]\ 
	 	 \begin{description}
	 	 	\item[hyperid] \verb!29394!
	 	 	\item[name] \verb!theta294!
	 	 	\item[short.name] \verb!theta294!
	 	 	\item[initial] \verb!1048576!
	 	 	\item[fixed] \verb!FALSE!
	 	 	\item[prior] \verb!none!
	 	 	\item[param] \verb!!
	 	 	\item[to.theta] \verb!function(x) x!
	 	 	\item[from.theta] \verb!function(x) x!
	 	 \end{description}
	 	\item[theta295]\ 
	 	 \begin{description}
	 	 	\item[hyperid] \verb!29395!
	 	 	\item[name] \verb!theta295!
	 	 	\item[short.name] \verb!theta295!
	 	 	\item[initial] \verb!1048576!
	 	 	\item[fixed] \verb!FALSE!
	 	 	\item[prior] \verb!none!
	 	 	\item[param] \verb!!
	 	 	\item[to.theta] \verb!function(x) x!
	 	 	\item[from.theta] \verb!function(x) x!
	 	 \end{description}
	 	\item[theta296]\ 
	 	 \begin{description}
	 	 	\item[hyperid] \verb!29396!
	 	 	\item[name] \verb!theta296!
	 	 	\item[short.name] \verb!theta296!
	 	 	\item[initial] \verb!1048576!
	 	 	\item[fixed] \verb!FALSE!
	 	 	\item[prior] \verb!none!
	 	 	\item[param] \verb!!
	 	 	\item[to.theta] \verb!function(x) x!
	 	 	\item[from.theta] \verb!function(x) x!
	 	 \end{description}
	 	\item[theta297]\ 
	 	 \begin{description}
	 	 	\item[hyperid] \verb!29397!
	 	 	\item[name] \verb!theta297!
	 	 	\item[short.name] \verb!theta297!
	 	 	\item[initial] \verb!1048576!
	 	 	\item[fixed] \verb!FALSE!
	 	 	\item[prior] \verb!none!
	 	 	\item[param] \verb!!
	 	 	\item[to.theta] \verb!function(x) x!
	 	 	\item[from.theta] \verb!function(x) x!
	 	 \end{description}
	 	\item[theta298]\ 
	 	 \begin{description}
	 	 	\item[hyperid] \verb!29398!
	 	 	\item[name] \verb!theta298!
	 	 	\item[short.name] \verb!theta298!
	 	 	\item[initial] \verb!1048576!
	 	 	\item[fixed] \verb!FALSE!
	 	 	\item[prior] \verb!none!
	 	 	\item[param] \verb!!
	 	 	\item[to.theta] \verb!function(x) x!
	 	 	\item[from.theta] \verb!function(x) x!
	 	 \end{description}
	 	\item[theta299]\ 
	 	 \begin{description}
	 	 	\item[hyperid] \verb!29399!
	 	 	\item[name] \verb!theta299!
	 	 	\item[short.name] \verb!theta299!
	 	 	\item[initial] \verb!1048576!
	 	 	\item[fixed] \verb!FALSE!
	 	 	\item[prior] \verb!none!
	 	 	\item[param] \verb!!
	 	 	\item[to.theta] \verb!function(x) x!
	 	 	\item[from.theta] \verb!function(x) x!
	 	 \end{description}
	 	\item[theta300]\ 
	 	 \begin{description}
	 	 	\item[hyperid] \verb!29400!
	 	 	\item[name] \verb!theta300!
	 	 	\item[short.name] \verb!theta300!
	 	 	\item[initial] \verb!1048576!
	 	 	\item[fixed] \verb!FALSE!
	 	 	\item[prior] \verb!none!
	 	 	\item[param] \verb!!
	 	 	\item[to.theta] \verb!function(x) x!
	 	 	\item[from.theta] \verb!function(x) x!
	 	 \end{description}
	 \end{description}
	\item[constr] \verb!FALSE!
	\item[nrow.ncol] \verb!FALSE!
	\item[augmented] \verb!TRUE!
	\item[aug.factor] \verb!1!
	\item[aug.constr] \verb!1 2 3 4 5 6 7 8 9 10 11 12 13 14 15 16 17 18 19 20 21 22 23 24!
	\item[n.div.by] \verb!-1!
	\item[n.required] \verb!TRUE!
	\item[set.default.values] \verb!TRUE!
	\item[pdf] \verb!iidkd!
\end{description}



\subsection*{Example}

Just simulate some data and estimate the parameters back. This is for
\texttt{order=}$4$.

{\small\verbatiminput{example-iidkd.R}}

\subsection*{Notes}

Note that for higher values of $k$ the only integration scheme that is
supported is \\
\texttt{control.inla=list(int.strategy="eb")}.

\end{document}
