\documentclass[a4paper,11pt]{article}
\usepackage[scale={0.8,0.9},centering,includeheadfoot]{geometry}
\usepackage{amstext}
\usepackage{listings}
\usepackage{verbatim}
\def\opening\null
\usepackage{block}
\begin{document}

\section*{Correlated random effects: \texttt{iidkd}}

This model is available for dimensions $k=2$, to $20$. We describe in
detail the case for $k=3$ as other ones are similar. This model do the
same as models \texttt{iid2d}, \texttt{iid3d}, \texttt{iid4d},
\texttt{iid5d}, but uses a different and more efficient parameterisation. 

\subsection*{Parametrization}

The $(k=3)$-dimensional Normal-Wishard model is used if one want to define
three vectors of ``random effects'', $u$ and $v$ and $w$, say, for which
$(u_{i}, v_{i}, w_i)$ are iid bivariate Normals
\begin{displaymath}
    \left(
      \begin{array}{c}
        u_{i}\\
        v_{i}\\
        w_{i}
      \end{array}\right)
    \sim \mathcal{N}\left(\mathbf{0}, \mathbf{W}^{-1}\right)
\end{displaymath}
where the  covariance matrix $\mathbf{W}^{-1}$ is parameterised as
$\mathbf{W}=\mathbf{L}\mathbf{L}^{T}$, where
\begin{equation}
    \label{precision}
    \mathbf{L} = \left(\begin{array}{ccc}
                         \exp(\theta_1) & & \\
                         \theta_4 & \exp(\theta_2) & \\
                         \theta_5 & \theta_6 & \exp(\theta_3)
      \end{array}\right)
\end{equation}
and $\theta_1, \theta_2, \theta_3, \theta_4, \theta_5, \theta_6$ can
take any value. The number of hyperparameters are $k(k+1)/2$, which is
$3$, $6$, $10$, $15$, $21$, $28$, $36$, $45$, $55, \ldots$, for
$k=2, 3, 4, 5, 6, 7, 8, 9, 10, \ldots$.

For these models the precision matrix $\mathbf{W}$ is Wishart
distributed
\begin{displaymath}
    \mathbf{W}
    \;\sim\;\text{Wishart}_{k}(r, \mathbf{R}^{-1}), 
\end{displaymath}
with density
\begin{displaymath}
    \pi(\mathbf{W}) = c^{-1} |\mathbf{W}|^{(r-(k+1))/2} \exp\left\{
      -\frac{1}{2}\text{Trace}(\mathbf{W}\mathbf{R})\right\}, \quad r > k+1
\end{displaymath}
and
\begin{displaymath}
    c = 2^{(rk)/2} |\mathbf{R}|^{-r/2} \pi^{(k(k-1))/4}\prod_{j=1}^{k}
    \Gamma((r+1-j)/2).
\end{displaymath}
Then,
\begin{displaymath}
    \text{E}(\mathbf{W}) = r\mathbf{R}^{-1}, \quad\text{and}\quad
    \text{E}(\mathbf{W}^{-1}) = \mathbf{R}/(r-(k+1)).
\end{displaymath}

\subsection*{Hyperparameters}

The hyperparameters are
$\theta_1, \theta_2, \theta_3, \theta_4, \theta_5, \theta_6$.

The prior-parameters are
\begin{displaymath}
    (r,R_{1}, R_{2}, R_{3}, R_{4}, R_{5}, R_{6})
\end{displaymath}
where 
\begin{displaymath}
    \mathbf{R}= \left(
      \begin{array}{ccc}
        R_{1} & R_{4} & R_{5}\\
        R_{4} & R_{2} & R_{6} \\
        R_{5} & R_{6} & R_{3}
      \end{array}\right)
\end{displaymath}

The {\tt inla} function reports posterior distribution for the
hyperparameters $\{\theta_i\}$, and the conversion into interpretable
quantities can be done using simulation as described below.

The prior for $\theta$ is {\bf fixed} to be {\tt wishartkd}, and
number of prior parameters required are $1 + k(k+1)/2$. By default the
prior-parameters are
\begin{displaymath}
    (r=100, \underbrace{1, \ldots, 1}_{k\;\text{times}}, 0, \ldots, 0)
\end{displaymath}


\subsection*{Specification}

The model \texttt{iidkd}
is specified as
\begin{verbatim}
    y ~ f(i, model="iidkd", order=3, n = <length>) + ...
\end{verbatim}
where $\text{order}=k=3$, and the \texttt{iidkd} model is represented
internally as one vector of length $n$,
\begin{displaymath}
    (u_{1}, u_{2} \ldots, u_{m}, v_{1}, v_{2}, \ldots, v_{m}, w_{1},
    w_{2}, \ldots, w_{m})
\end{displaymath}
where $n = 3m$, and $n$ is the (required) argument in
\texttt{f()}.

For this model the argument \texttt{constr=TRUE} is interpreted as $3$
sum-to-zero constraints
\begin{displaymath}
    \sum u_{i} = 0, \quad \sum v_{i} = 0 \quad\text{and}\quad \sum w_{i} = 0.
\end{displaymath}

\subsubsection*{Hyperparameter spesification and default values}

(\textbf{Note:} The value ``$1048576$'' is just a code for ``replace
this by the default value''. As the default value depends on
\texttt{order}, the was the easy way out for the moment.)

%% DO NOT EDIT!
%% This file is generated automatically from models.R
\begin{description}
	\item[doc] Gaussian random effect in dim=k with Wishart prior
	\item[hyper]\ 
	 \begin{description}
	 	\item[theta1]\ 
	 	 \begin{description}
	 	 	\item[hyperid] 29101
	 	 	\item[name] theta1
	 	 	\item[short.name] theta1
	 	 	\item[initial] 1048576
	 	 	\item[fixed] FALSE
	 	 	\item[prior] wishartkd
	 	 	\item[param] 11 1048576 1048576 1048576 1048576 1048576 1048576 1048576 1048576 1048576 1048576 1048576 1048576 1048576 1048576 1048576 1048576 1048576 1048576 1048576 1048576 1048576 1048576 1048576 1048576 1048576 1048576 1048576 1048576 1048576 1048576 1048576 1048576 1048576 1048576 1048576 1048576 1048576 1048576 1048576 1048576 1048576 1048576 1048576 1048576 1048576 1048576 1048576 1048576 1048576 1048576 1048576 1048576 1048576 1048576 1048576
	 	 	\item[to.theta] \verb!function(x) x!
	 	 	\item[from.theta] \verb!function(x) x!
	 	 \end{description}
	 	\item[theta2]\ 
	 	 \begin{description}
	 	 	\item[hyperid] 29102
	 	 	\item[name] theta2
	 	 	\item[short.name] theta2
	 	 	\item[initial] 1048576
	 	 	\item[fixed] FALSE
	 	 	\item[prior] none
	 	 	\item[param] 
	 	 	\item[to.theta] \verb!function(x) x!
	 	 	\item[from.theta] \verb!function(x) x!
	 	 \end{description}
	 	\item[theta3]\ 
	 	 \begin{description}
	 	 	\item[hyperid] 29103
	 	 	\item[name] theta3
	 	 	\item[short.name] theta3
	 	 	\item[initial] 1048576
	 	 	\item[fixed] FALSE
	 	 	\item[prior] none
	 	 	\item[param] 
	 	 	\item[to.theta] \verb!function(x) x!
	 	 	\item[from.theta] \verb!function(x) x!
	 	 \end{description}
	 	\item[theta4]\ 
	 	 \begin{description}
	 	 	\item[hyperid] 29104
	 	 	\item[name] theta4
	 	 	\item[short.name] theta4
	 	 	\item[initial] 1048576
	 	 	\item[fixed] FALSE
	 	 	\item[prior] none
	 	 	\item[param] 
	 	 	\item[to.theta] \verb!function(x) x!
	 	 	\item[from.theta] \verb!function(x) x!
	 	 \end{description}
	 	\item[theta5]\ 
	 	 \begin{description}
	 	 	\item[hyperid] 29105
	 	 	\item[name] theta5
	 	 	\item[short.name] theta5
	 	 	\item[initial] 1048576
	 	 	\item[fixed] FALSE
	 	 	\item[prior] none
	 	 	\item[param] 
	 	 	\item[to.theta] \verb!function(x) x!
	 	 	\item[from.theta] \verb!function(x) x!
	 	 \end{description}
	 	\item[theta6]\ 
	 	 \begin{description}
	 	 	\item[hyperid] 29106
	 	 	\item[name] theta6
	 	 	\item[short.name] theta6
	 	 	\item[initial] 1048576
	 	 	\item[fixed] FALSE
	 	 	\item[prior] none
	 	 	\item[param] 
	 	 	\item[to.theta] \verb!function(x) x!
	 	 	\item[from.theta] \verb!function(x) x!
	 	 \end{description}
	 	\item[theta7]\ 
	 	 \begin{description}
	 	 	\item[hyperid] 29107
	 	 	\item[name] theta7
	 	 	\item[short.name] theta7
	 	 	\item[initial] 1048576
	 	 	\item[fixed] FALSE
	 	 	\item[prior] none
	 	 	\item[param] 
	 	 	\item[to.theta] \verb!function(x) x!
	 	 	\item[from.theta] \verb!function(x) x!
	 	 \end{description}
	 	\item[theta8]\ 
	 	 \begin{description}
	 	 	\item[hyperid] 29108
	 	 	\item[name] theta8
	 	 	\item[short.name] theta8
	 	 	\item[initial] 1048576
	 	 	\item[fixed] FALSE
	 	 	\item[prior] none
	 	 	\item[param] 
	 	 	\item[to.theta] \verb!function(x) x!
	 	 	\item[from.theta] \verb!function(x) x!
	 	 \end{description}
	 	\item[theta9]\ 
	 	 \begin{description}
	 	 	\item[hyperid] 29109
	 	 	\item[name] theta9
	 	 	\item[short.name] theta9
	 	 	\item[initial] 1048576
	 	 	\item[fixed] FALSE
	 	 	\item[prior] none
	 	 	\item[param] 
	 	 	\item[to.theta] \verb!function(x) x!
	 	 	\item[from.theta] \verb!function(x) x!
	 	 \end{description}
	 	\item[theta10]\ 
	 	 \begin{description}
	 	 	\item[hyperid] 29110
	 	 	\item[name] theta10
	 	 	\item[short.name] theta10
	 	 	\item[initial] 1048576
	 	 	\item[fixed] FALSE
	 	 	\item[prior] none
	 	 	\item[param] 
	 	 	\item[to.theta] \verb!function(x) x!
	 	 	\item[from.theta] \verb!function(x) x!
	 	 \end{description}
	 	\item[theta11]\ 
	 	 \begin{description}
	 	 	\item[hyperid] 29111
	 	 	\item[name] theta11
	 	 	\item[short.name] theta11
	 	 	\item[initial] 1048576
	 	 	\item[fixed] FALSE
	 	 	\item[prior] none
	 	 	\item[param] 
	 	 	\item[to.theta] \verb!function(x) x!
	 	 	\item[from.theta] \verb!function(x) x!
	 	 \end{description}
	 	\item[theta12]\ 
	 	 \begin{description}
	 	 	\item[hyperid] 29112
	 	 	\item[name] theta12
	 	 	\item[short.name] theta12
	 	 	\item[initial] 1048576
	 	 	\item[fixed] FALSE
	 	 	\item[prior] none
	 	 	\item[param] 
	 	 	\item[to.theta] \verb!function(x) x!
	 	 	\item[from.theta] \verb!function(x) x!
	 	 \end{description}
	 	\item[theta13]\ 
	 	 \begin{description}
	 	 	\item[hyperid] 29113
	 	 	\item[name] theta13
	 	 	\item[short.name] theta13
	 	 	\item[initial] 1048576
	 	 	\item[fixed] FALSE
	 	 	\item[prior] none
	 	 	\item[param] 
	 	 	\item[to.theta] \verb!function(x) x!
	 	 	\item[from.theta] \verb!function(x) x!
	 	 \end{description}
	 	\item[theta14]\ 
	 	 \begin{description}
	 	 	\item[hyperid] 29114
	 	 	\item[name] theta14
	 	 	\item[short.name] theta14
	 	 	\item[initial] 1048576
	 	 	\item[fixed] FALSE
	 	 	\item[prior] none
	 	 	\item[param] 
	 	 	\item[to.theta] \verb!function(x) x!
	 	 	\item[from.theta] \verb!function(x) x!
	 	 \end{description}
	 	\item[theta15]\ 
	 	 \begin{description}
	 	 	\item[hyperid] 29115
	 	 	\item[name] theta15
	 	 	\item[short.name] theta15
	 	 	\item[initial] 1048576
	 	 	\item[fixed] FALSE
	 	 	\item[prior] none
	 	 	\item[param] 
	 	 	\item[to.theta] \verb!function(x) x!
	 	 	\item[from.theta] \verb!function(x) x!
	 	 \end{description}
	 	\item[theta16]\ 
	 	 \begin{description}
	 	 	\item[hyperid] 29116
	 	 	\item[name] theta16
	 	 	\item[short.name] theta16
	 	 	\item[initial] 1048576
	 	 	\item[fixed] FALSE
	 	 	\item[prior] none
	 	 	\item[param] 
	 	 	\item[to.theta] \verb!function(x) x!
	 	 	\item[from.theta] \verb!function(x) x!
	 	 \end{description}
	 	\item[theta17]\ 
	 	 \begin{description}
	 	 	\item[hyperid] 29117
	 	 	\item[name] theta17
	 	 	\item[short.name] theta17
	 	 	\item[initial] 1048576
	 	 	\item[fixed] FALSE
	 	 	\item[prior] none
	 	 	\item[param] 
	 	 	\item[to.theta] \verb!function(x) x!
	 	 	\item[from.theta] \verb!function(x) x!
	 	 \end{description}
	 	\item[theta18]\ 
	 	 \begin{description}
	 	 	\item[hyperid] 29118
	 	 	\item[name] theta18
	 	 	\item[short.name] theta18
	 	 	\item[initial] 1048576
	 	 	\item[fixed] FALSE
	 	 	\item[prior] none
	 	 	\item[param] 
	 	 	\item[to.theta] \verb!function(x) x!
	 	 	\item[from.theta] \verb!function(x) x!
	 	 \end{description}
	 	\item[theta19]\ 
	 	 \begin{description}
	 	 	\item[hyperid] 29119
	 	 	\item[name] theta19
	 	 	\item[short.name] theta19
	 	 	\item[initial] 1048576
	 	 	\item[fixed] FALSE
	 	 	\item[prior] none
	 	 	\item[param] 
	 	 	\item[to.theta] \verb!function(x) x!
	 	 	\item[from.theta] \verb!function(x) x!
	 	 \end{description}
	 	\item[theta20]\ 
	 	 \begin{description}
	 	 	\item[hyperid] 29120
	 	 	\item[name] theta20
	 	 	\item[short.name] theta20
	 	 	\item[initial] 1048576
	 	 	\item[fixed] FALSE
	 	 	\item[prior] none
	 	 	\item[param] 
	 	 	\item[to.theta] \verb!function(x) x!
	 	 	\item[from.theta] \verb!function(x) x!
	 	 \end{description}
	 	\item[theta21]\ 
	 	 \begin{description}
	 	 	\item[hyperid] 29121
	 	 	\item[name] theta21
	 	 	\item[short.name] theta21
	 	 	\item[initial] 1048576
	 	 	\item[fixed] FALSE
	 	 	\item[prior] none
	 	 	\item[param] 
	 	 	\item[to.theta] \verb!function(x) x!
	 	 	\item[from.theta] \verb!function(x) x!
	 	 \end{description}
	 	\item[theta22]\ 
	 	 \begin{description}
	 	 	\item[hyperid] 29122
	 	 	\item[name] theta22
	 	 	\item[short.name] theta22
	 	 	\item[initial] 1048576
	 	 	\item[fixed] FALSE
	 	 	\item[prior] none
	 	 	\item[param] 
	 	 	\item[to.theta] \verb!function(x) x!
	 	 	\item[from.theta] \verb!function(x) x!
	 	 \end{description}
	 	\item[theta23]\ 
	 	 \begin{description}
	 	 	\item[hyperid] 29123
	 	 	\item[name] theta23
	 	 	\item[short.name] theta23
	 	 	\item[initial] 1048576
	 	 	\item[fixed] FALSE
	 	 	\item[prior] none
	 	 	\item[param] 
	 	 	\item[to.theta] \verb!function(x) x!
	 	 	\item[from.theta] \verb!function(x) x!
	 	 \end{description}
	 	\item[theta24]\ 
	 	 \begin{description}
	 	 	\item[hyperid] 29124
	 	 	\item[name] theta24
	 	 	\item[short.name] theta24
	 	 	\item[initial] 1048576
	 	 	\item[fixed] FALSE
	 	 	\item[prior] none
	 	 	\item[param] 
	 	 	\item[to.theta] \verb!function(x) x!
	 	 	\item[from.theta] \verb!function(x) x!
	 	 \end{description}
	 	\item[theta25]\ 
	 	 \begin{description}
	 	 	\item[hyperid] 29125
	 	 	\item[name] theta25
	 	 	\item[short.name] theta25
	 	 	\item[initial] 1048576
	 	 	\item[fixed] FALSE
	 	 	\item[prior] none
	 	 	\item[param] 
	 	 	\item[to.theta] \verb!function(x) x!
	 	 	\item[from.theta] \verb!function(x) x!
	 	 \end{description}
	 	\item[theta26]\ 
	 	 \begin{description}
	 	 	\item[hyperid] 29126
	 	 	\item[name] theta26
	 	 	\item[short.name] theta26
	 	 	\item[initial] 1048576
	 	 	\item[fixed] FALSE
	 	 	\item[prior] none
	 	 	\item[param] 
	 	 	\item[to.theta] \verb!function(x) x!
	 	 	\item[from.theta] \verb!function(x) x!
	 	 \end{description}
	 	\item[theta27]\ 
	 	 \begin{description}
	 	 	\item[hyperid] 29127
	 	 	\item[name] theta27
	 	 	\item[short.name] theta27
	 	 	\item[initial] 1048576
	 	 	\item[fixed] FALSE
	 	 	\item[prior] none
	 	 	\item[param] 
	 	 	\item[to.theta] \verb!function(x) x!
	 	 	\item[from.theta] \verb!function(x) x!
	 	 \end{description}
	 	\item[theta28]\ 
	 	 \begin{description}
	 	 	\item[hyperid] 29128
	 	 	\item[name] theta28
	 	 	\item[short.name] theta28
	 	 	\item[initial] 1048576
	 	 	\item[fixed] FALSE
	 	 	\item[prior] none
	 	 	\item[param] 
	 	 	\item[to.theta] \verb!function(x) x!
	 	 	\item[from.theta] \verb!function(x) x!
	 	 \end{description}
	 	\item[theta29]\ 
	 	 \begin{description}
	 	 	\item[hyperid] 29129
	 	 	\item[name] theta29
	 	 	\item[short.name] theta29
	 	 	\item[initial] 1048576
	 	 	\item[fixed] FALSE
	 	 	\item[prior] none
	 	 	\item[param] 
	 	 	\item[to.theta] \verb!function(x) x!
	 	 	\item[from.theta] \verb!function(x) x!
	 	 \end{description}
	 	\item[theta30]\ 
	 	 \begin{description}
	 	 	\item[hyperid] 29130
	 	 	\item[name] theta30
	 	 	\item[short.name] theta30
	 	 	\item[initial] 1048576
	 	 	\item[fixed] FALSE
	 	 	\item[prior] none
	 	 	\item[param] 
	 	 	\item[to.theta] \verb!function(x) x!
	 	 	\item[from.theta] \verb!function(x) x!
	 	 \end{description}
	 	\item[theta31]\ 
	 	 \begin{description}
	 	 	\item[hyperid] 29131
	 	 	\item[name] theta31
	 	 	\item[short.name] theta31
	 	 	\item[initial] 1048576
	 	 	\item[fixed] FALSE
	 	 	\item[prior] none
	 	 	\item[param] 
	 	 	\item[to.theta] \verb!function(x) x!
	 	 	\item[from.theta] \verb!function(x) x!
	 	 \end{description}
	 	\item[theta32]\ 
	 	 \begin{description}
	 	 	\item[hyperid] 29132
	 	 	\item[name] theta32
	 	 	\item[short.name] theta32
	 	 	\item[initial] 1048576
	 	 	\item[fixed] FALSE
	 	 	\item[prior] none
	 	 	\item[param] 
	 	 	\item[to.theta] \verb!function(x) x!
	 	 	\item[from.theta] \verb!function(x) x!
	 	 \end{description}
	 	\item[theta33]\ 
	 	 \begin{description}
	 	 	\item[hyperid] 29133
	 	 	\item[name] theta33
	 	 	\item[short.name] theta33
	 	 	\item[initial] 1048576
	 	 	\item[fixed] FALSE
	 	 	\item[prior] none
	 	 	\item[param] 
	 	 	\item[to.theta] \verb!function(x) x!
	 	 	\item[from.theta] \verb!function(x) x!
	 	 \end{description}
	 	\item[theta34]\ 
	 	 \begin{description}
	 	 	\item[hyperid] 29134
	 	 	\item[name] theta34
	 	 	\item[short.name] theta34
	 	 	\item[initial] 1048576
	 	 	\item[fixed] FALSE
	 	 	\item[prior] none
	 	 	\item[param] 
	 	 	\item[to.theta] \verb!function(x) x!
	 	 	\item[from.theta] \verb!function(x) x!
	 	 \end{description}
	 	\item[theta35]\ 
	 	 \begin{description}
	 	 	\item[hyperid] 29135
	 	 	\item[name] theta35
	 	 	\item[short.name] theta35
	 	 	\item[initial] 1048576
	 	 	\item[fixed] FALSE
	 	 	\item[prior] none
	 	 	\item[param] 
	 	 	\item[to.theta] \verb!function(x) x!
	 	 	\item[from.theta] \verb!function(x) x!
	 	 \end{description}
	 	\item[theta36]\ 
	 	 \begin{description}
	 	 	\item[hyperid] 29136
	 	 	\item[name] theta36
	 	 	\item[short.name] theta36
	 	 	\item[initial] 1048576
	 	 	\item[fixed] FALSE
	 	 	\item[prior] none
	 	 	\item[param] 
	 	 	\item[to.theta] \verb!function(x) x!
	 	 	\item[from.theta] \verb!function(x) x!
	 	 \end{description}
	 	\item[theta37]\ 
	 	 \begin{description}
	 	 	\item[hyperid] 29137
	 	 	\item[name] theta37
	 	 	\item[short.name] theta37
	 	 	\item[initial] 1048576
	 	 	\item[fixed] FALSE
	 	 	\item[prior] none
	 	 	\item[param] 
	 	 	\item[to.theta] \verb!function(x) x!
	 	 	\item[from.theta] \verb!function(x) x!
	 	 \end{description}
	 	\item[theta38]\ 
	 	 \begin{description}
	 	 	\item[hyperid] 29138
	 	 	\item[name] theta38
	 	 	\item[short.name] theta38
	 	 	\item[initial] 1048576
	 	 	\item[fixed] FALSE
	 	 	\item[prior] none
	 	 	\item[param] 
	 	 	\item[to.theta] \verb!function(x) x!
	 	 	\item[from.theta] \verb!function(x) x!
	 	 \end{description}
	 	\item[theta39]\ 
	 	 \begin{description}
	 	 	\item[hyperid] 29139
	 	 	\item[name] theta39
	 	 	\item[short.name] theta39
	 	 	\item[initial] 1048576
	 	 	\item[fixed] FALSE
	 	 	\item[prior] none
	 	 	\item[param] 
	 	 	\item[to.theta] \verb!function(x) x!
	 	 	\item[from.theta] \verb!function(x) x!
	 	 \end{description}
	 	\item[theta40]\ 
	 	 \begin{description}
	 	 	\item[hyperid] 29140
	 	 	\item[name] theta40
	 	 	\item[short.name] theta40
	 	 	\item[initial] 1048576
	 	 	\item[fixed] FALSE
	 	 	\item[prior] none
	 	 	\item[param] 
	 	 	\item[to.theta] \verb!function(x) x!
	 	 	\item[from.theta] \verb!function(x) x!
	 	 \end{description}
	 	\item[theta41]\ 
	 	 \begin{description}
	 	 	\item[hyperid] 29141
	 	 	\item[name] theta41
	 	 	\item[short.name] theta41
	 	 	\item[initial] 1048576
	 	 	\item[fixed] FALSE
	 	 	\item[prior] none
	 	 	\item[param] 
	 	 	\item[to.theta] \verb!function(x) x!
	 	 	\item[from.theta] \verb!function(x) x!
	 	 \end{description}
	 	\item[theta42]\ 
	 	 \begin{description}
	 	 	\item[hyperid] 29142
	 	 	\item[name] theta42
	 	 	\item[short.name] theta42
	 	 	\item[initial] 1048576
	 	 	\item[fixed] FALSE
	 	 	\item[prior] none
	 	 	\item[param] 
	 	 	\item[to.theta] \verb!function(x) x!
	 	 	\item[from.theta] \verb!function(x) x!
	 	 \end{description}
	 	\item[theta43]\ 
	 	 \begin{description}
	 	 	\item[hyperid] 29143
	 	 	\item[name] theta43
	 	 	\item[short.name] theta43
	 	 	\item[initial] 1048576
	 	 	\item[fixed] FALSE
	 	 	\item[prior] none
	 	 	\item[param] 
	 	 	\item[to.theta] \verb!function(x) x!
	 	 	\item[from.theta] \verb!function(x) x!
	 	 \end{description}
	 	\item[theta44]\ 
	 	 \begin{description}
	 	 	\item[hyperid] 29144
	 	 	\item[name] theta44
	 	 	\item[short.name] theta44
	 	 	\item[initial] 1048576
	 	 	\item[fixed] FALSE
	 	 	\item[prior] none
	 	 	\item[param] 
	 	 	\item[to.theta] \verb!function(x) x!
	 	 	\item[from.theta] \verb!function(x) x!
	 	 \end{description}
	 	\item[theta45]\ 
	 	 \begin{description}
	 	 	\item[hyperid] 29145
	 	 	\item[name] theta45
	 	 	\item[short.name] theta45
	 	 	\item[initial] 1048576
	 	 	\item[fixed] FALSE
	 	 	\item[prior] none
	 	 	\item[param] 
	 	 	\item[to.theta] \verb!function(x) x!
	 	 	\item[from.theta] \verb!function(x) x!
	 	 \end{description}
	 	\item[theta46]\ 
	 	 \begin{description}
	 	 	\item[hyperid] 29146
	 	 	\item[name] theta46
	 	 	\item[short.name] theta46
	 	 	\item[initial] 1048576
	 	 	\item[fixed] FALSE
	 	 	\item[prior] none
	 	 	\item[param] 
	 	 	\item[to.theta] \verb!function(x) x!
	 	 	\item[from.theta] \verb!function(x) x!
	 	 \end{description}
	 	\item[theta47]\ 
	 	 \begin{description}
	 	 	\item[hyperid] 29147
	 	 	\item[name] theta47
	 	 	\item[short.name] theta47
	 	 	\item[initial] 1048576
	 	 	\item[fixed] FALSE
	 	 	\item[prior] none
	 	 	\item[param] 
	 	 	\item[to.theta] \verb!function(x) x!
	 	 	\item[from.theta] \verb!function(x) x!
	 	 \end{description}
	 	\item[theta48]\ 
	 	 \begin{description}
	 	 	\item[hyperid] 29148
	 	 	\item[name] theta48
	 	 	\item[short.name] theta48
	 	 	\item[initial] 1048576
	 	 	\item[fixed] FALSE
	 	 	\item[prior] none
	 	 	\item[param] 
	 	 	\item[to.theta] \verb!function(x) x!
	 	 	\item[from.theta] \verb!function(x) x!
	 	 \end{description}
	 	\item[theta49]\ 
	 	 \begin{description}
	 	 	\item[hyperid] 29149
	 	 	\item[name] theta49
	 	 	\item[short.name] theta49
	 	 	\item[initial] 1048576
	 	 	\item[fixed] FALSE
	 	 	\item[prior] none
	 	 	\item[param] 
	 	 	\item[to.theta] \verb!function(x) x!
	 	 	\item[from.theta] \verb!function(x) x!
	 	 \end{description}
	 	\item[theta50]\ 
	 	 \begin{description}
	 	 	\item[hyperid] 29150
	 	 	\item[name] theta50
	 	 	\item[short.name] theta50
	 	 	\item[initial] 1048576
	 	 	\item[fixed] FALSE
	 	 	\item[prior] none
	 	 	\item[param] 
	 	 	\item[to.theta] \verb!function(x) x!
	 	 	\item[from.theta] \verb!function(x) x!
	 	 \end{description}
	 	\item[theta51]\ 
	 	 \begin{description}
	 	 	\item[hyperid] 29151
	 	 	\item[name] theta51
	 	 	\item[short.name] theta51
	 	 	\item[initial] 1048576
	 	 	\item[fixed] FALSE
	 	 	\item[prior] none
	 	 	\item[param] 
	 	 	\item[to.theta] \verb!function(x) x!
	 	 	\item[from.theta] \verb!function(x) x!
	 	 \end{description}
	 	\item[theta52]\ 
	 	 \begin{description}
	 	 	\item[hyperid] 29152
	 	 	\item[name] theta52
	 	 	\item[short.name] theta52
	 	 	\item[initial] 1048576
	 	 	\item[fixed] FALSE
	 	 	\item[prior] none
	 	 	\item[param] 
	 	 	\item[to.theta] \verb!function(x) x!
	 	 	\item[from.theta] \verb!function(x) x!
	 	 \end{description}
	 	\item[theta53]\ 
	 	 \begin{description}
	 	 	\item[hyperid] 29153
	 	 	\item[name] theta53
	 	 	\item[short.name] theta53
	 	 	\item[initial] 1048576
	 	 	\item[fixed] FALSE
	 	 	\item[prior] none
	 	 	\item[param] 
	 	 	\item[to.theta] \verb!function(x) x!
	 	 	\item[from.theta] \verb!function(x) x!
	 	 \end{description}
	 	\item[theta54]\ 
	 	 \begin{description}
	 	 	\item[hyperid] 29154
	 	 	\item[name] theta54
	 	 	\item[short.name] theta54
	 	 	\item[initial] 1048576
	 	 	\item[fixed] FALSE
	 	 	\item[prior] none
	 	 	\item[param] 
	 	 	\item[to.theta] \verb!function(x) x!
	 	 	\item[from.theta] \verb!function(x) x!
	 	 \end{description}
	 	\item[theta55]\ 
	 	 \begin{description}
	 	 	\item[hyperid] 29155
	 	 	\item[name] theta55
	 	 	\item[short.name] theta55
	 	 	\item[initial] 1048576
	 	 	\item[fixed] FALSE
	 	 	\item[prior] none
	 	 	\item[param] 
	 	 	\item[to.theta] \verb!function(x) x!
	 	 	\item[from.theta] \verb!function(x) x!
	 	 \end{description}
	 \end{description}
	\item[constr] FALSE
	\item[nrow.ncol] FALSE
	\item[augmented] TRUE
	\item[aug.factor] 1
	\item[aug.constr] 1 2 3 4 5 6 7 8 9 10
	\item[n.div.by] -1
	\item[n.required] TRUE
	\item[set.default.values] TRUE
	\item[status] experimental
	\item[pdf] iidkd
\end{description}



\subsection*{Example}

Just simulate some data and estimate the parameters back. This is for
\texttt{order=}$4$.

{\small\verbatiminput{example-iidkd.R}}

\subsection*{Notes}

Note that for higher values of $k$ the only integration scheme that is
supported is \\
\texttt{control.inla=list(int.strategy="eb")}.

\end{document}
