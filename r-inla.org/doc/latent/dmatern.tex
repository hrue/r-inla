\documentclass[a4paper,11pt]{article}
\usepackage[scale={0.8,0.9},centering,includeheadfoot]{geometry}
\usepackage{amstext}
\usepackage{verbatim}
\usepackage{amsmath,amssymb}

\begin{document}
\bibliographystyle{apalike}

\section*{The dMatern model}
\subsection*{Parametrisation}

This model is the Gaussian field with a Mat\'ern correlation function,
directly, meaning \textbf{dense matrices}. This model is intended for
a low-dimension only. The correlation function is 
\begin{displaymath}
    \text{Corr}(d)
    = \frac{1}{2^{\nu-1}\Gamma(\nu)}
    \left(\kappa\, {d}\right)^{\nu}
    \;
    K_{\nu}( \kappa d), \qquad \alpha = \nu + \text{d}/2,
\end{displaymath}
where $K_{\nu}$ is the modified Bessel function and $\Gamma(\cdot)$ is
the Gamma-function. The range is \emph{defined} to be
\begin{displaymath}
    r = {\sqrt{8\nu}}/{\kappa}
\end{displaymath}
which about the distance where the covariance function becomes
about $0.1$.

\subsection*{Hyperparameters}

The hyperparameters are the precision parameter $\tau$, the range $r$
and the smoothness $\nu$, where the internal representation are
\begin{displaymath}
    \theta = (\log(\tau), \log(r), \log(\nu))
\end{displaymath}
The latent field has marginal variance $1/\tau$ and range (as defined
above) $r$.

We do \textbf{not recommend} to treat $\nu$ as random, and for this
reason it is default fixed. You can change its value by changing the
initial value.

\subsection*{Specification}

The {\tt dmatern} model is specified inside the {\tt f()} function as:
\begin{verbatim}
f(idx, model="dmatern", locations = L, hyper = <hyper>)
\end{verbatim}
where $L$ is a matrix of the locations for which the Gaussian field is
defined; row $L[i,]$ are the coordinates for the $i$'th location.
\verb|idx| represent the location indexing the corresponding row in
$L$, so $idx=3$ means location $L[idx,]$. \verb|idx| must be integers
1, 2, \ldots, nrow($L$), or \texttt{NA}.

\subsubsection*{Hyperparameter specification and default values}
%% DO NOT EDIT!
%% This file is generated automatically from models.R
\begin{description}
	\item[doc] Dense Matern covariance function
	\item[hyper]\ 
	 \begin{description}
	 	\item[theta1]\ 
	 	 \begin{description}
	 	 	\item[hyperid] 35101
	 	 	\item[name] log precision
	 	 	\item[short.name] prec
	 	 	\item[initial] 3
	 	 	\item[fixed] FALSE
	 	 	\item[prior] pc.prec
	 	 	\item[param] 1 0.01
	 	 	\item[to.theta] \verb!function(x) log(x)!
	 	 	\item[from.theta] \verb!function(x) exp(x)!
	 	 \end{description}
	 	\item[theta2]\ 
	 	 \begin{description}
	 	 	\item[hyperid] 35102
	 	 	\item[name] log range
	 	 	\item[short.name] range
	 	 	\item[initial] 0
	 	 	\item[fixed] FALSE
	 	 	\item[prior] loggamma
	 	 	\item[param] 0.1 0.1
	 	 	\item[to.theta] \verb!function(x) log(x)!
	 	 	\item[from.theta] \verb!function(x) exp(x)!
	 	 \end{description}
	 	\item[theta3]\ 
	 	 \begin{description}
	 	 	\item[hyperid] 35103
	 	 	\item[name] log nu
	 	 	\item[short.name] nu
	 	 	\item[initial] -0.693147180559945
	 	 	\item[fixed] TRUE
	 	 	\item[prior] loggamma
	 	 	\item[param] 0.5 1
	 	 	\item[to.theta] \verb!function(x) log(x)!
	 	 	\item[from.theta] \verb!function(x) exp(x)!
	 	 \end{description}
	 \end{description}
	\item[constr] FALSE
	\item[nrow.ncol] FALSE
	\item[augmented] FALSE
	\item[aug.factor] 1
	\item[aug.constr] 
	\item[n.div.by] 
	\item[n.required] TRUE
	\item[set.default.values] TRUE
	\item[pdf] dmatern
\end{description}



\subsection*{Example}

{\small\verbatiminput{dmatern-example.R}}

\subsection*{Notes}

Note that the above definition of range, might differ from the
definition in other packages. It is the same used for the SPDE-models.


\end{document}

% LocalWords:  dMatern Parametrisation ern covariance Hyperparameters dmatern
% LocalWords:  hyperparameters th idx nrow Hyperparameter SPDE
