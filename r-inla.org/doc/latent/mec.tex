\documentclass[a4paper,11pt]{article}
\usepackage[scale={0.8,0.9},centering,includeheadfoot]{geometry}
\usepackage{amstext}
\usepackage{listings}
\usepackage{verbatim}
\begin{document}

\section*{The Classical Measurement Error (MEC) model}

\subsection*{Parametrization}

This is an implementation of the classical ME model for a fixed
effect. It is best described by an example, let  the model be
\begin{displaymath}
    y = \beta x + \epsilon{}
\end{displaymath}
where $y$ is the response, $\beta$ the effect of the true covariate
$x$ with zero mean Gaussian noise $\epsilon$. The issue is that $x$ is
not observed directly, but only through $x_{\text{obs}}$, where
\begin{displaymath}
    x_{\text{obs}} = x + \nu{}
\end{displaymath}
where $\nu$ is zero mean Gaussian noise. Even though this setup is
possible to implement using basic features ("copy" and multiple
likelihoods), we provide the following model which replaces the above,
\begin{displaymath}
    y = u + \epsilon{}
\end{displaymath}
where $u$ has the correct distribution depending on various parameters:
$\beta$ has prior $\pi(\beta)$, $x$ is apriori ${\mathcal N}(\mu_{x}
{I}, \tau_{x} {I})$, and $s\times\tau_{\text{obs}}$ is the observation
precision for $x$ (ie $\text{Prec}(x_{\text{obs}}|x)$).\footnote{Note:
    The second argument in ${\mathcal N}(,)$ is the precision not the
    variance.} Here, $s$ is a vector of fixed scalings.



\subsection*{Hyperparameters}

This model has 4 hyperparameters, $\theta = (\theta_{1}, \theta_{2},
\theta_{3}, \theta_{4})$ where $\theta_{2}$, $\theta_{3}$ and
$\theta_{4}$ are default set to be fixed (ie defined to be known). The
values of $\theta_{2}, \theta_{3}$ and $\theta_{4}$ are set to mimic a
classical fixed effect, so they will always make sense. To achieve the
ME model, please use the appropriate choices for
(some of) these parameters!

The hyperparameter specification is as follows:
\begin{displaymath}
    \theta_{1} = \beta
\end{displaymath}
and the prior is defined on $\theta_{1}$,
\begin{displaymath}
    \theta_{2} = \log(\tau_{\text{obs}})
p\end{displaymath}
and the prior is defined on $\theta_{2}$,
\begin{displaymath}
    \theta_{3} = \mu_{x}
\end{displaymath}
and the prior is defined on $\theta_{3}$,
\begin{displaymath}
    \theta_{4} = \log(\tau_{\text{x}})
\end{displaymath}
and the prior is defined on $\theta_{4}$.


\subsection*{Specification}

The MEC is specified inside the {\tt f()}
function as
\begin{verbatim}
 f(x.obs, [<weights>,] model="mec", hyper = <hyper>, scale = <s>)
\end{verbatim}

The \texttt{x.obs} are the observed values of the unknown covariates
$x$, with the \emph{assumption}, that if two or more elements of
\texttt{x.obs} are \emph{identical}, then they refer to the
\emph{same} element in the true covariate $x$. The fixed scaling of
the observational precision is given in argument \texttt{scale}. If
the argument \texttt{scale} is not given, then $s$ is set to $1$.


\subsubsection*{Hyperparameter specification and default values}
\documentclass[a4paper,11pt]{article}
\usepackage[scale={0.8,0.9},centering,includeheadfoot]{geometry}
\usepackage{amstext}
\usepackage{listings}
\usepackage{verbatim}
\begin{document}

\section*{The Classical Measurement Error (MEC) model}

\subsection*{Parametrization}

This is an implementation of the classical ME model for a fixed
effect. It is best described by an example, let the model be
\begin{displaymath}
\eta = \beta x + \epsilon{}
\end{displaymath}
where $\eta$ is the linear predictor, $\beta$ the effect of the true covariate
$x$ with zero mean Gaussian noise $\epsilon$. The issue is that $x$ is
not observed directly, but only through $w$, where
\begin{displaymath}
w = x + u
\end{displaymath}
where $u$ is zero mean Gaussian noise. Even though this setup is
possible to implement using basic features ("copy" and multiple
likelihoods), we provide the following model which reparametrizes the above,
\begin{displaymath}
\eta = \nu + \epsilon{}
\end{displaymath}
where $\nu=\beta x$ has the correct distribution depending on various parameters:
$\beta$ has prior $\pi(\beta)$, and $x$ is apriori ${\mathcal N}(\mu_{x}
{I}, \tau_{x} {I})$\footnote{Note:
The second argument in ${\mathcal N}(,)$ is the precision not the
variance.}. The error is apriori $u\sim\mathcal{N}(0,\tau_u \mathbf{D})$, where $\tau_u$ is the observational precision of the error $\text{Prec}(u|x)$) with possible heteroscedasticy, encoded in the entries $d_i$ of the diagonal matrix $\mathbf{D}$. The vector $s$ contains the fixed scalings $s=(d_1,\ldots,d_n)$ (with $n$ the number of data points).



\subsection*{Hyperparameters}

This model has 4 hyperparameters, $\theta = (\theta_{1}, \theta_{2},
\theta_{3}, \theta_{4})$ where $\theta_{2}$, $\theta_{3}$ and
$\theta_{4}$ are default set to be fixed (ie defined to be known). The
values of $\theta_{2}, \theta_{3}$ and $\theta_{4}$ are set to mimic a
classical fixed effect, so they will always make sense. To achieve the
ME model, please use the appropriate choices for
(some of) these parameters!

The hyperparameter specification is as follows:
\begin{displaymath}
\theta_{1} = \beta
\end{displaymath}
and the prior is defined on $\theta_{1}$,
\begin{displaymath}
\theta_{2} = \log(\tau_u)
\end{displaymath}
and the prior is defined on $\theta_{2}$,
\begin{displaymath}
\theta_{3} = \mu_{x}
\end{displaymath}
and the prior is defined on $\theta_{3}$,
\begin{displaymath}
\theta_{4} = \log(\tau_{\text{x}})
\end{displaymath}
and the prior is defined on $\theta_{4}$.


\subsection*{Specification}

The MEC is specified inside the {\tt f()}
function as
\begin{verbatim}
f(w, [<weights>,] model="mec", scale = <s>, values= <w>, hyper = <hyper>)
\end{verbatim}

The \texttt{w} are the observed values of the true but unknown covariates
$x$, with the \emph{assumption}, that if two or more elements of
\texttt{w} are \emph{identical}, then they refer to the
\emph{same} element in the true covariate $x$. If data points with identical $w$ values belong to different $x$ values (e.g., different individuals), please add a \emph{tiny} random value to $w$ to make this difference obvious to the model.

The fixed scaling of
the observational precision is given in argument \texttt{scale}. If
the argument \texttt{scale} is not given, then $s$ is set to $1$.


\subsubsection*{Hyperparameter specification and default values}
\documentclass[a4paper,11pt]{article}
\usepackage[scale={0.8,0.9},centering,includeheadfoot]{geometry}
\usepackage{amstext}
\usepackage{listings}
\usepackage{verbatim}
\begin{document}

\section*{The Classical Measurement Error (MEC) model}

\subsection*{Parametrization}

This is an implementation of the classical ME model for a fixed
effect. It is best described by an example, let the model be
\begin{displaymath}
\eta = \beta x + \epsilon{}
\end{displaymath}
where $\eta$ is the linear predictor, $\beta$ the effect of the true covariate
$x$ with zero mean Gaussian noise $\epsilon$. The issue is that $x$ is
not observed directly, but only through $w$, where
\begin{displaymath}
w = x + u
\end{displaymath}
where $u$ is zero mean Gaussian noise. Even though this setup is
possible to implement using basic features ("copy" and multiple
likelihoods), we provide the following model which reparametrizes the above,
\begin{displaymath}
\eta = \nu + \epsilon{}
\end{displaymath}
where $\nu=\beta x$ has the correct distribution depending on various parameters:
$\beta$ has prior $\pi(\beta)$, and $x$ is apriori ${\mathcal N}(\mu_{x}
{I}, \tau_{x} {I})$\footnote{Note:
The second argument in ${\mathcal N}(,)$ is the precision not the
variance.}. The error is apriori $u\sim\mathcal{N}(0,\tau_u \mathbf{D})$, where $\tau_u$ is the observational precision of the error $\text{Prec}(u|x)$) with possible heteroscedasticy, encoded in the entries $d_i$ of the diagonal matrix $\mathbf{D}$. The vector $s$ contains the fixed scalings $s=(d_1,\ldots,d_n)$ (with $n$ the number of data points).



\subsection*{Hyperparameters}

This model has 4 hyperparameters, $\theta = (\theta_{1}, \theta_{2},
\theta_{3}, \theta_{4})$ where $\theta_{2}$, $\theta_{3}$ and
$\theta_{4}$ are default set to be fixed (ie defined to be known). The
values of $\theta_{2}, \theta_{3}$ and $\theta_{4}$ are set to mimic a
classical fixed effect, so they will always make sense. To achieve the
ME model, please use the appropriate choices for
(some of) these parameters!

The hyperparameter specification is as follows:
\begin{displaymath}
\theta_{1} = \beta
\end{displaymath}
and the prior is defined on $\theta_{1}$,
\begin{displaymath}
\theta_{2} = \log(\tau_u)
\end{displaymath}
and the prior is defined on $\theta_{2}$,
\begin{displaymath}
\theta_{3} = \mu_{x}
\end{displaymath}
and the prior is defined on $\theta_{3}$,
\begin{displaymath}
\theta_{4} = \log(\tau_{\text{x}})
\end{displaymath}
and the prior is defined on $\theta_{4}$.


\subsection*{Specification}

The MEC is specified inside the {\tt f()}
function as
\begin{verbatim}
f(w, [<weights>,] model="mec", scale = <s>, values= <w>, hyper = <hyper>)
\end{verbatim}

The \texttt{w} are the observed values of the true but unknown covariates
$x$, with the \emph{assumption}, that if two or more elements of
\texttt{w} are \emph{identical}, then they refer to the
\emph{same} element in the true covariate $x$. If data points with identical $w$ values belong to different $x$ values (e.g., different individuals), please add a \emph{tiny} random value to $w$ to make this difference obvious to the model.

The fixed scaling of
the observational precision is given in argument \texttt{scale}. If
the argument \texttt{scale} is not given, then $s$ is set to $1$.


\subsubsection*{Hyperparameter specification and default values}
\documentclass[a4paper,11pt]{article}
\usepackage[scale={0.8,0.9},centering,includeheadfoot]{geometry}
\usepackage{amstext}
\usepackage{listings}
\usepackage{verbatim}
\begin{document}

\section*{The Classical Measurement Error (MEC) model}

\subsection*{Parametrization}

This is an implementation of the classical ME model for a fixed
effect. It is best described by an example, let the model be
\begin{displaymath}
\eta = \beta x + \epsilon{}
\end{displaymath}
where $\eta$ is the linear predictor, $\beta$ the effect of the true covariate
$x$ with zero mean Gaussian noise $\epsilon$. The issue is that $x$ is
not observed directly, but only through $w$, where
\begin{displaymath}
w = x + u
\end{displaymath}
where $u$ is zero mean Gaussian noise. Even though this setup is
possible to implement using basic features ("copy" and multiple
likelihoods), we provide the following model which reparametrizes the above,
\begin{displaymath}
\eta = \nu + \epsilon{}
\end{displaymath}
where $\nu=\beta x$ has the correct distribution depending on various parameters:
$\beta$ has prior $\pi(\beta)$, and $x$ is apriori ${\mathcal N}(\mu_{x}
{I}, \tau_{x} {I})$\footnote{Note:
The second argument in ${\mathcal N}(,)$ is the precision not the
variance.}. The error is apriori $u\sim\mathcal{N}(0,\tau_u \mathbf{D})$, where $\tau_u$ is the observational precision of the error $\text{Prec}(u|x)$) with possible heteroscedasticy, encoded in the entries $d_i$ of the diagonal matrix $\mathbf{D}$. The vector $s$ contains the fixed scalings $s=(d_1,\ldots,d_n)$ (with $n$ the number of data points).



\subsection*{Hyperparameters}

This model has 4 hyperparameters, $\theta = (\theta_{1}, \theta_{2},
\theta_{3}, \theta_{4})$ where $\theta_{2}$, $\theta_{3}$ and
$\theta_{4}$ are default set to be fixed (ie defined to be known). The
values of $\theta_{2}, \theta_{3}$ and $\theta_{4}$ are set to mimic a
classical fixed effect, so they will always make sense. To achieve the
ME model, please use the appropriate choices for
(some of) these parameters!

The hyperparameter specification is as follows:
\begin{displaymath}
\theta_{1} = \beta
\end{displaymath}
and the prior is defined on $\theta_{1}$,
\begin{displaymath}
\theta_{2} = \log(\tau_u)
\end{displaymath}
and the prior is defined on $\theta_{2}$,
\begin{displaymath}
\theta_{3} = \mu_{x}
\end{displaymath}
and the prior is defined on $\theta_{3}$,
\begin{displaymath}
\theta_{4} = \log(\tau_{\text{x}})
\end{displaymath}
and the prior is defined on $\theta_{4}$.


\subsection*{Specification}

The MEC is specified inside the {\tt f()}
function as
\begin{verbatim}
f(w, [<weights>,] model="mec", scale = <s>, values= <w>, hyper = <hyper>)
\end{verbatim}

The \texttt{w} are the observed values of the true but unknown covariates
$x$, with the \emph{assumption}, that if two or more elements of
\texttt{w} are \emph{identical}, then they refer to the
\emph{same} element in the true covariate $x$. If data points with identical $w$ values belong to different $x$ values (e.g., different individuals), please add a \emph{tiny} random value to $w$ to make this difference obvious to the model.

The fixed scaling of
the observational precision is given in argument \texttt{scale}. If
the argument \texttt{scale} is not given, then $s$ is set to $1$.


\subsubsection*{Hyperparameter specification and default values}
\input{../hyper/latent/mec.tex}

\subsection*{Example}

\verbatiminput{mec-example.R}

\subsection*{Notes}

\begin{itemize}
\item \texttt{INLA} provides the posteriors of $\nu_i=\beta x_i$ and NOT $x_i$.
\item The posteriors of $\nu_i$ come (default) in the order given by the sorted
(from low to high) values of \texttt{w}. The entry \verb|$ID|
gives the mapping.
\item The option \verb|scale| defines the scaling in the same order as
argument \verb|values|. It is therefore adviced to also give
argument \verb|values| when \verb|scale| is used to be sure that
they are consistent.
\end{itemize}


\end{document}



%%% Local Variables:
%%% TeX-master: t
%%% End:


\subsection*{Example}

\verbatiminput{mec-example.R}

\subsection*{Notes}

\begin{itemize}
\item \texttt{INLA} provides the posteriors of $\nu_i=\beta x_i$ and NOT $x_i$.
\item The posteriors of $\nu_i$ come (default) in the order given by the sorted
(from low to high) values of \texttt{w}. The entry \verb|$ID|
gives the mapping.
\item The option \verb|scale| defines the scaling in the same order as
argument \verb|values|. It is therefore adviced to also give
argument \verb|values| when \verb|scale| is used to be sure that
they are consistent.
\end{itemize}


\end{document}



%%% Local Variables:
%%% TeX-master: t
%%% End:


\subsection*{Example}

\verbatiminput{mec-example.R}

\subsection*{Notes}

\begin{itemize}
\item \texttt{INLA} provides the posteriors of $\nu_i=\beta x_i$ and NOT $x_i$.
\item The posteriors of $\nu_i$ come (default) in the order given by the sorted
(from low to high) values of \texttt{w}. The entry \verb|$ID|
gives the mapping.
\item The option \verb|scale| defines the scaling in the same order as
argument \verb|values|. It is therefore adviced to also give
argument \verb|values| when \verb|scale| is used to be sure that
they are consistent.
\end{itemize}


\end{document}



%%% Local Variables:
%%% TeX-master: t
%%% End:


\subsection*{Example}

\verbatiminput{mec-example.R}

\subsection*{Notes}

\begin{itemize}
\item \texttt{INLA} provide the posterior of $u$ and NOT $x$.
\item The posterior of $u$ comes in the order given by the sorted
    (from low to high) values of \texttt{x.obs}. The entry \verb|$ID|
    gives the mapping.
\end{itemize}


\end{document}



%%% Local Variables: 
%%% TeX-master: t
%%% End: 

