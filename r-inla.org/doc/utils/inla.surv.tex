\documentclass[a4paper,11pt]{article}
\usepackage[scale={0.8,0.9},centering,includeheadfoot]{geometry}
\usepackage{amstext}
\usepackage{amsmath}
\usepackage{verbatim}

\begin{document}
\section*{inla.surv}
The routines in R-INLA work with objects of class "inla.surv", which is a data structure that combines times, censoring, truncation and subject number information. Such objects are constructed using the 'inla.surv' function. 'inla.surv' works for both 
\begin{itemize}
\item event time data(an event per subject)
\item longitudinal data(multiple events per subject)
\end{itemize}


\textbf{For event time data}, it takes four arguments: \\
an observation time, an event indicator(censoring indicator), an ending time for interval censored data and left  truncation time. This can be done using the following command:
\small
\begin{verbatim}
inla.surv(time, event, time2, truncation)
\end{verbatim}
\small
The observation time is the follow up time for right censored data and starting time for interval censored data. The event indicator can be coded as following: 
\begin{description}
\item$1$ for uncensored observation, 
\item$0$ for right censored observation, 
\item$2$ for left censored data and 
\item$3$ for interval censored data. 
\end{description}
The left truncation time if missing is considered $0$.\\


\textbf{For longitudinal data}, it takes three arguments: \\
an observation time, an event indicator(censoring indicator) and subject indicator. This can be done using the following command:
\small
\begin{verbatim}
inla.surv(time, event, subject=subject)
\end{verbatim}
\small
The observation time is the time of detecting an event, event is a censoring indicator and subject is subject indicator. The event indicator can be coded as following: 
\begin{description}
\item$1$ if an event is detected, 
\item$0$ if an event not detected 
\end{description}



The out put of 'inla.surv' is different for event time data(when there is one event per subject) and longitudinal data( multiple events per subject).
\begin{itemize}
\item event time data:\\
For,  such data, the out put of 'inla.surv' is a data frame consisting of 5 columns, the names of which can be seen using the command,
\small
\begin{verbatim}
names(inla.surv(time, event, time2, truncation))
\end{verbatim}
\small
resulting in:
\small
\begin{verbatim}
[1] "time"       "lower"      "upper"      "event"      "truncation"
\end{verbatim}
\small
\item longitudinal data: \\
For such data the out put of 'inla.surv' is a data frame consisting of 3 columns, the names of which can be seen using the command,
\small
\begin{verbatim}
names(inla.surv(time, event, time2, truncation, subject=subject))
\end{verbatim}
\small
resulting in:
\small
\begin{verbatim}
[1] "time"         "event"      "subject"
\end{verbatim}
\small
\end{itemize}
A print method is associated with the 'inla.surv'object that displays the objects in special format, with a '$+$' marking censoring observations. 









\end{document}